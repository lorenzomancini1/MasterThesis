\providecommand{\rootdir}{../..}
\providecommand{\fontpath}{\rootdir/fonts}
\documentclass[\rootdir/main.tex]{subfiles}
\addbibresource{\rootdir/references.bib}
\begin{document}
\chapter{Conclusions}\label{chap:conclusions}
The main goal of this thesis is to provide numerical results related to the properties of the Basins of Attractions for both the \acrlong{shm} and the \acrlong{mhm}.\\
As we have discussed, such models work as \acrlong{cam} in the sense that they are able to store some patterns -- through the introduction of a suitable energy function -- and retrieve them. The energy associates each memory (pattern) to a global or local free-energy minimum given that some conditions are verified, thus forming the so-called \acrlong{ba} for which patterns are attractors. A noisy memory tends towards the relative fixed point under a (quasi)-deterministic dynamics, clearly depending on the amount of noise. Our numerical simulations show that there is a sharp threshold on the allowed noise below which the fixed point of a certain basin can be reached. We refer to this threshold as the (average) radius of a \acrshort{ba}. Starting from finite size configurations, we measure the retrieval probability as a function of the spin-flip noise. This is a sigmoid-like function that, as the size of the configurations increases, tends towards a sharp step-function. Hence, finite size extrapolations allow us to infer the value of such critical noise.\\
We observe that the average radius is a decreasing function of the \emph{load parameter} $\alpha$, which governs the storage capacity. As $\alpha \to 0$ we obtain $p_c \to 0.5$, \ie the model is able to retrieve even an orthogonal configuration. On the other hand, when we approach $\alpha_c$ the radius goes continuously to zero, thus displaying a second order phase transition.\\
Furthermore, we find out that the radius is not isotropic, as suggested by simulations where the perturbation is performed along the direction of the most correlated memory. Indeed, in the first case the basins are wider.
However, this first set of simulations suggests that retrieval is computationally efficient given that enough information on the memories is available.

We also try to get some insights about the phase diagram of the \acrshort{mhm} in the small $\lambda$ region. Here the storage capacity grows exponentially with the size of the memories, thus making numerical simulations computationally costly. We try to compute the critical load parameter, $\alpha_c$ by analyzing the behaviour of the \emph{escape probability}. In other words, we look for the value of $\alpha$ above which any zero-temperature dynamics starting from one of the stored patterns moves away, \ie \acrshort{ba} disappear.
We find a very good result for the \acrshort{shm} which can be compared with other simulations and theoretical results present in the scientific literature. Nevertheless, for the \acrshort{mhm} we obtain a result which is slightly different to the $\alpha_c$ for which the radius of the basins disappear. This might be due to the fact that we are considering very small sizes, leading to relevant finite size effects. The same argument is still under investigation from a theoretical point of view (Lucibello and Mézard).

Finally, we introduce the \emph{matrix factorization problem} which is only related to the \acrlong{shm} as the modern one does not have a coupling matrix in the energy computation. Here, the question that arises is whether it is possible or not to infer the $M$ patterns that generated a given coupling matrix $J_{N \times N}$.
Since we have no information about memories -- \ie we do not have noisy version as before -- any minimization algorithm starts from a random configuration. The result is that, following this procedure, our dynamics gets trapped into states which tend to be orthogonal to all the stored memories as $N$ grows to infinite (\acrshort{sg}). However, we show some possible algorithms with the aim of finding global minima of the free energy in the $\alpha < 0.05$ phase. It seems that this is possible for a certain $N$ given that we let the dynamics explore the free energy landscape long enough and with a proper temperature. In this sense, we show -- not in a rigorous way -- that the number of sweeps required to factorize $J$ grows exponentially with $N$, thus making this a computationally hard problem.

Future developments certain include a more in-depth study of the phase diagram of the \acrlong{mhm} from both the analytical and numerical perspectives. Moreover, we need to repeat all the analysis from scratch for the \acrlong{chm} where some approaches might be different. Finally, some new and well-tuned algorithms should be explored for the matrix factoring problem.
\end{document}