\providecommand{\rootdir}{../..}
\providecommand{\imgpath}{\rootdir/images/chap2}
\providecommand{\fontpath}{\rootdir/fonts}
\documentclass[\rootdir/main.tex]{subfiles}
\addbibresource{\rootdir/references.bib}
\begin{document}
\chapter{Numerical simulations}\label{chap:numerical_sim}
Numerical simulations are very useful when analytical analysis becomes too hard or impossible even if some approximation are introduced. This could be the case of the Ising model with dimension $d > 3$ where no analytical solution exists and the results present in the literature are obtained by means of Monte Carlo simulations~\parencite[\eg][]{Cosme_2015,3d_ising}.\\
For our purpose, the key procedure to be performed is to retrieve stored patterns through energy minimization. However, at least in the binary cases, the energy landscape of the \acrlong{hm} is very complicated. For this reason we need to exploit some kind of stochastic dynamics rather than the update rules defined in \cref{eq:neur_dynamics}.\\
This is done through a class of methods called \acrfull{mcmc}. As we will see in \cref{sec:numsim_continuous}, the situation is different for the continuous case where the update rule is given by \cref{eq:update_rule_continuous}.

\section{Markov Chains}
A Markov Chain is defined as a \emph{stochastic process} where each state depends only on the previous one, \ie it satisfies the Markov Property.\\ 
In mathematical terms,  to have a finite set $S$ called state space and a sequence $x = \left(x_n \right)_{n \geq 0}$ with elements in $S$. We can set:
\begin{equation*}
    x_m^n = \left(x_m, x_{m+1}, \dots, x_n \right).
\end{equation*}
\begin{definition}[Markov Property]\label{def:markov_property}
    A Markov Chain with values in $S$ is a sequence $\left(X_n \right)_{n \geq 0}$ of $S$-valued random variables such that for each $n \geq 0$ and each sequence $x = \left(x_n \right)_{n \geq 0}$ with elements in $S$:
    \begin{equation}\label{eq:markov_property}
         \mathrm{P}\left(X_{n+1}=x_{n+1} \mid X_0^n=x_0^n\right)=\mathrm{P}\left(X_{n+1}=x_{n+1} \mid X_n=x_n\right).
\end{equation}
\end{definition}
In \cref{eq:markov_property} the conditional probabilities $\mathrm{P}_{yx} \coloneq \mathrm{P}\left(X_{n+1}=x \mid X_n=y\right)$ are ofter referred to as \emph{one step transition probabilities}. If $\mathrm{P}_{yx}$ does not depend on $n$, then the chain is said to be \emph{homogeneous}. If all the one step transition probabilities are known, one can define a $|S| \times |S|$ matrix called \emph{transition matrix} $P$.
\begin{definition}[Stationary distribution]
    A distribution $\symbf{\pi}$ on $S$ is called \emph{stationary} for an homogeneous Markov Chain if
    \begin{equation*}
        \symbf{\pi} P = \symbf{\pi},
    \end{equation*}
    where $P$ is the transition matrix of the chain.
\end{definition}

\begin{definition}[Reversible distribution and detailed balance]\label{def:detailed_balance}
    A distribution $\symbf{\pi}$ is called \emph{reversible} for a Markov Chain if, for $x,\,y \in S$ the \emph{detailed balance} condition holds:
    \begin{equation*}
        \pi_x P_{xy} = \pi_y P_{yx}
    \end{equation*}
\end{definition}
It can be shown that a reversible distribution is stationary.
\begin{definition}[Irreducible transition matrix]
    A transition matrix $P$ is called \emph{irreducible} if and only if for every $x \neq y$ there exist $n \geq 1$ such that $ \mathrm{P}_{xy}^n > 0$.
\end{definition}
The last definition just tells that $P$ is irreducible if and only if we can move from $x$ to $y$ in a certain number of steps $n$ bigger or equal than one.
\begin{definition}[Period of a state]
    Given a state $x \in S$ and a transition matrix $P$, the \emph{period} of $x$ is the greatest common divisor of $n$ such that $\mathrm{P}_{xx}^n > 0$.
\end{definition}
It is not hard to prove that if the Markov Chain is irreducible then all the states have the same period. In this last case, if the period is equal to $1$, the chain is called \emph{aperiodic}.
If $P$ is irreducible then there exist a unique stationary distribution $\symbf{\pi}$ and it has positive entries, \ie $\symbf{\pi}_x > 0 \forall x \in S$. Moreover, if it is also aperiodic, every initial distribution $\symbf{\nu}$ will tend to the stationary one $\symbf{\pi}$.\\
The goal of \acrlong{mcmc} is to \emph{sample} from a stationary distribution of a Markov Chain when direct sampling of i.i.d random variables is computationally hard.
In other words, our goal is to generate an irreducible Markov Chain $\left(X_n \right)_{n \geq 0}$ with a certain stationary distribution $\symbf{\pi}$ that is our target distribution.\\
For our simulations we relied on the well known Metropolis algorithm~\cite{metropolis}. 

\section{Metropolis algorithm}\label{sec:metr-lg}
This algorithm is based on the assumptions that there exist a stationary distribution and this distribution is unique. For the existence we have seed that \cref{def:detailed_balance} is a sufficient condition. For the second requirement we need aperiodicity and positive recurrence. Detailed balance can be rewritten as:
\begin{equation*}
    \frac{P_{xy}}{P_{yx}} = \frac{\pi_y}{\pi_{x}}.
\end{equation*}
The next step is to separate the transition probability into a product of two terms, the proposal distribution $g\left(y | x \right)$, \ie the conditional probability of proposing a new state $y$ given $x$, and the acceptance distribution $A(y,x)$, \ie probability of accepting the proposed state $y$:
\begin{equation*}
    P_{xy} = A(y, x)g(y | x)
\end{equation*}
Hence, from the previous equation it follows that:
\begin{equation*}
    \frac{A(y, x)}{A(x, y)} = \frac{\pi_y}{\pi_x} \frac{g(x | y)}{g(y | x)}.
\end{equation*}
We can choose $A(\cdot | \cdot)$ in multiple different ways. Usually, following the Metropolis choice we set:
\begin{equation}
    A(y , x) = \operatorname{min} \left(1,  \frac{\pi_y}{\pi_x} \frac{g(x | y)}{g(y | x)}\right)
\end{equation}
Thus, the Metropolis algorithm can be summarized as follows.
\begin{algorithm}
    \caption{General Metropolis algorithm}
    \label{alg:1}
    \begin{algorithmic}[1]
    \Require $x_0$, $t_{max} > 0$
        \For{$t = 1$ to $t_{max} + 1$}
        \State sample $x_{t} \sim g(x' | x_{t-1}) $ 
        \State sample $u \sim U(0, 1)$
        \If {$u \leq A(x', x_t)$}
            \State $x_{t-1} \gets x_{t} $
        \EndIf
        \EndFor
    \end{algorithmic}
\end{algorithm}
When dealing with binary states -- which is our case except for the continuous Hopfield -- a \emph{single spin-flip dynamics} can be applied. The underlying idea is pretty simple. At each step a random neuron is selected and if the energy variation that would occur by flipping that neuron (spin) is negative, we accept the new state with probability one. Instead, if it is positive we accept the new state with a suitable probability. It is important to note that this last step --that as we will show in a moment is strictly dependent on the temperature of the system-- is what introduces \emph{stochasticity} in our dynamics. If our configurations are made by $N$ neurons, the proposal distribution of the single spin-flip dynamics is just the uniform distribution $g(y | x) = 1/N$. On the other hand, if the equilibrium distribution is given by \cref{eq:prob_configuration} we have that:
\begin{equation*}
    \frac{A(y, x)}{A(x, y)} = e^{- \beta \Delta E},
\end{equation*}
where $y$ is the state in which a random neuron is flipped and $\Delta E$ is the consequent energy variation.\\
By setting the maximum between $A(x,\, y)$ and $A(y,\, x)$ to $1$, one gets~\cite{binder2010monte}:
\begin{equation}\label{eq:acceptance_prob}
A(x, \,y) = 
\left\{
\begin{alignedat}{3}
% R & L   &  R & L   &  R & L 
 & e^{-\beta \Delta E} \qquad & \Delta  E > 0 \\
 & 1 \qquad & \text{otherwise}
\end{alignedat}
\right.
\end{equation}
In the following sections we're going to show details on the Monte Carlo algorithms used for the simulations.
Some initial examples are also shown.

\section{Standard Hopfield Model}
As stated by \cref{eq:acceptance_prob} we need the enenrgy variation which derives from a spin-flip in a given configuration. For this purpose, whenever possible it is better to find an analytic expression for $\Delta E$, since in this way we can significantly reduce the computational cost. This is not true in general.\\
For the \acrlong{shm} this expression can be derived without much efforts. Let's say that we have a configuration $\symbf{\sigma} \in \{-1, +1 \}^N$ and we choose to flip the $k$-th neuron. We indicate with $\symbf{\sigma}_i$ and $\symbf{\sigma}_f$ the initial and final configurations respectively. Obviously those are equal except for the $k$-th element.
\begin{equation*}
\begin{split}
        E(\symbf{\sigma}_i) & = \frac{1}{2} \sum_{(i, j) \neq k} J_{ij} \sigma_i \sigma_j + \frac{1}{2}\left(\sum_j J_{kj} \sigma_k \sigma_j + \sum_i J_{ik} \sigma_i \sigma_k \right) \\
        & = \frac{1}{2} \sum_{(i, j) \neq k} J_{ij} \sigma_i \sigma_j + \sum_j J_{kj} \sigma_k \sigma_j,
\end{split}
\end{equation*}
where the terms inside the parentheses are equal since the matrix $J$ is symmetric.\\
Similarly, one gets:
\begin{equation*}
        E(\symbf{\sigma}_f) = \frac{1}{2} \sum_{(i, j) \neq k} J_{ij} \sigma_i \sigma_j + \sum_j J_{kj} \left(-\sigma_k\right) \sigma_j,
\end{equation*}
Hence, the variation in energy reads:
\begin{equation}\label{eq:energy_var_shm}
    \Delta E \equiv E(\symbf{\sigma}_f) - E(\symbf{\sigma}_i) = -2 \sigma_k \sum_j J_{kj} = -2 \sigma_k \sum_i J_{ik}.
\end{equation}

\begin{algorithm}
    \caption{Metropolis algorithm for the \acrlong{shm}}
    \label{alg:shm_metropolis}
    \begin{algorithmic}[1]
    \Require $\symbf{\sigma}$, $J$, $\beta$
        \State $N \gets |\symbf{\sigma}|$ 
        \State fliprate $\gets 0$
        \For{$i$ in random permutation $N$}
        \State Compute $\Delta E$ using \cref{eq:energy_var_shm}
        \If {$\Delta E < 0$ or $u \sim U(0,1) < e^{-\beta \Delta E}$}
            \State $\sigma[i] \gets - \sigma[i]$
            \State fliprate $\gets$ fliprate + 1
        \EndIf
        \EndFor
    \State \textbf{return} $\symbf{\sigma}$, $\text{fliprate} / N$
    \end{algorithmic}
\end{algorithm}

It is important to notice that the computational complexity of \cref{alg:shm_metropolis} is due to the computation of $\Delta E$. However, \cref{eq:energy_var_shm} allows the previous algorithm to be of order $O(N)$. Instead, if we had calculated this value by taking the difference between the final and the initial energies, we would have obtained an order complexity of the order $O(N^2)$. As we will see in \cref{sec:numsim_mh} this analytical trick does not lead to particular advantages.

The fliprate parameter just tells what fraction of flips we obtain in $N$ trials. This is useful to set up a stopping condition for the algorithm if the number of flips goes below a certain threshold. If it is $0$ it means that the state arrived to a minimimum (either local or global).
Summarizing everything, the procedure of minimization is quite simple: we choose a random neuron and we flip it, if the consequent $\Delta E$ is negative we accept the flip, otherwise we accept it with a probability that depends on the temperature ($\beta = 1/ T$). An high value of temperature makes the state able to ``jump out'' of local minima, whereas, for $T \to 0$ the dynamics becomes deterministic and even a very small barrier in the energy landscape traps our configuration.\\
All the steps described in \cref{alg:shm_metropolis} define what is usually addressed to as a Monte Carlo sweep. Shortly speaking, the full \acrlong{mcmc} algorithm executes the Metropolis algorithm for a certain number of times that is the number of sweeps that one wants to perform.

The full Monte Carlo algorithm is shown in \cref{alg:shm_mcmc}. For most of our computations we set the fliprate threshold to zero so that the algorithm stops if no more spin-flips occur. Moreover, one might need to allow the system to visit states with higher energy, thus making the dynamics more ``stochastic''. One common procedure is to gradually reduce such randomness as a function of the number of sweeps. Such method is called \acrfull{sa} since we start with a low $\beta$ (high $T$) and we ``cool down'' the dynamics until we reach a very high $\beta$ (low $T$). In this way the system should have enough momentum to avoid local minima and to reach the global ones. However, we will make use of \acrlong{sa} in \cref{chap:fact}.

\begin{algorithm}
    \caption{\acrlong{mcmc} for the \acrlong{shm}}
    \label{alg:shm_mcmc}
    \begin{algorithmic}[1]
    \Require $\symbf{\sigma}$, $J$, $\beta$, nsweeps, earlystop
        \State $\symbf{\sigma_{\text{rec}}} \gets \symbf{\sigma}$ 
        \For{$\_ = 1$ to nsweeps}
        \State $\symbf{\sigma_{\text{rec}}},\, \text{fliprate} \gets \text{metropolis}\left(\symbf{\sigma_{\text{rec}}},\, J,\, \beta \right)$  
        \If {fliprate $\leq$ earlystop}
            \State break
        \EndIf
        \EndFor
    \State \textbf{return} $\symbf{\sigma_{\text{rec}}}$
    \end{algorithmic}
\end{algorithm}

The spirit of the \acrlong{hm} lies in the fact of being able to retrieve a stored pattern whenever it is presented with a perturbed version of the original configuration.\\
Most of our numerical simulations make use of \cref{alg:shm_mcmc} where $\symbf{\sigma}$ is a noisy pattern. In this regard, we refer to $p$ as the \emph{perturb} probability (or spin-flip probability) used to perturb a pattern. Shortly speaking, given $\symbf{\xi}^{\mu}$, we flip each neuron with probability $p$. This also applies for the \acrlong{mhm} in the binary case. Instead for the continuous version we need to introduce a gaussian noise \cref{sec:numsim_continuous}.

\begin{comment}
We store $M = N \alpha$ patterns of length $N = 1000$. A random pattern is taken and perturbed with spin-flip probability $p = 0.2$ and the \acrlong{mcmc} is executed on the perturbed configuration for $\texttt{nsweeps} = 100$ with $\beta = 100$. Then the overlap between the retrieved state and the original one is measured. In \cref{fig:shm:p02:final_overlaps1} it can be clearly observed that there is a sharp peak centered at very high magnetization \ie $m \approx 1$. However, also a broad distribution centered at a lower value of $m$ appears.  
\begin{figure}
    \centering
    %% Creator: Matplotlib, PGF backend
%%
%% To include the figure in your LaTeX document, write
%%   \input{<filename>.pgf}
%%
%% Make sure the required packages are loaded in your preamble
%%   \usepackage{pgf}
%%
%% Also ensure that all the required font packages are loaded; for instance,
%% the lmodern package is sometimes necessary when using math font.
%%   \usepackage{lmodern}
%%
%% Figures using additional raster images can only be included by \input if
%% they are in the same directory as the main LaTeX file. For loading figures
%% from other directories you can use the `import` package
%%   \usepackage{import}
%%
%% and then include the figures with
%%   \import{<path to file>}{<filename>.pgf}
%%
%% Matplotlib used the following preamble
%%   
%%   \usepackage{fontspec}
%%   \makeatletter\@ifpackageloaded{underscore}{}{\usepackage[strings]{underscore}}\makeatother
%%
\begingroup%
\makeatletter%
\begin{pgfpicture}%
\pgfpathrectangle{\pgfpointorigin}{\pgfqpoint{5.905512in}{3.649807in}}%
\pgfusepath{use as bounding box, clip}%
\begin{pgfscope}%
\pgfsetbuttcap%
\pgfsetmiterjoin%
\definecolor{currentfill}{rgb}{1.000000,1.000000,1.000000}%
\pgfsetfillcolor{currentfill}%
\pgfsetlinewidth{0.000000pt}%
\definecolor{currentstroke}{rgb}{1.000000,1.000000,1.000000}%
\pgfsetstrokecolor{currentstroke}%
\pgfsetdash{}{0pt}%
\pgfpathmoveto{\pgfqpoint{0.000000in}{0.000000in}}%
\pgfpathlineto{\pgfqpoint{5.905512in}{0.000000in}}%
\pgfpathlineto{\pgfqpoint{5.905512in}{3.649807in}}%
\pgfpathlineto{\pgfqpoint{0.000000in}{3.649807in}}%
\pgfpathlineto{\pgfqpoint{0.000000in}{0.000000in}}%
\pgfpathclose%
\pgfusepath{fill}%
\end{pgfscope}%
\begin{pgfscope}%
\pgfsetbuttcap%
\pgfsetmiterjoin%
\definecolor{currentfill}{rgb}{1.000000,1.000000,1.000000}%
\pgfsetfillcolor{currentfill}%
\pgfsetlinewidth{0.000000pt}%
\definecolor{currentstroke}{rgb}{0.000000,0.000000,0.000000}%
\pgfsetstrokecolor{currentstroke}%
\pgfsetstrokeopacity{0.000000}%
\pgfsetdash{}{0pt}%
\pgfpathmoveto{\pgfqpoint{0.634445in}{0.549444in}}%
\pgfpathlineto{\pgfqpoint{5.755512in}{0.549444in}}%
\pgfpathlineto{\pgfqpoint{5.755512in}{3.499807in}}%
\pgfpathlineto{\pgfqpoint{0.634445in}{3.499807in}}%
\pgfpathlineto{\pgfqpoint{0.634445in}{0.549444in}}%
\pgfpathclose%
\pgfusepath{fill}%
\end{pgfscope}%
\begin{pgfscope}%
\pgfpathrectangle{\pgfqpoint{0.634445in}{0.549444in}}{\pgfqpoint{5.121067in}{2.950363in}}%
\pgfusepath{clip}%
\pgfsetbuttcap%
\pgfsetroundjoin%
\pgfsetlinewidth{0.803000pt}%
\definecolor{currentstroke}{rgb}{0.690196,0.690196,0.690196}%
\pgfsetstrokecolor{currentstroke}%
\pgfsetstrokeopacity{0.300000}%
\pgfsetdash{{2.960000pt}{1.280000pt}}{0.000000pt}%
\pgfpathmoveto{\pgfqpoint{1.420960in}{0.549444in}}%
\pgfpathlineto{\pgfqpoint{1.420960in}{3.499807in}}%
\pgfusepath{stroke}%
\end{pgfscope}%
\begin{pgfscope}%
\pgfsetbuttcap%
\pgfsetroundjoin%
\definecolor{currentfill}{rgb}{0.000000,0.000000,0.000000}%
\pgfsetfillcolor{currentfill}%
\pgfsetlinewidth{0.803000pt}%
\definecolor{currentstroke}{rgb}{0.000000,0.000000,0.000000}%
\pgfsetstrokecolor{currentstroke}%
\pgfsetdash{}{0pt}%
\pgfsys@defobject{currentmarker}{\pgfqpoint{0.000000in}{-0.048611in}}{\pgfqpoint{0.000000in}{0.000000in}}{%
\pgfpathmoveto{\pgfqpoint{0.000000in}{0.000000in}}%
\pgfpathlineto{\pgfqpoint{0.000000in}{-0.048611in}}%
\pgfusepath{stroke,fill}%
}%
\begin{pgfscope}%
\pgfsys@transformshift{1.420960in}{0.549444in}%
\pgfsys@useobject{currentmarker}{}%
\end{pgfscope}%
\end{pgfscope}%
\begin{pgfscope}%
\definecolor{textcolor}{rgb}{0.000000,0.000000,0.000000}%
\pgfsetstrokecolor{textcolor}%
\pgfsetfillcolor{textcolor}%
\pgftext[x=1.420960in,y=0.452222in,,top]{\color{textcolor}\rmfamily\fontsize{10.000000}{12.000000}\selectfont \(\displaystyle {0.2}\)}%
\end{pgfscope}%
\begin{pgfscope}%
\pgfpathrectangle{\pgfqpoint{0.634445in}{0.549444in}}{\pgfqpoint{5.121067in}{2.950363in}}%
\pgfusepath{clip}%
\pgfsetbuttcap%
\pgfsetroundjoin%
\pgfsetlinewidth{0.803000pt}%
\definecolor{currentstroke}{rgb}{0.690196,0.690196,0.690196}%
\pgfsetstrokecolor{currentstroke}%
\pgfsetstrokeopacity{0.300000}%
\pgfsetdash{{2.960000pt}{1.280000pt}}{0.000000pt}%
\pgfpathmoveto{\pgfqpoint{2.446404in}{0.549444in}}%
\pgfpathlineto{\pgfqpoint{2.446404in}{3.499807in}}%
\pgfusepath{stroke}%
\end{pgfscope}%
\begin{pgfscope}%
\pgfsetbuttcap%
\pgfsetroundjoin%
\definecolor{currentfill}{rgb}{0.000000,0.000000,0.000000}%
\pgfsetfillcolor{currentfill}%
\pgfsetlinewidth{0.803000pt}%
\definecolor{currentstroke}{rgb}{0.000000,0.000000,0.000000}%
\pgfsetstrokecolor{currentstroke}%
\pgfsetdash{}{0pt}%
\pgfsys@defobject{currentmarker}{\pgfqpoint{0.000000in}{-0.048611in}}{\pgfqpoint{0.000000in}{0.000000in}}{%
\pgfpathmoveto{\pgfqpoint{0.000000in}{0.000000in}}%
\pgfpathlineto{\pgfqpoint{0.000000in}{-0.048611in}}%
\pgfusepath{stroke,fill}%
}%
\begin{pgfscope}%
\pgfsys@transformshift{2.446404in}{0.549444in}%
\pgfsys@useobject{currentmarker}{}%
\end{pgfscope}%
\end{pgfscope}%
\begin{pgfscope}%
\definecolor{textcolor}{rgb}{0.000000,0.000000,0.000000}%
\pgfsetstrokecolor{textcolor}%
\pgfsetfillcolor{textcolor}%
\pgftext[x=2.446404in,y=0.452222in,,top]{\color{textcolor}\rmfamily\fontsize{10.000000}{12.000000}\selectfont \(\displaystyle {0.4}\)}%
\end{pgfscope}%
\begin{pgfscope}%
\pgfpathrectangle{\pgfqpoint{0.634445in}{0.549444in}}{\pgfqpoint{5.121067in}{2.950363in}}%
\pgfusepath{clip}%
\pgfsetbuttcap%
\pgfsetroundjoin%
\pgfsetlinewidth{0.803000pt}%
\definecolor{currentstroke}{rgb}{0.690196,0.690196,0.690196}%
\pgfsetstrokecolor{currentstroke}%
\pgfsetstrokeopacity{0.300000}%
\pgfsetdash{{2.960000pt}{1.280000pt}}{0.000000pt}%
\pgfpathmoveto{\pgfqpoint{3.471848in}{0.549444in}}%
\pgfpathlineto{\pgfqpoint{3.471848in}{3.499807in}}%
\pgfusepath{stroke}%
\end{pgfscope}%
\begin{pgfscope}%
\pgfsetbuttcap%
\pgfsetroundjoin%
\definecolor{currentfill}{rgb}{0.000000,0.000000,0.000000}%
\pgfsetfillcolor{currentfill}%
\pgfsetlinewidth{0.803000pt}%
\definecolor{currentstroke}{rgb}{0.000000,0.000000,0.000000}%
\pgfsetstrokecolor{currentstroke}%
\pgfsetdash{}{0pt}%
\pgfsys@defobject{currentmarker}{\pgfqpoint{0.000000in}{-0.048611in}}{\pgfqpoint{0.000000in}{0.000000in}}{%
\pgfpathmoveto{\pgfqpoint{0.000000in}{0.000000in}}%
\pgfpathlineto{\pgfqpoint{0.000000in}{-0.048611in}}%
\pgfusepath{stroke,fill}%
}%
\begin{pgfscope}%
\pgfsys@transformshift{3.471848in}{0.549444in}%
\pgfsys@useobject{currentmarker}{}%
\end{pgfscope}%
\end{pgfscope}%
\begin{pgfscope}%
\definecolor{textcolor}{rgb}{0.000000,0.000000,0.000000}%
\pgfsetstrokecolor{textcolor}%
\pgfsetfillcolor{textcolor}%
\pgftext[x=3.471848in,y=0.452222in,,top]{\color{textcolor}\rmfamily\fontsize{10.000000}{12.000000}\selectfont \(\displaystyle {0.6}\)}%
\end{pgfscope}%
\begin{pgfscope}%
\pgfpathrectangle{\pgfqpoint{0.634445in}{0.549444in}}{\pgfqpoint{5.121067in}{2.950363in}}%
\pgfusepath{clip}%
\pgfsetbuttcap%
\pgfsetroundjoin%
\pgfsetlinewidth{0.803000pt}%
\definecolor{currentstroke}{rgb}{0.690196,0.690196,0.690196}%
\pgfsetstrokecolor{currentstroke}%
\pgfsetstrokeopacity{0.300000}%
\pgfsetdash{{2.960000pt}{1.280000pt}}{0.000000pt}%
\pgfpathmoveto{\pgfqpoint{4.497292in}{0.549444in}}%
\pgfpathlineto{\pgfqpoint{4.497292in}{3.499807in}}%
\pgfusepath{stroke}%
\end{pgfscope}%
\begin{pgfscope}%
\pgfsetbuttcap%
\pgfsetroundjoin%
\definecolor{currentfill}{rgb}{0.000000,0.000000,0.000000}%
\pgfsetfillcolor{currentfill}%
\pgfsetlinewidth{0.803000pt}%
\definecolor{currentstroke}{rgb}{0.000000,0.000000,0.000000}%
\pgfsetstrokecolor{currentstroke}%
\pgfsetdash{}{0pt}%
\pgfsys@defobject{currentmarker}{\pgfqpoint{0.000000in}{-0.048611in}}{\pgfqpoint{0.000000in}{0.000000in}}{%
\pgfpathmoveto{\pgfqpoint{0.000000in}{0.000000in}}%
\pgfpathlineto{\pgfqpoint{0.000000in}{-0.048611in}}%
\pgfusepath{stroke,fill}%
}%
\begin{pgfscope}%
\pgfsys@transformshift{4.497292in}{0.549444in}%
\pgfsys@useobject{currentmarker}{}%
\end{pgfscope}%
\end{pgfscope}%
\begin{pgfscope}%
\definecolor{textcolor}{rgb}{0.000000,0.000000,0.000000}%
\pgfsetstrokecolor{textcolor}%
\pgfsetfillcolor{textcolor}%
\pgftext[x=4.497292in,y=0.452222in,,top]{\color{textcolor}\rmfamily\fontsize{10.000000}{12.000000}\selectfont \(\displaystyle {0.8}\)}%
\end{pgfscope}%
\begin{pgfscope}%
\pgfpathrectangle{\pgfqpoint{0.634445in}{0.549444in}}{\pgfqpoint{5.121067in}{2.950363in}}%
\pgfusepath{clip}%
\pgfsetbuttcap%
\pgfsetroundjoin%
\pgfsetlinewidth{0.803000pt}%
\definecolor{currentstroke}{rgb}{0.690196,0.690196,0.690196}%
\pgfsetstrokecolor{currentstroke}%
\pgfsetstrokeopacity{0.300000}%
\pgfsetdash{{2.960000pt}{1.280000pt}}{0.000000pt}%
\pgfpathmoveto{\pgfqpoint{5.522736in}{0.549444in}}%
\pgfpathlineto{\pgfqpoint{5.522736in}{3.499807in}}%
\pgfusepath{stroke}%
\end{pgfscope}%
\begin{pgfscope}%
\pgfsetbuttcap%
\pgfsetroundjoin%
\definecolor{currentfill}{rgb}{0.000000,0.000000,0.000000}%
\pgfsetfillcolor{currentfill}%
\pgfsetlinewidth{0.803000pt}%
\definecolor{currentstroke}{rgb}{0.000000,0.000000,0.000000}%
\pgfsetstrokecolor{currentstroke}%
\pgfsetdash{}{0pt}%
\pgfsys@defobject{currentmarker}{\pgfqpoint{0.000000in}{-0.048611in}}{\pgfqpoint{0.000000in}{0.000000in}}{%
\pgfpathmoveto{\pgfqpoint{0.000000in}{0.000000in}}%
\pgfpathlineto{\pgfqpoint{0.000000in}{-0.048611in}}%
\pgfusepath{stroke,fill}%
}%
\begin{pgfscope}%
\pgfsys@transformshift{5.522736in}{0.549444in}%
\pgfsys@useobject{currentmarker}{}%
\end{pgfscope}%
\end{pgfscope}%
\begin{pgfscope}%
\definecolor{textcolor}{rgb}{0.000000,0.000000,0.000000}%
\pgfsetstrokecolor{textcolor}%
\pgfsetfillcolor{textcolor}%
\pgftext[x=5.522736in,y=0.452222in,,top]{\color{textcolor}\rmfamily\fontsize{10.000000}{12.000000}\selectfont \(\displaystyle {1.0}\)}%
\end{pgfscope}%
\begin{pgfscope}%
\definecolor{textcolor}{rgb}{0.000000,0.000000,0.000000}%
\pgfsetstrokecolor{textcolor}%
\pgfsetfillcolor{textcolor}%
\pgftext[x=3.194978in,y=0.273333in,,top]{\color{textcolor}\rmfamily\fontsize{10.000000}{12.000000}\selectfont overlap}%
\end{pgfscope}%
\begin{pgfscope}%
\pgfpathrectangle{\pgfqpoint{0.634445in}{0.549444in}}{\pgfqpoint{5.121067in}{2.950363in}}%
\pgfusepath{clip}%
\pgfsetbuttcap%
\pgfsetroundjoin%
\pgfsetlinewidth{0.803000pt}%
\definecolor{currentstroke}{rgb}{0.690196,0.690196,0.690196}%
\pgfsetstrokecolor{currentstroke}%
\pgfsetstrokeopacity{0.300000}%
\pgfsetdash{{2.960000pt}{1.280000pt}}{0.000000pt}%
\pgfpathmoveto{\pgfqpoint{0.634445in}{0.549444in}}%
\pgfpathlineto{\pgfqpoint{5.755512in}{0.549444in}}%
\pgfusepath{stroke}%
\end{pgfscope}%
\begin{pgfscope}%
\pgfsetbuttcap%
\pgfsetroundjoin%
\definecolor{currentfill}{rgb}{0.000000,0.000000,0.000000}%
\pgfsetfillcolor{currentfill}%
\pgfsetlinewidth{0.803000pt}%
\definecolor{currentstroke}{rgb}{0.000000,0.000000,0.000000}%
\pgfsetstrokecolor{currentstroke}%
\pgfsetdash{}{0pt}%
\pgfsys@defobject{currentmarker}{\pgfqpoint{-0.048611in}{0.000000in}}{\pgfqpoint{-0.000000in}{0.000000in}}{%
\pgfpathmoveto{\pgfqpoint{-0.000000in}{0.000000in}}%
\pgfpathlineto{\pgfqpoint{-0.048611in}{0.000000in}}%
\pgfusepath{stroke,fill}%
}%
\begin{pgfscope}%
\pgfsys@transformshift{0.634445in}{0.549444in}%
\pgfsys@useobject{currentmarker}{}%
\end{pgfscope}%
\end{pgfscope}%
\begin{pgfscope}%
\definecolor{textcolor}{rgb}{0.000000,0.000000,0.000000}%
\pgfsetstrokecolor{textcolor}%
\pgfsetfillcolor{textcolor}%
\pgftext[x=0.467778in, y=0.501250in, left, base]{\color{textcolor}\rmfamily\fontsize{10.000000}{12.000000}\selectfont \(\displaystyle {0}\)}%
\end{pgfscope}%
\begin{pgfscope}%
\pgfpathrectangle{\pgfqpoint{0.634445in}{0.549444in}}{\pgfqpoint{5.121067in}{2.950363in}}%
\pgfusepath{clip}%
\pgfsetbuttcap%
\pgfsetroundjoin%
\pgfsetlinewidth{0.803000pt}%
\definecolor{currentstroke}{rgb}{0.690196,0.690196,0.690196}%
\pgfsetstrokecolor{currentstroke}%
\pgfsetstrokeopacity{0.300000}%
\pgfsetdash{{2.960000pt}{1.280000pt}}{0.000000pt}%
\pgfpathmoveto{\pgfqpoint{0.634445in}{1.111418in}}%
\pgfpathlineto{\pgfqpoint{5.755512in}{1.111418in}}%
\pgfusepath{stroke}%
\end{pgfscope}%
\begin{pgfscope}%
\pgfsetbuttcap%
\pgfsetroundjoin%
\definecolor{currentfill}{rgb}{0.000000,0.000000,0.000000}%
\pgfsetfillcolor{currentfill}%
\pgfsetlinewidth{0.803000pt}%
\definecolor{currentstroke}{rgb}{0.000000,0.000000,0.000000}%
\pgfsetstrokecolor{currentstroke}%
\pgfsetdash{}{0pt}%
\pgfsys@defobject{currentmarker}{\pgfqpoint{-0.048611in}{0.000000in}}{\pgfqpoint{-0.000000in}{0.000000in}}{%
\pgfpathmoveto{\pgfqpoint{-0.000000in}{0.000000in}}%
\pgfpathlineto{\pgfqpoint{-0.048611in}{0.000000in}}%
\pgfusepath{stroke,fill}%
}%
\begin{pgfscope}%
\pgfsys@transformshift{0.634445in}{1.111418in}%
\pgfsys@useobject{currentmarker}{}%
\end{pgfscope}%
\end{pgfscope}%
\begin{pgfscope}%
\definecolor{textcolor}{rgb}{0.000000,0.000000,0.000000}%
\pgfsetstrokecolor{textcolor}%
\pgfsetfillcolor{textcolor}%
\pgftext[x=0.398333in, y=1.063224in, left, base]{\color{textcolor}\rmfamily\fontsize{10.000000}{12.000000}\selectfont \(\displaystyle {50}\)}%
\end{pgfscope}%
\begin{pgfscope}%
\pgfpathrectangle{\pgfqpoint{0.634445in}{0.549444in}}{\pgfqpoint{5.121067in}{2.950363in}}%
\pgfusepath{clip}%
\pgfsetbuttcap%
\pgfsetroundjoin%
\pgfsetlinewidth{0.803000pt}%
\definecolor{currentstroke}{rgb}{0.690196,0.690196,0.690196}%
\pgfsetstrokecolor{currentstroke}%
\pgfsetstrokeopacity{0.300000}%
\pgfsetdash{{2.960000pt}{1.280000pt}}{0.000000pt}%
\pgfpathmoveto{\pgfqpoint{0.634445in}{1.673392in}}%
\pgfpathlineto{\pgfqpoint{5.755512in}{1.673392in}}%
\pgfusepath{stroke}%
\end{pgfscope}%
\begin{pgfscope}%
\pgfsetbuttcap%
\pgfsetroundjoin%
\definecolor{currentfill}{rgb}{0.000000,0.000000,0.000000}%
\pgfsetfillcolor{currentfill}%
\pgfsetlinewidth{0.803000pt}%
\definecolor{currentstroke}{rgb}{0.000000,0.000000,0.000000}%
\pgfsetstrokecolor{currentstroke}%
\pgfsetdash{}{0pt}%
\pgfsys@defobject{currentmarker}{\pgfqpoint{-0.048611in}{0.000000in}}{\pgfqpoint{-0.000000in}{0.000000in}}{%
\pgfpathmoveto{\pgfqpoint{-0.000000in}{0.000000in}}%
\pgfpathlineto{\pgfqpoint{-0.048611in}{0.000000in}}%
\pgfusepath{stroke,fill}%
}%
\begin{pgfscope}%
\pgfsys@transformshift{0.634445in}{1.673392in}%
\pgfsys@useobject{currentmarker}{}%
\end{pgfscope}%
\end{pgfscope}%
\begin{pgfscope}%
\definecolor{textcolor}{rgb}{0.000000,0.000000,0.000000}%
\pgfsetstrokecolor{textcolor}%
\pgfsetfillcolor{textcolor}%
\pgftext[x=0.328889in, y=1.625197in, left, base]{\color{textcolor}\rmfamily\fontsize{10.000000}{12.000000}\selectfont \(\displaystyle {100}\)}%
\end{pgfscope}%
\begin{pgfscope}%
\pgfpathrectangle{\pgfqpoint{0.634445in}{0.549444in}}{\pgfqpoint{5.121067in}{2.950363in}}%
\pgfusepath{clip}%
\pgfsetbuttcap%
\pgfsetroundjoin%
\pgfsetlinewidth{0.803000pt}%
\definecolor{currentstroke}{rgb}{0.690196,0.690196,0.690196}%
\pgfsetstrokecolor{currentstroke}%
\pgfsetstrokeopacity{0.300000}%
\pgfsetdash{{2.960000pt}{1.280000pt}}{0.000000pt}%
\pgfpathmoveto{\pgfqpoint{0.634445in}{2.235366in}}%
\pgfpathlineto{\pgfqpoint{5.755512in}{2.235366in}}%
\pgfusepath{stroke}%
\end{pgfscope}%
\begin{pgfscope}%
\pgfsetbuttcap%
\pgfsetroundjoin%
\definecolor{currentfill}{rgb}{0.000000,0.000000,0.000000}%
\pgfsetfillcolor{currentfill}%
\pgfsetlinewidth{0.803000pt}%
\definecolor{currentstroke}{rgb}{0.000000,0.000000,0.000000}%
\pgfsetstrokecolor{currentstroke}%
\pgfsetdash{}{0pt}%
\pgfsys@defobject{currentmarker}{\pgfqpoint{-0.048611in}{0.000000in}}{\pgfqpoint{-0.000000in}{0.000000in}}{%
\pgfpathmoveto{\pgfqpoint{-0.000000in}{0.000000in}}%
\pgfpathlineto{\pgfqpoint{-0.048611in}{0.000000in}}%
\pgfusepath{stroke,fill}%
}%
\begin{pgfscope}%
\pgfsys@transformshift{0.634445in}{2.235366in}%
\pgfsys@useobject{currentmarker}{}%
\end{pgfscope}%
\end{pgfscope}%
\begin{pgfscope}%
\definecolor{textcolor}{rgb}{0.000000,0.000000,0.000000}%
\pgfsetstrokecolor{textcolor}%
\pgfsetfillcolor{textcolor}%
\pgftext[x=0.328889in, y=2.187171in, left, base]{\color{textcolor}\rmfamily\fontsize{10.000000}{12.000000}\selectfont \(\displaystyle {150}\)}%
\end{pgfscope}%
\begin{pgfscope}%
\pgfpathrectangle{\pgfqpoint{0.634445in}{0.549444in}}{\pgfqpoint{5.121067in}{2.950363in}}%
\pgfusepath{clip}%
\pgfsetbuttcap%
\pgfsetroundjoin%
\pgfsetlinewidth{0.803000pt}%
\definecolor{currentstroke}{rgb}{0.690196,0.690196,0.690196}%
\pgfsetstrokecolor{currentstroke}%
\pgfsetstrokeopacity{0.300000}%
\pgfsetdash{{2.960000pt}{1.280000pt}}{0.000000pt}%
\pgfpathmoveto{\pgfqpoint{0.634445in}{2.797340in}}%
\pgfpathlineto{\pgfqpoint{5.755512in}{2.797340in}}%
\pgfusepath{stroke}%
\end{pgfscope}%
\begin{pgfscope}%
\pgfsetbuttcap%
\pgfsetroundjoin%
\definecolor{currentfill}{rgb}{0.000000,0.000000,0.000000}%
\pgfsetfillcolor{currentfill}%
\pgfsetlinewidth{0.803000pt}%
\definecolor{currentstroke}{rgb}{0.000000,0.000000,0.000000}%
\pgfsetstrokecolor{currentstroke}%
\pgfsetdash{}{0pt}%
\pgfsys@defobject{currentmarker}{\pgfqpoint{-0.048611in}{0.000000in}}{\pgfqpoint{-0.000000in}{0.000000in}}{%
\pgfpathmoveto{\pgfqpoint{-0.000000in}{0.000000in}}%
\pgfpathlineto{\pgfqpoint{-0.048611in}{0.000000in}}%
\pgfusepath{stroke,fill}%
}%
\begin{pgfscope}%
\pgfsys@transformshift{0.634445in}{2.797340in}%
\pgfsys@useobject{currentmarker}{}%
\end{pgfscope}%
\end{pgfscope}%
\begin{pgfscope}%
\definecolor{textcolor}{rgb}{0.000000,0.000000,0.000000}%
\pgfsetstrokecolor{textcolor}%
\pgfsetfillcolor{textcolor}%
\pgftext[x=0.328889in, y=2.749145in, left, base]{\color{textcolor}\rmfamily\fontsize{10.000000}{12.000000}\selectfont \(\displaystyle {200}\)}%
\end{pgfscope}%
\begin{pgfscope}%
\pgfpathrectangle{\pgfqpoint{0.634445in}{0.549444in}}{\pgfqpoint{5.121067in}{2.950363in}}%
\pgfusepath{clip}%
\pgfsetbuttcap%
\pgfsetroundjoin%
\pgfsetlinewidth{0.803000pt}%
\definecolor{currentstroke}{rgb}{0.690196,0.690196,0.690196}%
\pgfsetstrokecolor{currentstroke}%
\pgfsetstrokeopacity{0.300000}%
\pgfsetdash{{2.960000pt}{1.280000pt}}{0.000000pt}%
\pgfpathmoveto{\pgfqpoint{0.634445in}{3.359314in}}%
\pgfpathlineto{\pgfqpoint{5.755512in}{3.359314in}}%
\pgfusepath{stroke}%
\end{pgfscope}%
\begin{pgfscope}%
\pgfsetbuttcap%
\pgfsetroundjoin%
\definecolor{currentfill}{rgb}{0.000000,0.000000,0.000000}%
\pgfsetfillcolor{currentfill}%
\pgfsetlinewidth{0.803000pt}%
\definecolor{currentstroke}{rgb}{0.000000,0.000000,0.000000}%
\pgfsetstrokecolor{currentstroke}%
\pgfsetdash{}{0pt}%
\pgfsys@defobject{currentmarker}{\pgfqpoint{-0.048611in}{0.000000in}}{\pgfqpoint{-0.000000in}{0.000000in}}{%
\pgfpathmoveto{\pgfqpoint{-0.000000in}{0.000000in}}%
\pgfpathlineto{\pgfqpoint{-0.048611in}{0.000000in}}%
\pgfusepath{stroke,fill}%
}%
\begin{pgfscope}%
\pgfsys@transformshift{0.634445in}{3.359314in}%
\pgfsys@useobject{currentmarker}{}%
\end{pgfscope}%
\end{pgfscope}%
\begin{pgfscope}%
\definecolor{textcolor}{rgb}{0.000000,0.000000,0.000000}%
\pgfsetstrokecolor{textcolor}%
\pgfsetfillcolor{textcolor}%
\pgftext[x=0.328889in, y=3.311119in, left, base]{\color{textcolor}\rmfamily\fontsize{10.000000}{12.000000}\selectfont \(\displaystyle {250}\)}%
\end{pgfscope}%
\begin{pgfscope}%
\definecolor{textcolor}{rgb}{0.000000,0.000000,0.000000}%
\pgfsetstrokecolor{textcolor}%
\pgfsetfillcolor{textcolor}%
\pgftext[x=0.273333in,y=2.024626in,,bottom,rotate=90.000000]{\color{textcolor}\rmfamily\fontsize{10.000000}{12.000000}\selectfont counts}%
\end{pgfscope}%
\begin{pgfscope}%
\pgfpathrectangle{\pgfqpoint{0.634445in}{0.549444in}}{\pgfqpoint{5.121067in}{2.950363in}}%
\pgfusepath{clip}%
\pgfsetbuttcap%
\pgfsetmiterjoin%
\definecolor{currentfill}{rgb}{0.117647,0.564706,1.000000}%
\pgfsetfillcolor{currentfill}%
\pgfsetlinewidth{1.003750pt}%
\definecolor{currentstroke}{rgb}{0.000000,0.000000,0.000000}%
\pgfsetstrokecolor{currentstroke}%
\pgfsetdash{}{0pt}%
\pgfpathmoveto{\pgfqpoint{0.867221in}{0.549444in}}%
\pgfpathlineto{\pgfqpoint{0.867221in}{0.560684in}}%
\pgfpathlineto{\pgfqpoint{1.177588in}{0.560684in}}%
\pgfpathlineto{\pgfqpoint{1.177588in}{0.583163in}}%
\pgfpathlineto{\pgfqpoint{1.487956in}{0.583163in}}%
\pgfpathlineto{\pgfqpoint{1.487956in}{0.673078in}}%
\pgfpathlineto{\pgfqpoint{1.798324in}{0.673078in}}%
\pgfpathlineto{\pgfqpoint{1.798324in}{0.841671in}}%
\pgfpathlineto{\pgfqpoint{2.108692in}{0.841671in}}%
\pgfpathlineto{\pgfqpoint{2.108692in}{1.055221in}}%
\pgfpathlineto{\pgfqpoint{2.419059in}{1.055221in}}%
\pgfpathlineto{\pgfqpoint{2.419059in}{0.886628in}}%
\pgfpathlineto{\pgfqpoint{2.729427in}{0.886628in}}%
\pgfpathlineto{\pgfqpoint{2.729427in}{0.796713in}}%
\pgfpathlineto{\pgfqpoint{3.039795in}{0.796713in}}%
\pgfpathlineto{\pgfqpoint{3.039795in}{0.684318in}}%
\pgfpathlineto{\pgfqpoint{3.350162in}{0.684318in}}%
\pgfpathlineto{\pgfqpoint{3.350162in}{0.650599in}}%
\pgfpathlineto{\pgfqpoint{3.660530in}{0.650599in}}%
\pgfpathlineto{\pgfqpoint{3.660530in}{0.628121in}}%
\pgfpathlineto{\pgfqpoint{3.970898in}{0.628121in}}%
\pgfpathlineto{\pgfqpoint{3.970898in}{0.605642in}}%
\pgfpathlineto{\pgfqpoint{4.281265in}{0.605642in}}%
\pgfpathlineto{\pgfqpoint{4.281265in}{0.661839in}}%
\pgfpathlineto{\pgfqpoint{4.591633in}{0.661839in}}%
\pgfpathlineto{\pgfqpoint{4.591633in}{0.729276in}}%
\pgfpathlineto{\pgfqpoint{4.902001in}{0.729276in}}%
\pgfpathlineto{\pgfqpoint{4.902001in}{1.145136in}}%
\pgfpathlineto{\pgfqpoint{5.212368in}{1.145136in}}%
\pgfpathlineto{\pgfqpoint{5.212368in}{3.359314in}}%
\pgfpathlineto{\pgfqpoint{5.522736in}{3.359314in}}%
\pgfpathlineto{\pgfqpoint{5.522736in}{0.549444in}}%
\pgfusepath{stroke,fill}%
\end{pgfscope}%
\begin{pgfscope}%
\pgfsetrectcap%
\pgfsetmiterjoin%
\pgfsetlinewidth{0.803000pt}%
\definecolor{currentstroke}{rgb}{0.000000,0.000000,0.000000}%
\pgfsetstrokecolor{currentstroke}%
\pgfsetdash{}{0pt}%
\pgfpathmoveto{\pgfqpoint{0.634445in}{0.549444in}}%
\pgfpathlineto{\pgfqpoint{0.634445in}{3.499807in}}%
\pgfusepath{stroke}%
\end{pgfscope}%
\begin{pgfscope}%
\pgfsetrectcap%
\pgfsetmiterjoin%
\pgfsetlinewidth{0.803000pt}%
\definecolor{currentstroke}{rgb}{0.000000,0.000000,0.000000}%
\pgfsetstrokecolor{currentstroke}%
\pgfsetdash{}{0pt}%
\pgfpathmoveto{\pgfqpoint{0.634445in}{0.549444in}}%
\pgfpathlineto{\pgfqpoint{5.755512in}{0.549444in}}%
\pgfusepath{stroke}%
\end{pgfscope}%
\begin{pgfscope}%
\pgfsetbuttcap%
\pgfsetmiterjoin%
\definecolor{currentfill}{rgb}{1.000000,1.000000,1.000000}%
\pgfsetfillcolor{currentfill}%
\pgfsetfillopacity{0.800000}%
\pgfsetlinewidth{1.003750pt}%
\definecolor{currentstroke}{rgb}{0.800000,0.800000,0.800000}%
\pgfsetstrokecolor{currentstroke}%
\pgfsetstrokeopacity{0.800000}%
\pgfsetdash{}{0pt}%
\pgfpathmoveto{\pgfqpoint{0.731667in}{3.195085in}}%
\pgfpathlineto{\pgfqpoint{2.399306in}{3.195085in}}%
\pgfpathquadraticcurveto{\pgfqpoint{2.427084in}{3.195085in}}{\pgfqpoint{2.427084in}{3.222863in}}%
\pgfpathlineto{\pgfqpoint{2.427084in}{3.402585in}}%
\pgfpathquadraticcurveto{\pgfqpoint{2.427084in}{3.430363in}}{\pgfqpoint{2.399306in}{3.430363in}}%
\pgfpathlineto{\pgfqpoint{0.731667in}{3.430363in}}%
\pgfpathquadraticcurveto{\pgfqpoint{0.703889in}{3.430363in}}{\pgfqpoint{0.703889in}{3.402585in}}%
\pgfpathlineto{\pgfqpoint{0.703889in}{3.222863in}}%
\pgfpathquadraticcurveto{\pgfqpoint{0.703889in}{3.195085in}}{\pgfqpoint{0.731667in}{3.195085in}}%
\pgfpathlineto{\pgfqpoint{0.731667in}{3.195085in}}%
\pgfpathclose%
\pgfusepath{stroke,fill}%
\end{pgfscope}%
\begin{pgfscope}%
\pgfsetbuttcap%
\pgfsetmiterjoin%
\definecolor{currentfill}{rgb}{0.117647,0.564706,1.000000}%
\pgfsetfillcolor{currentfill}%
\pgfsetlinewidth{1.003750pt}%
\definecolor{currentstroke}{rgb}{0.000000,0.000000,0.000000}%
\pgfsetstrokecolor{currentstroke}%
\pgfsetdash{}{0pt}%
\pgfpathmoveto{\pgfqpoint{0.759445in}{3.277585in}}%
\pgfpathlineto{\pgfqpoint{1.037223in}{3.277585in}}%
\pgfpathlineto{\pgfqpoint{1.037223in}{3.374807in}}%
\pgfpathlineto{\pgfqpoint{0.759445in}{3.374807in}}%
\pgfpathlineto{\pgfqpoint{0.759445in}{3.277585in}}%
\pgfpathclose%
\pgfusepath{stroke,fill}%
\end{pgfscope}%
\begin{pgfscope}%
\definecolor{textcolor}{rgb}{0.000000,0.000000,0.000000}%
\pgfsetstrokecolor{textcolor}%
\pgfsetfillcolor{textcolor}%
\pgftext[x=1.148334in,y=3.277585in,left,base]{\color{textcolor}\rmfamily\fontsize{10.000000}{12.000000}\selectfont N = 1000, α = 0.135}%
\end{pgfscope}%
\end{pgfpicture}%
\makeatother%
\endgroup%

    \caption{Final overlaps after \acrshort{mcmc} at low temperature. Here, the spin-flip probability is $p = 0.2$ and $500$ independent trials are made.}
    \label{fig:shm:p02:final_overlaps1}
\end{figure}
\end{comment}

\section{Modern Hopfield}\label{sec:numsim_mh}
\subsection{Binary case}
In the \acrlong{mhm} things get more computationally heavy. Here there is no particular advantage in finding an analytical expression for the energy variation. Assuming that we have the usual configuration $\symbf{\sigma}$ for which we can compute the variation in energy after one spin-flip, we can first write the variation in the overlap, $\Delta m$ with one stored pattern $\symbf{\xi}^{\mu}$:
\begin{equation*}
    N \cdot \Delta m \equiv N \left( m_{fin} - m_{in} \right) = \left(\sum_{i \neq k}\xi_i^\mu \sigma_i + \xi_k^\mu (-\sigma_k)\right) - \left(\sum_{i \neq k}\xi_i^\mu \sigma_i + \xi_k^\mu \sigma_k\right)\\
= -2\xi_{k}^{\mu}\sigma_k, 
\end{equation*}
where $m_{fin}$ and $m_in$ are the final and initial overlaps respectively.\\
Hence, the energy variation is (referring to \cref{eq:modern_binary_energy}):
\begin{equation*}
\begin{split}
    \Delta E \equiv E_f - E_i & = - \frac{1}{\lambda} \log \left( \sum_{\mu = 1}^{M} \exp\left( \lambda \cdot m_{in}^\mu\right) \right)
+ \frac{1}{\lambda} \log \left( \sum_{\mu = 1}^{M} \exp\left( \lambda \cdot m_{fin}^\mu\right) \right)\\
& = - \frac{1}{\lambda} \left[\log \left(\sum_{\mu = 1}^{M} \exp\left(\lambda \cdot m_{fin}^\mu \right)\right)
- \log \left(\sum_{\mu = 1}^{M} \exp\left(\lambda \cdot m_{in}^\mu \right)\right)\right]\\
& = -\frac{1}{\lambda} \left[ \log \left( \frac{\sum_{\mu = 1}^{M} \exp\left(\lambda \cdot m_{fin}^\mu\right)}{\sum_{\mu = 1}^{M} \exp\left(\lambda \cdot m_{in}^\mu\right)} \right) \right].
\end{split}
\end{equation*}
So we obtain:
\begin{equation}\label{eq:numsim:mhm:energyvariation}
    \Delta E_{\text{spin-flip}} =  -\frac{1}{\lambda} \left[ \log \left( \frac{\sum_{\mu = 1}^{M} \exp\left(\lambda \cdot m_{in}^\mu \right)\exp\left(-2\lambda\xi_{k}^{\mu}\sigma_k \right) }{\sum_{\mu = 1}^{M} \exp\left(\lambda \cdot m_{in}^\mu\right)} \right) \right].
\end{equation}
Notice that here, the magnetization $m$ is not rescaled by $N$.\\
We have two problems:
\begin{itemize}
    \item since in the \acrlong{mhm} the storage capacity increases exponentially, \cref{eq:numsim:mhm:energyvariation} involves a sum of a number of terms that is exponential;
    \item the computation of the exponential might lead to overflow problems during computations, thus leaving us with a numerical unstable algorithm.
\end{itemize}
Furthermore, the first point suggest that there is no particular computational advantage in computing the energy variation with the above analytical expression. For this reason, it is sufficient to compute the energies before and after one single spin-flip and take the difference. Indeed, in both \cref{eq:numsim:mhm:energyvariation} and in the computation of the energy difference using \cref{eq:modern_binary_energy} it is involved a sum over an exponential number of terms.

The second problem is easily solved with the usual trick of subtracting the maximum from exponential.\\
For the first one we make use of the Kahan summation algorithm.

\subsubsection{Kahan summation and numerically stable log-sum-exp}
The representation of floating point numbers and mathematical operations performed among them inevitably generate numerical errors. Indeed, such numbers have a finite precision, \ie, a fixed number of significant digits. Thus, the value ``seen'' by a calculator differs from the real one by a certain quantity. Mathematical operations performed on numerical representations can enhance the effects of such errors.\\
The sum of $n$ floating point numbers leads to an error of the order $O(n)$ in the worst case and a root mean square error that goes like $O\left(\sqrt{n}\right)$ for inputs taken randomly~\cite{accuracy_floating_point}.
In the computation of $\Delta E$ we deal with sum of $M \sim e^{O(n)}$ terms and the consequent error in numerical simulations might be very big.\\
To reduce this problem we exploit the Kahan summation algorithm~\cite{kahan_truncation} based on a class of method that takes the name of \emph{compensated summation}~\cite{accuracy_floating_point}. Several similar algorithms that track the accumulated error in some kind of operations exist~\cite{deltasigma, bresenham}. Also, other variants of the Kahan summation might be used as well~\cite{kahan-babuska}.

\begin{algorithm}
    \caption{Kahan summation}
    \label{alg:kahan_summation}
    \begin{algorithmic}[1]
    \Require $x$
    \State sum $\gets 0$
    \State c $\gets 0$
    \For{$i = 1$ to $|x|$}
        \State $y \gets x[i]$ - c
        \State $t \gets $ sum + $y$
        \State c $\gets t$ - sum - $y$
        \State sum $\gets t$
    \EndFor
    \end{algorithmic}
\end{algorithm}

Overflow in the exponential is the other potential numerical problem. However, this can be simply solved with the usual trick of subtracting the maximum from the exponential argument. In particular, we deal with exponential of the overlaps between a configuration and the stored patterns, \ie, $\exp\left(\symbf{\sigma}\cdot \symbf{\xi}^{\mu}\right)$ where $\mu = 1,\dots, M$. Writing
\begin{equation*}
    a = \max_{\mu} \left(\symbf{\sigma} \cdot \symbf{\xi}^{\mu}\right),
\end{equation*}
one clearly gets $0 \leq \exp\left(\symbf{\sigma}\cdot \symbf{\xi}^{\mu} - a\right) \leq 1$.
Hence, following the work of~\textcite{numerically_stable} in ~\cref{alg:lse} we show a numerically stable computation of the log-sum-exp operator.

\begin{algorithm}
    \caption{Log-Sum-Exp algorithm}
    \label{alg:lse}
    \begin{algorithmic}[1]
    \Require $x$
    \State $s \gets 0$
    \State c $\gets 0$
    \For{$i = 1$ to $|x|$}
        \State $w[i] \gets \exp\left(x[i] - a\right)$
        \If{$i \neq k$}
            \State $s \gets s + w[i]$
        \EndIf
    \EndFor
    \textbf{return} $a + \log1p (s)$ 
    \end{algorithmic}
\end{algorithm}

In~\cref{alg:mhm_metropolis} we show the Metropolis version for the \acrlong{mhm}. The full \acrlong{mcmc} is the same as~\cref{alg:shm_mcmc}.


\begin{algorithm}
    \caption{Metropolis algorithm for the \acrlong{mhm}}
    \label{alg:mhm_metropolis}
    \begin{algorithmic}[1]
    \Require $\symbf{\sigma}$, $\symbf{\xi}$, $\beta$, $\lambda$
        \State $N \gets |\symbf{\sigma}|$ 
        \State fliprate $\gets 0$
        \State $E \gets energy\left(\symbf{\sigma}, \symbf{\xi}, \lambda\right)$
        \For{$i$ in random permutation $N$}
        \State $\sigma[i] \gets - \sigma[i]$
        \State $E_{new} \gets \text{energy}\left(\symbf{\sigma}, \symbf{\xi}, \lambda\right)$
        \State $\Delta E \gets E_{new} - E$
        \If {$\Delta E < 0$ or $u \sim U(0,1) < e^{-\beta \Delta E}$}
            \State $\sigma[i] \gets - \sigma[i]$
            \State fliprate $\gets$ fliprate + 1
            \State $E \gets E_{new}$
        \Else
            \State $\sigma[i] \gets - \sigma[i]$
        \EndIf
        \EndFor
    \State \textbf{return} $\symbf{\sigma}$, $\text{fliprate} / N$
    \end{algorithmic}
\end{algorithm}

\subsection{Continuous case}\label{sec:numsim_continuous}
The same considerations for the numerical stability argument hold also for the continuous case of the \acrlong{mhm}.\\
Here, patterns and configurations are generated by sampling from a standard normal distribution, \ie:
\begin{equation*}
    \xi_i^{\mu} \sim \mathcal{N}\left(0, 1\right), \qquad i = 1, \dots, N.
\end{equation*}
Here, the parameter used to perturb a stored pattern is governed by the variance of a gaussian noise. In mathematical terms:
\begin{equation*}
    \tilde{\xi}_i^{\mu} = \xi_i^{\mu} + \mathcal{N}\left(0, \delta^2\right),
\end{equation*}
where $\tilde{\xi}_i^{\mu}$ is the perturbed version of the $i$-th element belonging to the $\mu$-th pattern.\\
The key operation of the update rule is the $\operatorname{softmax}$ operation, as stated in~\cref{eq:continuous_energy}. In~\cref{alg:softmax} the implementation of such function is shown~\cite{numerically_stable}.

\begin{algorithm}
    \caption{Softmax algorithm}
    \label{alg:softmax}
    \begin{algorithmic}[1]
    \Require $x$
    \State $s \gets 0$
    \State c $\gets 0$
    \For{$i = 1$ to $|x|$}
        \State $w[i] \gets \exp\left(x[i] - a\right)$
        \If{$i \neq k$}
            \State $s \gets s + w[i]$
        \EndIf
    \EndFor
    \For{$i = 1$ to $|x|$}
        \State $g[i] \gets w[i]/(1+s)$
    \EndFor
    \State \textbf{return} $g$
\end{algorithmic}
\end{algorithm}

\begin{algorithm}
    \caption{Update function for the continuous \acrshort{mhm}}
    \label{alg:update_continuous_mhm}
    \begin{algorithmic}[1]
    \Require $\symbf{\sigma}$, $\symbf{\xi}$, $\lambda$
    \State $\symbf{\sigma}_{rec} \gets \symbf{\sigma}$
    \State $\symbf{\sigma}_{rec} \gets \symbf{\xi}  \operatorname{softmax}\left(\lambda \symbf{\xi}^T  \symbf{\sigma}_{rec}\right)$
    \State \textbf{return} $\symbf{\sigma}_{rec}$
\end{algorithmic}
\end{algorithm}

\subsubsection{Energy landscape}
Also in this case with real variables, the energy landscape is non trivial. Nevertheless, visualization methods are easier to apply with respect to the binary counterpart but the results are quite far from giving a satisfactory description of what really happens.
One simple method is to plot the energy on the plane defined by three patterns $\symbf{\xi}^1$, $\symbf{\xi}^2$ and $\symbf{\xi}^3$. Then, any new configuration belonging to the plane defined by such patterns can be written as:
\begin{equation*}
    \symbf{\sigma} = \symbf{\xi}^1 + \epsilon_1 \left(\symbf{\xi}^2 - \symbf{\xi}^1\right) + \epsilon_2 \left(\symbf{\xi}^3 - \symbf{\xi}^1\right)
\end{equation*}
\begin{figure}[hbt]
    \centering
    %% Creator: Matplotlib, PGF backend
%%
%% To include the figure in your LaTeX document, write
%%   \input{<filename>.pgf}
%%
%% Make sure the required packages are loaded in your preamble
%%   \usepackage{pgf}
%%
%% Also ensure that all the required font packages are loaded; for instance,
%% the lmodern package is sometimes necessary when using math font.
%%   \usepackage{lmodern}
%%
%% Figures using additional raster images can only be included by \input if
%% they are in the same directory as the main LaTeX file. For loading figures
%% from other directories you can use the `import` package
%%   \usepackage{import}
%%
%% and then include the figures with
%%   \import{<path to file>}{<filename>.pgf}
%%
%% Matplotlib used the following preamble
%%   
%%   \usepackage{fontspec}
%%   \makeatletter\@ifpackageloaded{underscore}{}{\usepackage[strings]{underscore}}\makeatother
%%
\begingroup%
\makeatletter%
\begin{pgfpicture}%
\pgfpathrectangle{\pgfpointorigin}{\pgfqpoint{5.905512in}{3.649807in}}%
\pgfusepath{use as bounding box, clip}%
\begin{pgfscope}%
\pgfsetbuttcap%
\pgfsetmiterjoin%
\definecolor{currentfill}{rgb}{1.000000,1.000000,1.000000}%
\pgfsetfillcolor{currentfill}%
\pgfsetlinewidth{0.000000pt}%
\definecolor{currentstroke}{rgb}{1.000000,1.000000,1.000000}%
\pgfsetstrokecolor{currentstroke}%
\pgfsetdash{}{0pt}%
\pgfpathmoveto{\pgfqpoint{0.000000in}{0.000000in}}%
\pgfpathlineto{\pgfqpoint{5.905512in}{0.000000in}}%
\pgfpathlineto{\pgfqpoint{5.905512in}{3.649807in}}%
\pgfpathlineto{\pgfqpoint{0.000000in}{3.649807in}}%
\pgfpathlineto{\pgfqpoint{0.000000in}{0.000000in}}%
\pgfpathclose%
\pgfusepath{fill}%
\end{pgfscope}%
\begin{pgfscope}%
\pgfsetbuttcap%
\pgfsetmiterjoin%
\definecolor{currentfill}{rgb}{1.000000,1.000000,1.000000}%
\pgfsetfillcolor{currentfill}%
\pgfsetlinewidth{0.000000pt}%
\definecolor{currentstroke}{rgb}{0.000000,0.000000,0.000000}%
\pgfsetstrokecolor{currentstroke}%
\pgfsetstrokeopacity{0.000000}%
\pgfsetdash{}{0pt}%
\pgfpathmoveto{\pgfqpoint{0.711606in}{0.549444in}}%
\pgfpathlineto{\pgfqpoint{5.666777in}{0.549444in}}%
\pgfpathlineto{\pgfqpoint{5.666777in}{3.451613in}}%
\pgfpathlineto{\pgfqpoint{0.711606in}{3.451613in}}%
\pgfpathlineto{\pgfqpoint{0.711606in}{0.549444in}}%
\pgfpathclose%
\pgfusepath{fill}%
\end{pgfscope}%
\begin{pgfscope}%
\pgfpathrectangle{\pgfqpoint{0.711606in}{0.549444in}}{\pgfqpoint{4.955171in}{2.902168in}}%
\pgfusepath{clip}%
\pgfsetbuttcap%
\pgfsetroundjoin%
\pgfsetlinewidth{0.803000pt}%
\definecolor{currentstroke}{rgb}{0.690196,0.690196,0.690196}%
\pgfsetstrokecolor{currentstroke}%
\pgfsetstrokeopacity{0.300000}%
\pgfsetdash{{2.960000pt}{1.280000pt}}{0.000000pt}%
\pgfpathmoveto{\pgfqpoint{0.711606in}{0.549444in}}%
\pgfpathlineto{\pgfqpoint{0.711606in}{3.451613in}}%
\pgfusepath{stroke}%
\end{pgfscope}%
\begin{pgfscope}%
\pgfsetbuttcap%
\pgfsetroundjoin%
\definecolor{currentfill}{rgb}{0.000000,0.000000,0.000000}%
\pgfsetfillcolor{currentfill}%
\pgfsetlinewidth{0.803000pt}%
\definecolor{currentstroke}{rgb}{0.000000,0.000000,0.000000}%
\pgfsetstrokecolor{currentstroke}%
\pgfsetdash{}{0pt}%
\pgfsys@defobject{currentmarker}{\pgfqpoint{0.000000in}{-0.048611in}}{\pgfqpoint{0.000000in}{0.000000in}}{%
\pgfpathmoveto{\pgfqpoint{0.000000in}{0.000000in}}%
\pgfpathlineto{\pgfqpoint{0.000000in}{-0.048611in}}%
\pgfusepath{stroke,fill}%
}%
\begin{pgfscope}%
\pgfsys@transformshift{0.711606in}{0.549444in}%
\pgfsys@useobject{currentmarker}{}%
\end{pgfscope}%
\end{pgfscope}%
\begin{pgfscope}%
\definecolor{textcolor}{rgb}{0.000000,0.000000,0.000000}%
\pgfsetstrokecolor{textcolor}%
\pgfsetfillcolor{textcolor}%
\pgftext[x=0.711606in,y=0.452222in,,top]{\color{textcolor}\rmfamily\fontsize{10.000000}{12.000000}\selectfont \(\displaystyle {\ensuremath{-}1.0}\)}%
\end{pgfscope}%
\begin{pgfscope}%
\pgfpathrectangle{\pgfqpoint{0.711606in}{0.549444in}}{\pgfqpoint{4.955171in}{2.902168in}}%
\pgfusepath{clip}%
\pgfsetbuttcap%
\pgfsetroundjoin%
\pgfsetlinewidth{0.803000pt}%
\definecolor{currentstroke}{rgb}{0.690196,0.690196,0.690196}%
\pgfsetstrokecolor{currentstroke}%
\pgfsetstrokeopacity{0.300000}%
\pgfsetdash{{2.960000pt}{1.280000pt}}{0.000000pt}%
\pgfpathmoveto{\pgfqpoint{1.537468in}{0.549444in}}%
\pgfpathlineto{\pgfqpoint{1.537468in}{3.451613in}}%
\pgfusepath{stroke}%
\end{pgfscope}%
\begin{pgfscope}%
\pgfsetbuttcap%
\pgfsetroundjoin%
\definecolor{currentfill}{rgb}{0.000000,0.000000,0.000000}%
\pgfsetfillcolor{currentfill}%
\pgfsetlinewidth{0.803000pt}%
\definecolor{currentstroke}{rgb}{0.000000,0.000000,0.000000}%
\pgfsetstrokecolor{currentstroke}%
\pgfsetdash{}{0pt}%
\pgfsys@defobject{currentmarker}{\pgfqpoint{0.000000in}{-0.048611in}}{\pgfqpoint{0.000000in}{0.000000in}}{%
\pgfpathmoveto{\pgfqpoint{0.000000in}{0.000000in}}%
\pgfpathlineto{\pgfqpoint{0.000000in}{-0.048611in}}%
\pgfusepath{stroke,fill}%
}%
\begin{pgfscope}%
\pgfsys@transformshift{1.537468in}{0.549444in}%
\pgfsys@useobject{currentmarker}{}%
\end{pgfscope}%
\end{pgfscope}%
\begin{pgfscope}%
\definecolor{textcolor}{rgb}{0.000000,0.000000,0.000000}%
\pgfsetstrokecolor{textcolor}%
\pgfsetfillcolor{textcolor}%
\pgftext[x=1.537468in,y=0.452222in,,top]{\color{textcolor}\rmfamily\fontsize{10.000000}{12.000000}\selectfont \(\displaystyle {\ensuremath{-}0.5}\)}%
\end{pgfscope}%
\begin{pgfscope}%
\pgfpathrectangle{\pgfqpoint{0.711606in}{0.549444in}}{\pgfqpoint{4.955171in}{2.902168in}}%
\pgfusepath{clip}%
\pgfsetbuttcap%
\pgfsetroundjoin%
\pgfsetlinewidth{0.803000pt}%
\definecolor{currentstroke}{rgb}{0.690196,0.690196,0.690196}%
\pgfsetstrokecolor{currentstroke}%
\pgfsetstrokeopacity{0.300000}%
\pgfsetdash{{2.960000pt}{1.280000pt}}{0.000000pt}%
\pgfpathmoveto{\pgfqpoint{2.363329in}{0.549444in}}%
\pgfpathlineto{\pgfqpoint{2.363329in}{3.451613in}}%
\pgfusepath{stroke}%
\end{pgfscope}%
\begin{pgfscope}%
\pgfsetbuttcap%
\pgfsetroundjoin%
\definecolor{currentfill}{rgb}{0.000000,0.000000,0.000000}%
\pgfsetfillcolor{currentfill}%
\pgfsetlinewidth{0.803000pt}%
\definecolor{currentstroke}{rgb}{0.000000,0.000000,0.000000}%
\pgfsetstrokecolor{currentstroke}%
\pgfsetdash{}{0pt}%
\pgfsys@defobject{currentmarker}{\pgfqpoint{0.000000in}{-0.048611in}}{\pgfqpoint{0.000000in}{0.000000in}}{%
\pgfpathmoveto{\pgfqpoint{0.000000in}{0.000000in}}%
\pgfpathlineto{\pgfqpoint{0.000000in}{-0.048611in}}%
\pgfusepath{stroke,fill}%
}%
\begin{pgfscope}%
\pgfsys@transformshift{2.363329in}{0.549444in}%
\pgfsys@useobject{currentmarker}{}%
\end{pgfscope}%
\end{pgfscope}%
\begin{pgfscope}%
\definecolor{textcolor}{rgb}{0.000000,0.000000,0.000000}%
\pgfsetstrokecolor{textcolor}%
\pgfsetfillcolor{textcolor}%
\pgftext[x=2.363329in,y=0.452222in,,top]{\color{textcolor}\rmfamily\fontsize{10.000000}{12.000000}\selectfont \(\displaystyle {0.0}\)}%
\end{pgfscope}%
\begin{pgfscope}%
\pgfpathrectangle{\pgfqpoint{0.711606in}{0.549444in}}{\pgfqpoint{4.955171in}{2.902168in}}%
\pgfusepath{clip}%
\pgfsetbuttcap%
\pgfsetroundjoin%
\pgfsetlinewidth{0.803000pt}%
\definecolor{currentstroke}{rgb}{0.690196,0.690196,0.690196}%
\pgfsetstrokecolor{currentstroke}%
\pgfsetstrokeopacity{0.300000}%
\pgfsetdash{{2.960000pt}{1.280000pt}}{0.000000pt}%
\pgfpathmoveto{\pgfqpoint{3.189191in}{0.549444in}}%
\pgfpathlineto{\pgfqpoint{3.189191in}{3.451613in}}%
\pgfusepath{stroke}%
\end{pgfscope}%
\begin{pgfscope}%
\pgfsetbuttcap%
\pgfsetroundjoin%
\definecolor{currentfill}{rgb}{0.000000,0.000000,0.000000}%
\pgfsetfillcolor{currentfill}%
\pgfsetlinewidth{0.803000pt}%
\definecolor{currentstroke}{rgb}{0.000000,0.000000,0.000000}%
\pgfsetstrokecolor{currentstroke}%
\pgfsetdash{}{0pt}%
\pgfsys@defobject{currentmarker}{\pgfqpoint{0.000000in}{-0.048611in}}{\pgfqpoint{0.000000in}{0.000000in}}{%
\pgfpathmoveto{\pgfqpoint{0.000000in}{0.000000in}}%
\pgfpathlineto{\pgfqpoint{0.000000in}{-0.048611in}}%
\pgfusepath{stroke,fill}%
}%
\begin{pgfscope}%
\pgfsys@transformshift{3.189191in}{0.549444in}%
\pgfsys@useobject{currentmarker}{}%
\end{pgfscope}%
\end{pgfscope}%
\begin{pgfscope}%
\definecolor{textcolor}{rgb}{0.000000,0.000000,0.000000}%
\pgfsetstrokecolor{textcolor}%
\pgfsetfillcolor{textcolor}%
\pgftext[x=3.189191in,y=0.452222in,,top]{\color{textcolor}\rmfamily\fontsize{10.000000}{12.000000}\selectfont \(\displaystyle {0.5}\)}%
\end{pgfscope}%
\begin{pgfscope}%
\pgfpathrectangle{\pgfqpoint{0.711606in}{0.549444in}}{\pgfqpoint{4.955171in}{2.902168in}}%
\pgfusepath{clip}%
\pgfsetbuttcap%
\pgfsetroundjoin%
\pgfsetlinewidth{0.803000pt}%
\definecolor{currentstroke}{rgb}{0.690196,0.690196,0.690196}%
\pgfsetstrokecolor{currentstroke}%
\pgfsetstrokeopacity{0.300000}%
\pgfsetdash{{2.960000pt}{1.280000pt}}{0.000000pt}%
\pgfpathmoveto{\pgfqpoint{4.015053in}{0.549444in}}%
\pgfpathlineto{\pgfqpoint{4.015053in}{3.451613in}}%
\pgfusepath{stroke}%
\end{pgfscope}%
\begin{pgfscope}%
\pgfsetbuttcap%
\pgfsetroundjoin%
\definecolor{currentfill}{rgb}{0.000000,0.000000,0.000000}%
\pgfsetfillcolor{currentfill}%
\pgfsetlinewidth{0.803000pt}%
\definecolor{currentstroke}{rgb}{0.000000,0.000000,0.000000}%
\pgfsetstrokecolor{currentstroke}%
\pgfsetdash{}{0pt}%
\pgfsys@defobject{currentmarker}{\pgfqpoint{0.000000in}{-0.048611in}}{\pgfqpoint{0.000000in}{0.000000in}}{%
\pgfpathmoveto{\pgfqpoint{0.000000in}{0.000000in}}%
\pgfpathlineto{\pgfqpoint{0.000000in}{-0.048611in}}%
\pgfusepath{stroke,fill}%
}%
\begin{pgfscope}%
\pgfsys@transformshift{4.015053in}{0.549444in}%
\pgfsys@useobject{currentmarker}{}%
\end{pgfscope}%
\end{pgfscope}%
\begin{pgfscope}%
\definecolor{textcolor}{rgb}{0.000000,0.000000,0.000000}%
\pgfsetstrokecolor{textcolor}%
\pgfsetfillcolor{textcolor}%
\pgftext[x=4.015053in,y=0.452222in,,top]{\color{textcolor}\rmfamily\fontsize{10.000000}{12.000000}\selectfont \(\displaystyle {1.0}\)}%
\end{pgfscope}%
\begin{pgfscope}%
\pgfpathrectangle{\pgfqpoint{0.711606in}{0.549444in}}{\pgfqpoint{4.955171in}{2.902168in}}%
\pgfusepath{clip}%
\pgfsetbuttcap%
\pgfsetroundjoin%
\pgfsetlinewidth{0.803000pt}%
\definecolor{currentstroke}{rgb}{0.690196,0.690196,0.690196}%
\pgfsetstrokecolor{currentstroke}%
\pgfsetstrokeopacity{0.300000}%
\pgfsetdash{{2.960000pt}{1.280000pt}}{0.000000pt}%
\pgfpathmoveto{\pgfqpoint{4.840915in}{0.549444in}}%
\pgfpathlineto{\pgfqpoint{4.840915in}{3.451613in}}%
\pgfusepath{stroke}%
\end{pgfscope}%
\begin{pgfscope}%
\pgfsetbuttcap%
\pgfsetroundjoin%
\definecolor{currentfill}{rgb}{0.000000,0.000000,0.000000}%
\pgfsetfillcolor{currentfill}%
\pgfsetlinewidth{0.803000pt}%
\definecolor{currentstroke}{rgb}{0.000000,0.000000,0.000000}%
\pgfsetstrokecolor{currentstroke}%
\pgfsetdash{}{0pt}%
\pgfsys@defobject{currentmarker}{\pgfqpoint{0.000000in}{-0.048611in}}{\pgfqpoint{0.000000in}{0.000000in}}{%
\pgfpathmoveto{\pgfqpoint{0.000000in}{0.000000in}}%
\pgfpathlineto{\pgfqpoint{0.000000in}{-0.048611in}}%
\pgfusepath{stroke,fill}%
}%
\begin{pgfscope}%
\pgfsys@transformshift{4.840915in}{0.549444in}%
\pgfsys@useobject{currentmarker}{}%
\end{pgfscope}%
\end{pgfscope}%
\begin{pgfscope}%
\definecolor{textcolor}{rgb}{0.000000,0.000000,0.000000}%
\pgfsetstrokecolor{textcolor}%
\pgfsetfillcolor{textcolor}%
\pgftext[x=4.840915in,y=0.452222in,,top]{\color{textcolor}\rmfamily\fontsize{10.000000}{12.000000}\selectfont \(\displaystyle {1.5}\)}%
\end{pgfscope}%
\begin{pgfscope}%
\pgfpathrectangle{\pgfqpoint{0.711606in}{0.549444in}}{\pgfqpoint{4.955171in}{2.902168in}}%
\pgfusepath{clip}%
\pgfsetbuttcap%
\pgfsetroundjoin%
\pgfsetlinewidth{0.803000pt}%
\definecolor{currentstroke}{rgb}{0.690196,0.690196,0.690196}%
\pgfsetstrokecolor{currentstroke}%
\pgfsetstrokeopacity{0.300000}%
\pgfsetdash{{2.960000pt}{1.280000pt}}{0.000000pt}%
\pgfpathmoveto{\pgfqpoint{5.666777in}{0.549444in}}%
\pgfpathlineto{\pgfqpoint{5.666777in}{3.451613in}}%
\pgfusepath{stroke}%
\end{pgfscope}%
\begin{pgfscope}%
\pgfsetbuttcap%
\pgfsetroundjoin%
\definecolor{currentfill}{rgb}{0.000000,0.000000,0.000000}%
\pgfsetfillcolor{currentfill}%
\pgfsetlinewidth{0.803000pt}%
\definecolor{currentstroke}{rgb}{0.000000,0.000000,0.000000}%
\pgfsetstrokecolor{currentstroke}%
\pgfsetdash{}{0pt}%
\pgfsys@defobject{currentmarker}{\pgfqpoint{0.000000in}{-0.048611in}}{\pgfqpoint{0.000000in}{0.000000in}}{%
\pgfpathmoveto{\pgfqpoint{0.000000in}{0.000000in}}%
\pgfpathlineto{\pgfqpoint{0.000000in}{-0.048611in}}%
\pgfusepath{stroke,fill}%
}%
\begin{pgfscope}%
\pgfsys@transformshift{5.666777in}{0.549444in}%
\pgfsys@useobject{currentmarker}{}%
\end{pgfscope}%
\end{pgfscope}%
\begin{pgfscope}%
\definecolor{textcolor}{rgb}{0.000000,0.000000,0.000000}%
\pgfsetstrokecolor{textcolor}%
\pgfsetfillcolor{textcolor}%
\pgftext[x=5.666777in,y=0.452222in,,top]{\color{textcolor}\rmfamily\fontsize{10.000000}{12.000000}\selectfont \(\displaystyle {2.0}\)}%
\end{pgfscope}%
\begin{pgfscope}%
\definecolor{textcolor}{rgb}{0.000000,0.000000,0.000000}%
\pgfsetstrokecolor{textcolor}%
\pgfsetfillcolor{textcolor}%
\pgftext[x=3.189191in,y=0.273333in,,top]{\color{textcolor}\rmfamily\fontsize{10.000000}{12.000000}\selectfont \(\displaystyle \epsilon_1\)}%
\end{pgfscope}%
\begin{pgfscope}%
\pgfpathrectangle{\pgfqpoint{0.711606in}{0.549444in}}{\pgfqpoint{4.955171in}{2.902168in}}%
\pgfusepath{clip}%
\pgfsetbuttcap%
\pgfsetroundjoin%
\pgfsetlinewidth{0.803000pt}%
\definecolor{currentstroke}{rgb}{0.690196,0.690196,0.690196}%
\pgfsetstrokecolor{currentstroke}%
\pgfsetstrokeopacity{0.300000}%
\pgfsetdash{{2.960000pt}{1.280000pt}}{0.000000pt}%
\pgfpathmoveto{\pgfqpoint{0.711606in}{0.549444in}}%
\pgfpathlineto{\pgfqpoint{5.666777in}{0.549444in}}%
\pgfusepath{stroke}%
\end{pgfscope}%
\begin{pgfscope}%
\pgfsetbuttcap%
\pgfsetroundjoin%
\definecolor{currentfill}{rgb}{0.000000,0.000000,0.000000}%
\pgfsetfillcolor{currentfill}%
\pgfsetlinewidth{0.803000pt}%
\definecolor{currentstroke}{rgb}{0.000000,0.000000,0.000000}%
\pgfsetstrokecolor{currentstroke}%
\pgfsetdash{}{0pt}%
\pgfsys@defobject{currentmarker}{\pgfqpoint{-0.048611in}{0.000000in}}{\pgfqpoint{-0.000000in}{0.000000in}}{%
\pgfpathmoveto{\pgfqpoint{-0.000000in}{0.000000in}}%
\pgfpathlineto{\pgfqpoint{-0.048611in}{0.000000in}}%
\pgfusepath{stroke,fill}%
}%
\begin{pgfscope}%
\pgfsys@transformshift{0.711606in}{0.549444in}%
\pgfsys@useobject{currentmarker}{}%
\end{pgfscope}%
\end{pgfscope}%
\begin{pgfscope}%
\definecolor{textcolor}{rgb}{0.000000,0.000000,0.000000}%
\pgfsetstrokecolor{textcolor}%
\pgfsetfillcolor{textcolor}%
\pgftext[x=0.328889in, y=0.501250in, left, base]{\color{textcolor}\rmfamily\fontsize{10.000000}{12.000000}\selectfont \(\displaystyle {\ensuremath{-}1.0}\)}%
\end{pgfscope}%
\begin{pgfscope}%
\pgfpathrectangle{\pgfqpoint{0.711606in}{0.549444in}}{\pgfqpoint{4.955171in}{2.902168in}}%
\pgfusepath{clip}%
\pgfsetbuttcap%
\pgfsetroundjoin%
\pgfsetlinewidth{0.803000pt}%
\definecolor{currentstroke}{rgb}{0.690196,0.690196,0.690196}%
\pgfsetstrokecolor{currentstroke}%
\pgfsetstrokeopacity{0.300000}%
\pgfsetdash{{2.960000pt}{1.280000pt}}{0.000000pt}%
\pgfpathmoveto{\pgfqpoint{0.711606in}{1.033139in}}%
\pgfpathlineto{\pgfqpoint{5.666777in}{1.033139in}}%
\pgfusepath{stroke}%
\end{pgfscope}%
\begin{pgfscope}%
\pgfsetbuttcap%
\pgfsetroundjoin%
\definecolor{currentfill}{rgb}{0.000000,0.000000,0.000000}%
\pgfsetfillcolor{currentfill}%
\pgfsetlinewidth{0.803000pt}%
\definecolor{currentstroke}{rgb}{0.000000,0.000000,0.000000}%
\pgfsetstrokecolor{currentstroke}%
\pgfsetdash{}{0pt}%
\pgfsys@defobject{currentmarker}{\pgfqpoint{-0.048611in}{0.000000in}}{\pgfqpoint{-0.000000in}{0.000000in}}{%
\pgfpathmoveto{\pgfqpoint{-0.000000in}{0.000000in}}%
\pgfpathlineto{\pgfqpoint{-0.048611in}{0.000000in}}%
\pgfusepath{stroke,fill}%
}%
\begin{pgfscope}%
\pgfsys@transformshift{0.711606in}{1.033139in}%
\pgfsys@useobject{currentmarker}{}%
\end{pgfscope}%
\end{pgfscope}%
\begin{pgfscope}%
\definecolor{textcolor}{rgb}{0.000000,0.000000,0.000000}%
\pgfsetstrokecolor{textcolor}%
\pgfsetfillcolor{textcolor}%
\pgftext[x=0.328889in, y=0.984944in, left, base]{\color{textcolor}\rmfamily\fontsize{10.000000}{12.000000}\selectfont \(\displaystyle {\ensuremath{-}0.5}\)}%
\end{pgfscope}%
\begin{pgfscope}%
\pgfpathrectangle{\pgfqpoint{0.711606in}{0.549444in}}{\pgfqpoint{4.955171in}{2.902168in}}%
\pgfusepath{clip}%
\pgfsetbuttcap%
\pgfsetroundjoin%
\pgfsetlinewidth{0.803000pt}%
\definecolor{currentstroke}{rgb}{0.690196,0.690196,0.690196}%
\pgfsetstrokecolor{currentstroke}%
\pgfsetstrokeopacity{0.300000}%
\pgfsetdash{{2.960000pt}{1.280000pt}}{0.000000pt}%
\pgfpathmoveto{\pgfqpoint{0.711606in}{1.516834in}}%
\pgfpathlineto{\pgfqpoint{5.666777in}{1.516834in}}%
\pgfusepath{stroke}%
\end{pgfscope}%
\begin{pgfscope}%
\pgfsetbuttcap%
\pgfsetroundjoin%
\definecolor{currentfill}{rgb}{0.000000,0.000000,0.000000}%
\pgfsetfillcolor{currentfill}%
\pgfsetlinewidth{0.803000pt}%
\definecolor{currentstroke}{rgb}{0.000000,0.000000,0.000000}%
\pgfsetstrokecolor{currentstroke}%
\pgfsetdash{}{0pt}%
\pgfsys@defobject{currentmarker}{\pgfqpoint{-0.048611in}{0.000000in}}{\pgfqpoint{-0.000000in}{0.000000in}}{%
\pgfpathmoveto{\pgfqpoint{-0.000000in}{0.000000in}}%
\pgfpathlineto{\pgfqpoint{-0.048611in}{0.000000in}}%
\pgfusepath{stroke,fill}%
}%
\begin{pgfscope}%
\pgfsys@transformshift{0.711606in}{1.516834in}%
\pgfsys@useobject{currentmarker}{}%
\end{pgfscope}%
\end{pgfscope}%
\begin{pgfscope}%
\definecolor{textcolor}{rgb}{0.000000,0.000000,0.000000}%
\pgfsetstrokecolor{textcolor}%
\pgfsetfillcolor{textcolor}%
\pgftext[x=0.436914in, y=1.468639in, left, base]{\color{textcolor}\rmfamily\fontsize{10.000000}{12.000000}\selectfont \(\displaystyle {0.0}\)}%
\end{pgfscope}%
\begin{pgfscope}%
\pgfpathrectangle{\pgfqpoint{0.711606in}{0.549444in}}{\pgfqpoint{4.955171in}{2.902168in}}%
\pgfusepath{clip}%
\pgfsetbuttcap%
\pgfsetroundjoin%
\pgfsetlinewidth{0.803000pt}%
\definecolor{currentstroke}{rgb}{0.690196,0.690196,0.690196}%
\pgfsetstrokecolor{currentstroke}%
\pgfsetstrokeopacity{0.300000}%
\pgfsetdash{{2.960000pt}{1.280000pt}}{0.000000pt}%
\pgfpathmoveto{\pgfqpoint{0.711606in}{2.000528in}}%
\pgfpathlineto{\pgfqpoint{5.666777in}{2.000528in}}%
\pgfusepath{stroke}%
\end{pgfscope}%
\begin{pgfscope}%
\pgfsetbuttcap%
\pgfsetroundjoin%
\definecolor{currentfill}{rgb}{0.000000,0.000000,0.000000}%
\pgfsetfillcolor{currentfill}%
\pgfsetlinewidth{0.803000pt}%
\definecolor{currentstroke}{rgb}{0.000000,0.000000,0.000000}%
\pgfsetstrokecolor{currentstroke}%
\pgfsetdash{}{0pt}%
\pgfsys@defobject{currentmarker}{\pgfqpoint{-0.048611in}{0.000000in}}{\pgfqpoint{-0.000000in}{0.000000in}}{%
\pgfpathmoveto{\pgfqpoint{-0.000000in}{0.000000in}}%
\pgfpathlineto{\pgfqpoint{-0.048611in}{0.000000in}}%
\pgfusepath{stroke,fill}%
}%
\begin{pgfscope}%
\pgfsys@transformshift{0.711606in}{2.000528in}%
\pgfsys@useobject{currentmarker}{}%
\end{pgfscope}%
\end{pgfscope}%
\begin{pgfscope}%
\definecolor{textcolor}{rgb}{0.000000,0.000000,0.000000}%
\pgfsetstrokecolor{textcolor}%
\pgfsetfillcolor{textcolor}%
\pgftext[x=0.436914in, y=1.952334in, left, base]{\color{textcolor}\rmfamily\fontsize{10.000000}{12.000000}\selectfont \(\displaystyle {0.5}\)}%
\end{pgfscope}%
\begin{pgfscope}%
\pgfpathrectangle{\pgfqpoint{0.711606in}{0.549444in}}{\pgfqpoint{4.955171in}{2.902168in}}%
\pgfusepath{clip}%
\pgfsetbuttcap%
\pgfsetroundjoin%
\pgfsetlinewidth{0.803000pt}%
\definecolor{currentstroke}{rgb}{0.690196,0.690196,0.690196}%
\pgfsetstrokecolor{currentstroke}%
\pgfsetstrokeopacity{0.300000}%
\pgfsetdash{{2.960000pt}{1.280000pt}}{0.000000pt}%
\pgfpathmoveto{\pgfqpoint{0.711606in}{2.484223in}}%
\pgfpathlineto{\pgfqpoint{5.666777in}{2.484223in}}%
\pgfusepath{stroke}%
\end{pgfscope}%
\begin{pgfscope}%
\pgfsetbuttcap%
\pgfsetroundjoin%
\definecolor{currentfill}{rgb}{0.000000,0.000000,0.000000}%
\pgfsetfillcolor{currentfill}%
\pgfsetlinewidth{0.803000pt}%
\definecolor{currentstroke}{rgb}{0.000000,0.000000,0.000000}%
\pgfsetstrokecolor{currentstroke}%
\pgfsetdash{}{0pt}%
\pgfsys@defobject{currentmarker}{\pgfqpoint{-0.048611in}{0.000000in}}{\pgfqpoint{-0.000000in}{0.000000in}}{%
\pgfpathmoveto{\pgfqpoint{-0.000000in}{0.000000in}}%
\pgfpathlineto{\pgfqpoint{-0.048611in}{0.000000in}}%
\pgfusepath{stroke,fill}%
}%
\begin{pgfscope}%
\pgfsys@transformshift{0.711606in}{2.484223in}%
\pgfsys@useobject{currentmarker}{}%
\end{pgfscope}%
\end{pgfscope}%
\begin{pgfscope}%
\definecolor{textcolor}{rgb}{0.000000,0.000000,0.000000}%
\pgfsetstrokecolor{textcolor}%
\pgfsetfillcolor{textcolor}%
\pgftext[x=0.436914in, y=2.436029in, left, base]{\color{textcolor}\rmfamily\fontsize{10.000000}{12.000000}\selectfont \(\displaystyle {1.0}\)}%
\end{pgfscope}%
\begin{pgfscope}%
\pgfpathrectangle{\pgfqpoint{0.711606in}{0.549444in}}{\pgfqpoint{4.955171in}{2.902168in}}%
\pgfusepath{clip}%
\pgfsetbuttcap%
\pgfsetroundjoin%
\pgfsetlinewidth{0.803000pt}%
\definecolor{currentstroke}{rgb}{0.690196,0.690196,0.690196}%
\pgfsetstrokecolor{currentstroke}%
\pgfsetstrokeopacity{0.300000}%
\pgfsetdash{{2.960000pt}{1.280000pt}}{0.000000pt}%
\pgfpathmoveto{\pgfqpoint{0.711606in}{2.967918in}}%
\pgfpathlineto{\pgfqpoint{5.666777in}{2.967918in}}%
\pgfusepath{stroke}%
\end{pgfscope}%
\begin{pgfscope}%
\pgfsetbuttcap%
\pgfsetroundjoin%
\definecolor{currentfill}{rgb}{0.000000,0.000000,0.000000}%
\pgfsetfillcolor{currentfill}%
\pgfsetlinewidth{0.803000pt}%
\definecolor{currentstroke}{rgb}{0.000000,0.000000,0.000000}%
\pgfsetstrokecolor{currentstroke}%
\pgfsetdash{}{0pt}%
\pgfsys@defobject{currentmarker}{\pgfqpoint{-0.048611in}{0.000000in}}{\pgfqpoint{-0.000000in}{0.000000in}}{%
\pgfpathmoveto{\pgfqpoint{-0.000000in}{0.000000in}}%
\pgfpathlineto{\pgfqpoint{-0.048611in}{0.000000in}}%
\pgfusepath{stroke,fill}%
}%
\begin{pgfscope}%
\pgfsys@transformshift{0.711606in}{2.967918in}%
\pgfsys@useobject{currentmarker}{}%
\end{pgfscope}%
\end{pgfscope}%
\begin{pgfscope}%
\definecolor{textcolor}{rgb}{0.000000,0.000000,0.000000}%
\pgfsetstrokecolor{textcolor}%
\pgfsetfillcolor{textcolor}%
\pgftext[x=0.436914in, y=2.919723in, left, base]{\color{textcolor}\rmfamily\fontsize{10.000000}{12.000000}\selectfont \(\displaystyle {1.5}\)}%
\end{pgfscope}%
\begin{pgfscope}%
\pgfpathrectangle{\pgfqpoint{0.711606in}{0.549444in}}{\pgfqpoint{4.955171in}{2.902168in}}%
\pgfusepath{clip}%
\pgfsetbuttcap%
\pgfsetroundjoin%
\pgfsetlinewidth{0.803000pt}%
\definecolor{currentstroke}{rgb}{0.690196,0.690196,0.690196}%
\pgfsetstrokecolor{currentstroke}%
\pgfsetstrokeopacity{0.300000}%
\pgfsetdash{{2.960000pt}{1.280000pt}}{0.000000pt}%
\pgfpathmoveto{\pgfqpoint{0.711606in}{3.451613in}}%
\pgfpathlineto{\pgfqpoint{5.666777in}{3.451613in}}%
\pgfusepath{stroke}%
\end{pgfscope}%
\begin{pgfscope}%
\pgfsetbuttcap%
\pgfsetroundjoin%
\definecolor{currentfill}{rgb}{0.000000,0.000000,0.000000}%
\pgfsetfillcolor{currentfill}%
\pgfsetlinewidth{0.803000pt}%
\definecolor{currentstroke}{rgb}{0.000000,0.000000,0.000000}%
\pgfsetstrokecolor{currentstroke}%
\pgfsetdash{}{0pt}%
\pgfsys@defobject{currentmarker}{\pgfqpoint{-0.048611in}{0.000000in}}{\pgfqpoint{-0.000000in}{0.000000in}}{%
\pgfpathmoveto{\pgfqpoint{-0.000000in}{0.000000in}}%
\pgfpathlineto{\pgfqpoint{-0.048611in}{0.000000in}}%
\pgfusepath{stroke,fill}%
}%
\begin{pgfscope}%
\pgfsys@transformshift{0.711606in}{3.451613in}%
\pgfsys@useobject{currentmarker}{}%
\end{pgfscope}%
\end{pgfscope}%
\begin{pgfscope}%
\definecolor{textcolor}{rgb}{0.000000,0.000000,0.000000}%
\pgfsetstrokecolor{textcolor}%
\pgfsetfillcolor{textcolor}%
\pgftext[x=0.436914in, y=3.403418in, left, base]{\color{textcolor}\rmfamily\fontsize{10.000000}{12.000000}\selectfont \(\displaystyle {2.0}\)}%
\end{pgfscope}%
\begin{pgfscope}%
\definecolor{textcolor}{rgb}{0.000000,0.000000,0.000000}%
\pgfsetstrokecolor{textcolor}%
\pgfsetfillcolor{textcolor}%
\pgftext[x=0.273333in,y=2.000528in,,bottom,rotate=90.000000]{\color{textcolor}\rmfamily\fontsize{10.000000}{12.000000}\selectfont \(\displaystyle \epsilon_2\)}%
\end{pgfscope}%
\begin{pgfscope}%
\pgfpathrectangle{\pgfqpoint{0.711606in}{0.549444in}}{\pgfqpoint{4.955171in}{2.902168in}}%
\pgfusepath{clip}%
\pgfsetbuttcap%
\pgfsetroundjoin%
\pgfsetlinewidth{1.003750pt}%
\definecolor{currentstroke}{rgb}{0.004547,0.003392,0.030909}%
\pgfsetstrokecolor{currentstroke}%
\pgfsetdash{}{0pt}%
\pgfpathmoveto{\pgfqpoint{2.474183in}{1.269294in}}%
\pgfpathlineto{\pgfqpoint{2.462713in}{1.270117in}}%
\pgfpathlineto{\pgfqpoint{2.440927in}{1.271770in}}%
\pgfpathlineto{\pgfqpoint{2.407671in}{1.275871in}}%
\pgfpathlineto{\pgfqpoint{2.374415in}{1.281388in}}%
\pgfpathlineto{\pgfqpoint{2.341159in}{1.288204in}}%
\pgfpathlineto{\pgfqpoint{2.335514in}{1.289595in}}%
\pgfpathlineto{\pgfqpoint{2.307902in}{1.296776in}}%
\pgfpathlineto{\pgfqpoint{2.274646in}{1.306621in}}%
\pgfpathlineto{\pgfqpoint{2.267328in}{1.309072in}}%
\pgfpathlineto{\pgfqpoint{2.241390in}{1.318259in}}%
\pgfpathlineto{\pgfqpoint{2.214926in}{1.328550in}}%
\pgfpathlineto{\pgfqpoint{2.208134in}{1.331352in}}%
\pgfpathlineto{\pgfqpoint{2.174878in}{1.346347in}}%
\pgfpathlineto{\pgfqpoint{2.171443in}{1.348027in}}%
\pgfpathlineto{\pgfqpoint{2.141622in}{1.363529in}}%
\pgfpathlineto{\pgfqpoint{2.134492in}{1.367505in}}%
\pgfpathlineto{\pgfqpoint{2.108365in}{1.383049in}}%
\pgfpathlineto{\pgfqpoint{2.102174in}{1.386983in}}%
\pgfpathlineto{\pgfqpoint{2.075109in}{1.405409in}}%
\pgfpathlineto{\pgfqpoint{2.073657in}{1.406460in}}%
\pgfpathlineto{\pgfqpoint{2.048899in}{1.425938in}}%
\pgfpathlineto{\pgfqpoint{2.041853in}{1.431976in}}%
\pgfpathlineto{\pgfqpoint{2.027083in}{1.445416in}}%
\pgfpathlineto{\pgfqpoint{2.008597in}{1.463846in}}%
\pgfpathlineto{\pgfqpoint{2.007604in}{1.464893in}}%
\pgfpathlineto{\pgfqpoint{1.991356in}{1.484371in}}%
\pgfpathlineto{\pgfqpoint{1.977003in}{1.503849in}}%
\pgfpathlineto{\pgfqpoint{1.975341in}{1.506553in}}%
\pgfpathlineto{\pgfqpoint{1.965599in}{1.523326in}}%
\pgfpathlineto{\pgfqpoint{1.956459in}{1.542804in}}%
\pgfpathlineto{\pgfqpoint{1.949638in}{1.562281in}}%
\pgfpathlineto{\pgfqpoint{1.945347in}{1.581759in}}%
\pgfpathlineto{\pgfqpoint{1.943827in}{1.601237in}}%
\pgfpathlineto{\pgfqpoint{1.945348in}{1.620714in}}%
\pgfpathlineto{\pgfqpoint{1.950217in}{1.640192in}}%
\pgfpathlineto{\pgfqpoint{1.958784in}{1.659670in}}%
\pgfpathlineto{\pgfqpoint{1.971449in}{1.679147in}}%
\pgfpathlineto{\pgfqpoint{1.975341in}{1.683675in}}%
\pgfpathlineto{\pgfqpoint{1.990335in}{1.698625in}}%
\pgfpathlineto{\pgfqpoint{2.008597in}{1.712856in}}%
\pgfpathlineto{\pgfqpoint{2.016678in}{1.718103in}}%
\pgfpathlineto{\pgfqpoint{2.041853in}{1.731517in}}%
\pgfpathlineto{\pgfqpoint{2.056081in}{1.737580in}}%
\pgfpathlineto{\pgfqpoint{2.075109in}{1.744456in}}%
\pgfpathlineto{\pgfqpoint{2.108365in}{1.753600in}}%
\pgfpathlineto{\pgfqpoint{2.126042in}{1.757058in}}%
\pgfpathlineto{\pgfqpoint{2.141622in}{1.759717in}}%
\pgfpathlineto{\pgfqpoint{2.174878in}{1.763299in}}%
\pgfpathlineto{\pgfqpoint{2.208134in}{1.764888in}}%
\pgfpathlineto{\pgfqpoint{2.241390in}{1.764662in}}%
\pgfpathlineto{\pgfqpoint{2.274646in}{1.762780in}}%
\pgfpathlineto{\pgfqpoint{2.307902in}{1.759380in}}%
\pgfpathlineto{\pgfqpoint{2.323784in}{1.757058in}}%
\pgfpathlineto{\pgfqpoint{2.341159in}{1.754374in}}%
\pgfpathlineto{\pgfqpoint{2.374415in}{1.747798in}}%
\pgfpathlineto{\pgfqpoint{2.407671in}{1.739971in}}%
\pgfpathlineto{\pgfqpoint{2.416353in}{1.737580in}}%
\pgfpathlineto{\pgfqpoint{2.440927in}{1.730432in}}%
\pgfpathlineto{\pgfqpoint{2.474183in}{1.719581in}}%
\pgfpathlineto{\pgfqpoint{2.478227in}{1.718103in}}%
\pgfpathlineto{\pgfqpoint{2.507440in}{1.706806in}}%
\pgfpathlineto{\pgfqpoint{2.526841in}{1.698625in}}%
\pgfpathlineto{\pgfqpoint{2.540696in}{1.692424in}}%
\pgfpathlineto{\pgfqpoint{2.568085in}{1.679147in}}%
\pgfpathlineto{\pgfqpoint{2.573952in}{1.676118in}}%
\pgfpathlineto{\pgfqpoint{2.603534in}{1.659670in}}%
\pgfpathlineto{\pgfqpoint{2.607208in}{1.657484in}}%
\pgfpathlineto{\pgfqpoint{2.634342in}{1.640192in}}%
\pgfpathlineto{\pgfqpoint{2.640464in}{1.635997in}}%
\pgfpathlineto{\pgfqpoint{2.661377in}{1.620714in}}%
\pgfpathlineto{\pgfqpoint{2.673720in}{1.610960in}}%
\pgfpathlineto{\pgfqpoint{2.685303in}{1.601237in}}%
\pgfpathlineto{\pgfqpoint{2.706606in}{1.581759in}}%
\pgfpathlineto{\pgfqpoint{2.706977in}{1.581379in}}%
\pgfpathlineto{\pgfqpoint{2.724543in}{1.562281in}}%
\pgfpathlineto{\pgfqpoint{2.740233in}{1.543277in}}%
\pgfpathlineto{\pgfqpoint{2.740603in}{1.542804in}}%
\pgfpathlineto{\pgfqpoint{2.753531in}{1.523326in}}%
\pgfpathlineto{\pgfqpoint{2.764408in}{1.503849in}}%
\pgfpathlineto{\pgfqpoint{2.773047in}{1.484371in}}%
\pgfpathlineto{\pgfqpoint{2.773489in}{1.482941in}}%
\pgfpathlineto{\pgfqpoint{2.778760in}{1.464893in}}%
\pgfpathlineto{\pgfqpoint{2.781960in}{1.445416in}}%
\pgfpathlineto{\pgfqpoint{2.782461in}{1.425938in}}%
\pgfpathlineto{\pgfqpoint{2.780012in}{1.406460in}}%
\pgfpathlineto{\pgfqpoint{2.774332in}{1.386983in}}%
\pgfpathlineto{\pgfqpoint{2.773489in}{1.385152in}}%
\pgfpathlineto{\pgfqpoint{2.764192in}{1.367505in}}%
\pgfpathlineto{\pgfqpoint{2.749536in}{1.348027in}}%
\pgfpathlineto{\pgfqpoint{2.740233in}{1.338598in}}%
\pgfpathlineto{\pgfqpoint{2.728567in}{1.328550in}}%
\pgfpathlineto{\pgfqpoint{2.706977in}{1.313740in}}%
\pgfpathlineto{\pgfqpoint{2.698712in}{1.309072in}}%
\pgfpathlineto{\pgfqpoint{2.673720in}{1.297345in}}%
\pgfpathlineto{\pgfqpoint{2.652692in}{1.289595in}}%
\pgfpathlineto{\pgfqpoint{2.640464in}{1.285740in}}%
\pgfpathlineto{\pgfqpoint{2.607208in}{1.277834in}}%
\pgfpathlineto{\pgfqpoint{2.573952in}{1.272467in}}%
\pgfpathlineto{\pgfqpoint{2.548794in}{1.270117in}}%
\pgfpathlineto{\pgfqpoint{2.540696in}{1.269450in}}%
\pgfpathlineto{\pgfqpoint{2.507440in}{1.268517in}}%
\pgfpathlineto{\pgfqpoint{2.474183in}{1.269294in}}%
\pgfusepath{stroke}%
\end{pgfscope}%
\begin{pgfscope}%
\pgfpathrectangle{\pgfqpoint{0.711606in}{0.549444in}}{\pgfqpoint{4.955171in}{2.902168in}}%
\pgfusepath{clip}%
\pgfsetbuttcap%
\pgfsetroundjoin%
\pgfsetlinewidth{1.003750pt}%
\definecolor{currentstroke}{rgb}{0.004547,0.003392,0.030909}%
\pgfsetstrokecolor{currentstroke}%
\pgfsetdash{}{0pt}%
\pgfpathmoveto{\pgfqpoint{4.070480in}{1.424382in}}%
\pgfpathlineto{\pgfqpoint{4.041229in}{1.425938in}}%
\pgfpathlineto{\pgfqpoint{4.037224in}{1.426186in}}%
\pgfpathlineto{\pgfqpoint{4.003968in}{1.432711in}}%
\pgfpathlineto{\pgfqpoint{3.970712in}{1.442564in}}%
\pgfpathlineto{\pgfqpoint{3.963532in}{1.445416in}}%
\pgfpathlineto{\pgfqpoint{3.937455in}{1.457481in}}%
\pgfpathlineto{\pgfqpoint{3.924278in}{1.464893in}}%
\pgfpathlineto{\pgfqpoint{3.904199in}{1.478419in}}%
\pgfpathlineto{\pgfqpoint{3.896696in}{1.484371in}}%
\pgfpathlineto{\pgfqpoint{3.877496in}{1.503849in}}%
\pgfpathlineto{\pgfqpoint{3.870943in}{1.513431in}}%
\pgfpathlineto{\pgfqpoint{3.865126in}{1.523326in}}%
\pgfpathlineto{\pgfqpoint{3.859772in}{1.542804in}}%
\pgfpathlineto{\pgfqpoint{3.861571in}{1.562281in}}%
\pgfpathlineto{\pgfqpoint{3.870943in}{1.579222in}}%
\pgfpathlineto{\pgfqpoint{3.873040in}{1.581759in}}%
\pgfpathlineto{\pgfqpoint{3.904199in}{1.600572in}}%
\pgfpathlineto{\pgfqpoint{3.906373in}{1.601237in}}%
\pgfpathlineto{\pgfqpoint{3.937455in}{1.607565in}}%
\pgfpathlineto{\pgfqpoint{3.970712in}{1.608754in}}%
\pgfpathlineto{\pgfqpoint{4.003968in}{1.605552in}}%
\pgfpathlineto{\pgfqpoint{4.024859in}{1.601237in}}%
\pgfpathlineto{\pgfqpoint{4.037224in}{1.598270in}}%
\pgfpathlineto{\pgfqpoint{4.070480in}{1.586634in}}%
\pgfpathlineto{\pgfqpoint{4.081391in}{1.581759in}}%
\pgfpathlineto{\pgfqpoint{4.103736in}{1.570032in}}%
\pgfpathlineto{\pgfqpoint{4.116030in}{1.562281in}}%
\pgfpathlineto{\pgfqpoint{4.136993in}{1.546269in}}%
\pgfpathlineto{\pgfqpoint{4.140879in}{1.542804in}}%
\pgfpathlineto{\pgfqpoint{4.157141in}{1.523326in}}%
\pgfpathlineto{\pgfqpoint{4.167929in}{1.503849in}}%
\pgfpathlineto{\pgfqpoint{4.170249in}{1.491478in}}%
\pgfpathlineto{\pgfqpoint{4.171390in}{1.484371in}}%
\pgfpathlineto{\pgfqpoint{4.170249in}{1.478736in}}%
\pgfpathlineto{\pgfqpoint{4.166299in}{1.464893in}}%
\pgfpathlineto{\pgfqpoint{4.148532in}{1.445416in}}%
\pgfpathlineto{\pgfqpoint{4.136993in}{1.438447in}}%
\pgfpathlineto{\pgfqpoint{4.103736in}{1.427365in}}%
\pgfpathlineto{\pgfqpoint{4.090654in}{1.425938in}}%
\pgfpathlineto{\pgfqpoint{4.070480in}{1.424382in}}%
\pgfpathclose%
\pgfusepath{stroke}%
\end{pgfscope}%
\begin{pgfscope}%
\pgfpathrectangle{\pgfqpoint{0.711606in}{0.549444in}}{\pgfqpoint{4.955171in}{2.902168in}}%
\pgfusepath{clip}%
\pgfsetbuttcap%
\pgfsetroundjoin%
\pgfsetlinewidth{1.003750pt}%
\definecolor{currentstroke}{rgb}{0.011663,0.009417,0.063460}%
\pgfsetstrokecolor{currentstroke}%
\pgfsetdash{}{0pt}%
\pgfpathmoveto{\pgfqpoint{2.440927in}{1.133370in}}%
\pgfpathlineto{\pgfqpoint{2.438208in}{1.133773in}}%
\pgfpathlineto{\pgfqpoint{2.407671in}{1.138465in}}%
\pgfpathlineto{\pgfqpoint{2.374415in}{1.144442in}}%
\pgfpathlineto{\pgfqpoint{2.341159in}{1.151232in}}%
\pgfpathlineto{\pgfqpoint{2.332391in}{1.153251in}}%
\pgfpathlineto{\pgfqpoint{2.307902in}{1.159088in}}%
\pgfpathlineto{\pgfqpoint{2.274646in}{1.167809in}}%
\pgfpathlineto{\pgfqpoint{2.257457in}{1.172729in}}%
\pgfpathlineto{\pgfqpoint{2.241390in}{1.177490in}}%
\pgfpathlineto{\pgfqpoint{2.208134in}{1.188137in}}%
\pgfpathlineto{\pgfqpoint{2.196302in}{1.192206in}}%
\pgfpathlineto{\pgfqpoint{2.174878in}{1.199840in}}%
\pgfpathlineto{\pgfqpoint{2.143563in}{1.211684in}}%
\pgfpathlineto{\pgfqpoint{2.141622in}{1.212446in}}%
\pgfpathlineto{\pgfqpoint{2.108365in}{1.226345in}}%
\pgfpathlineto{\pgfqpoint{2.097436in}{1.231162in}}%
\pgfpathlineto{\pgfqpoint{2.075109in}{1.241376in}}%
\pgfpathlineto{\pgfqpoint{2.055834in}{1.250639in}}%
\pgfpathlineto{\pgfqpoint{2.041853in}{1.257625in}}%
\pgfpathlineto{\pgfqpoint{2.018001in}{1.270117in}}%
\pgfpathlineto{\pgfqpoint{2.008597in}{1.275246in}}%
\pgfpathlineto{\pgfqpoint{1.983441in}{1.289595in}}%
\pgfpathlineto{\pgfqpoint{1.975341in}{1.294414in}}%
\pgfpathlineto{\pgfqpoint{1.951740in}{1.309072in}}%
\pgfpathlineto{\pgfqpoint{1.942084in}{1.315340in}}%
\pgfpathlineto{\pgfqpoint{1.922554in}{1.328550in}}%
\pgfpathlineto{\pgfqpoint{1.908828in}{1.338271in}}%
\pgfpathlineto{\pgfqpoint{1.895589in}{1.348027in}}%
\pgfpathlineto{\pgfqpoint{1.875572in}{1.363508in}}%
\pgfpathlineto{\pgfqpoint{1.870597in}{1.367505in}}%
\pgfpathlineto{\pgfqpoint{1.847654in}{1.386983in}}%
\pgfpathlineto{\pgfqpoint{1.842316in}{1.391792in}}%
\pgfpathlineto{\pgfqpoint{1.826630in}{1.406460in}}%
\pgfpathlineto{\pgfqpoint{1.809060in}{1.423877in}}%
\pgfpathlineto{\pgfqpoint{1.807054in}{1.425938in}}%
\pgfpathlineto{\pgfqpoint{1.789484in}{1.445416in}}%
\pgfpathlineto{\pgfqpoint{1.775804in}{1.461689in}}%
\pgfpathlineto{\pgfqpoint{1.773204in}{1.464893in}}%
\pgfpathlineto{\pgfqpoint{1.758805in}{1.484371in}}%
\pgfpathlineto{\pgfqpoint{1.745648in}{1.503849in}}%
\pgfpathlineto{\pgfqpoint{1.742547in}{1.509016in}}%
\pgfpathlineto{\pgfqpoint{1.734261in}{1.523326in}}%
\pgfpathlineto{\pgfqpoint{1.724368in}{1.542804in}}%
\pgfpathlineto{\pgfqpoint{1.715883in}{1.562281in}}%
\pgfpathlineto{\pgfqpoint{1.709291in}{1.580647in}}%
\pgfpathlineto{\pgfqpoint{1.708906in}{1.581759in}}%
\pgfpathlineto{\pgfqpoint{1.703770in}{1.601237in}}%
\pgfpathlineto{\pgfqpoint{1.700218in}{1.620714in}}%
\pgfpathlineto{\pgfqpoint{1.698338in}{1.640192in}}%
\pgfpathlineto{\pgfqpoint{1.698227in}{1.659670in}}%
\pgfpathlineto{\pgfqpoint{1.699987in}{1.679147in}}%
\pgfpathlineto{\pgfqpoint{1.703729in}{1.698625in}}%
\pgfpathlineto{\pgfqpoint{1.709291in}{1.717174in}}%
\pgfpathlineto{\pgfqpoint{1.709593in}{1.718103in}}%
\pgfpathlineto{\pgfqpoint{1.718200in}{1.737580in}}%
\pgfpathlineto{\pgfqpoint{1.729362in}{1.757058in}}%
\pgfpathlineto{\pgfqpoint{1.742547in}{1.775560in}}%
\pgfpathlineto{\pgfqpoint{1.743310in}{1.776536in}}%
\pgfpathlineto{\pgfqpoint{1.761396in}{1.796013in}}%
\pgfpathlineto{\pgfqpoint{1.775804in}{1.809158in}}%
\pgfpathlineto{\pgfqpoint{1.783520in}{1.815491in}}%
\pgfpathlineto{\pgfqpoint{1.809060in}{1.833658in}}%
\pgfpathlineto{\pgfqpoint{1.811133in}{1.834968in}}%
\pgfpathlineto{\pgfqpoint{1.842316in}{1.852358in}}%
\pgfpathlineto{\pgfqpoint{1.846597in}{1.854446in}}%
\pgfpathlineto{\pgfqpoint{1.875572in}{1.867089in}}%
\pgfpathlineto{\pgfqpoint{1.893862in}{1.873924in}}%
\pgfpathlineto{\pgfqpoint{1.908828in}{1.878984in}}%
\pgfpathlineto{\pgfqpoint{1.942084in}{1.888464in}}%
\pgfpathlineto{\pgfqpoint{1.963239in}{1.893401in}}%
\pgfpathlineto{\pgfqpoint{1.975341in}{1.895985in}}%
\pgfpathlineto{\pgfqpoint{2.008597in}{1.901648in}}%
\pgfpathlineto{\pgfqpoint{2.041853in}{1.905868in}}%
\pgfpathlineto{\pgfqpoint{2.075109in}{1.908735in}}%
\pgfpathlineto{\pgfqpoint{2.108365in}{1.910328in}}%
\pgfpathlineto{\pgfqpoint{2.141622in}{1.910724in}}%
\pgfpathlineto{\pgfqpoint{2.174878in}{1.909993in}}%
\pgfpathlineto{\pgfqpoint{2.208134in}{1.908197in}}%
\pgfpathlineto{\pgfqpoint{2.241390in}{1.905396in}}%
\pgfpathlineto{\pgfqpoint{2.274646in}{1.901645in}}%
\pgfpathlineto{\pgfqpoint{2.307902in}{1.896995in}}%
\pgfpathlineto{\pgfqpoint{2.329490in}{1.893401in}}%
\pgfpathlineto{\pgfqpoint{2.341159in}{1.891391in}}%
\pgfpathlineto{\pgfqpoint{2.374415in}{1.884751in}}%
\pgfpathlineto{\pgfqpoint{2.407671in}{1.877317in}}%
\pgfpathlineto{\pgfqpoint{2.421310in}{1.873924in}}%
\pgfpathlineto{\pgfqpoint{2.440927in}{1.868873in}}%
\pgfpathlineto{\pgfqpoint{2.474183in}{1.859509in}}%
\pgfpathlineto{\pgfqpoint{2.490770in}{1.854446in}}%
\pgfpathlineto{\pgfqpoint{2.507440in}{1.849177in}}%
\pgfpathlineto{\pgfqpoint{2.540696in}{1.837889in}}%
\pgfpathlineto{\pgfqpoint{2.548729in}{1.834968in}}%
\pgfpathlineto{\pgfqpoint{2.573952in}{1.825465in}}%
\pgfpathlineto{\pgfqpoint{2.598962in}{1.815491in}}%
\pgfpathlineto{\pgfqpoint{2.607208in}{1.812078in}}%
\pgfpathlineto{\pgfqpoint{2.640464in}{1.797526in}}%
\pgfpathlineto{\pgfqpoint{2.643739in}{1.796013in}}%
\pgfpathlineto{\pgfqpoint{2.673720in}{1.781631in}}%
\pgfpathlineto{\pgfqpoint{2.683851in}{1.776536in}}%
\pgfpathlineto{\pgfqpoint{2.706977in}{1.764439in}}%
\pgfpathlineto{\pgfqpoint{2.720465in}{1.757058in}}%
\pgfpathlineto{\pgfqpoint{2.740233in}{1.745790in}}%
\pgfpathlineto{\pgfqpoint{2.754026in}{1.737580in}}%
\pgfpathlineto{\pgfqpoint{2.773489in}{1.725493in}}%
\pgfpathlineto{\pgfqpoint{2.784906in}{1.718103in}}%
\pgfpathlineto{\pgfqpoint{2.806745in}{1.703323in}}%
\pgfpathlineto{\pgfqpoint{2.813418in}{1.698625in}}%
\pgfpathlineto{\pgfqpoint{2.839818in}{1.679147in}}%
\pgfpathlineto{\pgfqpoint{2.840001in}{1.679004in}}%
\pgfpathlineto{\pgfqpoint{2.863853in}{1.659670in}}%
\pgfpathlineto{\pgfqpoint{2.873258in}{1.651634in}}%
\pgfpathlineto{\pgfqpoint{2.886156in}{1.640192in}}%
\pgfpathlineto{\pgfqpoint{2.906514in}{1.621100in}}%
\pgfpathlineto{\pgfqpoint{2.906910in}{1.620714in}}%
\pgfpathlineto{\pgfqpoint{2.925540in}{1.601237in}}%
\pgfpathlineto{\pgfqpoint{2.939770in}{1.585356in}}%
\pgfpathlineto{\pgfqpoint{2.942882in}{1.581759in}}%
\pgfpathlineto{\pgfqpoint{2.958342in}{1.562281in}}%
\pgfpathlineto{\pgfqpoint{2.972588in}{1.542804in}}%
\pgfpathlineto{\pgfqpoint{2.973026in}{1.542133in}}%
\pgfpathlineto{\pgfqpoint{2.984898in}{1.523326in}}%
\pgfpathlineto{\pgfqpoint{2.995878in}{1.503849in}}%
\pgfpathlineto{\pgfqpoint{3.005480in}{1.484371in}}%
\pgfpathlineto{\pgfqpoint{3.006282in}{1.482420in}}%
\pgfpathlineto{\pgfqpoint{3.013242in}{1.464893in}}%
\pgfpathlineto{\pgfqpoint{3.019491in}{1.445416in}}%
\pgfpathlineto{\pgfqpoint{3.024187in}{1.425938in}}%
\pgfpathlineto{\pgfqpoint{3.027244in}{1.406460in}}%
\pgfpathlineto{\pgfqpoint{3.028567in}{1.386983in}}%
\pgfpathlineto{\pgfqpoint{3.028055in}{1.367505in}}%
\pgfpathlineto{\pgfqpoint{3.025598in}{1.348027in}}%
\pgfpathlineto{\pgfqpoint{3.021077in}{1.328550in}}%
\pgfpathlineto{\pgfqpoint{3.014363in}{1.309072in}}%
\pgfpathlineto{\pgfqpoint{3.006282in}{1.291636in}}%
\pgfpathlineto{\pgfqpoint{3.005251in}{1.289595in}}%
\pgfpathlineto{\pgfqpoint{2.992927in}{1.270117in}}%
\pgfpathlineto{\pgfqpoint{2.977729in}{1.250639in}}%
\pgfpathlineto{\pgfqpoint{2.973026in}{1.245539in}}%
\pgfpathlineto{\pgfqpoint{2.958393in}{1.231162in}}%
\pgfpathlineto{\pgfqpoint{2.939770in}{1.215541in}}%
\pgfpathlineto{\pgfqpoint{2.934639in}{1.211684in}}%
\pgfpathlineto{\pgfqpoint{2.906514in}{1.193241in}}%
\pgfpathlineto{\pgfqpoint{2.904730in}{1.192206in}}%
\pgfpathlineto{\pgfqpoint{2.873258in}{1.176024in}}%
\pgfpathlineto{\pgfqpoint{2.865882in}{1.172729in}}%
\pgfpathlineto{\pgfqpoint{2.840001in}{1.162341in}}%
\pgfpathlineto{\pgfqpoint{2.813334in}{1.153251in}}%
\pgfpathlineto{\pgfqpoint{2.806745in}{1.151212in}}%
\pgfpathlineto{\pgfqpoint{2.773489in}{1.142542in}}%
\pgfpathlineto{\pgfqpoint{2.740233in}{1.135568in}}%
\pgfpathlineto{\pgfqpoint{2.729330in}{1.133773in}}%
\pgfpathlineto{\pgfqpoint{2.706977in}{1.130393in}}%
\pgfpathlineto{\pgfqpoint{2.673720in}{1.126709in}}%
\pgfpathlineto{\pgfqpoint{2.640464in}{1.124322in}}%
\pgfpathlineto{\pgfqpoint{2.607208in}{1.123157in}}%
\pgfpathlineto{\pgfqpoint{2.573952in}{1.123143in}}%
\pgfpathlineto{\pgfqpoint{2.540696in}{1.124216in}}%
\pgfpathlineto{\pgfqpoint{2.507440in}{1.126314in}}%
\pgfpathlineto{\pgfqpoint{2.474183in}{1.129383in}}%
\pgfpathlineto{\pgfqpoint{2.440927in}{1.133370in}}%
\pgfpathclose%
\pgfusepath{stroke}%
\end{pgfscope}%
\begin{pgfscope}%
\pgfpathrectangle{\pgfqpoint{0.711606in}{0.549444in}}{\pgfqpoint{4.955171in}{2.902168in}}%
\pgfusepath{clip}%
\pgfsetbuttcap%
\pgfsetroundjoin%
\pgfsetlinewidth{1.003750pt}%
\definecolor{currentstroke}{rgb}{0.011663,0.009417,0.063460}%
\pgfsetstrokecolor{currentstroke}%
\pgfsetdash{}{0pt}%
\pgfpathmoveto{\pgfqpoint{4.070480in}{1.206993in}}%
\pgfpathlineto{\pgfqpoint{4.040811in}{1.211684in}}%
\pgfpathlineto{\pgfqpoint{4.037224in}{1.212276in}}%
\pgfpathlineto{\pgfqpoint{4.003968in}{1.218930in}}%
\pgfpathlineto{\pgfqpoint{3.970712in}{1.226564in}}%
\pgfpathlineto{\pgfqpoint{3.953009in}{1.231162in}}%
\pgfpathlineto{\pgfqpoint{3.937455in}{1.235378in}}%
\pgfpathlineto{\pgfqpoint{3.904199in}{1.245395in}}%
\pgfpathlineto{\pgfqpoint{3.888304in}{1.250639in}}%
\pgfpathlineto{\pgfqpoint{3.870943in}{1.256623in}}%
\pgfpathlineto{\pgfqpoint{3.837687in}{1.269011in}}%
\pgfpathlineto{\pgfqpoint{3.834944in}{1.270117in}}%
\pgfpathlineto{\pgfqpoint{3.804431in}{1.282977in}}%
\pgfpathlineto{\pgfqpoint{3.789695in}{1.289595in}}%
\pgfpathlineto{\pgfqpoint{3.771175in}{1.298310in}}%
\pgfpathlineto{\pgfqpoint{3.749629in}{1.309072in}}%
\pgfpathlineto{\pgfqpoint{3.737918in}{1.315216in}}%
\pgfpathlineto{\pgfqpoint{3.713894in}{1.328550in}}%
\pgfpathlineto{\pgfqpoint{3.704662in}{1.333945in}}%
\pgfpathlineto{\pgfqpoint{3.681815in}{1.348027in}}%
\pgfpathlineto{\pgfqpoint{3.671406in}{1.354802in}}%
\pgfpathlineto{\pgfqpoint{3.652851in}{1.367505in}}%
\pgfpathlineto{\pgfqpoint{3.638150in}{1.378166in}}%
\pgfpathlineto{\pgfqpoint{3.626562in}{1.386983in}}%
\pgfpathlineto{\pgfqpoint{3.604894in}{1.404508in}}%
\pgfpathlineto{\pgfqpoint{3.602588in}{1.406460in}}%
\pgfpathlineto{\pgfqpoint{3.581253in}{1.425938in}}%
\pgfpathlineto{\pgfqpoint{3.571637in}{1.435397in}}%
\pgfpathlineto{\pgfqpoint{3.561900in}{1.445416in}}%
\pgfpathlineto{\pgfqpoint{3.544523in}{1.464893in}}%
\pgfpathlineto{\pgfqpoint{3.538381in}{1.472511in}}%
\pgfpathlineto{\pgfqpoint{3.529228in}{1.484371in}}%
\pgfpathlineto{\pgfqpoint{3.515818in}{1.503849in}}%
\pgfpathlineto{\pgfqpoint{3.505125in}{1.521518in}}%
\pgfpathlineto{\pgfqpoint{3.504077in}{1.523326in}}%
\pgfpathlineto{\pgfqpoint{3.494631in}{1.542804in}}%
\pgfpathlineto{\pgfqpoint{3.486925in}{1.562281in}}%
\pgfpathlineto{\pgfqpoint{3.481080in}{1.581759in}}%
\pgfpathlineto{\pgfqpoint{3.477225in}{1.601237in}}%
\pgfpathlineto{\pgfqpoint{3.475504in}{1.620714in}}%
\pgfpathlineto{\pgfqpoint{3.476075in}{1.640192in}}%
\pgfpathlineto{\pgfqpoint{3.479112in}{1.659670in}}%
\pgfpathlineto{\pgfqpoint{3.484805in}{1.679147in}}%
\pgfpathlineto{\pgfqpoint{3.493368in}{1.698625in}}%
\pgfpathlineto{\pgfqpoint{3.505035in}{1.718103in}}%
\pgfpathlineto{\pgfqpoint{3.505125in}{1.718222in}}%
\pgfpathlineto{\pgfqpoint{3.521427in}{1.737580in}}%
\pgfpathlineto{\pgfqpoint{3.538381in}{1.753785in}}%
\pgfpathlineto{\pgfqpoint{3.542279in}{1.757058in}}%
\pgfpathlineto{\pgfqpoint{3.569943in}{1.776536in}}%
\pgfpathlineto{\pgfqpoint{3.571637in}{1.777560in}}%
\pgfpathlineto{\pgfqpoint{3.604894in}{1.794831in}}%
\pgfpathlineto{\pgfqpoint{3.607599in}{1.796013in}}%
\pgfpathlineto{\pgfqpoint{3.638150in}{1.807719in}}%
\pgfpathlineto{\pgfqpoint{3.663230in}{1.815491in}}%
\pgfpathlineto{\pgfqpoint{3.671406in}{1.817747in}}%
\pgfpathlineto{\pgfqpoint{3.704662in}{1.825034in}}%
\pgfpathlineto{\pgfqpoint{3.737918in}{1.830398in}}%
\pgfpathlineto{\pgfqpoint{3.771175in}{1.833988in}}%
\pgfpathlineto{\pgfqpoint{3.787676in}{1.834968in}}%
\pgfpathlineto{\pgfqpoint{3.804431in}{1.835873in}}%
\pgfpathlineto{\pgfqpoint{3.837687in}{1.836278in}}%
\pgfpathlineto{\pgfqpoint{3.870943in}{1.835369in}}%
\pgfpathlineto{\pgfqpoint{3.877055in}{1.834968in}}%
\pgfpathlineto{\pgfqpoint{3.904199in}{1.833112in}}%
\pgfpathlineto{\pgfqpoint{3.937455in}{1.829608in}}%
\pgfpathlineto{\pgfqpoint{3.970712in}{1.824967in}}%
\pgfpathlineto{\pgfqpoint{4.003968in}{1.819263in}}%
\pgfpathlineto{\pgfqpoint{4.022521in}{1.815491in}}%
\pgfpathlineto{\pgfqpoint{4.037224in}{1.812371in}}%
\pgfpathlineto{\pgfqpoint{4.070480in}{1.804252in}}%
\pgfpathlineto{\pgfqpoint{4.100772in}{1.796013in}}%
\pgfpathlineto{\pgfqpoint{4.103736in}{1.795171in}}%
\pgfpathlineto{\pgfqpoint{4.136993in}{1.784641in}}%
\pgfpathlineto{\pgfqpoint{4.160593in}{1.776536in}}%
\pgfpathlineto{\pgfqpoint{4.170249in}{1.773068in}}%
\pgfpathlineto{\pgfqpoint{4.203505in}{1.760142in}}%
\pgfpathlineto{\pgfqpoint{4.210883in}{1.757058in}}%
\pgfpathlineto{\pgfqpoint{4.236761in}{1.745741in}}%
\pgfpathlineto{\pgfqpoint{4.254296in}{1.737580in}}%
\pgfpathlineto{\pgfqpoint{4.270017in}{1.729908in}}%
\pgfpathlineto{\pgfqpoint{4.292834in}{1.718103in}}%
\pgfpathlineto{\pgfqpoint{4.303273in}{1.712425in}}%
\pgfpathlineto{\pgfqpoint{4.327282in}{1.698625in}}%
\pgfpathlineto{\pgfqpoint{4.336530in}{1.693023in}}%
\pgfpathlineto{\pgfqpoint{4.358268in}{1.679147in}}%
\pgfpathlineto{\pgfqpoint{4.369786in}{1.671376in}}%
\pgfpathlineto{\pgfqpoint{4.386295in}{1.659670in}}%
\pgfpathlineto{\pgfqpoint{4.403042in}{1.647078in}}%
\pgfpathlineto{\pgfqpoint{4.411775in}{1.640192in}}%
\pgfpathlineto{\pgfqpoint{4.434957in}{1.620714in}}%
\pgfpathlineto{\pgfqpoint{4.436298in}{1.619496in}}%
\pgfpathlineto{\pgfqpoint{4.455490in}{1.601237in}}%
\pgfpathlineto{\pgfqpoint{4.469554in}{1.586830in}}%
\pgfpathlineto{\pgfqpoint{4.474290in}{1.581759in}}%
\pgfpathlineto{\pgfqpoint{4.490822in}{1.562281in}}%
\pgfpathlineto{\pgfqpoint{4.502811in}{1.546712in}}%
\pgfpathlineto{\pgfqpoint{4.505692in}{1.542804in}}%
\pgfpathlineto{\pgfqpoint{4.518304in}{1.523326in}}%
\pgfpathlineto{\pgfqpoint{4.529307in}{1.503849in}}%
\pgfpathlineto{\pgfqpoint{4.536067in}{1.489577in}}%
\pgfpathlineto{\pgfqpoint{4.538428in}{1.484371in}}%
\pgfpathlineto{\pgfqpoint{4.545377in}{1.464893in}}%
\pgfpathlineto{\pgfqpoint{4.550460in}{1.445416in}}%
\pgfpathlineto{\pgfqpoint{4.553547in}{1.425938in}}%
\pgfpathlineto{\pgfqpoint{4.554497in}{1.406460in}}%
\pgfpathlineto{\pgfqpoint{4.553153in}{1.386983in}}%
\pgfpathlineto{\pgfqpoint{4.549346in}{1.367505in}}%
\pgfpathlineto{\pgfqpoint{4.542886in}{1.348027in}}%
\pgfpathlineto{\pgfqpoint{4.536067in}{1.333677in}}%
\pgfpathlineto{\pgfqpoint{4.533356in}{1.328550in}}%
\pgfpathlineto{\pgfqpoint{4.519866in}{1.309072in}}%
\pgfpathlineto{\pgfqpoint{4.502811in}{1.289753in}}%
\pgfpathlineto{\pgfqpoint{4.502653in}{1.289595in}}%
\pgfpathlineto{\pgfqpoint{4.479250in}{1.270117in}}%
\pgfpathlineto{\pgfqpoint{4.469554in}{1.263330in}}%
\pgfpathlineto{\pgfqpoint{4.448593in}{1.250639in}}%
\pgfpathlineto{\pgfqpoint{4.436298in}{1.244217in}}%
\pgfpathlineto{\pgfqpoint{4.406671in}{1.231162in}}%
\pgfpathlineto{\pgfqpoint{4.403042in}{1.229756in}}%
\pgfpathlineto{\pgfqpoint{4.369786in}{1.219030in}}%
\pgfpathlineto{\pgfqpoint{4.340880in}{1.211684in}}%
\pgfpathlineto{\pgfqpoint{4.336530in}{1.210694in}}%
\pgfpathlineto{\pgfqpoint{4.303273in}{1.204862in}}%
\pgfpathlineto{\pgfqpoint{4.270017in}{1.200799in}}%
\pgfpathlineto{\pgfqpoint{4.236761in}{1.198372in}}%
\pgfpathlineto{\pgfqpoint{4.203505in}{1.197464in}}%
\pgfpathlineto{\pgfqpoint{4.170249in}{1.197965in}}%
\pgfpathlineto{\pgfqpoint{4.136993in}{1.199778in}}%
\pgfpathlineto{\pgfqpoint{4.103736in}{1.202814in}}%
\pgfpathlineto{\pgfqpoint{4.070480in}{1.206993in}}%
\pgfpathclose%
\pgfusepath{stroke}%
\end{pgfscope}%
\begin{pgfscope}%
\pgfpathrectangle{\pgfqpoint{0.711606in}{0.549444in}}{\pgfqpoint{4.955171in}{2.902168in}}%
\pgfusepath{clip}%
\pgfsetbuttcap%
\pgfsetroundjoin%
\pgfsetlinewidth{1.003750pt}%
\definecolor{currentstroke}{rgb}{0.011663,0.009417,0.063460}%
\pgfsetstrokecolor{currentstroke}%
\pgfsetdash{}{0pt}%
\pgfpathmoveto{\pgfqpoint{2.341159in}{2.377608in}}%
\pgfpathlineto{\pgfqpoint{2.331819in}{2.380342in}}%
\pgfpathlineto{\pgfqpoint{2.307902in}{2.388192in}}%
\pgfpathlineto{\pgfqpoint{2.278943in}{2.399820in}}%
\pgfpathlineto{\pgfqpoint{2.274646in}{2.401783in}}%
\pgfpathlineto{\pgfqpoint{2.242230in}{2.419298in}}%
\pgfpathlineto{\pgfqpoint{2.241390in}{2.419824in}}%
\pgfpathlineto{\pgfqpoint{2.215194in}{2.438775in}}%
\pgfpathlineto{\pgfqpoint{2.208134in}{2.444855in}}%
\pgfpathlineto{\pgfqpoint{2.194411in}{2.458253in}}%
\pgfpathlineto{\pgfqpoint{2.178474in}{2.477731in}}%
\pgfpathlineto{\pgfqpoint{2.174878in}{2.484088in}}%
\pgfpathlineto{\pgfqpoint{2.168283in}{2.497208in}}%
\pgfpathlineto{\pgfqpoint{2.163315in}{2.516686in}}%
\pgfpathlineto{\pgfqpoint{2.163890in}{2.536163in}}%
\pgfpathlineto{\pgfqpoint{2.171149in}{2.555641in}}%
\pgfpathlineto{\pgfqpoint{2.174878in}{2.560611in}}%
\pgfpathlineto{\pgfqpoint{2.189816in}{2.575119in}}%
\pgfpathlineto{\pgfqpoint{2.208134in}{2.586184in}}%
\pgfpathlineto{\pgfqpoint{2.230322in}{2.594596in}}%
\pgfpathlineto{\pgfqpoint{2.241390in}{2.597642in}}%
\pgfpathlineto{\pgfqpoint{2.274646in}{2.602198in}}%
\pgfpathlineto{\pgfqpoint{2.307902in}{2.602635in}}%
\pgfpathlineto{\pgfqpoint{2.341159in}{2.599669in}}%
\pgfpathlineto{\pgfqpoint{2.369989in}{2.594596in}}%
\pgfpathlineto{\pgfqpoint{2.374415in}{2.593721in}}%
\pgfpathlineto{\pgfqpoint{2.407671in}{2.584064in}}%
\pgfpathlineto{\pgfqpoint{2.432352in}{2.575119in}}%
\pgfpathlineto{\pgfqpoint{2.440927in}{2.571601in}}%
\pgfpathlineto{\pgfqpoint{2.473500in}{2.555641in}}%
\pgfpathlineto{\pgfqpoint{2.474183in}{2.555255in}}%
\pgfpathlineto{\pgfqpoint{2.503278in}{2.536163in}}%
\pgfpathlineto{\pgfqpoint{2.507440in}{2.532943in}}%
\pgfpathlineto{\pgfqpoint{2.525873in}{2.516686in}}%
\pgfpathlineto{\pgfqpoint{2.540696in}{2.500762in}}%
\pgfpathlineto{\pgfqpoint{2.543643in}{2.497208in}}%
\pgfpathlineto{\pgfqpoint{2.555227in}{2.477731in}}%
\pgfpathlineto{\pgfqpoint{2.562218in}{2.458253in}}%
\pgfpathlineto{\pgfqpoint{2.563692in}{2.438775in}}%
\pgfpathlineto{\pgfqpoint{2.558464in}{2.419298in}}%
\pgfpathlineto{\pgfqpoint{2.544981in}{2.399820in}}%
\pgfpathlineto{\pgfqpoint{2.540696in}{2.396077in}}%
\pgfpathlineto{\pgfqpoint{2.514736in}{2.380342in}}%
\pgfpathlineto{\pgfqpoint{2.507440in}{2.377290in}}%
\pgfpathlineto{\pgfqpoint{2.474183in}{2.368876in}}%
\pgfpathlineto{\pgfqpoint{2.440927in}{2.365573in}}%
\pgfpathlineto{\pgfqpoint{2.407671in}{2.366396in}}%
\pgfpathlineto{\pgfqpoint{2.374415in}{2.370600in}}%
\pgfpathlineto{\pgfqpoint{2.341159in}{2.377608in}}%
\pgfusepath{stroke}%
\end{pgfscope}%
\begin{pgfscope}%
\pgfpathrectangle{\pgfqpoint{0.711606in}{0.549444in}}{\pgfqpoint{4.955171in}{2.902168in}}%
\pgfusepath{clip}%
\pgfsetbuttcap%
\pgfsetroundjoin%
\pgfsetlinewidth{1.003750pt}%
\definecolor{currentstroke}{rgb}{0.019373,0.015133,0.088767}%
\pgfsetstrokecolor{currentstroke}%
\pgfsetdash{}{0pt}%
\pgfpathmoveto{\pgfqpoint{2.440927in}{1.034393in}}%
\pgfpathlineto{\pgfqpoint{2.428432in}{1.036385in}}%
\pgfpathlineto{\pgfqpoint{2.407671in}{1.039786in}}%
\pgfpathlineto{\pgfqpoint{2.374415in}{1.045935in}}%
\pgfpathlineto{\pgfqpoint{2.341159in}{1.052726in}}%
\pgfpathlineto{\pgfqpoint{2.327184in}{1.055863in}}%
\pgfpathlineto{\pgfqpoint{2.307902in}{1.060311in}}%
\pgfpathlineto{\pgfqpoint{2.274646in}{1.068635in}}%
\pgfpathlineto{\pgfqpoint{2.249692in}{1.075340in}}%
\pgfpathlineto{\pgfqpoint{2.241390in}{1.077633in}}%
\pgfpathlineto{\pgfqpoint{2.208134in}{1.087489in}}%
\pgfpathlineto{\pgfqpoint{2.184797in}{1.094818in}}%
\pgfpathlineto{\pgfqpoint{2.174878in}{1.098021in}}%
\pgfpathlineto{\pgfqpoint{2.141622in}{1.109404in}}%
\pgfpathlineto{\pgfqpoint{2.128056in}{1.114296in}}%
\pgfpathlineto{\pgfqpoint{2.108365in}{1.121600in}}%
\pgfpathlineto{\pgfqpoint{2.077017in}{1.133773in}}%
\pgfpathlineto{\pgfqpoint{2.075109in}{1.134536in}}%
\pgfpathlineto{\pgfqpoint{2.041853in}{1.148501in}}%
\pgfpathlineto{\pgfqpoint{2.031007in}{1.153251in}}%
\pgfpathlineto{\pgfqpoint{2.008597in}{1.163359in}}%
\pgfpathlineto{\pgfqpoint{1.988626in}{1.172729in}}%
\pgfpathlineto{\pgfqpoint{1.975341in}{1.179154in}}%
\pgfpathlineto{\pgfqpoint{1.949362in}{1.192206in}}%
\pgfpathlineto{\pgfqpoint{1.942084in}{1.195979in}}%
\pgfpathlineto{\pgfqpoint{1.912879in}{1.211684in}}%
\pgfpathlineto{\pgfqpoint{1.908828in}{1.213934in}}%
\pgfpathlineto{\pgfqpoint{1.878888in}{1.231162in}}%
\pgfpathlineto{\pgfqpoint{1.875572in}{1.233135in}}%
\pgfpathlineto{\pgfqpoint{1.847139in}{1.250639in}}%
\pgfpathlineto{\pgfqpoint{1.842316in}{1.253713in}}%
\pgfpathlineto{\pgfqpoint{1.817414in}{1.270117in}}%
\pgfpathlineto{\pgfqpoint{1.809060in}{1.275821in}}%
\pgfpathlineto{\pgfqpoint{1.789521in}{1.289595in}}%
\pgfpathlineto{\pgfqpoint{1.775804in}{1.299632in}}%
\pgfpathlineto{\pgfqpoint{1.763294in}{1.309072in}}%
\pgfpathlineto{\pgfqpoint{1.742547in}{1.325347in}}%
\pgfpathlineto{\pgfqpoint{1.738586in}{1.328550in}}%
\pgfpathlineto{\pgfqpoint{1.715535in}{1.348027in}}%
\pgfpathlineto{\pgfqpoint{1.709291in}{1.353549in}}%
\pgfpathlineto{\pgfqpoint{1.693970in}{1.367505in}}%
\pgfpathlineto{\pgfqpoint{1.676035in}{1.384590in}}%
\pgfpathlineto{\pgfqpoint{1.673594in}{1.386983in}}%
\pgfpathlineto{\pgfqpoint{1.654811in}{1.406460in}}%
\pgfpathlineto{\pgfqpoint{1.642779in}{1.419609in}}%
\pgfpathlineto{\pgfqpoint{1.637149in}{1.425938in}}%
\pgfpathlineto{\pgfqpoint{1.620883in}{1.445416in}}%
\pgfpathlineto{\pgfqpoint{1.609523in}{1.459903in}}%
\pgfpathlineto{\pgfqpoint{1.605717in}{1.464893in}}%
\pgfpathlineto{\pgfqpoint{1.591970in}{1.484371in}}%
\pgfpathlineto{\pgfqpoint{1.579218in}{1.503849in}}%
\pgfpathlineto{\pgfqpoint{1.576266in}{1.508811in}}%
\pgfpathlineto{\pgfqpoint{1.567870in}{1.523326in}}%
\pgfpathlineto{\pgfqpoint{1.557695in}{1.542804in}}%
\pgfpathlineto{\pgfqpoint{1.548619in}{1.562281in}}%
\pgfpathlineto{\pgfqpoint{1.543010in}{1.576116in}}%
\pgfpathlineto{\pgfqpoint{1.540786in}{1.581759in}}%
\pgfpathlineto{\pgfqpoint{1.534340in}{1.601237in}}%
\pgfpathlineto{\pgfqpoint{1.529101in}{1.620714in}}%
\pgfpathlineto{\pgfqpoint{1.525122in}{1.640192in}}%
\pgfpathlineto{\pgfqpoint{1.522459in}{1.659670in}}%
\pgfpathlineto{\pgfqpoint{1.521172in}{1.679147in}}%
\pgfpathlineto{\pgfqpoint{1.521324in}{1.698625in}}%
\pgfpathlineto{\pgfqpoint{1.522982in}{1.718103in}}%
\pgfpathlineto{\pgfqpoint{1.526218in}{1.737580in}}%
\pgfpathlineto{\pgfqpoint{1.531108in}{1.757058in}}%
\pgfpathlineto{\pgfqpoint{1.537732in}{1.776536in}}%
\pgfpathlineto{\pgfqpoint{1.543010in}{1.788783in}}%
\pgfpathlineto{\pgfqpoint{1.546342in}{1.796013in}}%
\pgfpathlineto{\pgfqpoint{1.557249in}{1.815491in}}%
\pgfpathlineto{\pgfqpoint{1.570296in}{1.834968in}}%
\pgfpathlineto{\pgfqpoint{1.576266in}{1.842652in}}%
\pgfpathlineto{\pgfqpoint{1.586140in}{1.854446in}}%
\pgfpathlineto{\pgfqpoint{1.604882in}{1.873924in}}%
\pgfpathlineto{\pgfqpoint{1.609523in}{1.878203in}}%
\pgfpathlineto{\pgfqpoint{1.627415in}{1.893401in}}%
\pgfpathlineto{\pgfqpoint{1.642779in}{1.905032in}}%
\pgfpathlineto{\pgfqpoint{1.654113in}{1.912879in}}%
\pgfpathlineto{\pgfqpoint{1.676035in}{1.926567in}}%
\pgfpathlineto{\pgfqpoint{1.686265in}{1.932357in}}%
\pgfpathlineto{\pgfqpoint{1.709291in}{1.944226in}}%
\pgfpathlineto{\pgfqpoint{1.725750in}{1.951834in}}%
\pgfpathlineto{\pgfqpoint{1.742547in}{1.958963in}}%
\pgfpathlineto{\pgfqpoint{1.775428in}{1.971312in}}%
\pgfpathlineto{\pgfqpoint{1.775804in}{1.971442in}}%
\pgfpathlineto{\pgfqpoint{1.809060in}{1.981663in}}%
\pgfpathlineto{\pgfqpoint{1.842316in}{1.990423in}}%
\pgfpathlineto{\pgfqpoint{1.843941in}{1.990790in}}%
\pgfpathlineto{\pgfqpoint{1.875572in}{1.997447in}}%
\pgfpathlineto{\pgfqpoint{1.908828in}{2.003225in}}%
\pgfpathlineto{\pgfqpoint{1.942084in}{2.007838in}}%
\pgfpathlineto{\pgfqpoint{1.965005in}{2.010267in}}%
\pgfpathlineto{\pgfqpoint{1.975341in}{2.011296in}}%
\pgfpathlineto{\pgfqpoint{2.008597in}{2.013646in}}%
\pgfpathlineto{\pgfqpoint{2.041853in}{2.015045in}}%
\pgfpathlineto{\pgfqpoint{2.075109in}{2.015538in}}%
\pgfpathlineto{\pgfqpoint{2.108365in}{2.015166in}}%
\pgfpathlineto{\pgfqpoint{2.141622in}{2.013965in}}%
\pgfpathlineto{\pgfqpoint{2.174878in}{2.011963in}}%
\pgfpathlineto{\pgfqpoint{2.195082in}{2.010267in}}%
\pgfpathlineto{\pgfqpoint{2.208134in}{2.009151in}}%
\pgfpathlineto{\pgfqpoint{2.241390in}{2.005531in}}%
\pgfpathlineto{\pgfqpoint{2.274646in}{2.001180in}}%
\pgfpathlineto{\pgfqpoint{2.307902in}{1.996107in}}%
\pgfpathlineto{\pgfqpoint{2.338316in}{1.990790in}}%
\pgfpathlineto{\pgfqpoint{2.341159in}{1.990285in}}%
\pgfpathlineto{\pgfqpoint{2.374415in}{1.983653in}}%
\pgfpathlineto{\pgfqpoint{2.407671in}{1.976337in}}%
\pgfpathlineto{\pgfqpoint{2.428437in}{1.971312in}}%
\pgfpathlineto{\pgfqpoint{2.440927in}{1.968248in}}%
\pgfpathlineto{\pgfqpoint{2.474183in}{1.959397in}}%
\pgfpathlineto{\pgfqpoint{2.500575in}{1.951834in}}%
\pgfpathlineto{\pgfqpoint{2.507440in}{1.949831in}}%
\pgfpathlineto{\pgfqpoint{2.540696in}{1.939439in}}%
\pgfpathlineto{\pgfqpoint{2.562021in}{1.932357in}}%
\pgfpathlineto{\pgfqpoint{2.573952in}{1.928310in}}%
\pgfpathlineto{\pgfqpoint{2.607208in}{1.916379in}}%
\pgfpathlineto{\pgfqpoint{2.616449in}{1.912879in}}%
\pgfpathlineto{\pgfqpoint{2.640464in}{1.903567in}}%
\pgfpathlineto{\pgfqpoint{2.665505in}{1.893401in}}%
\pgfpathlineto{\pgfqpoint{2.673720in}{1.889975in}}%
\pgfpathlineto{\pgfqpoint{2.706977in}{1.875474in}}%
\pgfpathlineto{\pgfqpoint{2.710377in}{1.873924in}}%
\pgfpathlineto{\pgfqpoint{2.740233in}{1.859930in}}%
\pgfpathlineto{\pgfqpoint{2.751489in}{1.854446in}}%
\pgfpathlineto{\pgfqpoint{2.773489in}{1.843404in}}%
\pgfpathlineto{\pgfqpoint{2.789688in}{1.834968in}}%
\pgfpathlineto{\pgfqpoint{2.806745in}{1.825804in}}%
\pgfpathlineto{\pgfqpoint{2.825277in}{1.815491in}}%
\pgfpathlineto{\pgfqpoint{2.840001in}{1.807025in}}%
\pgfpathlineto{\pgfqpoint{2.858523in}{1.796013in}}%
\pgfpathlineto{\pgfqpoint{2.873258in}{1.786949in}}%
\pgfpathlineto{\pgfqpoint{2.889664in}{1.776536in}}%
\pgfpathlineto{\pgfqpoint{2.906514in}{1.765450in}}%
\pgfpathlineto{\pgfqpoint{2.918912in}{1.757058in}}%
\pgfpathlineto{\pgfqpoint{2.939770in}{1.742388in}}%
\pgfpathlineto{\pgfqpoint{2.946448in}{1.737580in}}%
\pgfpathlineto{\pgfqpoint{2.972371in}{1.718103in}}%
\pgfpathlineto{\pgfqpoint{2.973026in}{1.717587in}}%
\pgfpathlineto{\pgfqpoint{2.996655in}{1.698625in}}%
\pgfpathlineto{\pgfqpoint{3.006282in}{1.690508in}}%
\pgfpathlineto{\pgfqpoint{3.019668in}{1.679147in}}%
\pgfpathlineto{\pgfqpoint{3.039539in}{1.661224in}}%
\pgfpathlineto{\pgfqpoint{3.041295in}{1.659670in}}%
\pgfpathlineto{\pgfqpoint{3.061687in}{1.640192in}}%
\pgfpathlineto{\pgfqpoint{3.072795in}{1.628740in}}%
\pgfpathlineto{\pgfqpoint{3.080904in}{1.620714in}}%
\pgfpathlineto{\pgfqpoint{3.098633in}{1.601237in}}%
\pgfpathlineto{\pgfqpoint{3.106051in}{1.592303in}}%
\pgfpathlineto{\pgfqpoint{3.115457in}{1.581759in}}%
\pgfpathlineto{\pgfqpoint{3.130537in}{1.562281in}}%
\pgfpathlineto{\pgfqpoint{3.139307in}{1.549495in}}%
\pgfpathlineto{\pgfqpoint{3.144461in}{1.542804in}}%
\pgfpathlineto{\pgfqpoint{3.157107in}{1.523326in}}%
\pgfpathlineto{\pgfqpoint{3.167655in}{1.503849in}}%
\pgfpathlineto{\pgfqpoint{3.172563in}{1.493463in}}%
\pgfpathlineto{\pgfqpoint{3.177380in}{1.484371in}}%
\pgfpathlineto{\pgfqpoint{3.185701in}{1.464893in}}%
\pgfpathlineto{\pgfqpoint{3.192305in}{1.445416in}}%
\pgfpathlineto{\pgfqpoint{3.197582in}{1.425938in}}%
\pgfpathlineto{\pgfqpoint{3.201670in}{1.406460in}}%
\pgfpathlineto{\pgfqpoint{3.204587in}{1.386983in}}%
\pgfpathlineto{\pgfqpoint{3.205819in}{1.372870in}}%
\pgfpathlineto{\pgfqpoint{3.206296in}{1.367505in}}%
\pgfpathlineto{\pgfqpoint{3.206699in}{1.348027in}}%
\pgfpathlineto{\pgfqpoint{3.205819in}{1.330076in}}%
\pgfpathlineto{\pgfqpoint{3.205744in}{1.328550in}}%
\pgfpathlineto{\pgfqpoint{3.203316in}{1.309072in}}%
\pgfpathlineto{\pgfqpoint{3.199325in}{1.289595in}}%
\pgfpathlineto{\pgfqpoint{3.193674in}{1.270117in}}%
\pgfpathlineto{\pgfqpoint{3.186264in}{1.250639in}}%
\pgfpathlineto{\pgfqpoint{3.176991in}{1.231162in}}%
\pgfpathlineto{\pgfqpoint{3.172563in}{1.223410in}}%
\pgfpathlineto{\pgfqpoint{3.165392in}{1.211684in}}%
\pgfpathlineto{\pgfqpoint{3.151345in}{1.192206in}}%
\pgfpathlineto{\pgfqpoint{3.139307in}{1.177835in}}%
\pgfpathlineto{\pgfqpoint{3.134692in}{1.172729in}}%
\pgfpathlineto{\pgfqpoint{3.114667in}{1.153251in}}%
\pgfpathlineto{\pgfqpoint{3.106051in}{1.145818in}}%
\pgfpathlineto{\pgfqpoint{3.090861in}{1.133773in}}%
\pgfpathlineto{\pgfqpoint{3.072795in}{1.120951in}}%
\pgfpathlineto{\pgfqpoint{3.062512in}{1.114296in}}%
\pgfpathlineto{\pgfqpoint{3.039539in}{1.100838in}}%
\pgfpathlineto{\pgfqpoint{3.028164in}{1.094818in}}%
\pgfpathlineto{\pgfqpoint{3.006282in}{1.084242in}}%
\pgfpathlineto{\pgfqpoint{2.985663in}{1.075340in}}%
\pgfpathlineto{\pgfqpoint{2.973026in}{1.070321in}}%
\pgfpathlineto{\pgfqpoint{2.939770in}{1.058633in}}%
\pgfpathlineto{\pgfqpoint{2.930774in}{1.055863in}}%
\pgfpathlineto{\pgfqpoint{2.906514in}{1.048935in}}%
\pgfpathlineto{\pgfqpoint{2.873258in}{1.040791in}}%
\pgfpathlineto{\pgfqpoint{2.851898in}{1.036385in}}%
\pgfpathlineto{\pgfqpoint{2.840001in}{1.034094in}}%
\pgfpathlineto{\pgfqpoint{2.806745in}{1.028803in}}%
\pgfpathlineto{\pgfqpoint{2.773489in}{1.024638in}}%
\pgfpathlineto{\pgfqpoint{2.740233in}{1.021544in}}%
\pgfpathlineto{\pgfqpoint{2.706977in}{1.019470in}}%
\pgfpathlineto{\pgfqpoint{2.673720in}{1.018367in}}%
\pgfpathlineto{\pgfqpoint{2.640464in}{1.018191in}}%
\pgfpathlineto{\pgfqpoint{2.607208in}{1.018900in}}%
\pgfpathlineto{\pgfqpoint{2.573952in}{1.020453in}}%
\pgfpathlineto{\pgfqpoint{2.540696in}{1.022813in}}%
\pgfpathlineto{\pgfqpoint{2.507440in}{1.025945in}}%
\pgfpathlineto{\pgfqpoint{2.474183in}{1.029815in}}%
\pgfpathlineto{\pgfqpoint{2.440927in}{1.034393in}}%
\pgfpathclose%
\pgfusepath{stroke}%
\end{pgfscope}%
\begin{pgfscope}%
\pgfpathrectangle{\pgfqpoint{0.711606in}{0.549444in}}{\pgfqpoint{4.955171in}{2.902168in}}%
\pgfusepath{clip}%
\pgfsetbuttcap%
\pgfsetroundjoin%
\pgfsetlinewidth{1.003750pt}%
\definecolor{currentstroke}{rgb}{0.019373,0.015133,0.088767}%
\pgfsetstrokecolor{currentstroke}%
\pgfsetdash{}{0pt}%
\pgfpathmoveto{\pgfqpoint{4.236761in}{1.075059in}}%
\pgfpathlineto{\pgfqpoint{4.230139in}{1.075340in}}%
\pgfpathlineto{\pgfqpoint{4.203505in}{1.076509in}}%
\pgfpathlineto{\pgfqpoint{4.170249in}{1.078893in}}%
\pgfpathlineto{\pgfqpoint{4.136993in}{1.082149in}}%
\pgfpathlineto{\pgfqpoint{4.103736in}{1.086237in}}%
\pgfpathlineto{\pgfqpoint{4.070480in}{1.091116in}}%
\pgfpathlineto{\pgfqpoint{4.048738in}{1.094818in}}%
\pgfpathlineto{\pgfqpoint{4.037224in}{1.096840in}}%
\pgfpathlineto{\pgfqpoint{4.003968in}{1.103488in}}%
\pgfpathlineto{\pgfqpoint{3.970712in}{1.110845in}}%
\pgfpathlineto{\pgfqpoint{3.956549in}{1.114296in}}%
\pgfpathlineto{\pgfqpoint{3.937455in}{1.119094in}}%
\pgfpathlineto{\pgfqpoint{3.904199in}{1.128172in}}%
\pgfpathlineto{\pgfqpoint{3.885141in}{1.133773in}}%
\pgfpathlineto{\pgfqpoint{3.870943in}{1.138078in}}%
\pgfpathlineto{\pgfqpoint{3.837687in}{1.148874in}}%
\pgfpathlineto{\pgfqpoint{3.825021in}{1.153251in}}%
\pgfpathlineto{\pgfqpoint{3.804431in}{1.160596in}}%
\pgfpathlineto{\pgfqpoint{3.772182in}{1.172729in}}%
\pgfpathlineto{\pgfqpoint{3.771175in}{1.173120in}}%
\pgfpathlineto{\pgfqpoint{3.737918in}{1.186811in}}%
\pgfpathlineto{\pgfqpoint{3.725420in}{1.192206in}}%
\pgfpathlineto{\pgfqpoint{3.704662in}{1.201469in}}%
\pgfpathlineto{\pgfqpoint{3.682773in}{1.211684in}}%
\pgfpathlineto{\pgfqpoint{3.671406in}{1.217174in}}%
\pgfpathlineto{\pgfqpoint{3.643658in}{1.231162in}}%
\pgfpathlineto{\pgfqpoint{3.638150in}{1.234039in}}%
\pgfpathlineto{\pgfqpoint{3.607640in}{1.250639in}}%
\pgfpathlineto{\pgfqpoint{3.604894in}{1.252190in}}%
\pgfpathlineto{\pgfqpoint{3.574352in}{1.270117in}}%
\pgfpathlineto{\pgfqpoint{3.571637in}{1.271773in}}%
\pgfpathlineto{\pgfqpoint{3.543470in}{1.289595in}}%
\pgfpathlineto{\pgfqpoint{3.538381in}{1.292949in}}%
\pgfpathlineto{\pgfqpoint{3.514713in}{1.309072in}}%
\pgfpathlineto{\pgfqpoint{3.505125in}{1.315895in}}%
\pgfpathlineto{\pgfqpoint{3.487833in}{1.328550in}}%
\pgfpathlineto{\pgfqpoint{3.471869in}{1.340809in}}%
\pgfpathlineto{\pgfqpoint{3.462642in}{1.348027in}}%
\pgfpathlineto{\pgfqpoint{3.439114in}{1.367505in}}%
\pgfpathlineto{\pgfqpoint{3.438613in}{1.367946in}}%
\pgfpathlineto{\pgfqpoint{3.417194in}{1.386983in}}%
\pgfpathlineto{\pgfqpoint{3.405357in}{1.398261in}}%
\pgfpathlineto{\pgfqpoint{3.396624in}{1.406460in}}%
\pgfpathlineto{\pgfqpoint{3.377706in}{1.425938in}}%
\pgfpathlineto{\pgfqpoint{3.372100in}{1.432219in}}%
\pgfpathlineto{\pgfqpoint{3.359891in}{1.445416in}}%
\pgfpathlineto{\pgfqpoint{3.344004in}{1.464893in}}%
\pgfpathlineto{\pgfqpoint{3.338844in}{1.471919in}}%
\pgfpathlineto{\pgfqpoint{3.329072in}{1.484371in}}%
\pgfpathlineto{\pgfqpoint{3.316134in}{1.503849in}}%
\pgfpathlineto{\pgfqpoint{3.305588in}{1.522347in}}%
\pgfpathlineto{\pgfqpoint{3.304964in}{1.523326in}}%
\pgfpathlineto{\pgfqpoint{3.294589in}{1.542804in}}%
\pgfpathlineto{\pgfqpoint{3.286353in}{1.562281in}}%
\pgfpathlineto{\pgfqpoint{3.279717in}{1.581759in}}%
\pgfpathlineto{\pgfqpoint{3.274469in}{1.601237in}}%
\pgfpathlineto{\pgfqpoint{3.272332in}{1.611883in}}%
\pgfpathlineto{\pgfqpoint{3.270500in}{1.620714in}}%
\pgfpathlineto{\pgfqpoint{3.268000in}{1.640192in}}%
\pgfpathlineto{\pgfqpoint{3.267016in}{1.659670in}}%
\pgfpathlineto{\pgfqpoint{3.267549in}{1.679147in}}%
\pgfpathlineto{\pgfqpoint{3.269657in}{1.698625in}}%
\pgfpathlineto{\pgfqpoint{3.272332in}{1.712467in}}%
\pgfpathlineto{\pgfqpoint{3.273476in}{1.718103in}}%
\pgfpathlineto{\pgfqpoint{3.279282in}{1.737580in}}%
\pgfpathlineto{\pgfqpoint{3.287110in}{1.757058in}}%
\pgfpathlineto{\pgfqpoint{3.297090in}{1.776536in}}%
\pgfpathlineto{\pgfqpoint{3.305588in}{1.790102in}}%
\pgfpathlineto{\pgfqpoint{3.309594in}{1.796013in}}%
\pgfpathlineto{\pgfqpoint{3.325238in}{1.815491in}}%
\pgfpathlineto{\pgfqpoint{3.338844in}{1.829933in}}%
\pgfpathlineto{\pgfqpoint{3.344029in}{1.834968in}}%
\pgfpathlineto{\pgfqpoint{3.367023in}{1.854446in}}%
\pgfpathlineto{\pgfqpoint{3.372100in}{1.858249in}}%
\pgfpathlineto{\pgfqpoint{3.395234in}{1.873924in}}%
\pgfpathlineto{\pgfqpoint{3.405357in}{1.880045in}}%
\pgfpathlineto{\pgfqpoint{3.430056in}{1.893401in}}%
\pgfpathlineto{\pgfqpoint{3.438613in}{1.897580in}}%
\pgfpathlineto{\pgfqpoint{3.471869in}{1.911922in}}%
\pgfpathlineto{\pgfqpoint{3.474402in}{1.912879in}}%
\pgfpathlineto{\pgfqpoint{3.505125in}{1.923487in}}%
\pgfpathlineto{\pgfqpoint{3.535316in}{1.932357in}}%
\pgfpathlineto{\pgfqpoint{3.538381in}{1.933186in}}%
\pgfpathlineto{\pgfqpoint{3.571637in}{1.940827in}}%
\pgfpathlineto{\pgfqpoint{3.604894in}{1.947038in}}%
\pgfpathlineto{\pgfqpoint{3.637697in}{1.951834in}}%
\pgfpathlineto{\pgfqpoint{3.638150in}{1.951896in}}%
\pgfpathlineto{\pgfqpoint{3.671406in}{1.955299in}}%
\pgfpathlineto{\pgfqpoint{3.704662in}{1.957567in}}%
\pgfpathlineto{\pgfqpoint{3.737918in}{1.958761in}}%
\pgfpathlineto{\pgfqpoint{3.771175in}{1.958934in}}%
\pgfpathlineto{\pgfqpoint{3.804431in}{1.958137in}}%
\pgfpathlineto{\pgfqpoint{3.837687in}{1.956419in}}%
\pgfpathlineto{\pgfqpoint{3.870943in}{1.953824in}}%
\pgfpathlineto{\pgfqpoint{3.890108in}{1.951834in}}%
\pgfpathlineto{\pgfqpoint{3.904199in}{1.950325in}}%
\pgfpathlineto{\pgfqpoint{3.937455in}{1.945899in}}%
\pgfpathlineto{\pgfqpoint{3.970712in}{1.940685in}}%
\pgfpathlineto{\pgfqpoint{4.003968in}{1.934720in}}%
\pgfpathlineto{\pgfqpoint{4.015613in}{1.932357in}}%
\pgfpathlineto{\pgfqpoint{4.037224in}{1.927835in}}%
\pgfpathlineto{\pgfqpoint{4.070480in}{1.920131in}}%
\pgfpathlineto{\pgfqpoint{4.099220in}{1.912879in}}%
\pgfpathlineto{\pgfqpoint{4.103736in}{1.911703in}}%
\pgfpathlineto{\pgfqpoint{4.136993in}{1.902264in}}%
\pgfpathlineto{\pgfqpoint{4.166221in}{1.893401in}}%
\pgfpathlineto{\pgfqpoint{4.170249in}{1.892141in}}%
\pgfpathlineto{\pgfqpoint{4.203505in}{1.880970in}}%
\pgfpathlineto{\pgfqpoint{4.223322in}{1.873924in}}%
\pgfpathlineto{\pgfqpoint{4.236761in}{1.868989in}}%
\pgfpathlineto{\pgfqpoint{4.270017in}{1.856095in}}%
\pgfpathlineto{\pgfqpoint{4.274044in}{1.854446in}}%
\pgfpathlineto{\pgfqpoint{4.303273in}{1.842079in}}%
\pgfpathlineto{\pgfqpoint{4.319321in}{1.834968in}}%
\pgfpathlineto{\pgfqpoint{4.336530in}{1.827082in}}%
\pgfpathlineto{\pgfqpoint{4.360732in}{1.815491in}}%
\pgfpathlineto{\pgfqpoint{4.369786in}{1.811001in}}%
\pgfpathlineto{\pgfqpoint{4.398759in}{1.796013in}}%
\pgfpathlineto{\pgfqpoint{4.403042in}{1.793716in}}%
\pgfpathlineto{\pgfqpoint{4.433805in}{1.776536in}}%
\pgfpathlineto{\pgfqpoint{4.436298in}{1.775090in}}%
\pgfpathlineto{\pgfqpoint{4.466211in}{1.757058in}}%
\pgfpathlineto{\pgfqpoint{4.469554in}{1.754963in}}%
\pgfpathlineto{\pgfqpoint{4.496270in}{1.737580in}}%
\pgfpathlineto{\pgfqpoint{4.502811in}{1.733149in}}%
\pgfpathlineto{\pgfqpoint{4.524231in}{1.718103in}}%
\pgfpathlineto{\pgfqpoint{4.536067in}{1.709431in}}%
\pgfpathlineto{\pgfqpoint{4.550311in}{1.698625in}}%
\pgfpathlineto{\pgfqpoint{4.569323in}{1.683553in}}%
\pgfpathlineto{\pgfqpoint{4.574696in}{1.679147in}}%
\pgfpathlineto{\pgfqpoint{4.597294in}{1.659670in}}%
\pgfpathlineto{\pgfqpoint{4.602579in}{1.654864in}}%
\pgfpathlineto{\pgfqpoint{4.618187in}{1.640192in}}%
\pgfpathlineto{\pgfqpoint{4.635835in}{1.622716in}}%
\pgfpathlineto{\pgfqpoint{4.637793in}{1.620714in}}%
\pgfpathlineto{\pgfqpoint{4.655607in}{1.601237in}}%
\pgfpathlineto{\pgfqpoint{4.669091in}{1.585555in}}%
\pgfpathlineto{\pgfqpoint{4.672254in}{1.581759in}}%
\pgfpathlineto{\pgfqpoint{4.687240in}{1.562281in}}%
\pgfpathlineto{\pgfqpoint{4.701137in}{1.542804in}}%
\pgfpathlineto{\pgfqpoint{4.702348in}{1.540928in}}%
\pgfpathlineto{\pgfqpoint{4.713350in}{1.523326in}}%
\pgfpathlineto{\pgfqpoint{4.724343in}{1.503849in}}%
\pgfpathlineto{\pgfqpoint{4.734117in}{1.484371in}}%
\pgfpathlineto{\pgfqpoint{4.735604in}{1.480915in}}%
\pgfpathlineto{\pgfqpoint{4.742281in}{1.464893in}}%
\pgfpathlineto{\pgfqpoint{4.749082in}{1.445416in}}%
\pgfpathlineto{\pgfqpoint{4.754524in}{1.425938in}}%
\pgfpathlineto{\pgfqpoint{4.758541in}{1.406460in}}%
\pgfpathlineto{\pgfqpoint{4.761060in}{1.386983in}}%
\pgfpathlineto{\pgfqpoint{4.762002in}{1.367505in}}%
\pgfpathlineto{\pgfqpoint{4.761287in}{1.348027in}}%
\pgfpathlineto{\pgfqpoint{4.758823in}{1.328550in}}%
\pgfpathlineto{\pgfqpoint{4.754517in}{1.309072in}}%
\pgfpathlineto{\pgfqpoint{4.748266in}{1.289595in}}%
\pgfpathlineto{\pgfqpoint{4.739960in}{1.270117in}}%
\pgfpathlineto{\pgfqpoint{4.735604in}{1.261931in}}%
\pgfpathlineto{\pgfqpoint{4.729110in}{1.250639in}}%
\pgfpathlineto{\pgfqpoint{4.715532in}{1.231162in}}%
\pgfpathlineto{\pgfqpoint{4.702348in}{1.215252in}}%
\pgfpathlineto{\pgfqpoint{4.699125in}{1.211684in}}%
\pgfpathlineto{\pgfqpoint{4.678811in}{1.192206in}}%
\pgfpathlineto{\pgfqpoint{4.669091in}{1.184064in}}%
\pgfpathlineto{\pgfqpoint{4.654191in}{1.172729in}}%
\pgfpathlineto{\pgfqpoint{4.635835in}{1.160366in}}%
\pgfpathlineto{\pgfqpoint{4.624081in}{1.153251in}}%
\pgfpathlineto{\pgfqpoint{4.602579in}{1.141575in}}%
\pgfpathlineto{\pgfqpoint{4.586389in}{1.133773in}}%
\pgfpathlineto{\pgfqpoint{4.569323in}{1.126317in}}%
\pgfpathlineto{\pgfqpoint{4.537805in}{1.114296in}}%
\pgfpathlineto{\pgfqpoint{4.536067in}{1.113689in}}%
\pgfpathlineto{\pgfqpoint{4.502811in}{1.103583in}}%
\pgfpathlineto{\pgfqpoint{4.469554in}{1.095110in}}%
\pgfpathlineto{\pgfqpoint{4.468183in}{1.094818in}}%
\pgfpathlineto{\pgfqpoint{4.436298in}{1.088543in}}%
\pgfpathlineto{\pgfqpoint{4.403042in}{1.083338in}}%
\pgfpathlineto{\pgfqpoint{4.369786in}{1.079403in}}%
\pgfpathlineto{\pgfqpoint{4.336530in}{1.076669in}}%
\pgfpathlineto{\pgfqpoint{4.308979in}{1.075340in}}%
\pgfpathlineto{\pgfqpoint{4.303273in}{1.075084in}}%
\pgfpathlineto{\pgfqpoint{4.270017in}{1.074586in}}%
\pgfpathlineto{\pgfqpoint{4.236761in}{1.075059in}}%
\pgfpathclose%
\pgfusepath{stroke}%
\end{pgfscope}%
\begin{pgfscope}%
\pgfpathrectangle{\pgfqpoint{0.711606in}{0.549444in}}{\pgfqpoint{4.955171in}{2.902168in}}%
\pgfusepath{clip}%
\pgfsetbuttcap%
\pgfsetroundjoin%
\pgfsetlinewidth{1.003750pt}%
\definecolor{currentstroke}{rgb}{0.019373,0.015133,0.088767}%
\pgfsetstrokecolor{currentstroke}%
\pgfsetdash{}{0pt}%
\pgfpathmoveto{\pgfqpoint{2.440927in}{2.163101in}}%
\pgfpathlineto{\pgfqpoint{2.419174in}{2.166088in}}%
\pgfpathlineto{\pgfqpoint{2.407671in}{2.167730in}}%
\pgfpathlineto{\pgfqpoint{2.374415in}{2.173605in}}%
\pgfpathlineto{\pgfqpoint{2.341159in}{2.180478in}}%
\pgfpathlineto{\pgfqpoint{2.319678in}{2.185566in}}%
\pgfpathlineto{\pgfqpoint{2.307902in}{2.188465in}}%
\pgfpathlineto{\pgfqpoint{2.274646in}{2.197678in}}%
\pgfpathlineto{\pgfqpoint{2.250482in}{2.205044in}}%
\pgfpathlineto{\pgfqpoint{2.241390in}{2.207928in}}%
\pgfpathlineto{\pgfqpoint{2.208134in}{2.219477in}}%
\pgfpathlineto{\pgfqpoint{2.194694in}{2.224521in}}%
\pgfpathlineto{\pgfqpoint{2.174878in}{2.232276in}}%
\pgfpathlineto{\pgfqpoint{2.146860in}{2.243999in}}%
\pgfpathlineto{\pgfqpoint{2.141622in}{2.246290in}}%
\pgfpathlineto{\pgfqpoint{2.108365in}{2.261816in}}%
\pgfpathlineto{\pgfqpoint{2.105018in}{2.263477in}}%
\pgfpathlineto{\pgfqpoint{2.075109in}{2.279007in}}%
\pgfpathlineto{\pgfqpoint{2.067907in}{2.282954in}}%
\pgfpathlineto{\pgfqpoint{2.041853in}{2.297939in}}%
\pgfpathlineto{\pgfqpoint{2.034431in}{2.302432in}}%
\pgfpathlineto{\pgfqpoint{2.008597in}{2.318886in}}%
\pgfpathlineto{\pgfqpoint{2.004075in}{2.321909in}}%
\pgfpathlineto{\pgfqpoint{1.976497in}{2.341387in}}%
\pgfpathlineto{\pgfqpoint{1.975341in}{2.342256in}}%
\pgfpathlineto{\pgfqpoint{1.951739in}{2.360865in}}%
\pgfpathlineto{\pgfqpoint{1.942084in}{2.368966in}}%
\pgfpathlineto{\pgfqpoint{1.929126in}{2.380342in}}%
\pgfpathlineto{\pgfqpoint{1.908828in}{2.399386in}}%
\pgfpathlineto{\pgfqpoint{1.908385in}{2.399820in}}%
\pgfpathlineto{\pgfqpoint{1.890185in}{2.419298in}}%
\pgfpathlineto{\pgfqpoint{1.875572in}{2.436244in}}%
\pgfpathlineto{\pgfqpoint{1.873480in}{2.438775in}}%
\pgfpathlineto{\pgfqpoint{1.859073in}{2.458253in}}%
\pgfpathlineto{\pgfqpoint{1.846160in}{2.477731in}}%
\pgfpathlineto{\pgfqpoint{1.842316in}{2.484444in}}%
\pgfpathlineto{\pgfqpoint{1.835311in}{2.497208in}}%
\pgfpathlineto{\pgfqpoint{1.826326in}{2.516686in}}%
\pgfpathlineto{\pgfqpoint{1.819076in}{2.536163in}}%
\pgfpathlineto{\pgfqpoint{1.813678in}{2.555641in}}%
\pgfpathlineto{\pgfqpoint{1.810261in}{2.575119in}}%
\pgfpathlineto{\pgfqpoint{1.809060in}{2.593224in}}%
\pgfpathlineto{\pgfqpoint{1.808973in}{2.594596in}}%
\pgfpathlineto{\pgfqpoint{1.809060in}{2.596487in}}%
\pgfpathlineto{\pgfqpoint{1.809948in}{2.614074in}}%
\pgfpathlineto{\pgfqpoint{1.813376in}{2.633552in}}%
\pgfpathlineto{\pgfqpoint{1.819436in}{2.653029in}}%
\pgfpathlineto{\pgfqpoint{1.828336in}{2.672507in}}%
\pgfpathlineto{\pgfqpoint{1.840305in}{2.691985in}}%
\pgfpathlineto{\pgfqpoint{1.842316in}{2.694608in}}%
\pgfpathlineto{\pgfqpoint{1.856778in}{2.711462in}}%
\pgfpathlineto{\pgfqpoint{1.875572in}{2.729223in}}%
\pgfpathlineto{\pgfqpoint{1.877634in}{2.730940in}}%
\pgfpathlineto{\pgfqpoint{1.905338in}{2.750418in}}%
\pgfpathlineto{\pgfqpoint{1.908828in}{2.752528in}}%
\pgfpathlineto{\pgfqpoint{1.942084in}{2.769824in}}%
\pgfpathlineto{\pgfqpoint{1.942245in}{2.769895in}}%
\pgfpathlineto{\pgfqpoint{1.975341in}{2.782719in}}%
\pgfpathlineto{\pgfqpoint{1.996417in}{2.789373in}}%
\pgfpathlineto{\pgfqpoint{2.008597in}{2.792804in}}%
\pgfpathlineto{\pgfqpoint{2.041853in}{2.800278in}}%
\pgfpathlineto{\pgfqpoint{2.075109in}{2.805850in}}%
\pgfpathlineto{\pgfqpoint{2.101101in}{2.808850in}}%
\pgfpathlineto{\pgfqpoint{2.108365in}{2.809611in}}%
\pgfpathlineto{\pgfqpoint{2.141622in}{2.811658in}}%
\pgfpathlineto{\pgfqpoint{2.174878in}{2.812303in}}%
\pgfpathlineto{\pgfqpoint{2.208134in}{2.811641in}}%
\pgfpathlineto{\pgfqpoint{2.241390in}{2.809757in}}%
\pgfpathlineto{\pgfqpoint{2.251190in}{2.808850in}}%
\pgfpathlineto{\pgfqpoint{2.274646in}{2.806588in}}%
\pgfpathlineto{\pgfqpoint{2.307902in}{2.802225in}}%
\pgfpathlineto{\pgfqpoint{2.341159in}{2.796803in}}%
\pgfpathlineto{\pgfqpoint{2.374415in}{2.790387in}}%
\pgfpathlineto{\pgfqpoint{2.378926in}{2.789373in}}%
\pgfpathlineto{\pgfqpoint{2.407671in}{2.782636in}}%
\pgfpathlineto{\pgfqpoint{2.440927in}{2.773910in}}%
\pgfpathlineto{\pgfqpoint{2.454698in}{2.769895in}}%
\pgfpathlineto{\pgfqpoint{2.474183in}{2.763970in}}%
\pgfpathlineto{\pgfqpoint{2.507440in}{2.752941in}}%
\pgfpathlineto{\pgfqpoint{2.514433in}{2.750418in}}%
\pgfpathlineto{\pgfqpoint{2.540696in}{2.740526in}}%
\pgfpathlineto{\pgfqpoint{2.564479in}{2.730940in}}%
\pgfpathlineto{\pgfqpoint{2.573952in}{2.726947in}}%
\pgfpathlineto{\pgfqpoint{2.607208in}{2.711999in}}%
\pgfpathlineto{\pgfqpoint{2.608329in}{2.711462in}}%
\pgfpathlineto{\pgfqpoint{2.640464in}{2.695342in}}%
\pgfpathlineto{\pgfqpoint{2.646801in}{2.691985in}}%
\pgfpathlineto{\pgfqpoint{2.673720in}{2.677022in}}%
\pgfpathlineto{\pgfqpoint{2.681433in}{2.672507in}}%
\pgfpathlineto{\pgfqpoint{2.706977in}{2.656783in}}%
\pgfpathlineto{\pgfqpoint{2.712781in}{2.653029in}}%
\pgfpathlineto{\pgfqpoint{2.740233in}{2.634313in}}%
\pgfpathlineto{\pgfqpoint{2.741298in}{2.633552in}}%
\pgfpathlineto{\pgfqpoint{2.766920in}{2.614074in}}%
\pgfpathlineto{\pgfqpoint{2.773489in}{2.608761in}}%
\pgfpathlineto{\pgfqpoint{2.790215in}{2.594596in}}%
\pgfpathlineto{\pgfqpoint{2.806745in}{2.579659in}}%
\pgfpathlineto{\pgfqpoint{2.811553in}{2.575119in}}%
\pgfpathlineto{\pgfqpoint{2.830590in}{2.555641in}}%
\pgfpathlineto{\pgfqpoint{2.840001in}{2.545174in}}%
\pgfpathlineto{\pgfqpoint{2.847762in}{2.536163in}}%
\pgfpathlineto{\pgfqpoint{2.862962in}{2.516686in}}%
\pgfpathlineto{\pgfqpoint{2.873258in}{2.501988in}}%
\pgfpathlineto{\pgfqpoint{2.876468in}{2.497208in}}%
\pgfpathlineto{\pgfqpoint{2.887828in}{2.477731in}}%
\pgfpathlineto{\pgfqpoint{2.897571in}{2.458253in}}%
\pgfpathlineto{\pgfqpoint{2.905587in}{2.438775in}}%
\pgfpathlineto{\pgfqpoint{2.906514in}{2.435792in}}%
\pgfpathlineto{\pgfqpoint{2.911424in}{2.419298in}}%
\pgfpathlineto{\pgfqpoint{2.915339in}{2.399820in}}%
\pgfpathlineto{\pgfqpoint{2.917264in}{2.380342in}}%
\pgfpathlineto{\pgfqpoint{2.917063in}{2.360865in}}%
\pgfpathlineto{\pgfqpoint{2.914586in}{2.341387in}}%
\pgfpathlineto{\pgfqpoint{2.909667in}{2.321909in}}%
\pgfpathlineto{\pgfqpoint{2.906514in}{2.313647in}}%
\pgfpathlineto{\pgfqpoint{2.901782in}{2.302432in}}%
\pgfpathlineto{\pgfqpoint{2.890573in}{2.282954in}}%
\pgfpathlineto{\pgfqpoint{2.876023in}{2.263477in}}%
\pgfpathlineto{\pgfqpoint{2.873258in}{2.260441in}}%
\pgfpathlineto{\pgfqpoint{2.856399in}{2.243999in}}%
\pgfpathlineto{\pgfqpoint{2.840001in}{2.230732in}}%
\pgfpathlineto{\pgfqpoint{2.831225in}{2.224521in}}%
\pgfpathlineto{\pgfqpoint{2.806745in}{2.209719in}}%
\pgfpathlineto{\pgfqpoint{2.797719in}{2.205044in}}%
\pgfpathlineto{\pgfqpoint{2.773489in}{2.194087in}}%
\pgfpathlineto{\pgfqpoint{2.750860in}{2.185566in}}%
\pgfpathlineto{\pgfqpoint{2.740233in}{2.182015in}}%
\pgfpathlineto{\pgfqpoint{2.706977in}{2.172937in}}%
\pgfpathlineto{\pgfqpoint{2.674486in}{2.166088in}}%
\pgfpathlineto{\pgfqpoint{2.673720in}{2.165943in}}%
\pgfpathlineto{\pgfqpoint{2.640464in}{2.161207in}}%
\pgfpathlineto{\pgfqpoint{2.607208in}{2.158088in}}%
\pgfpathlineto{\pgfqpoint{2.573952in}{2.156471in}}%
\pgfpathlineto{\pgfqpoint{2.540696in}{2.156253in}}%
\pgfpathlineto{\pgfqpoint{2.507440in}{2.157340in}}%
\pgfpathlineto{\pgfqpoint{2.474183in}{2.159647in}}%
\pgfpathlineto{\pgfqpoint{2.440927in}{2.163101in}}%
\pgfpathclose%
\pgfusepath{stroke}%
\end{pgfscope}%
\begin{pgfscope}%
\pgfpathrectangle{\pgfqpoint{0.711606in}{0.549444in}}{\pgfqpoint{4.955171in}{2.902168in}}%
\pgfusepath{clip}%
\pgfsetbuttcap%
\pgfsetroundjoin%
\pgfsetlinewidth{1.003750pt}%
\definecolor{currentstroke}{rgb}{0.033385,0.023702,0.123397}%
\pgfsetstrokecolor{currentstroke}%
\pgfsetdash{}{0pt}%
\pgfpathmoveto{\pgfqpoint{2.573952in}{0.937058in}}%
\pgfpathlineto{\pgfqpoint{2.552251in}{0.938997in}}%
\pgfpathlineto{\pgfqpoint{2.540696in}{0.940054in}}%
\pgfpathlineto{\pgfqpoint{2.507440in}{0.943764in}}%
\pgfpathlineto{\pgfqpoint{2.474183in}{0.948094in}}%
\pgfpathlineto{\pgfqpoint{2.440927in}{0.953019in}}%
\pgfpathlineto{\pgfqpoint{2.407946in}{0.958475in}}%
\pgfpathlineto{\pgfqpoint{2.407671in}{0.958521in}}%
\pgfpathlineto{\pgfqpoint{2.374415in}{0.964792in}}%
\pgfpathlineto{\pgfqpoint{2.341159in}{0.971611in}}%
\pgfpathlineto{\pgfqpoint{2.312506in}{0.977952in}}%
\pgfpathlineto{\pgfqpoint{2.307902in}{0.978995in}}%
\pgfpathlineto{\pgfqpoint{2.274646in}{0.987132in}}%
\pgfpathlineto{\pgfqpoint{2.241390in}{0.995772in}}%
\pgfpathlineto{\pgfqpoint{2.235407in}{0.997430in}}%
\pgfpathlineto{\pgfqpoint{2.208134in}{1.005165in}}%
\pgfpathlineto{\pgfqpoint{2.174878in}{1.015094in}}%
\pgfpathlineto{\pgfqpoint{2.169131in}{1.016908in}}%
\pgfpathlineto{\pgfqpoint{2.141622in}{1.025794in}}%
\pgfpathlineto{\pgfqpoint{2.110240in}{1.036385in}}%
\pgfpathlineto{\pgfqpoint{2.108365in}{1.037033in}}%
\pgfpathlineto{\pgfqpoint{2.075109in}{1.049109in}}%
\pgfpathlineto{\pgfqpoint{2.057242in}{1.055863in}}%
\pgfpathlineto{\pgfqpoint{2.041853in}{1.061823in}}%
\pgfpathlineto{\pgfqpoint{2.008597in}{1.075203in}}%
\pgfpathlineto{\pgfqpoint{2.008270in}{1.075340in}}%
\pgfpathlineto{\pgfqpoint{1.975341in}{1.089496in}}%
\pgfpathlineto{\pgfqpoint{1.963383in}{1.094818in}}%
\pgfpathlineto{\pgfqpoint{1.942084in}{1.104540in}}%
\pgfpathlineto{\pgfqpoint{1.921414in}{1.114296in}}%
\pgfpathlineto{\pgfqpoint{1.908828in}{1.120392in}}%
\pgfpathlineto{\pgfqpoint{1.882085in}{1.133773in}}%
\pgfpathlineto{\pgfqpoint{1.875572in}{1.137120in}}%
\pgfpathlineto{\pgfqpoint{1.845150in}{1.153251in}}%
\pgfpathlineto{\pgfqpoint{1.842316in}{1.154796in}}%
\pgfpathlineto{\pgfqpoint{1.810396in}{1.172729in}}%
\pgfpathlineto{\pgfqpoint{1.809060in}{1.173501in}}%
\pgfpathlineto{\pgfqpoint{1.777631in}{1.192206in}}%
\pgfpathlineto{\pgfqpoint{1.775804in}{1.193326in}}%
\pgfpathlineto{\pgfqpoint{1.746687in}{1.211684in}}%
\pgfpathlineto{\pgfqpoint{1.742547in}{1.214373in}}%
\pgfpathlineto{\pgfqpoint{1.717414in}{1.231162in}}%
\pgfpathlineto{\pgfqpoint{1.709291in}{1.236757in}}%
\pgfpathlineto{\pgfqpoint{1.689679in}{1.250639in}}%
\pgfpathlineto{\pgfqpoint{1.676035in}{1.260608in}}%
\pgfpathlineto{\pgfqpoint{1.663360in}{1.270117in}}%
\pgfpathlineto{\pgfqpoint{1.642779in}{1.286072in}}%
\pgfpathlineto{\pgfqpoint{1.638350in}{1.289595in}}%
\pgfpathlineto{\pgfqpoint{1.614747in}{1.309072in}}%
\pgfpathlineto{\pgfqpoint{1.609523in}{1.313552in}}%
\pgfpathlineto{\pgfqpoint{1.592469in}{1.328550in}}%
\pgfpathlineto{\pgfqpoint{1.576266in}{1.343338in}}%
\pgfpathlineto{\pgfqpoint{1.571254in}{1.348027in}}%
\pgfpathlineto{\pgfqpoint{1.551326in}{1.367505in}}%
\pgfpathlineto{\pgfqpoint{1.543010in}{1.376000in}}%
\pgfpathlineto{\pgfqpoint{1.532519in}{1.386983in}}%
\pgfpathlineto{\pgfqpoint{1.514752in}{1.406460in}}%
\pgfpathlineto{\pgfqpoint{1.509754in}{1.412238in}}%
\pgfpathlineto{\pgfqpoint{1.498187in}{1.425938in}}%
\pgfpathlineto{\pgfqpoint{1.482583in}{1.445416in}}%
\pgfpathlineto{\pgfqpoint{1.476498in}{1.453489in}}%
\pgfpathlineto{\pgfqpoint{1.468104in}{1.464893in}}%
\pgfpathlineto{\pgfqpoint{1.454665in}{1.484371in}}%
\pgfpathlineto{\pgfqpoint{1.443242in}{1.502076in}}%
\pgfpathlineto{\pgfqpoint{1.442125in}{1.503849in}}%
\pgfpathlineto{\pgfqpoint{1.430850in}{1.523326in}}%
\pgfpathlineto{\pgfqpoint{1.420476in}{1.542804in}}%
\pgfpathlineto{\pgfqpoint{1.411034in}{1.562281in}}%
\pgfpathlineto{\pgfqpoint{1.409986in}{1.564718in}}%
\pgfpathlineto{\pgfqpoint{1.402825in}{1.581759in}}%
\pgfpathlineto{\pgfqpoint{1.395630in}{1.601237in}}%
\pgfpathlineto{\pgfqpoint{1.389446in}{1.620714in}}%
\pgfpathlineto{\pgfqpoint{1.384312in}{1.640192in}}%
\pgfpathlineto{\pgfqpoint{1.380268in}{1.659670in}}%
\pgfpathlineto{\pgfqpoint{1.377356in}{1.679147in}}%
\pgfpathlineto{\pgfqpoint{1.376729in}{1.686228in}}%
\pgfpathlineto{\pgfqpoint{1.375660in}{1.698625in}}%
\pgfpathlineto{\pgfqpoint{1.375165in}{1.718103in}}%
\pgfpathlineto{\pgfqpoint{1.375893in}{1.737580in}}%
\pgfpathlineto{\pgfqpoint{1.376729in}{1.745793in}}%
\pgfpathlineto{\pgfqpoint{1.377938in}{1.757058in}}%
\pgfpathlineto{\pgfqpoint{1.381393in}{1.776536in}}%
\pgfpathlineto{\pgfqpoint{1.386281in}{1.796013in}}%
\pgfpathlineto{\pgfqpoint{1.392664in}{1.815491in}}%
\pgfpathlineto{\pgfqpoint{1.400605in}{1.834968in}}%
\pgfpathlineto{\pgfqpoint{1.409986in}{1.854071in}}%
\pgfpathlineto{\pgfqpoint{1.410181in}{1.854446in}}%
\pgfpathlineto{\pgfqpoint{1.421958in}{1.873924in}}%
\pgfpathlineto{\pgfqpoint{1.435607in}{1.893401in}}%
\pgfpathlineto{\pgfqpoint{1.443242in}{1.903005in}}%
\pgfpathlineto{\pgfqpoint{1.451608in}{1.912879in}}%
\pgfpathlineto{\pgfqpoint{1.470163in}{1.932357in}}%
\pgfpathlineto{\pgfqpoint{1.476498in}{1.938344in}}%
\pgfpathlineto{\pgfqpoint{1.491794in}{1.951834in}}%
\pgfpathlineto{\pgfqpoint{1.509754in}{1.966150in}}%
\pgfpathlineto{\pgfqpoint{1.516730in}{1.971312in}}%
\pgfpathlineto{\pgfqpoint{1.543010in}{1.989051in}}%
\pgfpathlineto{\pgfqpoint{1.545803in}{1.990790in}}%
\pgfpathlineto{\pgfqpoint{1.576266in}{2.008232in}}%
\pgfpathlineto{\pgfqpoint{1.580147in}{2.010267in}}%
\pgfpathlineto{\pgfqpoint{1.609523in}{2.024527in}}%
\pgfpathlineto{\pgfqpoint{1.621354in}{2.029745in}}%
\pgfpathlineto{\pgfqpoint{1.642779in}{2.038538in}}%
\pgfpathlineto{\pgfqpoint{1.671723in}{2.049222in}}%
\pgfpathlineto{\pgfqpoint{1.676035in}{2.050711in}}%
\pgfpathlineto{\pgfqpoint{1.709291in}{2.061040in}}%
\pgfpathlineto{\pgfqpoint{1.737296in}{2.068700in}}%
\pgfpathlineto{\pgfqpoint{1.742547in}{2.070054in}}%
\pgfpathlineto{\pgfqpoint{1.775804in}{2.077616in}}%
\pgfpathlineto{\pgfqpoint{1.809060in}{2.084126in}}%
\pgfpathlineto{\pgfqpoint{1.833386in}{2.088178in}}%
\pgfpathlineto{\pgfqpoint{1.842316in}{2.089618in}}%
\pgfpathlineto{\pgfqpoint{1.875572in}{2.094187in}}%
\pgfpathlineto{\pgfqpoint{1.908828in}{2.098052in}}%
\pgfpathlineto{\pgfqpoint{1.942084in}{2.101347in}}%
\pgfpathlineto{\pgfqpoint{1.975341in}{2.104249in}}%
\pgfpathlineto{\pgfqpoint{2.008597in}{2.107007in}}%
\pgfpathlineto{\pgfqpoint{2.016264in}{2.107655in}}%
\pgfpathlineto{\pgfqpoint{2.041853in}{2.111793in}}%
\pgfpathlineto{\pgfqpoint{2.075109in}{2.123022in}}%
\pgfpathlineto{\pgfqpoint{2.081350in}{2.127133in}}%
\pgfpathlineto{\pgfqpoint{2.075109in}{2.132636in}}%
\pgfpathlineto{\pgfqpoint{2.064009in}{2.146611in}}%
\pgfpathlineto{\pgfqpoint{2.041853in}{2.159665in}}%
\pgfpathlineto{\pgfqpoint{2.032946in}{2.166088in}}%
\pgfpathlineto{\pgfqpoint{2.008597in}{2.179416in}}%
\pgfpathlineto{\pgfqpoint{1.998626in}{2.185566in}}%
\pgfpathlineto{\pgfqpoint{1.975341in}{2.198450in}}%
\pgfpathlineto{\pgfqpoint{1.964304in}{2.205044in}}%
\pgfpathlineto{\pgfqpoint{1.942084in}{2.217886in}}%
\pgfpathlineto{\pgfqpoint{1.931224in}{2.224521in}}%
\pgfpathlineto{\pgfqpoint{1.908828in}{2.238250in}}%
\pgfpathlineto{\pgfqpoint{1.899860in}{2.243999in}}%
\pgfpathlineto{\pgfqpoint{1.875572in}{2.259912in}}%
\pgfpathlineto{\pgfqpoint{1.870341in}{2.263477in}}%
\pgfpathlineto{\pgfqpoint{1.842648in}{2.282954in}}%
\pgfpathlineto{\pgfqpoint{1.842316in}{2.283198in}}%
\pgfpathlineto{\pgfqpoint{1.817019in}{2.302432in}}%
\pgfpathlineto{\pgfqpoint{1.809060in}{2.308747in}}%
\pgfpathlineto{\pgfqpoint{1.793024in}{2.321909in}}%
\pgfpathlineto{\pgfqpoint{1.775804in}{2.336700in}}%
\pgfpathlineto{\pgfqpoint{1.770523in}{2.341387in}}%
\pgfpathlineto{\pgfqpoint{1.749719in}{2.360865in}}%
\pgfpathlineto{\pgfqpoint{1.742547in}{2.367966in}}%
\pgfpathlineto{\pgfqpoint{1.730445in}{2.380342in}}%
\pgfpathlineto{\pgfqpoint{1.712466in}{2.399820in}}%
\pgfpathlineto{\pgfqpoint{1.709291in}{2.403512in}}%
\pgfpathlineto{\pgfqpoint{1.696140in}{2.419298in}}%
\pgfpathlineto{\pgfqpoint{1.680994in}{2.438775in}}%
\pgfpathlineto{\pgfqpoint{1.676035in}{2.445715in}}%
\pgfpathlineto{\pgfqpoint{1.667351in}{2.458253in}}%
\pgfpathlineto{\pgfqpoint{1.655039in}{2.477731in}}%
\pgfpathlineto{\pgfqpoint{1.643884in}{2.497208in}}%
\pgfpathlineto{\pgfqpoint{1.642779in}{2.499406in}}%
\pgfpathlineto{\pgfqpoint{1.634353in}{2.516686in}}%
\pgfpathlineto{\pgfqpoint{1.626102in}{2.536163in}}%
\pgfpathlineto{\pgfqpoint{1.619140in}{2.555641in}}%
\pgfpathlineto{\pgfqpoint{1.613531in}{2.575119in}}%
\pgfpathlineto{\pgfqpoint{1.609523in}{2.593768in}}%
\pgfpathlineto{\pgfqpoint{1.609350in}{2.594596in}}%
\pgfpathlineto{\pgfqpoint{1.606783in}{2.614074in}}%
\pgfpathlineto{\pgfqpoint{1.605713in}{2.633552in}}%
\pgfpathlineto{\pgfqpoint{1.606216in}{2.653029in}}%
\pgfpathlineto{\pgfqpoint{1.608371in}{2.672507in}}%
\pgfpathlineto{\pgfqpoint{1.609523in}{2.678340in}}%
\pgfpathlineto{\pgfqpoint{1.612411in}{2.691985in}}%
\pgfpathlineto{\pgfqpoint{1.618451in}{2.711462in}}%
\pgfpathlineto{\pgfqpoint{1.626537in}{2.730940in}}%
\pgfpathlineto{\pgfqpoint{1.636788in}{2.750418in}}%
\pgfpathlineto{\pgfqpoint{1.642779in}{2.759813in}}%
\pgfpathlineto{\pgfqpoint{1.649737in}{2.769895in}}%
\pgfpathlineto{\pgfqpoint{1.665663in}{2.789373in}}%
\pgfpathlineto{\pgfqpoint{1.676035in}{2.800273in}}%
\pgfpathlineto{\pgfqpoint{1.684948in}{2.808850in}}%
\pgfpathlineto{\pgfqpoint{1.708171in}{2.828328in}}%
\pgfpathlineto{\pgfqpoint{1.709291in}{2.829164in}}%
\pgfpathlineto{\pgfqpoint{1.736861in}{2.847806in}}%
\pgfpathlineto{\pgfqpoint{1.742547in}{2.851245in}}%
\pgfpathlineto{\pgfqpoint{1.772145in}{2.867283in}}%
\pgfpathlineto{\pgfqpoint{1.775804in}{2.869077in}}%
\pgfpathlineto{\pgfqpoint{1.809060in}{2.883564in}}%
\pgfpathlineto{\pgfqpoint{1.817442in}{2.886761in}}%
\pgfpathlineto{\pgfqpoint{1.842316in}{2.895439in}}%
\pgfpathlineto{\pgfqpoint{1.875572in}{2.905372in}}%
\pgfpathlineto{\pgfqpoint{1.878975in}{2.906239in}}%
\pgfpathlineto{\pgfqpoint{1.908828in}{2.913254in}}%
\pgfpathlineto{\pgfqpoint{1.942084in}{2.919663in}}%
\pgfpathlineto{\pgfqpoint{1.975341in}{2.924732in}}%
\pgfpathlineto{\pgfqpoint{1.983823in}{2.925716in}}%
\pgfpathlineto{\pgfqpoint{2.008597in}{2.928392in}}%
\pgfpathlineto{\pgfqpoint{2.041853in}{2.930871in}}%
\pgfpathlineto{\pgfqpoint{2.075109in}{2.932279in}}%
\pgfpathlineto{\pgfqpoint{2.108365in}{2.932671in}}%
\pgfpathlineto{\pgfqpoint{2.141622in}{2.932096in}}%
\pgfpathlineto{\pgfqpoint{2.174878in}{2.930601in}}%
\pgfpathlineto{\pgfqpoint{2.208134in}{2.928232in}}%
\pgfpathlineto{\pgfqpoint{2.234155in}{2.925716in}}%
\pgfpathlineto{\pgfqpoint{2.241390in}{2.924995in}}%
\pgfpathlineto{\pgfqpoint{2.274646in}{2.920809in}}%
\pgfpathlineto{\pgfqpoint{2.307902in}{2.915834in}}%
\pgfpathlineto{\pgfqpoint{2.341159in}{2.910109in}}%
\pgfpathlineto{\pgfqpoint{2.361020in}{2.906239in}}%
\pgfpathlineto{\pgfqpoint{2.374415in}{2.903548in}}%
\pgfpathlineto{\pgfqpoint{2.407671in}{2.896096in}}%
\pgfpathlineto{\pgfqpoint{2.440927in}{2.887973in}}%
\pgfpathlineto{\pgfqpoint{2.445464in}{2.886761in}}%
\pgfpathlineto{\pgfqpoint{2.474183in}{2.878854in}}%
\pgfpathlineto{\pgfqpoint{2.507440in}{2.869052in}}%
\pgfpathlineto{\pgfqpoint{2.513021in}{2.867283in}}%
\pgfpathlineto{\pgfqpoint{2.540696in}{2.858241in}}%
\pgfpathlineto{\pgfqpoint{2.570890in}{2.847806in}}%
\pgfpathlineto{\pgfqpoint{2.573952in}{2.846713in}}%
\pgfpathlineto{\pgfqpoint{2.607208in}{2.834104in}}%
\pgfpathlineto{\pgfqpoint{2.621690in}{2.828328in}}%
\pgfpathlineto{\pgfqpoint{2.640464in}{2.820595in}}%
\pgfpathlineto{\pgfqpoint{2.667670in}{2.808850in}}%
\pgfpathlineto{\pgfqpoint{2.673720in}{2.806150in}}%
\pgfpathlineto{\pgfqpoint{2.706977in}{2.790607in}}%
\pgfpathlineto{\pgfqpoint{2.709501in}{2.789373in}}%
\pgfpathlineto{\pgfqpoint{2.740233in}{2.773832in}}%
\pgfpathlineto{\pgfqpoint{2.747710in}{2.769895in}}%
\pgfpathlineto{\pgfqpoint{2.773489in}{2.755841in}}%
\pgfpathlineto{\pgfqpoint{2.783058in}{2.750418in}}%
\pgfpathlineto{\pgfqpoint{2.806745in}{2.736498in}}%
\pgfpathlineto{\pgfqpoint{2.815858in}{2.730940in}}%
\pgfpathlineto{\pgfqpoint{2.840001in}{2.715650in}}%
\pgfpathlineto{\pgfqpoint{2.846380in}{2.711462in}}%
\pgfpathlineto{\pgfqpoint{2.873258in}{2.693115in}}%
\pgfpathlineto{\pgfqpoint{2.874857in}{2.691985in}}%
\pgfpathlineto{\pgfqpoint{2.901228in}{2.672507in}}%
\pgfpathlineto{\pgfqpoint{2.906514in}{2.668426in}}%
\pgfpathlineto{\pgfqpoint{2.925785in}{2.653029in}}%
\pgfpathlineto{\pgfqpoint{2.939770in}{2.641339in}}%
\pgfpathlineto{\pgfqpoint{2.948784in}{2.633552in}}%
\pgfpathlineto{\pgfqpoint{2.970237in}{2.614074in}}%
\pgfpathlineto{\pgfqpoint{2.973026in}{2.611390in}}%
\pgfpathlineto{\pgfqpoint{2.989921in}{2.594596in}}%
\pgfpathlineto{\pgfqpoint{3.006282in}{2.577431in}}%
\pgfpathlineto{\pgfqpoint{3.008418in}{2.575119in}}%
\pgfpathlineto{\pgfqpoint{3.025187in}{2.555641in}}%
\pgfpathlineto{\pgfqpoint{3.039539in}{2.537862in}}%
\pgfpathlineto{\pgfqpoint{3.040868in}{2.536163in}}%
\pgfpathlineto{\pgfqpoint{3.054851in}{2.516686in}}%
\pgfpathlineto{\pgfqpoint{3.067736in}{2.497208in}}%
\pgfpathlineto{\pgfqpoint{3.072795in}{2.488734in}}%
\pgfpathlineto{\pgfqpoint{3.079163in}{2.477731in}}%
\pgfpathlineto{\pgfqpoint{3.089191in}{2.458253in}}%
\pgfpathlineto{\pgfqpoint{3.097995in}{2.438775in}}%
\pgfpathlineto{\pgfqpoint{3.105513in}{2.419298in}}%
\pgfpathlineto{\pgfqpoint{3.106051in}{2.417575in}}%
\pgfpathlineto{\pgfqpoint{3.111420in}{2.399820in}}%
\pgfpathlineto{\pgfqpoint{3.115943in}{2.380342in}}%
\pgfpathlineto{\pgfqpoint{3.119040in}{2.360865in}}%
\pgfpathlineto{\pgfqpoint{3.120642in}{2.341387in}}%
\pgfpathlineto{\pgfqpoint{3.120672in}{2.321909in}}%
\pgfpathlineto{\pgfqpoint{3.119050in}{2.302432in}}%
\pgfpathlineto{\pgfqpoint{3.115689in}{2.282954in}}%
\pgfpathlineto{\pgfqpoint{3.110498in}{2.263477in}}%
\pgfpathlineto{\pgfqpoint{3.106051in}{2.251230in}}%
\pgfpathlineto{\pgfqpoint{3.103226in}{2.243999in}}%
\pgfpathlineto{\pgfqpoint{3.093540in}{2.224521in}}%
\pgfpathlineto{\pgfqpoint{3.081558in}{2.205044in}}%
\pgfpathlineto{\pgfqpoint{3.072795in}{2.193114in}}%
\pgfpathlineto{\pgfqpoint{3.066777in}{2.185566in}}%
\pgfpathlineto{\pgfqpoint{3.048654in}{2.166088in}}%
\pgfpathlineto{\pgfqpoint{3.039539in}{2.157573in}}%
\pgfpathlineto{\pgfqpoint{3.026681in}{2.146611in}}%
\pgfpathlineto{\pgfqpoint{3.006282in}{2.131319in}}%
\pgfpathlineto{\pgfqpoint{3.000106in}{2.127133in}}%
\pgfpathlineto{\pgfqpoint{2.973026in}{2.110754in}}%
\pgfpathlineto{\pgfqpoint{2.967297in}{2.107655in}}%
\pgfpathlineto{\pgfqpoint{2.939770in}{2.094210in}}%
\pgfpathlineto{\pgfqpoint{2.925765in}{2.088178in}}%
\pgfpathlineto{\pgfqpoint{2.906514in}{2.080605in}}%
\pgfpathlineto{\pgfqpoint{2.873258in}{2.069243in}}%
\pgfpathlineto{\pgfqpoint{2.871425in}{2.068700in}}%
\pgfpathlineto{\pgfqpoint{2.840001in}{2.060031in}}%
\pgfpathlineto{\pgfqpoint{2.806745in}{2.052246in}}%
\pgfpathlineto{\pgfqpoint{2.791434in}{2.049222in}}%
\pgfpathlineto{\pgfqpoint{2.773489in}{2.045675in}}%
\pgfpathlineto{\pgfqpoint{2.740233in}{2.039881in}}%
\pgfpathlineto{\pgfqpoint{2.706977in}{2.034472in}}%
\pgfpathlineto{\pgfqpoint{2.678543in}{2.029745in}}%
\pgfpathlineto{\pgfqpoint{2.673720in}{2.028054in}}%
\pgfpathlineto{\pgfqpoint{2.645799in}{2.010267in}}%
\pgfpathlineto{\pgfqpoint{2.667496in}{1.990790in}}%
\pgfpathlineto{\pgfqpoint{2.673720in}{1.987918in}}%
\pgfpathlineto{\pgfqpoint{2.702868in}{1.971312in}}%
\pgfpathlineto{\pgfqpoint{2.706977in}{1.969497in}}%
\pgfpathlineto{\pgfqpoint{2.740233in}{1.952951in}}%
\pgfpathlineto{\pgfqpoint{2.742189in}{1.951834in}}%
\pgfpathlineto{\pgfqpoint{2.773489in}{1.936808in}}%
\pgfpathlineto{\pgfqpoint{2.781952in}{1.932357in}}%
\pgfpathlineto{\pgfqpoint{2.806745in}{1.920216in}}%
\pgfpathlineto{\pgfqpoint{2.820877in}{1.912879in}}%
\pgfpathlineto{\pgfqpoint{2.840001in}{1.903153in}}%
\pgfpathlineto{\pgfqpoint{2.858432in}{1.893401in}}%
\pgfpathlineto{\pgfqpoint{2.873258in}{1.885511in}}%
\pgfpathlineto{\pgfqpoint{2.894482in}{1.873924in}}%
\pgfpathlineto{\pgfqpoint{2.906514in}{1.867212in}}%
\pgfpathlineto{\pgfqpoint{2.929204in}{1.854446in}}%
\pgfpathlineto{\pgfqpoint{2.939770in}{1.848292in}}%
\pgfpathlineto{\pgfqpoint{2.963113in}{1.834968in}}%
\pgfpathlineto{\pgfqpoint{2.973026in}{1.829008in}}%
\pgfpathlineto{\pgfqpoint{2.997307in}{1.815491in}}%
\pgfpathlineto{\pgfqpoint{3.006282in}{1.810065in}}%
\pgfpathlineto{\pgfqpoint{3.034545in}{1.796013in}}%
\pgfpathlineto{\pgfqpoint{3.039539in}{1.793166in}}%
\pgfpathlineto{\pgfqpoint{3.072795in}{1.782788in}}%
\pgfpathlineto{\pgfqpoint{3.106051in}{1.791358in}}%
\pgfpathlineto{\pgfqpoint{3.110486in}{1.796013in}}%
\pgfpathlineto{\pgfqpoint{3.126552in}{1.815491in}}%
\pgfpathlineto{\pgfqpoint{3.139307in}{1.832593in}}%
\pgfpathlineto{\pgfqpoint{3.140860in}{1.834968in}}%
\pgfpathlineto{\pgfqpoint{3.154854in}{1.854446in}}%
\pgfpathlineto{\pgfqpoint{3.170413in}{1.873924in}}%
\pgfpathlineto{\pgfqpoint{3.172563in}{1.876349in}}%
\pgfpathlineto{\pgfqpoint{3.188461in}{1.893401in}}%
\pgfpathlineto{\pgfqpoint{3.205819in}{1.909920in}}%
\pgfpathlineto{\pgfqpoint{3.209158in}{1.912879in}}%
\pgfpathlineto{\pgfqpoint{3.233583in}{1.932357in}}%
\pgfpathlineto{\pgfqpoint{3.239076in}{1.936320in}}%
\pgfpathlineto{\pgfqpoint{3.262443in}{1.951834in}}%
\pgfpathlineto{\pgfqpoint{3.272332in}{1.957811in}}%
\pgfpathlineto{\pgfqpoint{3.296816in}{1.971312in}}%
\pgfpathlineto{\pgfqpoint{3.305588in}{1.975751in}}%
\pgfpathlineto{\pgfqpoint{3.338473in}{1.990790in}}%
\pgfpathlineto{\pgfqpoint{3.338844in}{1.990946in}}%
\pgfpathlineto{\pgfqpoint{3.372100in}{2.003649in}}%
\pgfpathlineto{\pgfqpoint{3.391684in}{2.010267in}}%
\pgfpathlineto{\pgfqpoint{3.405357in}{2.014568in}}%
\pgfpathlineto{\pgfqpoint{3.438613in}{2.023778in}}%
\pgfpathlineto{\pgfqpoint{3.463579in}{2.029745in}}%
\pgfpathlineto{\pgfqpoint{3.471869in}{2.031600in}}%
\pgfpathlineto{\pgfqpoint{3.505125in}{2.037977in}}%
\pgfpathlineto{\pgfqpoint{3.538381in}{2.043246in}}%
\pgfpathlineto{\pgfqpoint{3.571637in}{2.047459in}}%
\pgfpathlineto{\pgfqpoint{3.589809in}{2.049222in}}%
\pgfpathlineto{\pgfqpoint{3.604894in}{2.050603in}}%
\pgfpathlineto{\pgfqpoint{3.638150in}{2.052751in}}%
\pgfpathlineto{\pgfqpoint{3.671406in}{2.054020in}}%
\pgfpathlineto{\pgfqpoint{3.704662in}{2.054444in}}%
\pgfpathlineto{\pgfqpoint{3.737918in}{2.054060in}}%
\pgfpathlineto{\pgfqpoint{3.771175in}{2.052900in}}%
\pgfpathlineto{\pgfqpoint{3.804431in}{2.050996in}}%
\pgfpathlineto{\pgfqpoint{3.826848in}{2.049222in}}%
\pgfpathlineto{\pgfqpoint{3.837687in}{2.048343in}}%
\pgfpathlineto{\pgfqpoint{3.870943in}{2.044906in}}%
\pgfpathlineto{\pgfqpoint{3.904199in}{2.040788in}}%
\pgfpathlineto{\pgfqpoint{3.937455in}{2.036014in}}%
\pgfpathlineto{\pgfqpoint{3.970712in}{2.030610in}}%
\pgfpathlineto{\pgfqpoint{3.975448in}{2.029745in}}%
\pgfpathlineto{\pgfqpoint{4.003968in}{2.024401in}}%
\pgfpathlineto{\pgfqpoint{4.037224in}{2.017560in}}%
\pgfpathlineto{\pgfqpoint{4.069919in}{2.010267in}}%
\pgfpathlineto{\pgfqpoint{4.070480in}{2.010139in}}%
\pgfpathlineto{\pgfqpoint{4.103736in}{2.001861in}}%
\pgfpathlineto{\pgfqpoint{4.136993in}{1.993039in}}%
\pgfpathlineto{\pgfqpoint{4.144917in}{1.990790in}}%
\pgfpathlineto{\pgfqpoint{4.170249in}{1.983415in}}%
\pgfpathlineto{\pgfqpoint{4.203505in}{1.973190in}}%
\pgfpathlineto{\pgfqpoint{4.209270in}{1.971312in}}%
\pgfpathlineto{\pgfqpoint{4.236761in}{1.962122in}}%
\pgfpathlineto{\pgfqpoint{4.266149in}{1.951834in}}%
\pgfpathlineto{\pgfqpoint{4.270017in}{1.950444in}}%
\pgfpathlineto{\pgfqpoint{4.303273in}{1.937874in}}%
\pgfpathlineto{\pgfqpoint{4.317258in}{1.932357in}}%
\pgfpathlineto{\pgfqpoint{4.336530in}{1.924548in}}%
\pgfpathlineto{\pgfqpoint{4.364205in}{1.912879in}}%
\pgfpathlineto{\pgfqpoint{4.369786in}{1.910460in}}%
\pgfpathlineto{\pgfqpoint{4.403042in}{1.895457in}}%
\pgfpathlineto{\pgfqpoint{4.407427in}{1.893401in}}%
\pgfpathlineto{\pgfqpoint{4.436298in}{1.879484in}}%
\pgfpathlineto{\pgfqpoint{4.447438in}{1.873924in}}%
\pgfpathlineto{\pgfqpoint{4.469554in}{1.862564in}}%
\pgfpathlineto{\pgfqpoint{4.484835in}{1.854446in}}%
\pgfpathlineto{\pgfqpoint{4.502811in}{1.844611in}}%
\pgfpathlineto{\pgfqpoint{4.519870in}{1.834968in}}%
\pgfpathlineto{\pgfqpoint{4.536067in}{1.825531in}}%
\pgfpathlineto{\pgfqpoint{4.552763in}{1.815491in}}%
\pgfpathlineto{\pgfqpoint{4.569323in}{1.805215in}}%
\pgfpathlineto{\pgfqpoint{4.583707in}{1.796013in}}%
\pgfpathlineto{\pgfqpoint{4.602579in}{1.783542in}}%
\pgfpathlineto{\pgfqpoint{4.612873in}{1.776536in}}%
\pgfpathlineto{\pgfqpoint{4.635835in}{1.760373in}}%
\pgfpathlineto{\pgfqpoint{4.640412in}{1.757058in}}%
\pgfpathlineto{\pgfqpoint{4.666345in}{1.737580in}}%
\pgfpathlineto{\pgfqpoint{4.669091in}{1.735435in}}%
\pgfpathlineto{\pgfqpoint{4.690662in}{1.718103in}}%
\pgfpathlineto{\pgfqpoint{4.702348in}{1.708350in}}%
\pgfpathlineto{\pgfqpoint{4.713683in}{1.698625in}}%
\pgfpathlineto{\pgfqpoint{4.735507in}{1.679147in}}%
\pgfpathlineto{\pgfqpoint{4.735604in}{1.679056in}}%
\pgfpathlineto{\pgfqpoint{4.755725in}{1.659670in}}%
\pgfpathlineto{\pgfqpoint{4.768860in}{1.646449in}}%
\pgfpathlineto{\pgfqpoint{4.774913in}{1.640192in}}%
\pgfpathlineto{\pgfqpoint{4.792790in}{1.620714in}}%
\pgfpathlineto{\pgfqpoint{4.802116in}{1.609991in}}%
\pgfpathlineto{\pgfqpoint{4.809533in}{1.601237in}}%
\pgfpathlineto{\pgfqpoint{4.825068in}{1.581759in}}%
\pgfpathlineto{\pgfqpoint{4.835372in}{1.568005in}}%
\pgfpathlineto{\pgfqpoint{4.839551in}{1.562281in}}%
\pgfpathlineto{\pgfqpoint{4.852742in}{1.542804in}}%
\pgfpathlineto{\pgfqpoint{4.864998in}{1.523326in}}%
\pgfpathlineto{\pgfqpoint{4.868629in}{1.517001in}}%
\pgfpathlineto{\pgfqpoint{4.875985in}{1.503849in}}%
\pgfpathlineto{\pgfqpoint{4.885849in}{1.484371in}}%
\pgfpathlineto{\pgfqpoint{4.894688in}{1.464893in}}%
\pgfpathlineto{\pgfqpoint{4.901885in}{1.446842in}}%
\pgfpathlineto{\pgfqpoint{4.902439in}{1.445416in}}%
\pgfpathlineto{\pgfqpoint{4.908840in}{1.425938in}}%
\pgfpathlineto{\pgfqpoint{4.914124in}{1.406460in}}%
\pgfpathlineto{\pgfqpoint{4.918244in}{1.386983in}}%
\pgfpathlineto{\pgfqpoint{4.921153in}{1.367505in}}%
\pgfpathlineto{\pgfqpoint{4.922801in}{1.348027in}}%
\pgfpathlineto{\pgfqpoint{4.923133in}{1.328550in}}%
\pgfpathlineto{\pgfqpoint{4.922093in}{1.309072in}}%
\pgfpathlineto{\pgfqpoint{4.919622in}{1.289595in}}%
\pgfpathlineto{\pgfqpoint{4.915656in}{1.270117in}}%
\pgfpathlineto{\pgfqpoint{4.910128in}{1.250639in}}%
\pgfpathlineto{\pgfqpoint{4.902967in}{1.231162in}}%
\pgfpathlineto{\pgfqpoint{4.901885in}{1.228754in}}%
\pgfpathlineto{\pgfqpoint{4.893724in}{1.211684in}}%
\pgfpathlineto{\pgfqpoint{4.882540in}{1.192206in}}%
\pgfpathlineto{\pgfqpoint{4.869371in}{1.172729in}}%
\pgfpathlineto{\pgfqpoint{4.868629in}{1.171765in}}%
\pgfpathlineto{\pgfqpoint{4.853348in}{1.153251in}}%
\pgfpathlineto{\pgfqpoint{4.835372in}{1.134219in}}%
\pgfpathlineto{\pgfqpoint{4.834919in}{1.133773in}}%
\pgfpathlineto{\pgfqpoint{4.812828in}{1.114296in}}%
\pgfpathlineto{\pgfqpoint{4.802116in}{1.105824in}}%
\pgfpathlineto{\pgfqpoint{4.787026in}{1.094818in}}%
\pgfpathlineto{\pgfqpoint{4.768860in}{1.082810in}}%
\pgfpathlineto{\pgfqpoint{4.756519in}{1.075340in}}%
\pgfpathlineto{\pgfqpoint{4.735604in}{1.063766in}}%
\pgfpathlineto{\pgfqpoint{4.719872in}{1.055863in}}%
\pgfpathlineto{\pgfqpoint{4.702348in}{1.047753in}}%
\pgfpathlineto{\pgfqpoint{4.675008in}{1.036385in}}%
\pgfpathlineto{\pgfqpoint{4.669091in}{1.034105in}}%
\pgfpathlineto{\pgfqpoint{4.635835in}{1.022628in}}%
\pgfpathlineto{\pgfqpoint{4.616994in}{1.016908in}}%
\pgfpathlineto{\pgfqpoint{4.602579in}{1.012824in}}%
\pgfpathlineto{\pgfqpoint{4.569323in}{1.004605in}}%
\pgfpathlineto{\pgfqpoint{4.536067in}{0.997618in}}%
\pgfpathlineto{\pgfqpoint{4.535010in}{0.997430in}}%
\pgfpathlineto{\pgfqpoint{4.502811in}{0.992055in}}%
\pgfpathlineto{\pgfqpoint{4.469554in}{0.987559in}}%
\pgfpathlineto{\pgfqpoint{4.436298in}{0.984071in}}%
\pgfpathlineto{\pgfqpoint{4.403042in}{0.981546in}}%
\pgfpathlineto{\pgfqpoint{4.369786in}{0.979943in}}%
\pgfpathlineto{\pgfqpoint{4.336530in}{0.979221in}}%
\pgfpathlineto{\pgfqpoint{4.303273in}{0.979343in}}%
\pgfpathlineto{\pgfqpoint{4.270017in}{0.980273in}}%
\pgfpathlineto{\pgfqpoint{4.236761in}{0.981977in}}%
\pgfpathlineto{\pgfqpoint{4.203505in}{0.984423in}}%
\pgfpathlineto{\pgfqpoint{4.170249in}{0.987583in}}%
\pgfpathlineto{\pgfqpoint{4.136993in}{0.991426in}}%
\pgfpathlineto{\pgfqpoint{4.103736in}{0.995927in}}%
\pgfpathlineto{\pgfqpoint{4.094086in}{0.997430in}}%
\pgfpathlineto{\pgfqpoint{4.070480in}{1.001201in}}%
\pgfpathlineto{\pgfqpoint{4.037224in}{1.007160in}}%
\pgfpathlineto{\pgfqpoint{4.003968in}{1.013719in}}%
\pgfpathlineto{\pgfqpoint{3.989220in}{1.016908in}}%
\pgfpathlineto{\pgfqpoint{3.970712in}{1.021011in}}%
\pgfpathlineto{\pgfqpoint{3.937455in}{1.028997in}}%
\pgfpathlineto{\pgfqpoint{3.908716in}{1.036385in}}%
\pgfpathlineto{\pgfqpoint{3.904199in}{1.037576in}}%
\pgfpathlineto{\pgfqpoint{3.870943in}{1.046988in}}%
\pgfpathlineto{\pgfqpoint{3.841254in}{1.055863in}}%
\pgfpathlineto{\pgfqpoint{3.837687in}{1.056957in}}%
\pgfpathlineto{\pgfqpoint{3.804431in}{1.067790in}}%
\pgfpathlineto{\pgfqpoint{3.782341in}{1.075340in}}%
\pgfpathlineto{\pgfqpoint{3.771175in}{1.079259in}}%
\pgfpathlineto{\pgfqpoint{3.737918in}{1.091506in}}%
\pgfpathlineto{\pgfqpoint{3.729331in}{1.094818in}}%
\pgfpathlineto{\pgfqpoint{3.704662in}{1.104587in}}%
\pgfpathlineto{\pgfqpoint{3.681100in}{1.114296in}}%
\pgfpathlineto{\pgfqpoint{3.671406in}{1.118401in}}%
\pgfpathlineto{\pgfqpoint{3.638150in}{1.133033in}}%
\pgfpathlineto{\pgfqpoint{3.636527in}{1.133773in}}%
\pgfpathlineto{\pgfqpoint{3.604894in}{1.148636in}}%
\pgfpathlineto{\pgfqpoint{3.595337in}{1.153251in}}%
\pgfpathlineto{\pgfqpoint{3.571637in}{1.165065in}}%
\pgfpathlineto{\pgfqpoint{3.556534in}{1.172729in}}%
\pgfpathlineto{\pgfqpoint{3.538381in}{1.182282in}}%
\pgfpathlineto{\pgfqpoint{3.519502in}{1.192206in}}%
\pgfpathlineto{\pgfqpoint{3.505125in}{1.200120in}}%
\pgfpathlineto{\pgfqpoint{3.483262in}{1.211684in}}%
\pgfpathlineto{\pgfqpoint{3.471869in}{1.218118in}}%
\pgfpathlineto{\pgfqpoint{3.445792in}{1.231162in}}%
\pgfpathlineto{\pgfqpoint{3.438613in}{1.235145in}}%
\pgfpathlineto{\pgfqpoint{3.405357in}{1.248351in}}%
\pgfpathlineto{\pgfqpoint{3.372100in}{1.249289in}}%
\pgfpathlineto{\pgfqpoint{3.351490in}{1.231162in}}%
\pgfpathlineto{\pgfqpoint{3.338844in}{1.215196in}}%
\pgfpathlineto{\pgfqpoint{3.336915in}{1.211684in}}%
\pgfpathlineto{\pgfqpoint{3.325811in}{1.192206in}}%
\pgfpathlineto{\pgfqpoint{3.314016in}{1.172729in}}%
\pgfpathlineto{\pgfqpoint{3.305588in}{1.160039in}}%
\pgfpathlineto{\pgfqpoint{3.301004in}{1.153251in}}%
\pgfpathlineto{\pgfqpoint{3.286189in}{1.133773in}}%
\pgfpathlineto{\pgfqpoint{3.272332in}{1.117597in}}%
\pgfpathlineto{\pgfqpoint{3.269334in}{1.114296in}}%
\pgfpathlineto{\pgfqpoint{3.249645in}{1.094818in}}%
\pgfpathlineto{\pgfqpoint{3.239076in}{1.085401in}}%
\pgfpathlineto{\pgfqpoint{3.226967in}{1.075340in}}%
\pgfpathlineto{\pgfqpoint{3.205819in}{1.059404in}}%
\pgfpathlineto{\pgfqpoint{3.200749in}{1.055863in}}%
\pgfpathlineto{\pgfqpoint{3.172563in}{1.037856in}}%
\pgfpathlineto{\pgfqpoint{3.170062in}{1.036385in}}%
\pgfpathlineto{\pgfqpoint{3.139307in}{1.019717in}}%
\pgfpathlineto{\pgfqpoint{3.133634in}{1.016908in}}%
\pgfpathlineto{\pgfqpoint{3.106051in}{1.004242in}}%
\pgfpathlineto{\pgfqpoint{3.089670in}{0.997430in}}%
\pgfpathlineto{\pgfqpoint{3.072795in}{0.990889in}}%
\pgfpathlineto{\pgfqpoint{3.039539in}{0.979318in}}%
\pgfpathlineto{\pgfqpoint{3.035156in}{0.977952in}}%
\pgfpathlineto{\pgfqpoint{3.006282in}{0.969518in}}%
\pgfpathlineto{\pgfqpoint{2.973026in}{0.961001in}}%
\pgfpathlineto{\pgfqpoint{2.961723in}{0.958475in}}%
\pgfpathlineto{\pgfqpoint{2.939770in}{0.953853in}}%
\pgfpathlineto{\pgfqpoint{2.906514in}{0.947868in}}%
\pgfpathlineto{\pgfqpoint{2.873258in}{0.942889in}}%
\pgfpathlineto{\pgfqpoint{2.841047in}{0.938997in}}%
\pgfpathlineto{\pgfqpoint{2.840001in}{0.938877in}}%
\pgfpathlineto{\pgfqpoint{2.806745in}{0.935905in}}%
\pgfpathlineto{\pgfqpoint{2.773489in}{0.933779in}}%
\pgfpathlineto{\pgfqpoint{2.740233in}{0.932467in}}%
\pgfpathlineto{\pgfqpoint{2.706977in}{0.931937in}}%
\pgfpathlineto{\pgfqpoint{2.673720in}{0.932159in}}%
\pgfpathlineto{\pgfqpoint{2.640464in}{0.933104in}}%
\pgfpathlineto{\pgfqpoint{2.607208in}{0.934746in}}%
\pgfpathlineto{\pgfqpoint{2.573952in}{0.937058in}}%
\pgfpathclose%
\pgfusepath{stroke}%
\end{pgfscope}%
\begin{pgfscope}%
\pgfpathrectangle{\pgfqpoint{0.711606in}{0.549444in}}{\pgfqpoint{4.955171in}{2.902168in}}%
\pgfusepath{clip}%
\pgfsetbuttcap%
\pgfsetroundjoin%
\pgfsetlinewidth{1.003750pt}%
\definecolor{currentstroke}{rgb}{0.051644,0.032474,0.159254}%
\pgfsetstrokecolor{currentstroke}%
\pgfsetdash{}{0pt}%
\pgfpathmoveto{\pgfqpoint{2.640464in}{0.859851in}}%
\pgfpathlineto{\pgfqpoint{2.573952in}{0.865013in}}%
\pgfpathlineto{\pgfqpoint{2.507440in}{0.872502in}}%
\pgfpathlineto{\pgfqpoint{2.440927in}{0.882181in}}%
\pgfpathlineto{\pgfqpoint{2.374415in}{0.894179in}}%
\pgfpathlineto{\pgfqpoint{2.307902in}{0.908356in}}%
\pgfpathlineto{\pgfqpoint{2.241390in}{0.924726in}}%
\pgfpathlineto{\pgfqpoint{2.174878in}{0.943355in}}%
\pgfpathlineto{\pgfqpoint{2.108365in}{0.964310in}}%
\pgfpathlineto{\pgfqpoint{2.041853in}{0.987663in}}%
\pgfpathlineto{\pgfqpoint{1.967056in}{1.016908in}}%
\pgfpathlineto{\pgfqpoint{1.908828in}{1.042050in}}%
\pgfpathlineto{\pgfqpoint{1.838543in}{1.075340in}}%
\pgfpathlineto{\pgfqpoint{1.764679in}{1.114296in}}%
\pgfpathlineto{\pgfqpoint{1.709291in}{1.146423in}}%
\pgfpathlineto{\pgfqpoint{1.667217in}{1.172729in}}%
\pgfpathlineto{\pgfqpoint{1.609523in}{1.211803in}}%
\pgfpathlineto{\pgfqpoint{1.557668in}{1.250639in}}%
\pgfpathlineto{\pgfqpoint{1.509754in}{1.289935in}}%
\pgfpathlineto{\pgfqpoint{1.467280in}{1.328550in}}%
\pgfpathlineto{\pgfqpoint{1.428487in}{1.367505in}}%
\pgfpathlineto{\pgfqpoint{1.393625in}{1.406460in}}%
\pgfpathlineto{\pgfqpoint{1.362571in}{1.445416in}}%
\pgfpathlineto{\pgfqpoint{1.335210in}{1.484371in}}%
\pgfpathlineto{\pgfqpoint{1.310217in}{1.525629in}}%
\pgfpathlineto{\pgfqpoint{1.291569in}{1.562281in}}%
\pgfpathlineto{\pgfqpoint{1.275137in}{1.601237in}}%
\pgfpathlineto{\pgfqpoint{1.262658in}{1.640192in}}%
\pgfpathlineto{\pgfqpoint{1.253913in}{1.679147in}}%
\pgfpathlineto{\pgfqpoint{1.249158in}{1.718103in}}%
\pgfpathlineto{\pgfqpoint{1.248670in}{1.757058in}}%
\pgfpathlineto{\pgfqpoint{1.250121in}{1.776536in}}%
\pgfpathlineto{\pgfqpoint{1.252756in}{1.796013in}}%
\pgfpathlineto{\pgfqpoint{1.256617in}{1.815491in}}%
\pgfpathlineto{\pgfqpoint{1.261749in}{1.834968in}}%
\pgfpathlineto{\pgfqpoint{1.268200in}{1.854446in}}%
\pgfpathlineto{\pgfqpoint{1.276961in}{1.875932in}}%
\pgfpathlineto{\pgfqpoint{1.285581in}{1.893401in}}%
\pgfpathlineto{\pgfqpoint{1.296722in}{1.912879in}}%
\pgfpathlineto{\pgfqpoint{1.310217in}{1.933382in}}%
\pgfpathlineto{\pgfqpoint{1.324463in}{1.951834in}}%
\pgfpathlineto{\pgfqpoint{1.343473in}{1.973573in}}%
\pgfpathlineto{\pgfqpoint{1.376729in}{2.005070in}}%
\pgfpathlineto{\pgfqpoint{1.382882in}{2.010267in}}%
\pgfpathlineto{\pgfqpoint{1.409986in}{2.031142in}}%
\pgfpathlineto{\pgfqpoint{1.443242in}{2.053075in}}%
\pgfpathlineto{\pgfqpoint{1.476498in}{2.071984in}}%
\pgfpathlineto{\pgfqpoint{1.509754in}{2.088453in}}%
\pgfpathlineto{\pgfqpoint{1.555638in}{2.107655in}}%
\pgfpathlineto{\pgfqpoint{1.609523in}{2.126455in}}%
\pgfpathlineto{\pgfqpoint{1.642779in}{2.136313in}}%
\pgfpathlineto{\pgfqpoint{1.748933in}{2.166088in}}%
\pgfpathlineto{\pgfqpoint{1.774532in}{2.185566in}}%
\pgfpathlineto{\pgfqpoint{1.767648in}{2.205044in}}%
\pgfpathlineto{\pgfqpoint{1.742547in}{2.229452in}}%
\pgfpathlineto{\pgfqpoint{1.701735in}{2.263477in}}%
\pgfpathlineto{\pgfqpoint{1.655721in}{2.302432in}}%
\pgfpathlineto{\pgfqpoint{1.634265in}{2.321909in}}%
\pgfpathlineto{\pgfqpoint{1.595050in}{2.360865in}}%
\pgfpathlineto{\pgfqpoint{1.560549in}{2.399820in}}%
\pgfpathlineto{\pgfqpoint{1.530638in}{2.438775in}}%
\pgfpathlineto{\pgfqpoint{1.505154in}{2.477731in}}%
\pgfpathlineto{\pgfqpoint{1.484227in}{2.516686in}}%
\pgfpathlineto{\pgfqpoint{1.475324in}{2.536163in}}%
\pgfpathlineto{\pgfqpoint{1.461232in}{2.575119in}}%
\pgfpathlineto{\pgfqpoint{1.455838in}{2.594596in}}%
\pgfpathlineto{\pgfqpoint{1.451603in}{2.614074in}}%
\pgfpathlineto{\pgfqpoint{1.448577in}{2.633552in}}%
\pgfpathlineto{\pgfqpoint{1.446809in}{2.653029in}}%
\pgfpathlineto{\pgfqpoint{1.446353in}{2.672507in}}%
\pgfpathlineto{\pgfqpoint{1.447266in}{2.691985in}}%
\pgfpathlineto{\pgfqpoint{1.449606in}{2.711462in}}%
\pgfpathlineto{\pgfqpoint{1.453437in}{2.730940in}}%
\pgfpathlineto{\pgfqpoint{1.458826in}{2.750418in}}%
\pgfpathlineto{\pgfqpoint{1.465844in}{2.769895in}}%
\pgfpathlineto{\pgfqpoint{1.476498in}{2.793000in}}%
\pgfpathlineto{\pgfqpoint{1.485490in}{2.808850in}}%
\pgfpathlineto{\pgfqpoint{1.498486in}{2.828328in}}%
\pgfpathlineto{\pgfqpoint{1.513763in}{2.847806in}}%
\pgfpathlineto{\pgfqpoint{1.543010in}{2.877787in}}%
\pgfpathlineto{\pgfqpoint{1.553184in}{2.886761in}}%
\pgfpathlineto{\pgfqpoint{1.577943in}{2.906239in}}%
\pgfpathlineto{\pgfqpoint{1.609523in}{2.927022in}}%
\pgfpathlineto{\pgfqpoint{1.642779in}{2.945385in}}%
\pgfpathlineto{\pgfqpoint{1.685416in}{2.964672in}}%
\pgfpathlineto{\pgfqpoint{1.709291in}{2.973873in}}%
\pgfpathlineto{\pgfqpoint{1.742547in}{2.985167in}}%
\pgfpathlineto{\pgfqpoint{1.775804in}{2.994571in}}%
\pgfpathlineto{\pgfqpoint{1.813554in}{3.003627in}}%
\pgfpathlineto{\pgfqpoint{1.875572in}{3.014757in}}%
\pgfpathlineto{\pgfqpoint{1.942084in}{3.022567in}}%
\pgfpathlineto{\pgfqpoint{1.975341in}{3.024930in}}%
\pgfpathlineto{\pgfqpoint{2.041853in}{3.027011in}}%
\pgfpathlineto{\pgfqpoint{2.108365in}{3.025861in}}%
\pgfpathlineto{\pgfqpoint{2.174878in}{3.021680in}}%
\pgfpathlineto{\pgfqpoint{2.241390in}{3.014543in}}%
\pgfpathlineto{\pgfqpoint{2.314463in}{3.003627in}}%
\pgfpathlineto{\pgfqpoint{2.374415in}{2.992179in}}%
\pgfpathlineto{\pgfqpoint{2.440927in}{2.976959in}}%
\pgfpathlineto{\pgfqpoint{2.507440in}{2.959028in}}%
\pgfpathlineto{\pgfqpoint{2.573952in}{2.938286in}}%
\pgfpathlineto{\pgfqpoint{2.640464in}{2.914629in}}%
\pgfpathlineto{\pgfqpoint{2.709728in}{2.886761in}}%
\pgfpathlineto{\pgfqpoint{2.773489in}{2.857733in}}%
\pgfpathlineto{\pgfqpoint{2.831663in}{2.828328in}}%
\pgfpathlineto{\pgfqpoint{2.873258in}{2.805344in}}%
\pgfpathlineto{\pgfqpoint{2.931637in}{2.769895in}}%
\pgfpathlineto{\pgfqpoint{2.973026in}{2.742191in}}%
\pgfpathlineto{\pgfqpoint{3.015158in}{2.711462in}}%
\pgfpathlineto{\pgfqpoint{3.063249in}{2.672507in}}%
\pgfpathlineto{\pgfqpoint{3.085227in}{2.653029in}}%
\pgfpathlineto{\pgfqpoint{3.125382in}{2.614074in}}%
\pgfpathlineto{\pgfqpoint{3.143708in}{2.594596in}}%
\pgfpathlineto{\pgfqpoint{3.176672in}{2.555641in}}%
\pgfpathlineto{\pgfqpoint{3.205819in}{2.515619in}}%
\pgfpathlineto{\pgfqpoint{3.228929in}{2.477731in}}%
\pgfpathlineto{\pgfqpoint{3.239358in}{2.458253in}}%
\pgfpathlineto{\pgfqpoint{3.256451in}{2.419298in}}%
\pgfpathlineto{\pgfqpoint{3.269269in}{2.380342in}}%
\pgfpathlineto{\pgfqpoint{3.273902in}{2.360865in}}%
\pgfpathlineto{\pgfqpoint{3.277260in}{2.341387in}}%
\pgfpathlineto{\pgfqpoint{3.279416in}{2.321909in}}%
\pgfpathlineto{\pgfqpoint{3.280327in}{2.302432in}}%
\pgfpathlineto{\pgfqpoint{3.279957in}{2.282954in}}%
\pgfpathlineto{\pgfqpoint{3.278274in}{2.263477in}}%
\pgfpathlineto{\pgfqpoint{3.272332in}{2.230749in}}%
\pgfpathlineto{\pgfqpoint{3.270871in}{2.224521in}}%
\pgfpathlineto{\pgfqpoint{3.258212in}{2.185566in}}%
\pgfpathlineto{\pgfqpoint{3.233792in}{2.127133in}}%
\pgfpathlineto{\pgfqpoint{3.229192in}{2.107655in}}%
\pgfpathlineto{\pgfqpoint{3.239076in}{2.091236in}}%
\pgfpathlineto{\pgfqpoint{3.272332in}{2.090662in}}%
\pgfpathlineto{\pgfqpoint{3.338844in}{2.101937in}}%
\pgfpathlineto{\pgfqpoint{3.405357in}{2.113942in}}%
\pgfpathlineto{\pgfqpoint{3.471869in}{2.123763in}}%
\pgfpathlineto{\pgfqpoint{3.538381in}{2.130643in}}%
\pgfpathlineto{\pgfqpoint{3.604894in}{2.134420in}}%
\pgfpathlineto{\pgfqpoint{3.671406in}{2.135276in}}%
\pgfpathlineto{\pgfqpoint{3.737918in}{2.133366in}}%
\pgfpathlineto{\pgfqpoint{3.804431in}{2.128868in}}%
\pgfpathlineto{\pgfqpoint{3.870943in}{2.121776in}}%
\pgfpathlineto{\pgfqpoint{3.937455in}{2.112302in}}%
\pgfpathlineto{\pgfqpoint{4.003968in}{2.100428in}}%
\pgfpathlineto{\pgfqpoint{4.070480in}{2.086259in}}%
\pgfpathlineto{\pgfqpoint{4.140608in}{2.068700in}}%
\pgfpathlineto{\pgfqpoint{4.208243in}{2.049222in}}%
\pgfpathlineto{\pgfqpoint{4.270017in}{2.029230in}}%
\pgfpathlineto{\pgfqpoint{4.336530in}{2.005114in}}%
\pgfpathlineto{\pgfqpoint{4.403042in}{1.978302in}}%
\pgfpathlineto{\pgfqpoint{4.469554in}{1.948585in}}%
\pgfpathlineto{\pgfqpoint{4.541260in}{1.912879in}}%
\pgfpathlineto{\pgfqpoint{4.602579in}{1.878963in}}%
\pgfpathlineto{\pgfqpoint{4.643430in}{1.854446in}}%
\pgfpathlineto{\pgfqpoint{4.703111in}{1.815491in}}%
\pgfpathlineto{\pgfqpoint{4.756739in}{1.776536in}}%
\pgfpathlineto{\pgfqpoint{4.805363in}{1.737580in}}%
\pgfpathlineto{\pgfqpoint{4.849018in}{1.698625in}}%
\pgfpathlineto{\pgfqpoint{4.888298in}{1.659670in}}%
\pgfpathlineto{\pgfqpoint{4.923349in}{1.620714in}}%
\pgfpathlineto{\pgfqpoint{4.954311in}{1.581759in}}%
\pgfpathlineto{\pgfqpoint{4.981321in}{1.542804in}}%
\pgfpathlineto{\pgfqpoint{5.004506in}{1.503849in}}%
\pgfpathlineto{\pgfqpoint{5.023617in}{1.464893in}}%
\pgfpathlineto{\pgfqpoint{5.038935in}{1.425938in}}%
\pgfpathlineto{\pgfqpoint{5.050074in}{1.386983in}}%
\pgfpathlineto{\pgfqpoint{5.057136in}{1.348027in}}%
\pgfpathlineto{\pgfqpoint{5.059045in}{1.328550in}}%
\pgfpathlineto{\pgfqpoint{5.059820in}{1.309072in}}%
\pgfpathlineto{\pgfqpoint{5.059418in}{1.289595in}}%
\pgfpathlineto{\pgfqpoint{5.057797in}{1.270117in}}%
\pgfpathlineto{\pgfqpoint{5.054908in}{1.250639in}}%
\pgfpathlineto{\pgfqpoint{5.050703in}{1.231162in}}%
\pgfpathlineto{\pgfqpoint{5.045130in}{1.211684in}}%
\pgfpathlineto{\pgfqpoint{5.034909in}{1.184728in}}%
\pgfpathlineto{\pgfqpoint{5.029446in}{1.172729in}}%
\pgfpathlineto{\pgfqpoint{5.019012in}{1.153251in}}%
\pgfpathlineto{\pgfqpoint{5.001653in}{1.126349in}}%
\pgfpathlineto{\pgfqpoint{4.992632in}{1.114296in}}%
\pgfpathlineto{\pgfqpoint{4.968397in}{1.086511in}}%
\pgfpathlineto{\pgfqpoint{4.957220in}{1.075340in}}%
\pgfpathlineto{\pgfqpoint{4.935141in}{1.055479in}}%
\pgfpathlineto{\pgfqpoint{4.901885in}{1.030398in}}%
\pgfpathlineto{\pgfqpoint{4.868629in}{1.009237in}}%
\pgfpathlineto{\pgfqpoint{4.835372in}{0.991147in}}%
\pgfpathlineto{\pgfqpoint{4.802116in}{0.975511in}}%
\pgfpathlineto{\pgfqpoint{4.759180in}{0.958475in}}%
\pgfpathlineto{\pgfqpoint{4.735604in}{0.950352in}}%
\pgfpathlineto{\pgfqpoint{4.698333in}{0.938997in}}%
\pgfpathlineto{\pgfqpoint{4.635835in}{0.923833in}}%
\pgfpathlineto{\pgfqpoint{4.602579in}{0.917398in}}%
\pgfpathlineto{\pgfqpoint{4.536067in}{0.907696in}}%
\pgfpathlineto{\pgfqpoint{4.469554in}{0.901509in}}%
\pgfpathlineto{\pgfqpoint{4.403042in}{0.898621in}}%
\pgfpathlineto{\pgfqpoint{4.336530in}{0.898682in}}%
\pgfpathlineto{\pgfqpoint{4.270017in}{0.901485in}}%
\pgfpathlineto{\pgfqpoint{4.203505in}{0.906928in}}%
\pgfpathlineto{\pgfqpoint{4.136993in}{0.914780in}}%
\pgfpathlineto{\pgfqpoint{4.070480in}{0.925062in}}%
\pgfpathlineto{\pgfqpoint{3.997734in}{0.938997in}}%
\pgfpathlineto{\pgfqpoint{3.937455in}{0.952712in}}%
\pgfpathlineto{\pgfqpoint{3.870943in}{0.970104in}}%
\pgfpathlineto{\pgfqpoint{3.804431in}{0.989911in}}%
\pgfpathlineto{\pgfqpoint{3.737918in}{1.012201in}}%
\pgfpathlineto{\pgfqpoint{3.671406in}{1.037024in}}%
\pgfpathlineto{\pgfqpoint{3.578964in}{1.075340in}}%
\pgfpathlineto{\pgfqpoint{3.538381in}{1.092462in}}%
\pgfpathlineto{\pgfqpoint{3.505125in}{1.104668in}}%
\pgfpathlineto{\pgfqpoint{3.471869in}{1.111724in}}%
\pgfpathlineto{\pgfqpoint{3.438613in}{1.105884in}}%
\pgfpathlineto{\pgfqpoint{3.422828in}{1.094818in}}%
\pgfpathlineto{\pgfqpoint{3.400180in}{1.075340in}}%
\pgfpathlineto{\pgfqpoint{3.359198in}{1.036385in}}%
\pgfpathlineto{\pgfqpoint{3.335917in}{1.016908in}}%
\pgfpathlineto{\pgfqpoint{3.305588in}{0.994816in}}%
\pgfpathlineto{\pgfqpoint{3.272332in}{0.973983in}}%
\pgfpathlineto{\pgfqpoint{3.239076in}{0.955949in}}%
\pgfpathlineto{\pgfqpoint{3.202920in}{0.938997in}}%
\pgfpathlineto{\pgfqpoint{3.153358in}{0.919519in}}%
\pgfpathlineto{\pgfqpoint{3.106051in}{0.904102in}}%
\pgfpathlineto{\pgfqpoint{3.072795in}{0.894942in}}%
\pgfpathlineto{\pgfqpoint{3.006282in}{0.880045in}}%
\pgfpathlineto{\pgfqpoint{2.939770in}{0.869360in}}%
\pgfpathlineto{\pgfqpoint{2.873258in}{0.862058in}}%
\pgfpathlineto{\pgfqpoint{2.806745in}{0.858003in}}%
\pgfpathlineto{\pgfqpoint{2.740233in}{0.856795in}}%
\pgfpathlineto{\pgfqpoint{2.673720in}{0.858212in}}%
\pgfpathlineto{\pgfqpoint{2.640464in}{0.859851in}}%
\pgfpathlineto{\pgfqpoint{2.640464in}{0.859851in}}%
\pgfusepath{stroke}%
\end{pgfscope}%
\begin{pgfscope}%
\pgfpathrectangle{\pgfqpoint{0.711606in}{0.549444in}}{\pgfqpoint{4.955171in}{2.902168in}}%
\pgfusepath{clip}%
\pgfsetbuttcap%
\pgfsetroundjoin%
\pgfsetlinewidth{1.003750pt}%
\definecolor{currentstroke}{rgb}{0.051644,0.032474,0.159254}%
\pgfsetstrokecolor{currentstroke}%
\pgfsetdash{}{0pt}%
\pgfpathmoveto{\pgfqpoint{3.006282in}{1.945092in}}%
\pgfpathlineto{\pgfqpoint{3.017044in}{1.951834in}}%
\pgfpathlineto{\pgfqpoint{3.006282in}{1.962164in}}%
\pgfpathlineto{\pgfqpoint{2.988721in}{1.951834in}}%
\pgfpathlineto{\pgfqpoint{3.006282in}{1.945092in}}%
\pgfpathclose%
\pgfusepath{stroke}%
\end{pgfscope}%
\begin{pgfscope}%
\pgfpathrectangle{\pgfqpoint{0.711606in}{0.549444in}}{\pgfqpoint{4.955171in}{2.902168in}}%
\pgfusepath{clip}%
\pgfsetbuttcap%
\pgfsetroundjoin%
\pgfsetlinewidth{1.003750pt}%
\definecolor{currentstroke}{rgb}{0.066331,0.038504,0.186962}%
\pgfsetstrokecolor{currentstroke}%
\pgfsetdash{}{0pt}%
\pgfpathmoveto{\pgfqpoint{2.573952in}{0.800590in}}%
\pgfpathlineto{\pgfqpoint{2.555508in}{0.802653in}}%
\pgfpathlineto{\pgfqpoint{2.540696in}{0.804342in}}%
\pgfpathlineto{\pgfqpoint{2.507440in}{0.808653in}}%
\pgfpathlineto{\pgfqpoint{2.474183in}{0.813447in}}%
\pgfpathlineto{\pgfqpoint{2.440927in}{0.818710in}}%
\pgfpathlineto{\pgfqpoint{2.421111in}{0.822131in}}%
\pgfpathlineto{\pgfqpoint{2.407671in}{0.824495in}}%
\pgfpathlineto{\pgfqpoint{2.374415in}{0.830832in}}%
\pgfpathlineto{\pgfqpoint{2.341159in}{0.837607in}}%
\pgfpathlineto{\pgfqpoint{2.322759in}{0.841609in}}%
\pgfpathlineto{\pgfqpoint{2.307902in}{0.844901in}}%
\pgfpathlineto{\pgfqpoint{2.274646in}{0.852732in}}%
\pgfpathlineto{\pgfqpoint{2.241390in}{0.860973in}}%
\pgfpathlineto{\pgfqpoint{2.240957in}{0.861086in}}%
\pgfpathlineto{\pgfqpoint{2.208134in}{0.869856in}}%
\pgfpathlineto{\pgfqpoint{2.174878in}{0.879134in}}%
\pgfpathlineto{\pgfqpoint{2.169996in}{0.880564in}}%
\pgfpathlineto{\pgfqpoint{2.141622in}{0.889032in}}%
\pgfpathlineto{\pgfqpoint{2.108365in}{0.899347in}}%
\pgfpathlineto{\pgfqpoint{2.106221in}{0.900042in}}%
\pgfpathlineto{\pgfqpoint{2.075109in}{0.910318in}}%
\pgfpathlineto{\pgfqpoint{2.048195in}{0.919519in}}%
\pgfpathlineto{\pgfqpoint{2.041853in}{0.921730in}}%
\pgfpathlineto{\pgfqpoint{2.008597in}{0.933770in}}%
\pgfpathlineto{\pgfqpoint{1.994636in}{0.938997in}}%
\pgfpathlineto{\pgfqpoint{1.975341in}{0.946364in}}%
\pgfpathlineto{\pgfqpoint{1.944584in}{0.958475in}}%
\pgfpathlineto{\pgfqpoint{1.942084in}{0.959479in}}%
\pgfpathlineto{\pgfqpoint{1.908828in}{0.973292in}}%
\pgfpathlineto{\pgfqpoint{1.897933in}{0.977952in}}%
\pgfpathlineto{\pgfqpoint{1.875572in}{0.987711in}}%
\pgfpathlineto{\pgfqpoint{1.853914in}{0.997430in}}%
\pgfpathlineto{\pgfqpoint{1.842316in}{1.002742in}}%
\pgfpathlineto{\pgfqpoint{1.812216in}{1.016908in}}%
\pgfpathlineto{\pgfqpoint{1.809060in}{1.018424in}}%
\pgfpathlineto{\pgfqpoint{1.775804in}{1.034850in}}%
\pgfpathlineto{\pgfqpoint{1.772776in}{1.036385in}}%
\pgfpathlineto{\pgfqpoint{1.742547in}{1.052043in}}%
\pgfpathlineto{\pgfqpoint{1.735352in}{1.055863in}}%
\pgfpathlineto{\pgfqpoint{1.709291in}{1.070002in}}%
\pgfpathlineto{\pgfqpoint{1.699685in}{1.075340in}}%
\pgfpathlineto{\pgfqpoint{1.676035in}{1.088778in}}%
\pgfpathlineto{\pgfqpoint{1.665653in}{1.094818in}}%
\pgfpathlineto{\pgfqpoint{1.642779in}{1.108429in}}%
\pgfpathlineto{\pgfqpoint{1.633144in}{1.114296in}}%
\pgfpathlineto{\pgfqpoint{1.609523in}{1.129015in}}%
\pgfpathlineto{\pgfqpoint{1.602057in}{1.133773in}}%
\pgfpathlineto{\pgfqpoint{1.576266in}{1.150604in}}%
\pgfpathlineto{\pgfqpoint{1.572298in}{1.153251in}}%
\pgfpathlineto{\pgfqpoint{1.543810in}{1.172729in}}%
\pgfpathlineto{\pgfqpoint{1.543010in}{1.173290in}}%
\pgfpathlineto{\pgfqpoint{1.516654in}{1.192206in}}%
\pgfpathlineto{\pgfqpoint{1.509754in}{1.197288in}}%
\pgfpathlineto{\pgfqpoint{1.490619in}{1.211684in}}%
\pgfpathlineto{\pgfqpoint{1.476498in}{1.222594in}}%
\pgfpathlineto{\pgfqpoint{1.465637in}{1.231162in}}%
\pgfpathlineto{\pgfqpoint{1.443242in}{1.249316in}}%
\pgfpathlineto{\pgfqpoint{1.441643in}{1.250639in}}%
\pgfpathlineto{\pgfqpoint{1.418843in}{1.270117in}}%
\pgfpathlineto{\pgfqpoint{1.409986in}{1.277914in}}%
\pgfpathlineto{\pgfqpoint{1.396983in}{1.289595in}}%
\pgfpathlineto{\pgfqpoint{1.376729in}{1.308346in}}%
\pgfpathlineto{\pgfqpoint{1.375960in}{1.309072in}}%
\pgfpathlineto{\pgfqpoint{1.356093in}{1.328550in}}%
\pgfpathlineto{\pgfqpoint{1.343473in}{1.341337in}}%
\pgfpathlineto{\pgfqpoint{1.336999in}{1.348027in}}%
\pgfpathlineto{\pgfqpoint{1.318862in}{1.367505in}}%
\pgfpathlineto{\pgfqpoint{1.310217in}{1.377160in}}%
\pgfpathlineto{\pgfqpoint{1.301591in}{1.386983in}}%
\pgfpathlineto{\pgfqpoint{1.285187in}{1.406460in}}%
\pgfpathlineto{\pgfqpoint{1.276961in}{1.416670in}}%
\pgfpathlineto{\pgfqpoint{1.269636in}{1.425938in}}%
\pgfpathlineto{\pgfqpoint{1.254966in}{1.445416in}}%
\pgfpathlineto{\pgfqpoint{1.243705in}{1.461107in}}%
\pgfpathlineto{\pgfqpoint{1.241039in}{1.464893in}}%
\pgfpathlineto{\pgfqpoint{1.228103in}{1.484371in}}%
\pgfpathlineto{\pgfqpoint{1.215859in}{1.503849in}}%
\pgfpathlineto{\pgfqpoint{1.210448in}{1.513036in}}%
\pgfpathlineto{\pgfqpoint{1.204503in}{1.523326in}}%
\pgfpathlineto{\pgfqpoint{1.194020in}{1.542804in}}%
\pgfpathlineto{\pgfqpoint{1.184278in}{1.562281in}}%
\pgfpathlineto{\pgfqpoint{1.177192in}{1.577685in}}%
\pgfpathlineto{\pgfqpoint{1.175353in}{1.581759in}}%
\pgfpathlineto{\pgfqpoint{1.167397in}{1.601237in}}%
\pgfpathlineto{\pgfqpoint{1.160232in}{1.620714in}}%
\pgfpathlineto{\pgfqpoint{1.153881in}{1.640192in}}%
\pgfpathlineto{\pgfqpoint{1.148369in}{1.659670in}}%
\pgfpathlineto{\pgfqpoint{1.143936in}{1.678256in}}%
\pgfpathlineto{\pgfqpoint{1.143728in}{1.679147in}}%
\pgfpathlineto{\pgfqpoint{1.140082in}{1.698625in}}%
\pgfpathlineto{\pgfqpoint{1.137329in}{1.718103in}}%
\pgfpathlineto{\pgfqpoint{1.135497in}{1.737580in}}%
\pgfpathlineto{\pgfqpoint{1.134614in}{1.757058in}}%
\pgfpathlineto{\pgfqpoint{1.134709in}{1.776536in}}%
\pgfpathlineto{\pgfqpoint{1.135813in}{1.796013in}}%
\pgfpathlineto{\pgfqpoint{1.137957in}{1.815491in}}%
\pgfpathlineto{\pgfqpoint{1.141176in}{1.834968in}}%
\pgfpathlineto{\pgfqpoint{1.143936in}{1.847438in}}%
\pgfpathlineto{\pgfqpoint{1.145556in}{1.854446in}}%
\pgfpathlineto{\pgfqpoint{1.151211in}{1.873924in}}%
\pgfpathlineto{\pgfqpoint{1.158094in}{1.893401in}}%
\pgfpathlineto{\pgfqpoint{1.166246in}{1.912879in}}%
\pgfpathlineto{\pgfqpoint{1.175712in}{1.932357in}}%
\pgfpathlineto{\pgfqpoint{1.177192in}{1.935047in}}%
\pgfpathlineto{\pgfqpoint{1.186876in}{1.951834in}}%
\pgfpathlineto{\pgfqpoint{1.199563in}{1.971312in}}%
\pgfpathlineto{\pgfqpoint{1.210448in}{1.986273in}}%
\pgfpathlineto{\pgfqpoint{1.213904in}{1.990790in}}%
\pgfpathlineto{\pgfqpoint{1.230318in}{2.010267in}}%
\pgfpathlineto{\pgfqpoint{1.243705in}{2.024691in}}%
\pgfpathlineto{\pgfqpoint{1.248654in}{2.029745in}}%
\pgfpathlineto{\pgfqpoint{1.269425in}{2.049222in}}%
\pgfpathlineto{\pgfqpoint{1.276961in}{2.055737in}}%
\pgfpathlineto{\pgfqpoint{1.292842in}{2.068700in}}%
\pgfpathlineto{\pgfqpoint{1.310217in}{2.081818in}}%
\pgfpathlineto{\pgfqpoint{1.319171in}{2.088178in}}%
\pgfpathlineto{\pgfqpoint{1.343473in}{2.104236in}}%
\pgfpathlineto{\pgfqpoint{1.348994in}{2.107655in}}%
\pgfpathlineto{\pgfqpoint{1.376729in}{2.123716in}}%
\pgfpathlineto{\pgfqpoint{1.383054in}{2.127133in}}%
\pgfpathlineto{\pgfqpoint{1.409986in}{2.140816in}}%
\pgfpathlineto{\pgfqpoint{1.422258in}{2.146611in}}%
\pgfpathlineto{\pgfqpoint{1.443242in}{2.156012in}}%
\pgfpathlineto{\pgfqpoint{1.467448in}{2.166088in}}%
\pgfpathlineto{\pgfqpoint{1.476498in}{2.169759in}}%
\pgfpathlineto{\pgfqpoint{1.509754in}{2.182535in}}%
\pgfpathlineto{\pgfqpoint{1.517987in}{2.185566in}}%
\pgfpathlineto{\pgfqpoint{1.543010in}{2.196155in}}%
\pgfpathlineto{\pgfqpoint{1.562253in}{2.205044in}}%
\pgfpathlineto{\pgfqpoint{1.576266in}{2.219603in}}%
\pgfpathlineto{\pgfqpoint{1.579969in}{2.224521in}}%
\pgfpathlineto{\pgfqpoint{1.576266in}{2.238048in}}%
\pgfpathlineto{\pgfqpoint{1.574877in}{2.243999in}}%
\pgfpathlineto{\pgfqpoint{1.559598in}{2.263477in}}%
\pgfpathlineto{\pgfqpoint{1.543010in}{2.280361in}}%
\pgfpathlineto{\pgfqpoint{1.540606in}{2.282954in}}%
\pgfpathlineto{\pgfqpoint{1.520878in}{2.302432in}}%
\pgfpathlineto{\pgfqpoint{1.509754in}{2.313377in}}%
\pgfpathlineto{\pgfqpoint{1.501329in}{2.321909in}}%
\pgfpathlineto{\pgfqpoint{1.482504in}{2.341387in}}%
\pgfpathlineto{\pgfqpoint{1.476498in}{2.347839in}}%
\pgfpathlineto{\pgfqpoint{1.464656in}{2.360865in}}%
\pgfpathlineto{\pgfqpoint{1.447668in}{2.380342in}}%
\pgfpathlineto{\pgfqpoint{1.443242in}{2.385685in}}%
\pgfpathlineto{\pgfqpoint{1.431794in}{2.399820in}}%
\pgfpathlineto{\pgfqpoint{1.416807in}{2.419298in}}%
\pgfpathlineto{\pgfqpoint{1.409986in}{2.428701in}}%
\pgfpathlineto{\pgfqpoint{1.402839in}{2.438775in}}%
\pgfpathlineto{\pgfqpoint{1.389880in}{2.458253in}}%
\pgfpathlineto{\pgfqpoint{1.377735in}{2.477731in}}%
\pgfpathlineto{\pgfqpoint{1.376729in}{2.479481in}}%
\pgfpathlineto{\pgfqpoint{1.366774in}{2.497208in}}%
\pgfpathlineto{\pgfqpoint{1.356697in}{2.516686in}}%
\pgfpathlineto{\pgfqpoint{1.347500in}{2.536163in}}%
\pgfpathlineto{\pgfqpoint{1.343473in}{2.545684in}}%
\pgfpathlineto{\pgfqpoint{1.339355in}{2.555641in}}%
\pgfpathlineto{\pgfqpoint{1.332261in}{2.575119in}}%
\pgfpathlineto{\pgfqpoint{1.326117in}{2.594596in}}%
\pgfpathlineto{\pgfqpoint{1.320954in}{2.614074in}}%
\pgfpathlineto{\pgfqpoint{1.316809in}{2.633552in}}%
\pgfpathlineto{\pgfqpoint{1.313717in}{2.653029in}}%
\pgfpathlineto{\pgfqpoint{1.311717in}{2.672507in}}%
\pgfpathlineto{\pgfqpoint{1.310848in}{2.691985in}}%
\pgfpathlineto{\pgfqpoint{1.311154in}{2.711462in}}%
\pgfpathlineto{\pgfqpoint{1.312678in}{2.730940in}}%
\pgfpathlineto{\pgfqpoint{1.315466in}{2.750418in}}%
\pgfpathlineto{\pgfqpoint{1.319568in}{2.769895in}}%
\pgfpathlineto{\pgfqpoint{1.325035in}{2.789373in}}%
\pgfpathlineto{\pgfqpoint{1.331920in}{2.808850in}}%
\pgfpathlineto{\pgfqpoint{1.340282in}{2.828328in}}%
\pgfpathlineto{\pgfqpoint{1.343473in}{2.834666in}}%
\pgfpathlineto{\pgfqpoint{1.350460in}{2.847806in}}%
\pgfpathlineto{\pgfqpoint{1.362450in}{2.867283in}}%
\pgfpathlineto{\pgfqpoint{1.376192in}{2.886761in}}%
\pgfpathlineto{\pgfqpoint{1.376729in}{2.887443in}}%
\pgfpathlineto{\pgfqpoint{1.392452in}{2.906239in}}%
\pgfpathlineto{\pgfqpoint{1.409986in}{2.924905in}}%
\pgfpathlineto{\pgfqpoint{1.410797in}{2.925716in}}%
\pgfpathlineto{\pgfqpoint{1.432239in}{2.945194in}}%
\pgfpathlineto{\pgfqpoint{1.443242in}{2.954276in}}%
\pgfpathlineto{\pgfqpoint{1.456729in}{2.964672in}}%
\pgfpathlineto{\pgfqpoint{1.476498in}{2.978625in}}%
\pgfpathlineto{\pgfqpoint{1.484922in}{2.984149in}}%
\pgfpathlineto{\pgfqpoint{1.509754in}{2.999168in}}%
\pgfpathlineto{\pgfqpoint{1.517736in}{3.003627in}}%
\pgfpathlineto{\pgfqpoint{1.543010in}{3.016728in}}%
\pgfpathlineto{\pgfqpoint{1.556421in}{3.023104in}}%
\pgfpathlineto{\pgfqpoint{1.576266in}{3.031906in}}%
\pgfpathlineto{\pgfqpoint{1.602723in}{3.042582in}}%
\pgfpathlineto{\pgfqpoint{1.609523in}{3.045153in}}%
\pgfpathlineto{\pgfqpoint{1.642779in}{3.056569in}}%
\pgfpathlineto{\pgfqpoint{1.660626in}{3.062060in}}%
\pgfpathlineto{\pgfqpoint{1.676035in}{3.066524in}}%
\pgfpathlineto{\pgfqpoint{1.709291in}{3.075099in}}%
\pgfpathlineto{\pgfqpoint{1.737832in}{3.081537in}}%
\pgfpathlineto{\pgfqpoint{1.742547in}{3.082543in}}%
\pgfpathlineto{\pgfqpoint{1.775804in}{3.088733in}}%
\pgfpathlineto{\pgfqpoint{1.809060in}{3.093973in}}%
\pgfpathlineto{\pgfqpoint{1.842316in}{3.098299in}}%
\pgfpathlineto{\pgfqpoint{1.868419in}{3.101015in}}%
\pgfpathlineto{\pgfqpoint{1.875572in}{3.101722in}}%
\pgfpathlineto{\pgfqpoint{1.908828in}{3.104236in}}%
\pgfpathlineto{\pgfqpoint{1.942084in}{3.105971in}}%
\pgfpathlineto{\pgfqpoint{1.975341in}{3.106957in}}%
\pgfpathlineto{\pgfqpoint{2.008597in}{3.107221in}}%
\pgfpathlineto{\pgfqpoint{2.041853in}{3.106788in}}%
\pgfpathlineto{\pgfqpoint{2.075109in}{3.105683in}}%
\pgfpathlineto{\pgfqpoint{2.108365in}{3.103930in}}%
\pgfpathlineto{\pgfqpoint{2.141622in}{3.101552in}}%
\pgfpathlineto{\pgfqpoint{2.147550in}{3.101015in}}%
\pgfpathlineto{\pgfqpoint{2.174878in}{3.098485in}}%
\pgfpathlineto{\pgfqpoint{2.208134in}{3.094799in}}%
\pgfpathlineto{\pgfqpoint{2.241390in}{3.090532in}}%
\pgfpathlineto{\pgfqpoint{2.274646in}{3.085704in}}%
\pgfpathlineto{\pgfqpoint{2.300389in}{3.081537in}}%
\pgfpathlineto{\pgfqpoint{2.307902in}{3.080294in}}%
\pgfpathlineto{\pgfqpoint{2.341159in}{3.074205in}}%
\pgfpathlineto{\pgfqpoint{2.374415in}{3.067598in}}%
\pgfpathlineto{\pgfqpoint{2.400259in}{3.062060in}}%
\pgfpathlineto{\pgfqpoint{2.407671in}{3.060436in}}%
\pgfpathlineto{\pgfqpoint{2.440927in}{3.052591in}}%
\pgfpathlineto{\pgfqpoint{2.474183in}{3.044267in}}%
\pgfpathlineto{\pgfqpoint{2.480505in}{3.042582in}}%
\pgfpathlineto{\pgfqpoint{2.507440in}{3.035243in}}%
\pgfpathlineto{\pgfqpoint{2.540696in}{3.025706in}}%
\pgfpathlineto{\pgfqpoint{2.549296in}{3.023104in}}%
\pgfpathlineto{\pgfqpoint{2.573952in}{3.015479in}}%
\pgfpathlineto{\pgfqpoint{2.607208in}{3.004732in}}%
\pgfpathlineto{\pgfqpoint{2.610467in}{3.003627in}}%
\pgfpathlineto{\pgfqpoint{2.640464in}{2.993223in}}%
\pgfpathlineto{\pgfqpoint{2.665637in}{2.984149in}}%
\pgfpathlineto{\pgfqpoint{2.673720in}{2.981168in}}%
\pgfpathlineto{\pgfqpoint{2.706977in}{2.968393in}}%
\pgfpathlineto{\pgfqpoint{2.716305in}{2.964672in}}%
\pgfpathlineto{\pgfqpoint{2.740233in}{2.954902in}}%
\pgfpathlineto{\pgfqpoint{2.763213in}{2.945194in}}%
\pgfpathlineto{\pgfqpoint{2.773489in}{2.940749in}}%
\pgfpathlineto{\pgfqpoint{2.806745in}{2.925882in}}%
\pgfpathlineto{\pgfqpoint{2.807104in}{2.925716in}}%
\pgfpathlineto{\pgfqpoint{2.840001in}{2.910117in}}%
\pgfpathlineto{\pgfqpoint{2.847936in}{2.906239in}}%
\pgfpathlineto{\pgfqpoint{2.873258in}{2.893555in}}%
\pgfpathlineto{\pgfqpoint{2.886427in}{2.886761in}}%
\pgfpathlineto{\pgfqpoint{2.906514in}{2.876135in}}%
\pgfpathlineto{\pgfqpoint{2.922775in}{2.867283in}}%
\pgfpathlineto{\pgfqpoint{2.939770in}{2.857791in}}%
\pgfpathlineto{\pgfqpoint{2.957156in}{2.847806in}}%
\pgfpathlineto{\pgfqpoint{2.973026in}{2.838447in}}%
\pgfpathlineto{\pgfqpoint{2.989728in}{2.828328in}}%
\pgfpathlineto{\pgfqpoint{3.006282in}{2.818022in}}%
\pgfpathlineto{\pgfqpoint{3.020632in}{2.808850in}}%
\pgfpathlineto{\pgfqpoint{3.039539in}{2.796425in}}%
\pgfpathlineto{\pgfqpoint{3.049996in}{2.789373in}}%
\pgfpathlineto{\pgfqpoint{3.072795in}{2.773551in}}%
\pgfpathlineto{\pgfqpoint{3.077933in}{2.769895in}}%
\pgfpathlineto{\pgfqpoint{3.104490in}{2.750418in}}%
\pgfpathlineto{\pgfqpoint{3.106051in}{2.749234in}}%
\pgfpathlineto{\pgfqpoint{3.129585in}{2.730940in}}%
\pgfpathlineto{\pgfqpoint{3.139307in}{2.723138in}}%
\pgfpathlineto{\pgfqpoint{3.153509in}{2.711462in}}%
\pgfpathlineto{\pgfqpoint{3.172563in}{2.695272in}}%
\pgfpathlineto{\pgfqpoint{3.176342in}{2.691985in}}%
\pgfpathlineto{\pgfqpoint{3.197887in}{2.672507in}}%
\pgfpathlineto{\pgfqpoint{3.205819in}{2.665057in}}%
\pgfpathlineto{\pgfqpoint{3.218332in}{2.653029in}}%
\pgfpathlineto{\pgfqpoint{3.237838in}{2.633552in}}%
\pgfpathlineto{\pgfqpoint{3.239076in}{2.632256in}}%
\pgfpathlineto{\pgfqpoint{3.256050in}{2.614074in}}%
\pgfpathlineto{\pgfqpoint{3.272332in}{2.595900in}}%
\pgfpathlineto{\pgfqpoint{3.273474in}{2.594596in}}%
\pgfpathlineto{\pgfqpoint{3.289637in}{2.575119in}}%
\pgfpathlineto{\pgfqpoint{3.305057in}{2.555641in}}%
\pgfpathlineto{\pgfqpoint{3.305588in}{2.554927in}}%
\pgfpathlineto{\pgfqpoint{3.319231in}{2.536163in}}%
\pgfpathlineto{\pgfqpoint{3.332605in}{2.516686in}}%
\pgfpathlineto{\pgfqpoint{3.338844in}{2.506959in}}%
\pgfpathlineto{\pgfqpoint{3.344961in}{2.497208in}}%
\pgfpathlineto{\pgfqpoint{3.356293in}{2.477731in}}%
\pgfpathlineto{\pgfqpoint{3.366779in}{2.458253in}}%
\pgfpathlineto{\pgfqpoint{3.372100in}{2.447412in}}%
\pgfpathlineto{\pgfqpoint{3.376248in}{2.438775in}}%
\pgfpathlineto{\pgfqpoint{3.384662in}{2.419298in}}%
\pgfpathlineto{\pgfqpoint{3.392169in}{2.399820in}}%
\pgfpathlineto{\pgfqpoint{3.398745in}{2.380342in}}%
\pgfpathlineto{\pgfqpoint{3.404372in}{2.360865in}}%
\pgfpathlineto{\pgfqpoint{3.405357in}{2.356717in}}%
\pgfpathlineto{\pgfqpoint{3.408933in}{2.341387in}}%
\pgfpathlineto{\pgfqpoint{3.412593in}{2.321909in}}%
\pgfpathlineto{\pgfqpoint{3.415517in}{2.302432in}}%
\pgfpathlineto{\pgfqpoint{3.418026in}{2.282954in}}%
\pgfpathlineto{\pgfqpoint{3.420874in}{2.263477in}}%
\pgfpathlineto{\pgfqpoint{3.425888in}{2.243999in}}%
\pgfpathlineto{\pgfqpoint{3.438126in}{2.224521in}}%
\pgfpathlineto{\pgfqpoint{3.438613in}{2.224132in}}%
\pgfpathlineto{\pgfqpoint{3.471869in}{2.211411in}}%
\pgfpathlineto{\pgfqpoint{3.505125in}{2.208055in}}%
\pgfpathlineto{\pgfqpoint{3.538381in}{2.207377in}}%
\pgfpathlineto{\pgfqpoint{3.571637in}{2.207383in}}%
\pgfpathlineto{\pgfqpoint{3.604894in}{2.207355in}}%
\pgfpathlineto{\pgfqpoint{3.638150in}{2.206992in}}%
\pgfpathlineto{\pgfqpoint{3.671406in}{2.206161in}}%
\pgfpathlineto{\pgfqpoint{3.698761in}{2.205044in}}%
\pgfpathlineto{\pgfqpoint{3.704662in}{2.204799in}}%
\pgfpathlineto{\pgfqpoint{3.737918in}{2.202843in}}%
\pgfpathlineto{\pgfqpoint{3.771175in}{2.200329in}}%
\pgfpathlineto{\pgfqpoint{3.804431in}{2.197267in}}%
\pgfpathlineto{\pgfqpoint{3.837687in}{2.193669in}}%
\pgfpathlineto{\pgfqpoint{3.870943in}{2.189551in}}%
\pgfpathlineto{\pgfqpoint{3.899571in}{2.185566in}}%
\pgfpathlineto{\pgfqpoint{3.904199in}{2.184909in}}%
\pgfpathlineto{\pgfqpoint{3.937455in}{2.179641in}}%
\pgfpathlineto{\pgfqpoint{3.970712in}{2.173886in}}%
\pgfpathlineto{\pgfqpoint{4.003968in}{2.167659in}}%
\pgfpathlineto{\pgfqpoint{4.011722in}{2.166088in}}%
\pgfpathlineto{\pgfqpoint{4.037224in}{2.160820in}}%
\pgfpathlineto{\pgfqpoint{4.070480in}{2.153479in}}%
\pgfpathlineto{\pgfqpoint{4.099802in}{2.146611in}}%
\pgfpathlineto{\pgfqpoint{4.103736in}{2.145671in}}%
\pgfpathlineto{\pgfqpoint{4.136993in}{2.137216in}}%
\pgfpathlineto{\pgfqpoint{4.170249in}{2.128341in}}%
\pgfpathlineto{\pgfqpoint{4.174539in}{2.127133in}}%
\pgfpathlineto{\pgfqpoint{4.203505in}{2.118812in}}%
\pgfpathlineto{\pgfqpoint{4.236761in}{2.108845in}}%
\pgfpathlineto{\pgfqpoint{4.240545in}{2.107655in}}%
\pgfpathlineto{\pgfqpoint{4.270017in}{2.098205in}}%
\pgfpathlineto{\pgfqpoint{4.300144in}{2.088178in}}%
\pgfpathlineto{\pgfqpoint{4.303273in}{2.087115in}}%
\pgfpathlineto{\pgfqpoint{4.336530in}{2.075332in}}%
\pgfpathlineto{\pgfqpoint{4.354600in}{2.068700in}}%
\pgfpathlineto{\pgfqpoint{4.369786in}{2.063011in}}%
\pgfpathlineto{\pgfqpoint{4.403042in}{2.050125in}}%
\pgfpathlineto{\pgfqpoint{4.405288in}{2.049222in}}%
\pgfpathlineto{\pgfqpoint{4.436298in}{2.036501in}}%
\pgfpathlineto{\pgfqpoint{4.452274in}{2.029745in}}%
\pgfpathlineto{\pgfqpoint{4.469554in}{2.022280in}}%
\pgfpathlineto{\pgfqpoint{4.496550in}{2.010267in}}%
\pgfpathlineto{\pgfqpoint{4.502811in}{2.007420in}}%
\pgfpathlineto{\pgfqpoint{4.536067in}{1.991844in}}%
\pgfpathlineto{\pgfqpoint{4.538253in}{1.990790in}}%
\pgfpathlineto{\pgfqpoint{4.569323in}{1.975468in}}%
\pgfpathlineto{\pgfqpoint{4.577527in}{1.971312in}}%
\pgfpathlineto{\pgfqpoint{4.602579in}{1.958334in}}%
\pgfpathlineto{\pgfqpoint{4.614804in}{1.951834in}}%
\pgfpathlineto{\pgfqpoint{4.635835in}{1.940392in}}%
\pgfpathlineto{\pgfqpoint{4.650234in}{1.932357in}}%
\pgfpathlineto{\pgfqpoint{4.669091in}{1.921583in}}%
\pgfpathlineto{\pgfqpoint{4.683954in}{1.912879in}}%
\pgfpathlineto{\pgfqpoint{4.702348in}{1.901844in}}%
\pgfpathlineto{\pgfqpoint{4.716085in}{1.893401in}}%
\pgfpathlineto{\pgfqpoint{4.735604in}{1.881106in}}%
\pgfpathlineto{\pgfqpoint{4.746739in}{1.873924in}}%
\pgfpathlineto{\pgfqpoint{4.768860in}{1.859291in}}%
\pgfpathlineto{\pgfqpoint{4.776018in}{1.854446in}}%
\pgfpathlineto{\pgfqpoint{4.802116in}{1.836316in}}%
\pgfpathlineto{\pgfqpoint{4.804013in}{1.834968in}}%
\pgfpathlineto{\pgfqpoint{4.830654in}{1.815491in}}%
\pgfpathlineto{\pgfqpoint{4.835372in}{1.811941in}}%
\pgfpathlineto{\pgfqpoint{4.856078in}{1.796013in}}%
\pgfpathlineto{\pgfqpoint{4.868629in}{1.786079in}}%
\pgfpathlineto{\pgfqpoint{4.880426in}{1.776536in}}%
\pgfpathlineto{\pgfqpoint{4.901885in}{1.758660in}}%
\pgfpathlineto{\pgfqpoint{4.903767in}{1.757058in}}%
\pgfpathlineto{\pgfqpoint{4.925874in}{1.737580in}}%
\pgfpathlineto{\pgfqpoint{4.935141in}{1.729145in}}%
\pgfpathlineto{\pgfqpoint{4.947018in}{1.718103in}}%
\pgfpathlineto{\pgfqpoint{4.967282in}{1.698625in}}%
\pgfpathlineto{\pgfqpoint{4.968397in}{1.697509in}}%
\pgfpathlineto{\pgfqpoint{4.986357in}{1.679147in}}%
\pgfpathlineto{\pgfqpoint{5.001653in}{1.662943in}}%
\pgfpathlineto{\pgfqpoint{5.004680in}{1.659670in}}%
\pgfpathlineto{\pgfqpoint{5.021904in}{1.640192in}}%
\pgfpathlineto{\pgfqpoint{5.034909in}{1.624869in}}%
\pgfpathlineto{\pgfqpoint{5.038365in}{1.620714in}}%
\pgfpathlineto{\pgfqpoint{5.053771in}{1.601237in}}%
\pgfpathlineto{\pgfqpoint{5.068166in}{1.582184in}}%
\pgfpathlineto{\pgfqpoint{5.068481in}{1.581759in}}%
\pgfpathlineto{\pgfqpoint{5.082067in}{1.562281in}}%
\pgfpathlineto{\pgfqpoint{5.094937in}{1.542804in}}%
\pgfpathlineto{\pgfqpoint{5.101422in}{1.532340in}}%
\pgfpathlineto{\pgfqpoint{5.106897in}{1.523326in}}%
\pgfpathlineto{\pgfqpoint{5.117918in}{1.503849in}}%
\pgfpathlineto{\pgfqpoint{5.128168in}{1.484371in}}%
\pgfpathlineto{\pgfqpoint{5.134678in}{1.470917in}}%
\pgfpathlineto{\pgfqpoint{5.137535in}{1.464893in}}%
\pgfpathlineto{\pgfqpoint{5.145908in}{1.445416in}}%
\pgfpathlineto{\pgfqpoint{5.153456in}{1.425938in}}%
\pgfpathlineto{\pgfqpoint{5.160152in}{1.406460in}}%
\pgfpathlineto{\pgfqpoint{5.165970in}{1.386983in}}%
\pgfpathlineto{\pgfqpoint{5.167934in}{1.379143in}}%
\pgfpathlineto{\pgfqpoint{5.170791in}{1.367505in}}%
\pgfpathlineto{\pgfqpoint{5.174642in}{1.348027in}}%
\pgfpathlineto{\pgfqpoint{5.177555in}{1.328550in}}%
\pgfpathlineto{\pgfqpoint{5.179499in}{1.309072in}}%
\pgfpathlineto{\pgfqpoint{5.180442in}{1.289595in}}%
\pgfpathlineto{\pgfqpoint{5.180352in}{1.270117in}}%
\pgfpathlineto{\pgfqpoint{5.179195in}{1.250639in}}%
\pgfpathlineto{\pgfqpoint{5.176935in}{1.231162in}}%
\pgfpathlineto{\pgfqpoint{5.173534in}{1.211684in}}%
\pgfpathlineto{\pgfqpoint{5.168954in}{1.192206in}}%
\pgfpathlineto{\pgfqpoint{5.167934in}{1.188751in}}%
\pgfpathlineto{\pgfqpoint{5.162983in}{1.172729in}}%
\pgfpathlineto{\pgfqpoint{5.155664in}{1.153251in}}%
\pgfpathlineto{\pgfqpoint{5.146982in}{1.133773in}}%
\pgfpathlineto{\pgfqpoint{5.136888in}{1.114296in}}%
\pgfpathlineto{\pgfqpoint{5.134678in}{1.110535in}}%
\pgfpathlineto{\pgfqpoint{5.124966in}{1.094818in}}%
\pgfpathlineto{\pgfqpoint{5.111367in}{1.075340in}}%
\pgfpathlineto{\pgfqpoint{5.101422in}{1.062585in}}%
\pgfpathlineto{\pgfqpoint{5.095891in}{1.055863in}}%
\pgfpathlineto{\pgfqpoint{5.078194in}{1.036385in}}%
\pgfpathlineto{\pgfqpoint{5.068166in}{1.026345in}}%
\pgfpathlineto{\pgfqpoint{5.058179in}{1.016908in}}%
\pgfpathlineto{\pgfqpoint{5.035634in}{0.997430in}}%
\pgfpathlineto{\pgfqpoint{5.034909in}{0.996850in}}%
\pgfpathlineto{\pgfqpoint{5.009749in}{0.977952in}}%
\pgfpathlineto{\pgfqpoint{5.001653in}{0.972328in}}%
\pgfpathlineto{\pgfqpoint{4.980347in}{0.958475in}}%
\pgfpathlineto{\pgfqpoint{4.968397in}{0.951247in}}%
\pgfpathlineto{\pgfqpoint{4.946652in}{0.938997in}}%
\pgfpathlineto{\pgfqpoint{4.935141in}{0.932935in}}%
\pgfpathlineto{\pgfqpoint{4.907639in}{0.919519in}}%
\pgfpathlineto{\pgfqpoint{4.901885in}{0.916884in}}%
\pgfpathlineto{\pgfqpoint{4.868629in}{0.902821in}}%
\pgfpathlineto{\pgfqpoint{4.861469in}{0.900042in}}%
\pgfpathlineto{\pgfqpoint{4.835372in}{0.890486in}}%
\pgfpathlineto{\pgfqpoint{4.805517in}{0.880564in}}%
\pgfpathlineto{\pgfqpoint{4.802116in}{0.879495in}}%
\pgfpathlineto{\pgfqpoint{4.768860in}{0.869978in}}%
\pgfpathlineto{\pgfqpoint{4.735604in}{0.861492in}}%
\pgfpathlineto{\pgfqpoint{4.733828in}{0.861086in}}%
\pgfpathlineto{\pgfqpoint{4.702348in}{0.854256in}}%
\pgfpathlineto{\pgfqpoint{4.669091in}{0.847944in}}%
\pgfpathlineto{\pgfqpoint{4.635835in}{0.842507in}}%
\pgfpathlineto{\pgfqpoint{4.629399in}{0.841609in}}%
\pgfpathlineto{\pgfqpoint{4.602579in}{0.838039in}}%
\pgfpathlineto{\pgfqpoint{4.569323in}{0.834383in}}%
\pgfpathlineto{\pgfqpoint{4.536067in}{0.831479in}}%
\pgfpathlineto{\pgfqpoint{4.502811in}{0.829304in}}%
\pgfpathlineto{\pgfqpoint{4.469554in}{0.827830in}}%
\pgfpathlineto{\pgfqpoint{4.436298in}{0.827036in}}%
\pgfpathlineto{\pgfqpoint{4.403042in}{0.826899in}}%
\pgfpathlineto{\pgfqpoint{4.369786in}{0.827396in}}%
\pgfpathlineto{\pgfqpoint{4.336530in}{0.828508in}}%
\pgfpathlineto{\pgfqpoint{4.303273in}{0.830214in}}%
\pgfpathlineto{\pgfqpoint{4.270017in}{0.832496in}}%
\pgfpathlineto{\pgfqpoint{4.236761in}{0.835335in}}%
\pgfpathlineto{\pgfqpoint{4.203505in}{0.838715in}}%
\pgfpathlineto{\pgfqpoint{4.178913in}{0.841609in}}%
\pgfpathlineto{\pgfqpoint{4.170249in}{0.842649in}}%
\pgfpathlineto{\pgfqpoint{4.136993in}{0.847193in}}%
\pgfpathlineto{\pgfqpoint{4.103736in}{0.852242in}}%
\pgfpathlineto{\pgfqpoint{4.070480in}{0.857779in}}%
\pgfpathlineto{\pgfqpoint{4.052268in}{0.861086in}}%
\pgfpathlineto{\pgfqpoint{4.037224in}{0.863873in}}%
\pgfpathlineto{\pgfqpoint{4.003968in}{0.870537in}}%
\pgfpathlineto{\pgfqpoint{3.970712in}{0.877656in}}%
\pgfpathlineto{\pgfqpoint{3.958002in}{0.880564in}}%
\pgfpathlineto{\pgfqpoint{3.937455in}{0.885358in}}%
\pgfpathlineto{\pgfqpoint{3.904199in}{0.893585in}}%
\pgfpathlineto{\pgfqpoint{3.879437in}{0.900042in}}%
\pgfpathlineto{\pgfqpoint{3.870943in}{0.902301in}}%
\pgfpathlineto{\pgfqpoint{3.837687in}{0.911632in}}%
\pgfpathlineto{\pgfqpoint{3.810789in}{0.919519in}}%
\pgfpathlineto{\pgfqpoint{3.804431in}{0.921421in}}%
\pgfpathlineto{\pgfqpoint{3.771175in}{0.931846in}}%
\pgfpathlineto{\pgfqpoint{3.749210in}{0.938997in}}%
\pgfpathlineto{\pgfqpoint{3.737918in}{0.942750in}}%
\pgfpathlineto{\pgfqpoint{3.704662in}{0.954208in}}%
\pgfpathlineto{\pgfqpoint{3.692590in}{0.958475in}}%
\pgfpathlineto{\pgfqpoint{3.671406in}{0.966144in}}%
\pgfpathlineto{\pgfqpoint{3.639075in}{0.977952in}}%
\pgfpathlineto{\pgfqpoint{3.638150in}{0.978300in}}%
\pgfpathlineto{\pgfqpoint{3.604894in}{0.990429in}}%
\pgfpathlineto{\pgfqpoint{3.583017in}{0.997430in}}%
\pgfpathlineto{\pgfqpoint{3.571637in}{1.001328in}}%
\pgfpathlineto{\pgfqpoint{3.538381in}{1.008570in}}%
\pgfpathlineto{\pgfqpoint{3.505125in}{1.007054in}}%
\pgfpathlineto{\pgfqpoint{3.483256in}{0.997430in}}%
\pgfpathlineto{\pgfqpoint{3.471869in}{0.991596in}}%
\pgfpathlineto{\pgfqpoint{3.452956in}{0.977952in}}%
\pgfpathlineto{\pgfqpoint{3.438613in}{0.967616in}}%
\pgfpathlineto{\pgfqpoint{3.426298in}{0.958475in}}%
\pgfpathlineto{\pgfqpoint{3.405357in}{0.943802in}}%
\pgfpathlineto{\pgfqpoint{3.398201in}{0.938997in}}%
\pgfpathlineto{\pgfqpoint{3.372100in}{0.922607in}}%
\pgfpathlineto{\pgfqpoint{3.366870in}{0.919519in}}%
\pgfpathlineto{\pgfqpoint{3.338844in}{0.904023in}}%
\pgfpathlineto{\pgfqpoint{3.331121in}{0.900042in}}%
\pgfpathlineto{\pgfqpoint{3.305588in}{0.887668in}}%
\pgfpathlineto{\pgfqpoint{3.289767in}{0.880564in}}%
\pgfpathlineto{\pgfqpoint{3.272332in}{0.873179in}}%
\pgfpathlineto{\pgfqpoint{3.241317in}{0.861086in}}%
\pgfpathlineto{\pgfqpoint{3.239076in}{0.860259in}}%
\pgfpathlineto{\pgfqpoint{3.205819in}{0.848947in}}%
\pgfpathlineto{\pgfqpoint{3.182071in}{0.841609in}}%
\pgfpathlineto{\pgfqpoint{3.172563in}{0.838818in}}%
\pgfpathlineto{\pgfqpoint{3.139307in}{0.829951in}}%
\pgfpathlineto{\pgfqpoint{3.106477in}{0.822131in}}%
\pgfpathlineto{\pgfqpoint{3.106051in}{0.822034in}}%
\pgfpathlineto{\pgfqpoint{3.072795in}{0.815269in}}%
\pgfpathlineto{\pgfqpoint{3.039539in}{0.809346in}}%
\pgfpathlineto{\pgfqpoint{3.006282in}{0.804237in}}%
\pgfpathlineto{\pgfqpoint{2.994198in}{0.802653in}}%
\pgfpathlineto{\pgfqpoint{2.973026in}{0.800001in}}%
\pgfpathlineto{\pgfqpoint{2.939770in}{0.796547in}}%
\pgfpathlineto{\pgfqpoint{2.906514in}{0.793800in}}%
\pgfpathlineto{\pgfqpoint{2.873258in}{0.791736in}}%
\pgfpathlineto{\pgfqpoint{2.840001in}{0.790334in}}%
\pgfpathlineto{\pgfqpoint{2.806745in}{0.789573in}}%
\pgfpathlineto{\pgfqpoint{2.773489in}{0.789432in}}%
\pgfpathlineto{\pgfqpoint{2.740233in}{0.789892in}}%
\pgfpathlineto{\pgfqpoint{2.706977in}{0.790935in}}%
\pgfpathlineto{\pgfqpoint{2.673720in}{0.792543in}}%
\pgfpathlineto{\pgfqpoint{2.640464in}{0.794699in}}%
\pgfpathlineto{\pgfqpoint{2.607208in}{0.797387in}}%
\pgfpathlineto{\pgfqpoint{2.573952in}{0.800590in}}%
\pgfpathclose%
\pgfusepath{stroke}%
\end{pgfscope}%
\begin{pgfscope}%
\pgfpathrectangle{\pgfqpoint{0.711606in}{0.549444in}}{\pgfqpoint{4.955171in}{2.902168in}}%
\pgfusepath{clip}%
\pgfsetbuttcap%
\pgfsetroundjoin%
\pgfsetlinewidth{1.003750pt}%
\definecolor{currentstroke}{rgb}{0.087411,0.044556,0.224813}%
\pgfsetstrokecolor{currentstroke}%
\pgfsetdash{}{0pt}%
\pgfpathmoveto{\pgfqpoint{2.573952in}{0.741832in}}%
\pgfpathlineto{\pgfqpoint{2.507440in}{0.750319in}}%
\pgfpathlineto{\pgfqpoint{2.423438in}{0.763698in}}%
\pgfpathlineto{\pgfqpoint{2.374415in}{0.772867in}}%
\pgfpathlineto{\pgfqpoint{2.307902in}{0.786889in}}%
\pgfpathlineto{\pgfqpoint{2.241390in}{0.802749in}}%
\pgfpathlineto{\pgfqpoint{2.169301in}{0.822131in}}%
\pgfpathlineto{\pgfqpoint{2.103919in}{0.841609in}}%
\pgfpathlineto{\pgfqpoint{2.041853in}{0.861833in}}%
\pgfpathlineto{\pgfqpoint{1.975341in}{0.885526in}}%
\pgfpathlineto{\pgfqpoint{1.908828in}{0.911304in}}%
\pgfpathlineto{\pgfqpoint{1.842316in}{0.939229in}}%
\pgfpathlineto{\pgfqpoint{1.758532in}{0.977952in}}%
\pgfpathlineto{\pgfqpoint{1.709291in}{1.002560in}}%
\pgfpathlineto{\pgfqpoint{1.642779in}{1.038244in}}%
\pgfpathlineto{\pgfqpoint{1.576266in}{1.077068in}}%
\pgfpathlineto{\pgfqpoint{1.509754in}{1.119456in}}%
\pgfpathlineto{\pgfqpoint{1.460764in}{1.153251in}}%
\pgfpathlineto{\pgfqpoint{1.408232in}{1.192206in}}%
\pgfpathlineto{\pgfqpoint{1.359932in}{1.231162in}}%
\pgfpathlineto{\pgfqpoint{1.310217in}{1.274744in}}%
\pgfpathlineto{\pgfqpoint{1.274176in}{1.309072in}}%
\pgfpathlineto{\pgfqpoint{1.236569in}{1.348027in}}%
\pgfpathlineto{\pgfqpoint{1.202206in}{1.386983in}}%
\pgfpathlineto{\pgfqpoint{1.171004in}{1.425938in}}%
\pgfpathlineto{\pgfqpoint{1.142881in}{1.464893in}}%
\pgfpathlineto{\pgfqpoint{1.117950in}{1.503849in}}%
\pgfpathlineto{\pgfqpoint{1.096047in}{1.542804in}}%
\pgfpathlineto{\pgfqpoint{1.077066in}{1.581759in}}%
\pgfpathlineto{\pgfqpoint{1.061383in}{1.620714in}}%
\pgfpathlineto{\pgfqpoint{1.048672in}{1.659670in}}%
\pgfpathlineto{\pgfqpoint{1.039237in}{1.698625in}}%
\pgfpathlineto{\pgfqpoint{1.033158in}{1.737580in}}%
\pgfpathlineto{\pgfqpoint{1.030500in}{1.776536in}}%
\pgfpathlineto{\pgfqpoint{1.031460in}{1.815491in}}%
\pgfpathlineto{\pgfqpoint{1.036252in}{1.854446in}}%
\pgfpathlineto{\pgfqpoint{1.045135in}{1.893401in}}%
\pgfpathlineto{\pgfqpoint{1.058715in}{1.932357in}}%
\pgfpathlineto{\pgfqpoint{1.067266in}{1.951834in}}%
\pgfpathlineto{\pgfqpoint{1.077424in}{1.971993in}}%
\pgfpathlineto{\pgfqpoint{1.088438in}{1.990790in}}%
\pgfpathlineto{\pgfqpoint{1.110680in}{2.023337in}}%
\pgfpathlineto{\pgfqpoint{1.115545in}{2.029745in}}%
\pgfpathlineto{\pgfqpoint{1.143936in}{2.062650in}}%
\pgfpathlineto{\pgfqpoint{1.149710in}{2.068700in}}%
\pgfpathlineto{\pgfqpoint{1.177192in}{2.094758in}}%
\pgfpathlineto{\pgfqpoint{1.210448in}{2.121951in}}%
\pgfpathlineto{\pgfqpoint{1.243705in}{2.145514in}}%
\pgfpathlineto{\pgfqpoint{1.276961in}{2.166141in}}%
\pgfpathlineto{\pgfqpoint{1.312417in}{2.185566in}}%
\pgfpathlineto{\pgfqpoint{1.391816in}{2.224521in}}%
\pgfpathlineto{\pgfqpoint{1.419173in}{2.243999in}}%
\pgfpathlineto{\pgfqpoint{1.424850in}{2.263477in}}%
\pgfpathlineto{\pgfqpoint{1.416078in}{2.282954in}}%
\pgfpathlineto{\pgfqpoint{1.383996in}{2.321909in}}%
\pgfpathlineto{\pgfqpoint{1.343473in}{2.368234in}}%
\pgfpathlineto{\pgfqpoint{1.310217in}{2.410217in}}%
\pgfpathlineto{\pgfqpoint{1.289858in}{2.438775in}}%
\pgfpathlineto{\pgfqpoint{1.265095in}{2.477731in}}%
\pgfpathlineto{\pgfqpoint{1.243692in}{2.516686in}}%
\pgfpathlineto{\pgfqpoint{1.226016in}{2.555641in}}%
\pgfpathlineto{\pgfqpoint{1.211637in}{2.594596in}}%
\pgfpathlineto{\pgfqpoint{1.201064in}{2.633552in}}%
\pgfpathlineto{\pgfqpoint{1.194162in}{2.672507in}}%
\pgfpathlineto{\pgfqpoint{1.191131in}{2.711462in}}%
\pgfpathlineto{\pgfqpoint{1.192227in}{2.750418in}}%
\pgfpathlineto{\pgfqpoint{1.194410in}{2.769895in}}%
\pgfpathlineto{\pgfqpoint{1.197731in}{2.789373in}}%
\pgfpathlineto{\pgfqpoint{1.202230in}{2.808850in}}%
\pgfpathlineto{\pgfqpoint{1.210448in}{2.835359in}}%
\pgfpathlineto{\pgfqpoint{1.215088in}{2.847806in}}%
\pgfpathlineto{\pgfqpoint{1.223676in}{2.867283in}}%
\pgfpathlineto{\pgfqpoint{1.233674in}{2.886761in}}%
\pgfpathlineto{\pgfqpoint{1.245191in}{2.906239in}}%
\pgfpathlineto{\pgfqpoint{1.258682in}{2.925716in}}%
\pgfpathlineto{\pgfqpoint{1.276961in}{2.948854in}}%
\pgfpathlineto{\pgfqpoint{1.291269in}{2.964672in}}%
\pgfpathlineto{\pgfqpoint{1.310754in}{2.984149in}}%
\pgfpathlineto{\pgfqpoint{1.343473in}{3.011875in}}%
\pgfpathlineto{\pgfqpoint{1.376729in}{3.035823in}}%
\pgfpathlineto{\pgfqpoint{1.409986in}{3.056409in}}%
\pgfpathlineto{\pgfqpoint{1.443242in}{3.074290in}}%
\pgfpathlineto{\pgfqpoint{1.476498in}{3.089964in}}%
\pgfpathlineto{\pgfqpoint{1.509754in}{3.103812in}}%
\pgfpathlineto{\pgfqpoint{1.556685in}{3.120493in}}%
\pgfpathlineto{\pgfqpoint{1.609523in}{3.136098in}}%
\pgfpathlineto{\pgfqpoint{1.642779in}{3.144347in}}%
\pgfpathlineto{\pgfqpoint{1.709291in}{3.157759in}}%
\pgfpathlineto{\pgfqpoint{1.742547in}{3.163010in}}%
\pgfpathlineto{\pgfqpoint{1.809060in}{3.170999in}}%
\pgfpathlineto{\pgfqpoint{1.875572in}{3.175979in}}%
\pgfpathlineto{\pgfqpoint{1.942084in}{3.178152in}}%
\pgfpathlineto{\pgfqpoint{2.008597in}{3.177702in}}%
\pgfpathlineto{\pgfqpoint{2.075109in}{3.174797in}}%
\pgfpathlineto{\pgfqpoint{2.141622in}{3.169593in}}%
\pgfpathlineto{\pgfqpoint{2.208134in}{3.162231in}}%
\pgfpathlineto{\pgfqpoint{2.274646in}{3.152642in}}%
\pgfpathlineto{\pgfqpoint{2.346455in}{3.139970in}}%
\pgfpathlineto{\pgfqpoint{2.407671in}{3.127185in}}%
\pgfpathlineto{\pgfqpoint{2.474183in}{3.111291in}}%
\pgfpathlineto{\pgfqpoint{2.540696in}{3.093277in}}%
\pgfpathlineto{\pgfqpoint{2.607208in}{3.073082in}}%
\pgfpathlineto{\pgfqpoint{2.673720in}{3.050644in}}%
\pgfpathlineto{\pgfqpoint{2.747302in}{3.023104in}}%
\pgfpathlineto{\pgfqpoint{2.806745in}{2.998615in}}%
\pgfpathlineto{\pgfqpoint{2.881769in}{2.964672in}}%
\pgfpathlineto{\pgfqpoint{2.939770in}{2.935889in}}%
\pgfpathlineto{\pgfqpoint{3.006282in}{2.899915in}}%
\pgfpathlineto{\pgfqpoint{3.072795in}{2.860322in}}%
\pgfpathlineto{\pgfqpoint{3.122067in}{2.828328in}}%
\pgfpathlineto{\pgfqpoint{3.177342in}{2.789373in}}%
\pgfpathlineto{\pgfqpoint{3.227687in}{2.750418in}}%
\pgfpathlineto{\pgfqpoint{3.273770in}{2.711462in}}%
\pgfpathlineto{\pgfqpoint{3.315544in}{2.672507in}}%
\pgfpathlineto{\pgfqpoint{3.353490in}{2.633552in}}%
\pgfpathlineto{\pgfqpoint{3.387723in}{2.594596in}}%
\pgfpathlineto{\pgfqpoint{3.418355in}{2.555641in}}%
\pgfpathlineto{\pgfqpoint{3.445492in}{2.516686in}}%
\pgfpathlineto{\pgfqpoint{3.471869in}{2.472775in}}%
\pgfpathlineto{\pgfqpoint{3.489278in}{2.438775in}}%
\pgfpathlineto{\pgfqpoint{3.506247in}{2.399820in}}%
\pgfpathlineto{\pgfqpoint{3.538381in}{2.315773in}}%
\pgfpathlineto{\pgfqpoint{3.548033in}{2.302432in}}%
\pgfpathlineto{\pgfqpoint{3.571637in}{2.286220in}}%
\pgfpathlineto{\pgfqpoint{3.581144in}{2.282954in}}%
\pgfpathlineto{\pgfqpoint{3.604894in}{2.277514in}}%
\pgfpathlineto{\pgfqpoint{3.638150in}{2.273414in}}%
\pgfpathlineto{\pgfqpoint{3.804431in}{2.259126in}}%
\pgfpathlineto{\pgfqpoint{3.870943in}{2.250830in}}%
\pgfpathlineto{\pgfqpoint{3.937455in}{2.240637in}}%
\pgfpathlineto{\pgfqpoint{4.023532in}{2.224521in}}%
\pgfpathlineto{\pgfqpoint{4.070480in}{2.214384in}}%
\pgfpathlineto{\pgfqpoint{4.136993in}{2.198378in}}%
\pgfpathlineto{\pgfqpoint{4.203505in}{2.180414in}}%
\pgfpathlineto{\pgfqpoint{4.270017in}{2.160442in}}%
\pgfpathlineto{\pgfqpoint{4.336530in}{2.138407in}}%
\pgfpathlineto{\pgfqpoint{4.403042in}{2.114255in}}%
\pgfpathlineto{\pgfqpoint{4.469554in}{2.087922in}}%
\pgfpathlineto{\pgfqpoint{4.536067in}{2.059089in}}%
\pgfpathlineto{\pgfqpoint{4.602579in}{2.027828in}}%
\pgfpathlineto{\pgfqpoint{4.674582in}{1.990790in}}%
\pgfpathlineto{\pgfqpoint{4.743700in}{1.951834in}}%
\pgfpathlineto{\pgfqpoint{4.807005in}{1.912879in}}%
\pgfpathlineto{\pgfqpoint{4.868629in}{1.871445in}}%
\pgfpathlineto{\pgfqpoint{4.918426in}{1.834968in}}%
\pgfpathlineto{\pgfqpoint{4.968397in}{1.795394in}}%
\pgfpathlineto{\pgfqpoint{5.012681in}{1.757058in}}%
\pgfpathlineto{\pgfqpoint{5.054045in}{1.718103in}}%
\pgfpathlineto{\pgfqpoint{5.091868in}{1.679147in}}%
\pgfpathlineto{\pgfqpoint{5.126248in}{1.640192in}}%
\pgfpathlineto{\pgfqpoint{5.157280in}{1.601237in}}%
\pgfpathlineto{\pgfqpoint{5.185055in}{1.562281in}}%
\pgfpathlineto{\pgfqpoint{5.209659in}{1.523326in}}%
\pgfpathlineto{\pgfqpoint{5.234447in}{1.477534in}}%
\pgfpathlineto{\pgfqpoint{5.249190in}{1.445416in}}%
\pgfpathlineto{\pgfqpoint{5.264199in}{1.406460in}}%
\pgfpathlineto{\pgfqpoint{5.275794in}{1.367505in}}%
\pgfpathlineto{\pgfqpoint{5.284007in}{1.328550in}}%
\pgfpathlineto{\pgfqpoint{5.288740in}{1.289595in}}%
\pgfpathlineto{\pgfqpoint{5.289783in}{1.250639in}}%
\pgfpathlineto{\pgfqpoint{5.286909in}{1.211684in}}%
\pgfpathlineto{\pgfqpoint{5.279871in}{1.172729in}}%
\pgfpathlineto{\pgfqpoint{5.268401in}{1.133773in}}%
\pgfpathlineto{\pgfqpoint{5.260691in}{1.114296in}}%
\pgfpathlineto{\pgfqpoint{5.251692in}{1.094818in}}%
\pgfpathlineto{\pgfqpoint{5.234447in}{1.063699in}}%
\pgfpathlineto{\pgfqpoint{5.229555in}{1.055863in}}%
\pgfpathlineto{\pgfqpoint{5.216021in}{1.036385in}}%
\pgfpathlineto{\pgfqpoint{5.200962in}{1.016908in}}%
\pgfpathlineto{\pgfqpoint{5.167934in}{0.981200in}}%
\pgfpathlineto{\pgfqpoint{5.164608in}{0.977952in}}%
\pgfpathlineto{\pgfqpoint{5.134678in}{0.951520in}}%
\pgfpathlineto{\pgfqpoint{5.101422in}{0.926256in}}%
\pgfpathlineto{\pgfqpoint{5.068166in}{0.904368in}}%
\pgfpathlineto{\pgfqpoint{5.034909in}{0.885226in}}%
\pgfpathlineto{\pgfqpoint{5.001653in}{0.868347in}}%
\pgfpathlineto{\pgfqpoint{4.968397in}{0.853355in}}%
\pgfpathlineto{\pgfqpoint{4.935141in}{0.839952in}}%
\pgfpathlineto{\pgfqpoint{4.883477in}{0.822131in}}%
\pgfpathlineto{\pgfqpoint{4.835372in}{0.808148in}}%
\pgfpathlineto{\pgfqpoint{4.802116in}{0.799803in}}%
\pgfpathlineto{\pgfqpoint{4.735604in}{0.786075in}}%
\pgfpathlineto{\pgfqpoint{4.669091in}{0.775832in}}%
\pgfpathlineto{\pgfqpoint{4.602579in}{0.768601in}}%
\pgfpathlineto{\pgfqpoint{4.525604in}{0.763698in}}%
\pgfpathlineto{\pgfqpoint{4.469554in}{0.762214in}}%
\pgfpathlineto{\pgfqpoint{4.403042in}{0.762676in}}%
\pgfpathlineto{\pgfqpoint{4.336530in}{0.765388in}}%
\pgfpathlineto{\pgfqpoint{4.270017in}{0.770265in}}%
\pgfpathlineto{\pgfqpoint{4.203505in}{0.777153in}}%
\pgfpathlineto{\pgfqpoint{4.136993in}{0.786019in}}%
\pgfpathlineto{\pgfqpoint{4.070480in}{0.796889in}}%
\pgfpathlineto{\pgfqpoint{4.003968in}{0.809682in}}%
\pgfpathlineto{\pgfqpoint{3.937455in}{0.824366in}}%
\pgfpathlineto{\pgfqpoint{3.868705in}{0.841609in}}%
\pgfpathlineto{\pgfqpoint{3.799534in}{0.861086in}}%
\pgfpathlineto{\pgfqpoint{3.704662in}{0.891003in}}%
\pgfpathlineto{\pgfqpoint{3.638150in}{0.912399in}}%
\pgfpathlineto{\pgfqpoint{3.604894in}{0.920944in}}%
\pgfpathlineto{\pgfqpoint{3.571637in}{0.924369in}}%
\pgfpathlineto{\pgfqpoint{3.538381in}{0.917843in}}%
\pgfpathlineto{\pgfqpoint{3.503812in}{0.900042in}}%
\pgfpathlineto{\pgfqpoint{3.438485in}{0.861086in}}%
\pgfpathlineto{\pgfqpoint{3.400893in}{0.841609in}}%
\pgfpathlineto{\pgfqpoint{3.357828in}{0.822131in}}%
\pgfpathlineto{\pgfqpoint{3.305588in}{0.801845in}}%
\pgfpathlineto{\pgfqpoint{3.239076in}{0.780755in}}%
\pgfpathlineto{\pgfqpoint{3.171440in}{0.763698in}}%
\pgfpathlineto{\pgfqpoint{3.106051in}{0.750921in}}%
\pgfpathlineto{\pgfqpoint{3.039539in}{0.741041in}}%
\pgfpathlineto{\pgfqpoint{2.973026in}{0.734113in}}%
\pgfpathlineto{\pgfqpoint{2.906514in}{0.729766in}}%
\pgfpathlineto{\pgfqpoint{2.840001in}{0.727849in}}%
\pgfpathlineto{\pgfqpoint{2.773489in}{0.728222in}}%
\pgfpathlineto{\pgfqpoint{2.706977in}{0.730755in}}%
\pgfpathlineto{\pgfqpoint{2.640464in}{0.735329in}}%
\pgfpathlineto{\pgfqpoint{2.573952in}{0.741832in}}%
\pgfpathlineto{\pgfqpoint{2.573952in}{0.741832in}}%
\pgfusepath{stroke}%
\end{pgfscope}%
\begin{pgfscope}%
\pgfpathrectangle{\pgfqpoint{0.711606in}{0.549444in}}{\pgfqpoint{4.955171in}{2.902168in}}%
\pgfusepath{clip}%
\pgfsetbuttcap%
\pgfsetroundjoin%
\pgfsetlinewidth{1.003750pt}%
\definecolor{currentstroke}{rgb}{0.104551,0.047008,0.253430}%
\pgfsetstrokecolor{currentstroke}%
\pgfsetdash{}{0pt}%
\pgfpathmoveto{\pgfqpoint{2.607208in}{0.683708in}}%
\pgfpathlineto{\pgfqpoint{2.588541in}{0.685788in}}%
\pgfpathlineto{\pgfqpoint{2.573952in}{0.687439in}}%
\pgfpathlineto{\pgfqpoint{2.540696in}{0.691652in}}%
\pgfpathlineto{\pgfqpoint{2.507440in}{0.696282in}}%
\pgfpathlineto{\pgfqpoint{2.474183in}{0.701319in}}%
\pgfpathlineto{\pgfqpoint{2.450088in}{0.705265in}}%
\pgfpathlineto{\pgfqpoint{2.440927in}{0.706790in}}%
\pgfpathlineto{\pgfqpoint{2.407671in}{0.712754in}}%
\pgfpathlineto{\pgfqpoint{2.374415in}{0.719103in}}%
\pgfpathlineto{\pgfqpoint{2.346559in}{0.724743in}}%
\pgfpathlineto{\pgfqpoint{2.341159in}{0.725854in}}%
\pgfpathlineto{\pgfqpoint{2.307902in}{0.733117in}}%
\pgfpathlineto{\pgfqpoint{2.274646in}{0.740743in}}%
\pgfpathlineto{\pgfqpoint{2.260221in}{0.744221in}}%
\pgfpathlineto{\pgfqpoint{2.241390in}{0.748834in}}%
\pgfpathlineto{\pgfqpoint{2.208134in}{0.757360in}}%
\pgfpathlineto{\pgfqpoint{2.184423in}{0.763698in}}%
\pgfpathlineto{\pgfqpoint{2.174878in}{0.766291in}}%
\pgfpathlineto{\pgfqpoint{2.141622in}{0.775718in}}%
\pgfpathlineto{\pgfqpoint{2.116250in}{0.783176in}}%
\pgfpathlineto{\pgfqpoint{2.108365in}{0.785532in}}%
\pgfpathlineto{\pgfqpoint{2.075109in}{0.795857in}}%
\pgfpathlineto{\pgfqpoint{2.053936in}{0.802653in}}%
\pgfpathlineto{\pgfqpoint{2.041853in}{0.806596in}}%
\pgfpathlineto{\pgfqpoint{2.008597in}{0.817819in}}%
\pgfpathlineto{\pgfqpoint{1.996222in}{0.822131in}}%
\pgfpathlineto{\pgfqpoint{1.975341in}{0.829529in}}%
\pgfpathlineto{\pgfqpoint{1.942189in}{0.841609in}}%
\pgfpathlineto{\pgfqpoint{1.942084in}{0.841647in}}%
\pgfpathlineto{\pgfqpoint{1.908828in}{0.854374in}}%
\pgfpathlineto{\pgfqpoint{1.891743in}{0.861086in}}%
\pgfpathlineto{\pgfqpoint{1.875572in}{0.867549in}}%
\pgfpathlineto{\pgfqpoint{1.843826in}{0.880564in}}%
\pgfpathlineto{\pgfqpoint{1.842316in}{0.881194in}}%
\pgfpathlineto{\pgfqpoint{1.809060in}{0.895459in}}%
\pgfpathlineto{\pgfqpoint{1.798636in}{0.900042in}}%
\pgfpathlineto{\pgfqpoint{1.775804in}{0.910255in}}%
\pgfpathlineto{\pgfqpoint{1.755574in}{0.919519in}}%
\pgfpathlineto{\pgfqpoint{1.742547in}{0.925592in}}%
\pgfpathlineto{\pgfqpoint{1.714443in}{0.938997in}}%
\pgfpathlineto{\pgfqpoint{1.709291in}{0.941499in}}%
\pgfpathlineto{\pgfqpoint{1.676035in}{0.958022in}}%
\pgfpathlineto{\pgfqpoint{1.675145in}{0.958475in}}%
\pgfpathlineto{\pgfqpoint{1.642779in}{0.975235in}}%
\pgfpathlineto{\pgfqpoint{1.637641in}{0.977952in}}%
\pgfpathlineto{\pgfqpoint{1.609523in}{0.993104in}}%
\pgfpathlineto{\pgfqpoint{1.601660in}{0.997430in}}%
\pgfpathlineto{\pgfqpoint{1.576266in}{1.011668in}}%
\pgfpathlineto{\pgfqpoint{1.567110in}{1.016908in}}%
\pgfpathlineto{\pgfqpoint{1.543010in}{1.030967in}}%
\pgfpathlineto{\pgfqpoint{1.533906in}{1.036385in}}%
\pgfpathlineto{\pgfqpoint{1.509754in}{1.051045in}}%
\pgfpathlineto{\pgfqpoint{1.501970in}{1.055863in}}%
\pgfpathlineto{\pgfqpoint{1.476498in}{1.071948in}}%
\pgfpathlineto{\pgfqpoint{1.471229in}{1.075340in}}%
\pgfpathlineto{\pgfqpoint{1.443242in}{1.093729in}}%
\pgfpathlineto{\pgfqpoint{1.441616in}{1.094818in}}%
\pgfpathlineto{\pgfqpoint{1.413157in}{1.114296in}}%
\pgfpathlineto{\pgfqpoint{1.409986in}{1.116515in}}%
\pgfpathlineto{\pgfqpoint{1.385777in}{1.133773in}}%
\pgfpathlineto{\pgfqpoint{1.376729in}{1.140367in}}%
\pgfpathlineto{\pgfqpoint{1.359373in}{1.153251in}}%
\pgfpathlineto{\pgfqpoint{1.343473in}{1.165322in}}%
\pgfpathlineto{\pgfqpoint{1.333892in}{1.172729in}}%
\pgfpathlineto{\pgfqpoint{1.310217in}{1.191457in}}%
\pgfpathlineto{\pgfqpoint{1.309286in}{1.192206in}}%
\pgfpathlineto{\pgfqpoint{1.285738in}{1.211684in}}%
\pgfpathlineto{\pgfqpoint{1.276961in}{1.219126in}}%
\pgfpathlineto{\pgfqpoint{1.263014in}{1.231162in}}%
\pgfpathlineto{\pgfqpoint{1.243705in}{1.248248in}}%
\pgfpathlineto{\pgfqpoint{1.241049in}{1.250639in}}%
\pgfpathlineto{\pgfqpoint{1.220047in}{1.270117in}}%
\pgfpathlineto{\pgfqpoint{1.210448in}{1.279271in}}%
\pgfpathlineto{\pgfqpoint{1.199807in}{1.289595in}}%
\pgfpathlineto{\pgfqpoint{1.180301in}{1.309072in}}%
\pgfpathlineto{\pgfqpoint{1.177192in}{1.312283in}}%
\pgfpathlineto{\pgfqpoint{1.161704in}{1.328550in}}%
\pgfpathlineto{\pgfqpoint{1.143936in}{1.347771in}}%
\pgfpathlineto{\pgfqpoint{1.143703in}{1.348027in}}%
\pgfpathlineto{\pgfqpoint{1.126662in}{1.367505in}}%
\pgfpathlineto{\pgfqpoint{1.110680in}{1.386370in}}%
\pgfpathlineto{\pgfqpoint{1.110169in}{1.386983in}}%
\pgfpathlineto{\pgfqpoint{1.094609in}{1.406460in}}%
\pgfpathlineto{\pgfqpoint{1.079603in}{1.425938in}}%
\pgfpathlineto{\pgfqpoint{1.077424in}{1.428900in}}%
\pgfpathlineto{\pgfqpoint{1.065470in}{1.445416in}}%
\pgfpathlineto{\pgfqpoint{1.051966in}{1.464893in}}%
\pgfpathlineto{\pgfqpoint{1.044168in}{1.476695in}}%
\pgfpathlineto{\pgfqpoint{1.039178in}{1.484371in}}%
\pgfpathlineto{\pgfqpoint{1.027174in}{1.503849in}}%
\pgfpathlineto{\pgfqpoint{1.015781in}{1.523326in}}%
\pgfpathlineto{\pgfqpoint{1.010911in}{1.532177in}}%
\pgfpathlineto{\pgfqpoint{1.005159in}{1.542804in}}%
\pgfpathlineto{\pgfqpoint{0.995287in}{1.562281in}}%
\pgfpathlineto{\pgfqpoint{0.986060in}{1.581759in}}%
\pgfpathlineto{\pgfqpoint{0.977655in}{1.600880in}}%
\pgfpathlineto{\pgfqpoint{0.977501in}{1.601237in}}%
\pgfpathlineto{\pgfqpoint{0.969813in}{1.620714in}}%
\pgfpathlineto{\pgfqpoint{0.962809in}{1.640192in}}%
\pgfpathlineto{\pgfqpoint{0.956506in}{1.659670in}}%
\pgfpathlineto{\pgfqpoint{0.950922in}{1.679147in}}%
\pgfpathlineto{\pgfqpoint{0.946075in}{1.698625in}}%
\pgfpathlineto{\pgfqpoint{0.944399in}{1.706649in}}%
\pgfpathlineto{\pgfqpoint{0.942046in}{1.718103in}}%
\pgfpathlineto{\pgfqpoint{0.938819in}{1.737580in}}%
\pgfpathlineto{\pgfqpoint{0.936370in}{1.757058in}}%
\pgfpathlineto{\pgfqpoint{0.934720in}{1.776536in}}%
\pgfpathlineto{\pgfqpoint{0.933890in}{1.796013in}}%
\pgfpathlineto{\pgfqpoint{0.933903in}{1.815491in}}%
\pgfpathlineto{\pgfqpoint{0.934780in}{1.834968in}}%
\pgfpathlineto{\pgfqpoint{0.936545in}{1.854446in}}%
\pgfpathlineto{\pgfqpoint{0.939223in}{1.873924in}}%
\pgfpathlineto{\pgfqpoint{0.942839in}{1.893401in}}%
\pgfpathlineto{\pgfqpoint{0.944399in}{1.900079in}}%
\pgfpathlineto{\pgfqpoint{0.947503in}{1.912879in}}%
\pgfpathlineto{\pgfqpoint{0.953234in}{1.932357in}}%
\pgfpathlineto{\pgfqpoint{0.960016in}{1.951834in}}%
\pgfpathlineto{\pgfqpoint{0.967881in}{1.971312in}}%
\pgfpathlineto{\pgfqpoint{0.976861in}{1.990790in}}%
\pgfpathlineto{\pgfqpoint{0.977655in}{1.992331in}}%
\pgfpathlineto{\pgfqpoint{0.987276in}{2.010267in}}%
\pgfpathlineto{\pgfqpoint{0.998941in}{2.029745in}}%
\pgfpathlineto{\pgfqpoint{1.010911in}{2.047789in}}%
\pgfpathlineto{\pgfqpoint{1.011903in}{2.049222in}}%
\pgfpathlineto{\pgfqpoint{1.026606in}{2.068700in}}%
\pgfpathlineto{\pgfqpoint{1.042712in}{2.088178in}}%
\pgfpathlineto{\pgfqpoint{1.044168in}{2.089809in}}%
\pgfpathlineto{\pgfqpoint{1.060825in}{2.107655in}}%
\pgfpathlineto{\pgfqpoint{1.077424in}{2.124072in}}%
\pgfpathlineto{\pgfqpoint{1.080669in}{2.127133in}}%
\pgfpathlineto{\pgfqpoint{1.102806in}{2.146611in}}%
\pgfpathlineto{\pgfqpoint{1.110680in}{2.153082in}}%
\pgfpathlineto{\pgfqpoint{1.127320in}{2.166088in}}%
\pgfpathlineto{\pgfqpoint{1.143936in}{2.178260in}}%
\pgfpathlineto{\pgfqpoint{1.154449in}{2.185566in}}%
\pgfpathlineto{\pgfqpoint{1.177192in}{2.200475in}}%
\pgfpathlineto{\pgfqpoint{1.184548in}{2.205044in}}%
\pgfpathlineto{\pgfqpoint{1.210448in}{2.220427in}}%
\pgfpathlineto{\pgfqpoint{1.217691in}{2.224521in}}%
\pgfpathlineto{\pgfqpoint{1.243705in}{2.239270in}}%
\pgfpathlineto{\pgfqpoint{1.252199in}{2.243999in}}%
\pgfpathlineto{\pgfqpoint{1.276961in}{2.260787in}}%
\pgfpathlineto{\pgfqpoint{1.280610in}{2.263477in}}%
\pgfpathlineto{\pgfqpoint{1.292188in}{2.282954in}}%
\pgfpathlineto{\pgfqpoint{1.288180in}{2.302432in}}%
\pgfpathlineto{\pgfqpoint{1.276961in}{2.320520in}}%
\pgfpathlineto{\pgfqpoint{1.276160in}{2.321909in}}%
\pgfpathlineto{\pgfqpoint{1.261240in}{2.341387in}}%
\pgfpathlineto{\pgfqpoint{1.245454in}{2.360865in}}%
\pgfpathlineto{\pgfqpoint{1.243705in}{2.363031in}}%
\pgfpathlineto{\pgfqpoint{1.230029in}{2.380342in}}%
\pgfpathlineto{\pgfqpoint{1.215037in}{2.399820in}}%
\pgfpathlineto{\pgfqpoint{1.210448in}{2.406041in}}%
\pgfpathlineto{\pgfqpoint{1.200853in}{2.419298in}}%
\pgfpathlineto{\pgfqpoint{1.187400in}{2.438775in}}%
\pgfpathlineto{\pgfqpoint{1.177192in}{2.454326in}}%
\pgfpathlineto{\pgfqpoint{1.174662in}{2.458253in}}%
\pgfpathlineto{\pgfqpoint{1.162860in}{2.477731in}}%
\pgfpathlineto{\pgfqpoint{1.151741in}{2.497208in}}%
\pgfpathlineto{\pgfqpoint{1.143936in}{2.511837in}}%
\pgfpathlineto{\pgfqpoint{1.141396in}{2.516686in}}%
\pgfpathlineto{\pgfqpoint{1.131970in}{2.536163in}}%
\pgfpathlineto{\pgfqpoint{1.123274in}{2.555641in}}%
\pgfpathlineto{\pgfqpoint{1.115327in}{2.575119in}}%
\pgfpathlineto{\pgfqpoint{1.110680in}{2.587776in}}%
\pgfpathlineto{\pgfqpoint{1.108221in}{2.594596in}}%
\pgfpathlineto{\pgfqpoint{1.102018in}{2.614074in}}%
\pgfpathlineto{\pgfqpoint{1.096611in}{2.633552in}}%
\pgfpathlineto{\pgfqpoint{1.092025in}{2.653029in}}%
\pgfpathlineto{\pgfqpoint{1.088282in}{2.672507in}}%
\pgfpathlineto{\pgfqpoint{1.085409in}{2.691985in}}%
\pgfpathlineto{\pgfqpoint{1.083429in}{2.711462in}}%
\pgfpathlineto{\pgfqpoint{1.082371in}{2.730940in}}%
\pgfpathlineto{\pgfqpoint{1.082261in}{2.750418in}}%
\pgfpathlineto{\pgfqpoint{1.083130in}{2.769895in}}%
\pgfpathlineto{\pgfqpoint{1.085006in}{2.789373in}}%
\pgfpathlineto{\pgfqpoint{1.087921in}{2.808850in}}%
\pgfpathlineto{\pgfqpoint{1.091908in}{2.828328in}}%
\pgfpathlineto{\pgfqpoint{1.097002in}{2.847806in}}%
\pgfpathlineto{\pgfqpoint{1.103236in}{2.867283in}}%
\pgfpathlineto{\pgfqpoint{1.110649in}{2.886761in}}%
\pgfpathlineto{\pgfqpoint{1.110680in}{2.886831in}}%
\pgfpathlineto{\pgfqpoint{1.119568in}{2.906239in}}%
\pgfpathlineto{\pgfqpoint{1.129794in}{2.925716in}}%
\pgfpathlineto{\pgfqpoint{1.141372in}{2.945194in}}%
\pgfpathlineto{\pgfqpoint{1.143936in}{2.949078in}}%
\pgfpathlineto{\pgfqpoint{1.154724in}{2.964672in}}%
\pgfpathlineto{\pgfqpoint{1.169680in}{2.984149in}}%
\pgfpathlineto{\pgfqpoint{1.177192in}{2.993063in}}%
\pgfpathlineto{\pgfqpoint{1.186550in}{3.003627in}}%
\pgfpathlineto{\pgfqpoint{1.205413in}{3.023104in}}%
\pgfpathlineto{\pgfqpoint{1.210448in}{3.027904in}}%
\pgfpathlineto{\pgfqpoint{1.226688in}{3.042582in}}%
\pgfpathlineto{\pgfqpoint{1.243705in}{3.056784in}}%
\pgfpathlineto{\pgfqpoint{1.250391in}{3.062060in}}%
\pgfpathlineto{\pgfqpoint{1.276961in}{3.081532in}}%
\pgfpathlineto{\pgfqpoint{1.276968in}{3.081537in}}%
\pgfpathlineto{\pgfqpoint{1.307133in}{3.101015in}}%
\pgfpathlineto{\pgfqpoint{1.310217in}{3.102881in}}%
\pgfpathlineto{\pgfqpoint{1.341268in}{3.120493in}}%
\pgfpathlineto{\pgfqpoint{1.343473in}{3.121669in}}%
\pgfpathlineto{\pgfqpoint{1.376729in}{3.138257in}}%
\pgfpathlineto{\pgfqpoint{1.380418in}{3.139970in}}%
\pgfpathlineto{\pgfqpoint{1.409986in}{3.152942in}}%
\pgfpathlineto{\pgfqpoint{1.426018in}{3.159448in}}%
\pgfpathlineto{\pgfqpoint{1.443242in}{3.166070in}}%
\pgfpathlineto{\pgfqpoint{1.476498in}{3.177833in}}%
\pgfpathlineto{\pgfqpoint{1.479855in}{3.178926in}}%
\pgfpathlineto{\pgfqpoint{1.509754in}{3.188179in}}%
\pgfpathlineto{\pgfqpoint{1.543010in}{3.197509in}}%
\pgfpathlineto{\pgfqpoint{1.546515in}{3.198403in}}%
\pgfpathlineto{\pgfqpoint{1.576266in}{3.205641in}}%
\pgfpathlineto{\pgfqpoint{1.609523in}{3.212872in}}%
\pgfpathlineto{\pgfqpoint{1.635507in}{3.217881in}}%
\pgfpathlineto{\pgfqpoint{1.642779in}{3.219221in}}%
\pgfpathlineto{\pgfqpoint{1.676035in}{3.224627in}}%
\pgfpathlineto{\pgfqpoint{1.709291in}{3.229279in}}%
\pgfpathlineto{\pgfqpoint{1.742547in}{3.233204in}}%
\pgfpathlineto{\pgfqpoint{1.775804in}{3.236423in}}%
\pgfpathlineto{\pgfqpoint{1.787978in}{3.237359in}}%
\pgfpathlineto{\pgfqpoint{1.809060in}{3.238913in}}%
\pgfpathlineto{\pgfqpoint{1.842316in}{3.240734in}}%
\pgfpathlineto{\pgfqpoint{1.875572in}{3.241936in}}%
\pgfpathlineto{\pgfqpoint{1.908828in}{3.242536in}}%
\pgfpathlineto{\pgfqpoint{1.942084in}{3.242552in}}%
\pgfpathlineto{\pgfqpoint{1.975341in}{3.242002in}}%
\pgfpathlineto{\pgfqpoint{2.008597in}{3.240901in}}%
\pgfpathlineto{\pgfqpoint{2.041853in}{3.239267in}}%
\pgfpathlineto{\pgfqpoint{2.071292in}{3.237359in}}%
\pgfpathlineto{\pgfqpoint{2.075109in}{3.237106in}}%
\pgfpathlineto{\pgfqpoint{2.108365in}{3.234374in}}%
\pgfpathlineto{\pgfqpoint{2.141622in}{3.231140in}}%
\pgfpathlineto{\pgfqpoint{2.174878in}{3.227417in}}%
\pgfpathlineto{\pgfqpoint{2.208134in}{3.223221in}}%
\pgfpathlineto{\pgfqpoint{2.241390in}{3.218563in}}%
\pgfpathlineto{\pgfqpoint{2.245801in}{3.217881in}}%
\pgfpathlineto{\pgfqpoint{2.274646in}{3.213336in}}%
\pgfpathlineto{\pgfqpoint{2.307902in}{3.207645in}}%
\pgfpathlineto{\pgfqpoint{2.341159in}{3.201521in}}%
\pgfpathlineto{\pgfqpoint{2.356925in}{3.198403in}}%
\pgfpathlineto{\pgfqpoint{2.374415in}{3.194882in}}%
\pgfpathlineto{\pgfqpoint{2.407671in}{3.187740in}}%
\pgfpathlineto{\pgfqpoint{2.440927in}{3.180193in}}%
\pgfpathlineto{\pgfqpoint{2.446193in}{3.178926in}}%
\pgfpathlineto{\pgfqpoint{2.474183in}{3.172068in}}%
\pgfpathlineto{\pgfqpoint{2.507440in}{3.163519in}}%
\pgfpathlineto{\pgfqpoint{2.522527in}{3.159448in}}%
\pgfpathlineto{\pgfqpoint{2.540696in}{3.154456in}}%
\pgfpathlineto{\pgfqpoint{2.573952in}{3.144906in}}%
\pgfpathlineto{\pgfqpoint{2.590432in}{3.139970in}}%
\pgfpathlineto{\pgfqpoint{2.607208in}{3.134854in}}%
\pgfpathlineto{\pgfqpoint{2.640464in}{3.124305in}}%
\pgfpathlineto{\pgfqpoint{2.652027in}{3.120493in}}%
\pgfpathlineto{\pgfqpoint{2.673720in}{3.113208in}}%
\pgfpathlineto{\pgfqpoint{2.706977in}{3.101663in}}%
\pgfpathlineto{\pgfqpoint{2.708774in}{3.101015in}}%
\pgfpathlineto{\pgfqpoint{2.740233in}{3.089466in}}%
\pgfpathlineto{\pgfqpoint{2.761184in}{3.081537in}}%
\pgfpathlineto{\pgfqpoint{2.773489in}{3.076792in}}%
\pgfpathlineto{\pgfqpoint{2.806745in}{3.063570in}}%
\pgfpathlineto{\pgfqpoint{2.810424in}{3.062060in}}%
\pgfpathlineto{\pgfqpoint{2.840001in}{3.049688in}}%
\pgfpathlineto{\pgfqpoint{2.856529in}{3.042582in}}%
\pgfpathlineto{\pgfqpoint{2.873258in}{3.035249in}}%
\pgfpathlineto{\pgfqpoint{2.900234in}{3.023104in}}%
\pgfpathlineto{\pgfqpoint{2.906514in}{3.020221in}}%
\pgfpathlineto{\pgfqpoint{2.939770in}{3.004538in}}%
\pgfpathlineto{\pgfqpoint{2.941652in}{3.003627in}}%
\pgfpathlineto{\pgfqpoint{2.973026in}{2.988122in}}%
\pgfpathlineto{\pgfqpoint{2.980871in}{2.984149in}}%
\pgfpathlineto{\pgfqpoint{3.006282in}{2.971016in}}%
\pgfpathlineto{\pgfqpoint{3.018269in}{2.964672in}}%
\pgfpathlineto{\pgfqpoint{3.039539in}{2.953177in}}%
\pgfpathlineto{\pgfqpoint{3.053972in}{2.945194in}}%
\pgfpathlineto{\pgfqpoint{3.072795in}{2.934559in}}%
\pgfpathlineto{\pgfqpoint{3.088095in}{2.925716in}}%
\pgfpathlineto{\pgfqpoint{3.106051in}{2.915110in}}%
\pgfpathlineto{\pgfqpoint{3.120741in}{2.906239in}}%
\pgfpathlineto{\pgfqpoint{3.139307in}{2.894774in}}%
\pgfpathlineto{\pgfqpoint{3.152005in}{2.886761in}}%
\pgfpathlineto{\pgfqpoint{3.172563in}{2.873490in}}%
\pgfpathlineto{\pgfqpoint{3.181976in}{2.867283in}}%
\pgfpathlineto{\pgfqpoint{3.205819in}{2.851192in}}%
\pgfpathlineto{\pgfqpoint{3.210733in}{2.847806in}}%
\pgfpathlineto{\pgfqpoint{3.238326in}{2.828328in}}%
\pgfpathlineto{\pgfqpoint{3.239076in}{2.827785in}}%
\pgfpathlineto{\pgfqpoint{3.264663in}{2.808850in}}%
\pgfpathlineto{\pgfqpoint{3.272332in}{2.803031in}}%
\pgfpathlineto{\pgfqpoint{3.289968in}{2.789373in}}%
\pgfpathlineto{\pgfqpoint{3.305588in}{2.776959in}}%
\pgfpathlineto{\pgfqpoint{3.314301in}{2.769895in}}%
\pgfpathlineto{\pgfqpoint{3.337685in}{2.750418in}}%
\pgfpathlineto{\pgfqpoint{3.338844in}{2.749421in}}%
\pgfpathlineto{\pgfqpoint{3.359925in}{2.730940in}}%
\pgfpathlineto{\pgfqpoint{3.372100in}{2.719958in}}%
\pgfpathlineto{\pgfqpoint{3.381339in}{2.711462in}}%
\pgfpathlineto{\pgfqpoint{3.401873in}{2.691985in}}%
\pgfpathlineto{\pgfqpoint{3.405357in}{2.688562in}}%
\pgfpathlineto{\pgfqpoint{3.421389in}{2.672507in}}%
\pgfpathlineto{\pgfqpoint{3.438613in}{2.654703in}}%
\pgfpathlineto{\pgfqpoint{3.440202in}{2.653029in}}%
\pgfpathlineto{\pgfqpoint{3.457969in}{2.633552in}}%
\pgfpathlineto{\pgfqpoint{3.471869in}{2.617765in}}%
\pgfpathlineto{\pgfqpoint{3.475060in}{2.614074in}}%
\pgfpathlineto{\pgfqpoint{3.491175in}{2.594596in}}%
\pgfpathlineto{\pgfqpoint{3.505125in}{2.577056in}}%
\pgfpathlineto{\pgfqpoint{3.506638in}{2.575119in}}%
\pgfpathlineto{\pgfqpoint{3.521105in}{2.555641in}}%
\pgfpathlineto{\pgfqpoint{3.534940in}{2.536163in}}%
\pgfpathlineto{\pgfqpoint{3.538381in}{2.531043in}}%
\pgfpathlineto{\pgfqpoint{3.547862in}{2.516686in}}%
\pgfpathlineto{\pgfqpoint{3.560043in}{2.497208in}}%
\pgfpathlineto{\pgfqpoint{3.571569in}{2.477731in}}%
\pgfpathlineto{\pgfqpoint{3.571637in}{2.477607in}}%
\pgfpathlineto{\pgfqpoint{3.582171in}{2.458253in}}%
\pgfpathlineto{\pgfqpoint{3.592191in}{2.438775in}}%
\pgfpathlineto{\pgfqpoint{3.601754in}{2.419298in}}%
\pgfpathlineto{\pgfqpoint{3.604894in}{2.412693in}}%
\pgfpathlineto{\pgfqpoint{3.611157in}{2.399820in}}%
\pgfpathlineto{\pgfqpoint{3.621510in}{2.380342in}}%
\pgfpathlineto{\pgfqpoint{3.635238in}{2.360865in}}%
\pgfpathlineto{\pgfqpoint{3.638150in}{2.357966in}}%
\pgfpathlineto{\pgfqpoint{3.663291in}{2.341387in}}%
\pgfpathlineto{\pgfqpoint{3.671406in}{2.338069in}}%
\pgfpathlineto{\pgfqpoint{3.704662in}{2.329742in}}%
\pgfpathlineto{\pgfqpoint{3.737918in}{2.324522in}}%
\pgfpathlineto{\pgfqpoint{3.758235in}{2.321909in}}%
\pgfpathlineto{\pgfqpoint{3.771175in}{2.320320in}}%
\pgfpathlineto{\pgfqpoint{3.804431in}{2.316224in}}%
\pgfpathlineto{\pgfqpoint{3.837687in}{2.311921in}}%
\pgfpathlineto{\pgfqpoint{3.870943in}{2.307294in}}%
\pgfpathlineto{\pgfqpoint{3.903278in}{2.302432in}}%
\pgfpathlineto{\pgfqpoint{3.904199in}{2.302291in}}%
\pgfpathlineto{\pgfqpoint{3.937455in}{2.296771in}}%
\pgfpathlineto{\pgfqpoint{3.970712in}{2.290853in}}%
\pgfpathlineto{\pgfqpoint{4.003968in}{2.284542in}}%
\pgfpathlineto{\pgfqpoint{4.011803in}{2.282954in}}%
\pgfpathlineto{\pgfqpoint{4.037224in}{2.277717in}}%
\pgfpathlineto{\pgfqpoint{4.070480in}{2.270468in}}%
\pgfpathlineto{\pgfqpoint{4.100959in}{2.263477in}}%
\pgfpathlineto{\pgfqpoint{4.103736in}{2.262829in}}%
\pgfpathlineto{\pgfqpoint{4.136993in}{2.254638in}}%
\pgfpathlineto{\pgfqpoint{4.170249in}{2.246086in}}%
\pgfpathlineto{\pgfqpoint{4.177992in}{2.243999in}}%
\pgfpathlineto{\pgfqpoint{4.203505in}{2.237006in}}%
\pgfpathlineto{\pgfqpoint{4.236761in}{2.227525in}}%
\pgfpathlineto{\pgfqpoint{4.246874in}{2.224521in}}%
\pgfpathlineto{\pgfqpoint{4.270017in}{2.217529in}}%
\pgfpathlineto{\pgfqpoint{4.303273in}{2.207120in}}%
\pgfpathlineto{\pgfqpoint{4.309661in}{2.205044in}}%
\pgfpathlineto{\pgfqpoint{4.336530in}{2.196160in}}%
\pgfpathlineto{\pgfqpoint{4.367593in}{2.185566in}}%
\pgfpathlineto{\pgfqpoint{4.369786in}{2.184805in}}%
\pgfpathlineto{\pgfqpoint{4.403042in}{2.172852in}}%
\pgfpathlineto{\pgfqpoint{4.421317in}{2.166088in}}%
\pgfpathlineto{\pgfqpoint{4.436298in}{2.160446in}}%
\pgfpathlineto{\pgfqpoint{4.469554in}{2.147557in}}%
\pgfpathlineto{\pgfqpoint{4.471923in}{2.146611in}}%
\pgfpathlineto{\pgfqpoint{4.502811in}{2.134047in}}%
\pgfpathlineto{\pgfqpoint{4.519370in}{2.127133in}}%
\pgfpathlineto{\pgfqpoint{4.536067in}{2.120036in}}%
\pgfpathlineto{\pgfqpoint{4.564463in}{2.107655in}}%
\pgfpathlineto{\pgfqpoint{4.569323in}{2.105497in}}%
\pgfpathlineto{\pgfqpoint{4.602579in}{2.090339in}}%
\pgfpathlineto{\pgfqpoint{4.607203in}{2.088178in}}%
\pgfpathlineto{\pgfqpoint{4.635835in}{2.074545in}}%
\pgfpathlineto{\pgfqpoint{4.647827in}{2.068700in}}%
\pgfpathlineto{\pgfqpoint{4.669091in}{2.058140in}}%
\pgfpathlineto{\pgfqpoint{4.686643in}{2.049222in}}%
\pgfpathlineto{\pgfqpoint{4.702348in}{2.041090in}}%
\pgfpathlineto{\pgfqpoint{4.723772in}{2.029745in}}%
\pgfpathlineto{\pgfqpoint{4.735604in}{2.023356in}}%
\pgfpathlineto{\pgfqpoint{4.759321in}{2.010267in}}%
\pgfpathlineto{\pgfqpoint{4.768860in}{2.004897in}}%
\pgfpathlineto{\pgfqpoint{4.793392in}{1.990790in}}%
\pgfpathlineto{\pgfqpoint{4.802116in}{1.985670in}}%
\pgfpathlineto{\pgfqpoint{4.826076in}{1.971312in}}%
\pgfpathlineto{\pgfqpoint{4.835372in}{1.965624in}}%
\pgfpathlineto{\pgfqpoint{4.857457in}{1.951834in}}%
\pgfpathlineto{\pgfqpoint{4.868629in}{1.944709in}}%
\pgfpathlineto{\pgfqpoint{4.887613in}{1.932357in}}%
\pgfpathlineto{\pgfqpoint{4.901885in}{1.922867in}}%
\pgfpathlineto{\pgfqpoint{4.916615in}{1.912879in}}%
\pgfpathlineto{\pgfqpoint{4.935141in}{1.900037in}}%
\pgfpathlineto{\pgfqpoint{4.944530in}{1.893401in}}%
\pgfpathlineto{\pgfqpoint{4.968397in}{1.876149in}}%
\pgfpathlineto{\pgfqpoint{4.971418in}{1.873924in}}%
\pgfpathlineto{\pgfqpoint{4.997214in}{1.854446in}}%
\pgfpathlineto{\pgfqpoint{5.001653in}{1.851009in}}%
\pgfpathlineto{\pgfqpoint{5.021986in}{1.834968in}}%
\pgfpathlineto{\pgfqpoint{5.034909in}{1.824521in}}%
\pgfpathlineto{\pgfqpoint{5.045875in}{1.815491in}}%
\pgfpathlineto{\pgfqpoint{5.068166in}{1.796668in}}%
\pgfpathlineto{\pgfqpoint{5.068927in}{1.796013in}}%
\pgfpathlineto{\pgfqpoint{5.090908in}{1.776536in}}%
\pgfpathlineto{\pgfqpoint{5.101422in}{1.766963in}}%
\pgfpathlineto{\pgfqpoint{5.112107in}{1.757058in}}%
\pgfpathlineto{\pgfqpoint{5.132529in}{1.737580in}}%
\pgfpathlineto{\pgfqpoint{5.134678in}{1.735460in}}%
\pgfpathlineto{\pgfqpoint{5.151966in}{1.718103in}}%
\pgfpathlineto{\pgfqpoint{5.167934in}{1.701589in}}%
\pgfpathlineto{\pgfqpoint{5.170750in}{1.698625in}}%
\pgfpathlineto{\pgfqpoint{5.188591in}{1.679147in}}%
\pgfpathlineto{\pgfqpoint{5.201190in}{1.664924in}}%
\pgfpathlineto{\pgfqpoint{5.205765in}{1.659670in}}%
\pgfpathlineto{\pgfqpoint{5.222062in}{1.640192in}}%
\pgfpathlineto{\pgfqpoint{5.234447in}{1.624828in}}%
\pgfpathlineto{\pgfqpoint{5.237706in}{1.620714in}}%
\pgfpathlineto{\pgfqpoint{5.252461in}{1.601237in}}%
\pgfpathlineto{\pgfqpoint{5.266625in}{1.581759in}}%
\pgfpathlineto{\pgfqpoint{5.267703in}{1.580198in}}%
\pgfpathlineto{\pgfqpoint{5.279863in}{1.562281in}}%
\pgfpathlineto{\pgfqpoint{5.292461in}{1.542804in}}%
\pgfpathlineto{\pgfqpoint{5.300959in}{1.528941in}}%
\pgfpathlineto{\pgfqpoint{5.304343in}{1.523326in}}%
\pgfpathlineto{\pgfqpoint{5.315376in}{1.503849in}}%
\pgfpathlineto{\pgfqpoint{5.325760in}{1.484371in}}%
\pgfpathlineto{\pgfqpoint{5.334215in}{1.467401in}}%
\pgfpathlineto{\pgfqpoint{5.335443in}{1.464893in}}%
\pgfpathlineto{\pgfqpoint{5.344241in}{1.445416in}}%
\pgfpathlineto{\pgfqpoint{5.352350in}{1.425938in}}%
\pgfpathlineto{\pgfqpoint{5.359750in}{1.406460in}}%
\pgfpathlineto{\pgfqpoint{5.366424in}{1.386983in}}%
\pgfpathlineto{\pgfqpoint{5.367471in}{1.383517in}}%
\pgfpathlineto{\pgfqpoint{5.372225in}{1.367505in}}%
\pgfpathlineto{\pgfqpoint{5.377250in}{1.348027in}}%
\pgfpathlineto{\pgfqpoint{5.381505in}{1.328550in}}%
\pgfpathlineto{\pgfqpoint{5.384971in}{1.309072in}}%
\pgfpathlineto{\pgfqpoint{5.387624in}{1.289595in}}%
\pgfpathlineto{\pgfqpoint{5.389444in}{1.270117in}}%
\pgfpathlineto{\pgfqpoint{5.390408in}{1.250639in}}%
\pgfpathlineto{\pgfqpoint{5.390490in}{1.231162in}}%
\pgfpathlineto{\pgfqpoint{5.389666in}{1.211684in}}%
\pgfpathlineto{\pgfqpoint{5.387911in}{1.192206in}}%
\pgfpathlineto{\pgfqpoint{5.385198in}{1.172729in}}%
\pgfpathlineto{\pgfqpoint{5.381498in}{1.153251in}}%
\pgfpathlineto{\pgfqpoint{5.376784in}{1.133773in}}%
\pgfpathlineto{\pgfqpoint{5.371024in}{1.114296in}}%
\pgfpathlineto{\pgfqpoint{5.367471in}{1.104124in}}%
\pgfpathlineto{\pgfqpoint{5.364088in}{1.094818in}}%
\pgfpathlineto{\pgfqpoint{5.355898in}{1.075340in}}%
\pgfpathlineto{\pgfqpoint{5.346525in}{1.055863in}}%
\pgfpathlineto{\pgfqpoint{5.335930in}{1.036385in}}%
\pgfpathlineto{\pgfqpoint{5.334215in}{1.033541in}}%
\pgfpathlineto{\pgfqpoint{5.323743in}{1.016908in}}%
\pgfpathlineto{\pgfqpoint{5.310148in}{0.997430in}}%
\pgfpathlineto{\pgfqpoint{5.300959in}{0.985442in}}%
\pgfpathlineto{\pgfqpoint{5.294952in}{0.977952in}}%
\pgfpathlineto{\pgfqpoint{5.277931in}{0.958475in}}%
\pgfpathlineto{\pgfqpoint{5.267703in}{0.947698in}}%
\pgfpathlineto{\pgfqpoint{5.259038in}{0.938997in}}%
\pgfpathlineto{\pgfqpoint{5.238075in}{0.919519in}}%
\pgfpathlineto{\pgfqpoint{5.234447in}{0.916377in}}%
\pgfpathlineto{\pgfqpoint{5.214598in}{0.900042in}}%
\pgfpathlineto{\pgfqpoint{5.201190in}{0.889754in}}%
\pgfpathlineto{\pgfqpoint{5.188549in}{0.880564in}}%
\pgfpathlineto{\pgfqpoint{5.167934in}{0.866527in}}%
\pgfpathlineto{\pgfqpoint{5.159476in}{0.861086in}}%
\pgfpathlineto{\pgfqpoint{5.134678in}{0.846087in}}%
\pgfpathlineto{\pgfqpoint{5.126814in}{0.841609in}}%
\pgfpathlineto{\pgfqpoint{5.101422in}{0.827964in}}%
\pgfpathlineto{\pgfqpoint{5.089845in}{0.822131in}}%
\pgfpathlineto{\pgfqpoint{5.068166in}{0.811789in}}%
\pgfpathlineto{\pgfqpoint{5.047653in}{0.802653in}}%
\pgfpathlineto{\pgfqpoint{5.034909in}{0.797265in}}%
\pgfpathlineto{\pgfqpoint{5.001653in}{0.784191in}}%
\pgfpathlineto{\pgfqpoint{4.998876in}{0.783176in}}%
\pgfpathlineto{\pgfqpoint{4.968397in}{0.772566in}}%
\pgfpathlineto{\pgfqpoint{4.940671in}{0.763698in}}%
\pgfpathlineto{\pgfqpoint{4.935141in}{0.762010in}}%
\pgfpathlineto{\pgfqpoint{4.901885in}{0.752656in}}%
\pgfpathlineto{\pgfqpoint{4.868821in}{0.744221in}}%
\pgfpathlineto{\pgfqpoint{4.868629in}{0.744174in}}%
\pgfpathlineto{\pgfqpoint{4.835372in}{0.736770in}}%
\pgfpathlineto{\pgfqpoint{4.802116in}{0.730148in}}%
\pgfpathlineto{\pgfqpoint{4.771495in}{0.724743in}}%
\pgfpathlineto{\pgfqpoint{4.768860in}{0.724296in}}%
\pgfpathlineto{\pgfqpoint{4.735604in}{0.719315in}}%
\pgfpathlineto{\pgfqpoint{4.702348in}{0.715019in}}%
\pgfpathlineto{\pgfqpoint{4.669091in}{0.711388in}}%
\pgfpathlineto{\pgfqpoint{4.635835in}{0.708402in}}%
\pgfpathlineto{\pgfqpoint{4.602579in}{0.706042in}}%
\pgfpathlineto{\pgfqpoint{4.587910in}{0.705265in}}%
\pgfpathlineto{\pgfqpoint{4.569323in}{0.704317in}}%
\pgfpathlineto{\pgfqpoint{4.536067in}{0.703186in}}%
\pgfpathlineto{\pgfqpoint{4.502811in}{0.702613in}}%
\pgfpathlineto{\pgfqpoint{4.469554in}{0.702581in}}%
\pgfpathlineto{\pgfqpoint{4.436298in}{0.703076in}}%
\pgfpathlineto{\pgfqpoint{4.403042in}{0.704084in}}%
\pgfpathlineto{\pgfqpoint{4.377033in}{0.705265in}}%
\pgfpathlineto{\pgfqpoint{4.369786in}{0.705600in}}%
\pgfpathlineto{\pgfqpoint{4.336530in}{0.707645in}}%
\pgfpathlineto{\pgfqpoint{4.303273in}{0.710175in}}%
\pgfpathlineto{\pgfqpoint{4.270017in}{0.713176in}}%
\pgfpathlineto{\pgfqpoint{4.236761in}{0.716635in}}%
\pgfpathlineto{\pgfqpoint{4.203505in}{0.720541in}}%
\pgfpathlineto{\pgfqpoint{4.171324in}{0.724743in}}%
\pgfpathlineto{\pgfqpoint{4.170249in}{0.724886in}}%
\pgfpathlineto{\pgfqpoint{4.136993in}{0.729773in}}%
\pgfpathlineto{\pgfqpoint{4.103736in}{0.735081in}}%
\pgfpathlineto{\pgfqpoint{4.070480in}{0.740800in}}%
\pgfpathlineto{\pgfqpoint{4.051955in}{0.744221in}}%
\pgfpathlineto{\pgfqpoint{4.037224in}{0.746987in}}%
\pgfpathlineto{\pgfqpoint{4.003968in}{0.753659in}}%
\pgfpathlineto{\pgfqpoint{3.970712in}{0.760717in}}%
\pgfpathlineto{\pgfqpoint{3.957443in}{0.763698in}}%
\pgfpathlineto{\pgfqpoint{3.937455in}{0.768265in}}%
\pgfpathlineto{\pgfqpoint{3.904199in}{0.776261in}}%
\pgfpathlineto{\pgfqpoint{3.876714in}{0.783176in}}%
\pgfpathlineto{\pgfqpoint{3.870943in}{0.784653in}}%
\pgfpathlineto{\pgfqpoint{3.837687in}{0.793573in}}%
\pgfpathlineto{\pgfqpoint{3.805019in}{0.802653in}}%
\pgfpathlineto{\pgfqpoint{3.804431in}{0.802820in}}%
\pgfpathlineto{\pgfqpoint{3.771175in}{0.812575in}}%
\pgfpathlineto{\pgfqpoint{3.739128in}{0.822131in}}%
\pgfpathlineto{\pgfqpoint{3.737918in}{0.822500in}}%
\pgfpathlineto{\pgfqpoint{3.704662in}{0.832514in}}%
\pgfpathlineto{\pgfqpoint{3.671835in}{0.841609in}}%
\pgfpathlineto{\pgfqpoint{3.671406in}{0.841733in}}%
\pgfpathlineto{\pgfqpoint{3.638150in}{0.848550in}}%
\pgfpathlineto{\pgfqpoint{3.604894in}{0.849514in}}%
\pgfpathlineto{\pgfqpoint{3.573857in}{0.841609in}}%
\pgfpathlineto{\pgfqpoint{3.571637in}{0.840961in}}%
\pgfpathlineto{\pgfqpoint{3.538381in}{0.824927in}}%
\pgfpathlineto{\pgfqpoint{3.533245in}{0.822131in}}%
\pgfpathlineto{\pgfqpoint{3.505125in}{0.807272in}}%
\pgfpathlineto{\pgfqpoint{3.496208in}{0.802653in}}%
\pgfpathlineto{\pgfqpoint{3.471869in}{0.790639in}}%
\pgfpathlineto{\pgfqpoint{3.455889in}{0.783176in}}%
\pgfpathlineto{\pgfqpoint{3.438613in}{0.775500in}}%
\pgfpathlineto{\pgfqpoint{3.410218in}{0.763698in}}%
\pgfpathlineto{\pgfqpoint{3.405357in}{0.761773in}}%
\pgfpathlineto{\pgfqpoint{3.372100in}{0.749472in}}%
\pgfpathlineto{\pgfqpoint{3.356781in}{0.744221in}}%
\pgfpathlineto{\pgfqpoint{3.338844in}{0.738347in}}%
\pgfpathlineto{\pgfqpoint{3.305588in}{0.728302in}}%
\pgfpathlineto{\pgfqpoint{3.292771in}{0.724743in}}%
\pgfpathlineto{\pgfqpoint{3.272332in}{0.719307in}}%
\pgfpathlineto{\pgfqpoint{3.239076in}{0.711235in}}%
\pgfpathlineto{\pgfqpoint{3.211879in}{0.705265in}}%
\pgfpathlineto{\pgfqpoint{3.205819in}{0.703988in}}%
\pgfpathlineto{\pgfqpoint{3.172563in}{0.697655in}}%
\pgfpathlineto{\pgfqpoint{3.139307in}{0.692033in}}%
\pgfpathlineto{\pgfqpoint{3.106051in}{0.687101in}}%
\pgfpathlineto{\pgfqpoint{3.095872in}{0.685788in}}%
\pgfpathlineto{\pgfqpoint{3.072795in}{0.682921in}}%
\pgfpathlineto{\pgfqpoint{3.039539in}{0.679407in}}%
\pgfpathlineto{\pgfqpoint{3.006282in}{0.676504in}}%
\pgfpathlineto{\pgfqpoint{2.973026in}{0.674196in}}%
\pgfpathlineto{\pgfqpoint{2.939770in}{0.672466in}}%
\pgfpathlineto{\pgfqpoint{2.906514in}{0.671297in}}%
\pgfpathlineto{\pgfqpoint{2.873258in}{0.670676in}}%
\pgfpathlineto{\pgfqpoint{2.840001in}{0.670586in}}%
\pgfpathlineto{\pgfqpoint{2.806745in}{0.671013in}}%
\pgfpathlineto{\pgfqpoint{2.773489in}{0.671944in}}%
\pgfpathlineto{\pgfqpoint{2.740233in}{0.673366in}}%
\pgfpathlineto{\pgfqpoint{2.706977in}{0.675265in}}%
\pgfpathlineto{\pgfqpoint{2.673720in}{0.677630in}}%
\pgfpathlineto{\pgfqpoint{2.640464in}{0.680448in}}%
\pgfpathlineto{\pgfqpoint{2.607208in}{0.683708in}}%
\pgfpathclose%
\pgfusepath{stroke}%
\end{pgfscope}%
\begin{pgfscope}%
\pgfpathrectangle{\pgfqpoint{0.711606in}{0.549444in}}{\pgfqpoint{4.955171in}{2.902168in}}%
\pgfusepath{clip}%
\pgfsetbuttcap%
\pgfsetroundjoin%
\pgfsetlinewidth{1.003750pt}%
\definecolor{currentstroke}{rgb}{0.129285,0.047293,0.290788}%
\pgfsetstrokecolor{currentstroke}%
\pgfsetdash{}{0pt}%
\pgfpathmoveto{\pgfqpoint{2.673720in}{0.626091in}}%
\pgfpathlineto{\pgfqpoint{2.659935in}{0.627355in}}%
\pgfpathlineto{\pgfqpoint{2.640464in}{0.629167in}}%
\pgfpathlineto{\pgfqpoint{2.607208in}{0.632688in}}%
\pgfpathlineto{\pgfqpoint{2.573952in}{0.636616in}}%
\pgfpathlineto{\pgfqpoint{2.540696in}{0.640941in}}%
\pgfpathlineto{\pgfqpoint{2.507440in}{0.645654in}}%
\pgfpathlineto{\pgfqpoint{2.499785in}{0.646832in}}%
\pgfpathlineto{\pgfqpoint{2.474183in}{0.650835in}}%
\pgfpathlineto{\pgfqpoint{2.440927in}{0.656420in}}%
\pgfpathlineto{\pgfqpoint{2.407671in}{0.662373in}}%
\pgfpathlineto{\pgfqpoint{2.386983in}{0.666310in}}%
\pgfpathlineto{\pgfqpoint{2.374415in}{0.668739in}}%
\pgfpathlineto{\pgfqpoint{2.341159in}{0.675553in}}%
\pgfpathlineto{\pgfqpoint{2.307902in}{0.682714in}}%
\pgfpathlineto{\pgfqpoint{2.294339in}{0.685788in}}%
\pgfpathlineto{\pgfqpoint{2.274646in}{0.690318in}}%
\pgfpathlineto{\pgfqpoint{2.241390in}{0.698330in}}%
\pgfpathlineto{\pgfqpoint{2.213773in}{0.705265in}}%
\pgfpathlineto{\pgfqpoint{2.208134in}{0.706703in}}%
\pgfpathlineto{\pgfqpoint{2.174878in}{0.715565in}}%
\pgfpathlineto{\pgfqpoint{2.141623in}{0.724743in}}%
\pgfpathlineto{\pgfqpoint{2.141622in}{0.724743in}}%
\pgfpathlineto{\pgfqpoint{2.108365in}{0.734454in}}%
\pgfpathlineto{\pgfqpoint{2.075943in}{0.744221in}}%
\pgfpathlineto{\pgfqpoint{2.075109in}{0.744476in}}%
\pgfpathlineto{\pgfqpoint{2.041853in}{0.755033in}}%
\pgfpathlineto{\pgfqpoint{2.015342in}{0.763698in}}%
\pgfpathlineto{\pgfqpoint{2.008597in}{0.765937in}}%
\pgfpathlineto{\pgfqpoint{1.975341in}{0.777340in}}%
\pgfpathlineto{\pgfqpoint{1.958802in}{0.783176in}}%
\pgfpathlineto{\pgfqpoint{1.942084in}{0.789167in}}%
\pgfpathlineto{\pgfqpoint{1.908828in}{0.801414in}}%
\pgfpathlineto{\pgfqpoint{1.905560in}{0.802653in}}%
\pgfpathlineto{\pgfqpoint{1.875572in}{0.814206in}}%
\pgfpathlineto{\pgfqpoint{1.855500in}{0.822131in}}%
\pgfpathlineto{\pgfqpoint{1.842316in}{0.827421in}}%
\pgfpathlineto{\pgfqpoint{1.809060in}{0.841093in}}%
\pgfpathlineto{\pgfqpoint{1.807839in}{0.841609in}}%
\pgfpathlineto{\pgfqpoint{1.775804in}{0.855353in}}%
\pgfpathlineto{\pgfqpoint{1.762740in}{0.861086in}}%
\pgfpathlineto{\pgfqpoint{1.742547in}{0.870095in}}%
\pgfpathlineto{\pgfqpoint{1.719596in}{0.880564in}}%
\pgfpathlineto{\pgfqpoint{1.709291in}{0.885344in}}%
\pgfpathlineto{\pgfqpoint{1.678283in}{0.900042in}}%
\pgfpathlineto{\pgfqpoint{1.676035in}{0.901125in}}%
\pgfpathlineto{\pgfqpoint{1.642779in}{0.917521in}}%
\pgfpathlineto{\pgfqpoint{1.638810in}{0.919519in}}%
\pgfpathlineto{\pgfqpoint{1.609523in}{0.934522in}}%
\pgfpathlineto{\pgfqpoint{1.600961in}{0.938997in}}%
\pgfpathlineto{\pgfqpoint{1.576266in}{0.952131in}}%
\pgfpathlineto{\pgfqpoint{1.564572in}{0.958475in}}%
\pgfpathlineto{\pgfqpoint{1.543010in}{0.970380in}}%
\pgfpathlineto{\pgfqpoint{1.529561in}{0.977952in}}%
\pgfpathlineto{\pgfqpoint{1.509754in}{0.989306in}}%
\pgfpathlineto{\pgfqpoint{1.495849in}{0.997430in}}%
\pgfpathlineto{\pgfqpoint{1.476498in}{1.008945in}}%
\pgfpathlineto{\pgfqpoint{1.463364in}{1.016908in}}%
\pgfpathlineto{\pgfqpoint{1.443242in}{1.029337in}}%
\pgfpathlineto{\pgfqpoint{1.432039in}{1.036385in}}%
\pgfpathlineto{\pgfqpoint{1.409986in}{1.050527in}}%
\pgfpathlineto{\pgfqpoint{1.401813in}{1.055863in}}%
\pgfpathlineto{\pgfqpoint{1.376729in}{1.072561in}}%
\pgfpathlineto{\pgfqpoint{1.372627in}{1.075340in}}%
\pgfpathlineto{\pgfqpoint{1.344453in}{1.094818in}}%
\pgfpathlineto{\pgfqpoint{1.343473in}{1.095510in}}%
\pgfpathlineto{\pgfqpoint{1.317347in}{1.114296in}}%
\pgfpathlineto{\pgfqpoint{1.310217in}{1.119531in}}%
\pgfpathlineto{\pgfqpoint{1.291149in}{1.133773in}}%
\pgfpathlineto{\pgfqpoint{1.276961in}{1.144599in}}%
\pgfpathlineto{\pgfqpoint{1.265812in}{1.153251in}}%
\pgfpathlineto{\pgfqpoint{1.243705in}{1.170785in}}%
\pgfpathlineto{\pgfqpoint{1.241295in}{1.172729in}}%
\pgfpathlineto{\pgfqpoint{1.217735in}{1.192206in}}%
\pgfpathlineto{\pgfqpoint{1.210448in}{1.198376in}}%
\pgfpathlineto{\pgfqpoint{1.194990in}{1.211684in}}%
\pgfpathlineto{\pgfqpoint{1.177192in}{1.227370in}}%
\pgfpathlineto{\pgfqpoint{1.172960in}{1.231162in}}%
\pgfpathlineto{\pgfqpoint{1.151800in}{1.250639in}}%
\pgfpathlineto{\pgfqpoint{1.143936in}{1.258074in}}%
\pgfpathlineto{\pgfqpoint{1.131402in}{1.270117in}}%
\pgfpathlineto{\pgfqpoint{1.111652in}{1.289595in}}%
\pgfpathlineto{\pgfqpoint{1.110680in}{1.290585in}}%
\pgfpathlineto{\pgfqpoint{1.092813in}{1.309072in}}%
\pgfpathlineto{\pgfqpoint{1.077424in}{1.325439in}}%
\pgfpathlineto{\pgfqpoint{1.074544in}{1.328550in}}%
\pgfpathlineto{\pgfqpoint{1.057122in}{1.348027in}}%
\pgfpathlineto{\pgfqpoint{1.044168in}{1.362965in}}%
\pgfpathlineto{\pgfqpoint{1.040291in}{1.367505in}}%
\pgfpathlineto{\pgfqpoint{1.024265in}{1.386983in}}%
\pgfpathlineto{\pgfqpoint{1.010911in}{1.403772in}}%
\pgfpathlineto{\pgfqpoint{1.008806in}{1.406460in}}%
\pgfpathlineto{\pgfqpoint{0.994176in}{1.425938in}}%
\pgfpathlineto{\pgfqpoint{0.980081in}{1.445416in}}%
\pgfpathlineto{\pgfqpoint{0.977655in}{1.448927in}}%
\pgfpathlineto{\pgfqpoint{0.966792in}{1.464893in}}%
\pgfpathlineto{\pgfqpoint{0.954112in}{1.484371in}}%
\pgfpathlineto{\pgfqpoint{0.944399in}{1.500012in}}%
\pgfpathlineto{\pgfqpoint{0.942053in}{1.503849in}}%
\pgfpathlineto{\pgfqpoint{0.930787in}{1.523326in}}%
\pgfpathlineto{\pgfqpoint{0.920105in}{1.542804in}}%
\pgfpathlineto{\pgfqpoint{0.911143in}{1.560129in}}%
\pgfpathlineto{\pgfqpoint{0.910047in}{1.562281in}}%
\pgfpathlineto{\pgfqpoint{0.900795in}{1.581759in}}%
\pgfpathlineto{\pgfqpoint{0.892160in}{1.601237in}}%
\pgfpathlineto{\pgfqpoint{0.884155in}{1.620714in}}%
\pgfpathlineto{\pgfqpoint{0.877887in}{1.637325in}}%
\pgfpathlineto{\pgfqpoint{0.876821in}{1.640192in}}%
\pgfpathlineto{\pgfqpoint{0.870281in}{1.659670in}}%
\pgfpathlineto{\pgfqpoint{0.864406in}{1.679147in}}%
\pgfpathlineto{\pgfqpoint{0.859211in}{1.698625in}}%
\pgfpathlineto{\pgfqpoint{0.854712in}{1.718103in}}%
\pgfpathlineto{\pgfqpoint{0.850928in}{1.737580in}}%
\pgfpathlineto{\pgfqpoint{0.847875in}{1.757058in}}%
\pgfpathlineto{\pgfqpoint{0.845572in}{1.776536in}}%
\pgfpathlineto{\pgfqpoint{0.844630in}{1.788523in}}%
\pgfpathlineto{\pgfqpoint{0.844052in}{1.796013in}}%
\pgfpathlineto{\pgfqpoint{0.843324in}{1.815491in}}%
\pgfpathlineto{\pgfqpoint{0.843384in}{1.834968in}}%
\pgfpathlineto{\pgfqpoint{0.844253in}{1.854446in}}%
\pgfpathlineto{\pgfqpoint{0.844630in}{1.858803in}}%
\pgfpathlineto{\pgfqpoint{0.845985in}{1.873924in}}%
\pgfpathlineto{\pgfqpoint{0.848599in}{1.893401in}}%
\pgfpathlineto{\pgfqpoint{0.852109in}{1.912879in}}%
\pgfpathlineto{\pgfqpoint{0.856539in}{1.932357in}}%
\pgfpathlineto{\pgfqpoint{0.861913in}{1.951834in}}%
\pgfpathlineto{\pgfqpoint{0.868258in}{1.971312in}}%
\pgfpathlineto{\pgfqpoint{0.875599in}{1.990790in}}%
\pgfpathlineto{\pgfqpoint{0.877887in}{1.996152in}}%
\pgfpathlineto{\pgfqpoint{0.884135in}{2.010267in}}%
\pgfpathlineto{\pgfqpoint{0.893822in}{2.029745in}}%
\pgfpathlineto{\pgfqpoint{0.904628in}{2.049222in}}%
\pgfpathlineto{\pgfqpoint{0.911143in}{2.059882in}}%
\pgfpathlineto{\pgfqpoint{0.916747in}{2.068700in}}%
\pgfpathlineto{\pgfqpoint{0.930286in}{2.088178in}}%
\pgfpathlineto{\pgfqpoint{0.944399in}{2.106754in}}%
\pgfpathlineto{\pgfqpoint{0.945113in}{2.107655in}}%
\pgfpathlineto{\pgfqpoint{0.961734in}{2.127133in}}%
\pgfpathlineto{\pgfqpoint{0.977655in}{2.144363in}}%
\pgfpathlineto{\pgfqpoint{0.979824in}{2.146611in}}%
\pgfpathlineto{\pgfqpoint{0.999941in}{2.166088in}}%
\pgfpathlineto{\pgfqpoint{1.010911in}{2.176002in}}%
\pgfpathlineto{\pgfqpoint{1.021989in}{2.185566in}}%
\pgfpathlineto{\pgfqpoint{1.044168in}{2.203509in}}%
\pgfpathlineto{\pgfqpoint{1.046156in}{2.205044in}}%
\pgfpathlineto{\pgfqpoint{1.072879in}{2.224521in}}%
\pgfpathlineto{\pgfqpoint{1.077424in}{2.227674in}}%
\pgfpathlineto{\pgfqpoint{1.102047in}{2.243999in}}%
\pgfpathlineto{\pgfqpoint{1.110680in}{2.249622in}}%
\pgfpathlineto{\pgfqpoint{1.132551in}{2.263477in}}%
\pgfpathlineto{\pgfqpoint{1.143936in}{2.271545in}}%
\pgfpathlineto{\pgfqpoint{1.159393in}{2.282954in}}%
\pgfpathlineto{\pgfqpoint{1.173313in}{2.302432in}}%
\pgfpathlineto{\pgfqpoint{1.172179in}{2.321909in}}%
\pgfpathlineto{\pgfqpoint{1.162325in}{2.341387in}}%
\pgfpathlineto{\pgfqpoint{1.148898in}{2.360865in}}%
\pgfpathlineto{\pgfqpoint{1.143936in}{2.367461in}}%
\pgfpathlineto{\pgfqpoint{1.134505in}{2.380342in}}%
\pgfpathlineto{\pgfqpoint{1.120143in}{2.399820in}}%
\pgfpathlineto{\pgfqpoint{1.110680in}{2.412976in}}%
\pgfpathlineto{\pgfqpoint{1.106217in}{2.419298in}}%
\pgfpathlineto{\pgfqpoint{1.093022in}{2.438775in}}%
\pgfpathlineto{\pgfqpoint{1.080400in}{2.458253in}}%
\pgfpathlineto{\pgfqpoint{1.077424in}{2.463112in}}%
\pgfpathlineto{\pgfqpoint{1.068620in}{2.477731in}}%
\pgfpathlineto{\pgfqpoint{1.057543in}{2.497208in}}%
\pgfpathlineto{\pgfqpoint{1.047113in}{2.516686in}}%
\pgfpathlineto{\pgfqpoint{1.044168in}{2.522598in}}%
\pgfpathlineto{\pgfqpoint{1.037523in}{2.536163in}}%
\pgfpathlineto{\pgfqpoint{1.028678in}{2.555641in}}%
\pgfpathlineto{\pgfqpoint{1.020522in}{2.575119in}}%
\pgfpathlineto{\pgfqpoint{1.013073in}{2.594596in}}%
\pgfpathlineto{\pgfqpoint{1.010911in}{2.600896in}}%
\pgfpathlineto{\pgfqpoint{1.006467in}{2.614074in}}%
\pgfpathlineto{\pgfqpoint{1.000645in}{2.633552in}}%
\pgfpathlineto{\pgfqpoint{0.995572in}{2.653029in}}%
\pgfpathlineto{\pgfqpoint{0.991267in}{2.672507in}}%
\pgfpathlineto{\pgfqpoint{0.987751in}{2.691985in}}%
\pgfpathlineto{\pgfqpoint{0.985046in}{2.711462in}}%
\pgfpathlineto{\pgfqpoint{0.983174in}{2.730940in}}%
\pgfpathlineto{\pgfqpoint{0.982157in}{2.750418in}}%
\pgfpathlineto{\pgfqpoint{0.982021in}{2.769895in}}%
\pgfpathlineto{\pgfqpoint{0.982789in}{2.789373in}}%
\pgfpathlineto{\pgfqpoint{0.984488in}{2.808850in}}%
\pgfpathlineto{\pgfqpoint{0.987144in}{2.828328in}}%
\pgfpathlineto{\pgfqpoint{0.990785in}{2.847806in}}%
\pgfpathlineto{\pgfqpoint{0.995440in}{2.867283in}}%
\pgfpathlineto{\pgfqpoint{1.001139in}{2.886761in}}%
\pgfpathlineto{\pgfqpoint{1.007913in}{2.906239in}}%
\pgfpathlineto{\pgfqpoint{1.010911in}{2.913696in}}%
\pgfpathlineto{\pgfqpoint{1.015944in}{2.925716in}}%
\pgfpathlineto{\pgfqpoint{1.025249in}{2.945194in}}%
\pgfpathlineto{\pgfqpoint{1.035774in}{2.964672in}}%
\pgfpathlineto{\pgfqpoint{1.044168in}{2.978589in}}%
\pgfpathlineto{\pgfqpoint{1.047668in}{2.984149in}}%
\pgfpathlineto{\pgfqpoint{1.061185in}{3.003627in}}%
\pgfpathlineto{\pgfqpoint{1.076100in}{3.023104in}}%
\pgfpathlineto{\pgfqpoint{1.077424in}{3.024697in}}%
\pgfpathlineto{\pgfqpoint{1.092988in}{3.042582in}}%
\pgfpathlineto{\pgfqpoint{1.110680in}{3.061223in}}%
\pgfpathlineto{\pgfqpoint{1.111513in}{3.062060in}}%
\pgfpathlineto{\pgfqpoint{1.132378in}{3.081537in}}%
\pgfpathlineto{\pgfqpoint{1.143936in}{3.091549in}}%
\pgfpathlineto{\pgfqpoint{1.155438in}{3.101015in}}%
\pgfpathlineto{\pgfqpoint{1.177192in}{3.117707in}}%
\pgfpathlineto{\pgfqpoint{1.181024in}{3.120493in}}%
\pgfpathlineto{\pgfqpoint{1.209617in}{3.139970in}}%
\pgfpathlineto{\pgfqpoint{1.210448in}{3.140503in}}%
\pgfpathlineto{\pgfqpoint{1.241754in}{3.159448in}}%
\pgfpathlineto{\pgfqpoint{1.243705in}{3.160561in}}%
\pgfpathlineto{\pgfqpoint{1.276961in}{3.178416in}}%
\pgfpathlineto{\pgfqpoint{1.277973in}{3.178926in}}%
\pgfpathlineto{\pgfqpoint{1.310217in}{3.194308in}}%
\pgfpathlineto{\pgfqpoint{1.319414in}{3.198403in}}%
\pgfpathlineto{\pgfqpoint{1.343473in}{3.208576in}}%
\pgfpathlineto{\pgfqpoint{1.367165in}{3.217881in}}%
\pgfpathlineto{\pgfqpoint{1.376729in}{3.221457in}}%
\pgfpathlineto{\pgfqpoint{1.409986in}{3.232990in}}%
\pgfpathlineto{\pgfqpoint{1.423644in}{3.237359in}}%
\pgfpathlineto{\pgfqpoint{1.443242in}{3.243342in}}%
\pgfpathlineto{\pgfqpoint{1.476498in}{3.252646in}}%
\pgfpathlineto{\pgfqpoint{1.492921in}{3.256836in}}%
\pgfpathlineto{\pgfqpoint{1.509754in}{3.260946in}}%
\pgfpathlineto{\pgfqpoint{1.543010in}{3.268305in}}%
\pgfpathlineto{\pgfqpoint{1.576266in}{3.274884in}}%
\pgfpathlineto{\pgfqpoint{1.584366in}{3.276314in}}%
\pgfpathlineto{\pgfqpoint{1.609523in}{3.280576in}}%
\pgfpathlineto{\pgfqpoint{1.642779in}{3.285518in}}%
\pgfpathlineto{\pgfqpoint{1.676035in}{3.289775in}}%
\pgfpathlineto{\pgfqpoint{1.709291in}{3.293369in}}%
\pgfpathlineto{\pgfqpoint{1.736559in}{3.295791in}}%
\pgfpathlineto{\pgfqpoint{1.742547in}{3.296303in}}%
\pgfpathlineto{\pgfqpoint{1.775804in}{3.298563in}}%
\pgfpathlineto{\pgfqpoint{1.809060in}{3.300234in}}%
\pgfpathlineto{\pgfqpoint{1.842316in}{3.301332in}}%
\pgfpathlineto{\pgfqpoint{1.875572in}{3.301873in}}%
\pgfpathlineto{\pgfqpoint{1.908828in}{3.301873in}}%
\pgfpathlineto{\pgfqpoint{1.942084in}{3.301347in}}%
\pgfpathlineto{\pgfqpoint{1.975341in}{3.300308in}}%
\pgfpathlineto{\pgfqpoint{2.008597in}{3.298772in}}%
\pgfpathlineto{\pgfqpoint{2.041853in}{3.296750in}}%
\pgfpathlineto{\pgfqpoint{2.054557in}{3.295791in}}%
\pgfpathlineto{\pgfqpoint{2.075109in}{3.294215in}}%
\pgfpathlineto{\pgfqpoint{2.108365in}{3.291184in}}%
\pgfpathlineto{\pgfqpoint{2.141622in}{3.287696in}}%
\pgfpathlineto{\pgfqpoint{2.174878in}{3.283761in}}%
\pgfpathlineto{\pgfqpoint{2.208134in}{3.279391in}}%
\pgfpathlineto{\pgfqpoint{2.229424in}{3.276314in}}%
\pgfpathlineto{\pgfqpoint{2.241390in}{3.274555in}}%
\pgfpathlineto{\pgfqpoint{2.274646in}{3.269218in}}%
\pgfpathlineto{\pgfqpoint{2.307902in}{3.263471in}}%
\pgfpathlineto{\pgfqpoint{2.341159in}{3.257325in}}%
\pgfpathlineto{\pgfqpoint{2.343629in}{3.256836in}}%
\pgfpathlineto{\pgfqpoint{2.374415in}{3.250637in}}%
\pgfpathlineto{\pgfqpoint{2.407671in}{3.243551in}}%
\pgfpathlineto{\pgfqpoint{2.435226in}{3.237359in}}%
\pgfpathlineto{\pgfqpoint{2.440927in}{3.236056in}}%
\pgfpathlineto{\pgfqpoint{2.474183in}{3.228029in}}%
\pgfpathlineto{\pgfqpoint{2.507440in}{3.219640in}}%
\pgfpathlineto{\pgfqpoint{2.514085in}{3.217881in}}%
\pgfpathlineto{\pgfqpoint{2.540696in}{3.210719in}}%
\pgfpathlineto{\pgfqpoint{2.573952in}{3.201404in}}%
\pgfpathlineto{\pgfqpoint{2.584227in}{3.198403in}}%
\pgfpathlineto{\pgfqpoint{2.607208in}{3.191578in}}%
\pgfpathlineto{\pgfqpoint{2.640464in}{3.181338in}}%
\pgfpathlineto{\pgfqpoint{2.648007in}{3.178926in}}%
\pgfpathlineto{\pgfqpoint{2.673720in}{3.170560in}}%
\pgfpathlineto{\pgfqpoint{2.706828in}{3.159448in}}%
\pgfpathlineto{\pgfqpoint{2.706977in}{3.159397in}}%
\pgfpathlineto{\pgfqpoint{2.740233in}{3.147622in}}%
\pgfpathlineto{\pgfqpoint{2.761223in}{3.139970in}}%
\pgfpathlineto{\pgfqpoint{2.773489in}{3.135421in}}%
\pgfpathlineto{\pgfqpoint{2.806745in}{3.122713in}}%
\pgfpathlineto{\pgfqpoint{2.812384in}{3.120493in}}%
\pgfpathlineto{\pgfqpoint{2.840001in}{3.109423in}}%
\pgfpathlineto{\pgfqpoint{2.860435in}{3.101015in}}%
\pgfpathlineto{\pgfqpoint{2.873258in}{3.095644in}}%
\pgfpathlineto{\pgfqpoint{2.906081in}{3.081537in}}%
\pgfpathlineto{\pgfqpoint{2.906514in}{3.081348in}}%
\pgfpathlineto{\pgfqpoint{2.939770in}{3.066384in}}%
\pgfpathlineto{\pgfqpoint{2.949151in}{3.062060in}}%
\pgfpathlineto{\pgfqpoint{2.973026in}{3.050851in}}%
\pgfpathlineto{\pgfqpoint{2.990229in}{3.042582in}}%
\pgfpathlineto{\pgfqpoint{3.006282in}{3.034721in}}%
\pgfpathlineto{\pgfqpoint{3.029465in}{3.023104in}}%
\pgfpathlineto{\pgfqpoint{3.039539in}{3.017960in}}%
\pgfpathlineto{\pgfqpoint{3.066981in}{3.003627in}}%
\pgfpathlineto{\pgfqpoint{3.072795in}{3.000531in}}%
\pgfpathlineto{\pgfqpoint{3.102891in}{2.984149in}}%
\pgfpathlineto{\pgfqpoint{3.106051in}{2.982395in}}%
\pgfpathlineto{\pgfqpoint{3.137295in}{2.964672in}}%
\pgfpathlineto{\pgfqpoint{3.139307in}{2.963507in}}%
\pgfpathlineto{\pgfqpoint{3.170290in}{2.945194in}}%
\pgfpathlineto{\pgfqpoint{3.172563in}{2.943822in}}%
\pgfpathlineto{\pgfqpoint{3.201960in}{2.925716in}}%
\pgfpathlineto{\pgfqpoint{3.205819in}{2.923289in}}%
\pgfpathlineto{\pgfqpoint{3.232385in}{2.906239in}}%
\pgfpathlineto{\pgfqpoint{3.239076in}{2.901851in}}%
\pgfpathlineto{\pgfqpoint{3.261638in}{2.886761in}}%
\pgfpathlineto{\pgfqpoint{3.272332in}{2.879451in}}%
\pgfpathlineto{\pgfqpoint{3.289788in}{2.867283in}}%
\pgfpathlineto{\pgfqpoint{3.305588in}{2.856021in}}%
\pgfpathlineto{\pgfqpoint{3.316896in}{2.847806in}}%
\pgfpathlineto{\pgfqpoint{3.338844in}{2.831490in}}%
\pgfpathlineto{\pgfqpoint{3.343019in}{2.828328in}}%
\pgfpathlineto{\pgfqpoint{3.368104in}{2.808850in}}%
\pgfpathlineto{\pgfqpoint{3.372100in}{2.805666in}}%
\pgfpathlineto{\pgfqpoint{3.392172in}{2.789373in}}%
\pgfpathlineto{\pgfqpoint{3.405357in}{2.778400in}}%
\pgfpathlineto{\pgfqpoint{3.415392in}{2.769895in}}%
\pgfpathlineto{\pgfqpoint{3.437787in}{2.750418in}}%
\pgfpathlineto{\pgfqpoint{3.438613in}{2.749677in}}%
\pgfpathlineto{\pgfqpoint{3.459131in}{2.730940in}}%
\pgfpathlineto{\pgfqpoint{3.471869in}{2.718986in}}%
\pgfpathlineto{\pgfqpoint{3.479746in}{2.711462in}}%
\pgfpathlineto{\pgfqpoint{3.499522in}{2.691985in}}%
\pgfpathlineto{\pgfqpoint{3.505125in}{2.686282in}}%
\pgfpathlineto{\pgfqpoint{3.518424in}{2.672507in}}%
\pgfpathlineto{\pgfqpoint{3.536654in}{2.653029in}}%
\pgfpathlineto{\pgfqpoint{3.538381in}{2.651112in}}%
\pgfpathlineto{\pgfqpoint{3.553930in}{2.633552in}}%
\pgfpathlineto{\pgfqpoint{3.570614in}{2.614074in}}%
\pgfpathlineto{\pgfqpoint{3.571637in}{2.612828in}}%
\pgfpathlineto{\pgfqpoint{3.586352in}{2.594596in}}%
\pgfpathlineto{\pgfqpoint{3.601498in}{2.575119in}}%
\pgfpathlineto{\pgfqpoint{3.604894in}{2.570549in}}%
\pgfpathlineto{\pgfqpoint{3.615789in}{2.555641in}}%
\pgfpathlineto{\pgfqpoint{3.629422in}{2.536163in}}%
\pgfpathlineto{\pgfqpoint{3.638150in}{2.523104in}}%
\pgfpathlineto{\pgfqpoint{3.642378in}{2.516686in}}%
\pgfpathlineto{\pgfqpoint{3.654590in}{2.497208in}}%
\pgfpathlineto{\pgfqpoint{3.666329in}{2.477731in}}%
\pgfpathlineto{\pgfqpoint{3.671406in}{2.468991in}}%
\pgfpathlineto{\pgfqpoint{3.677680in}{2.458253in}}%
\pgfpathlineto{\pgfqpoint{3.689238in}{2.438775in}}%
\pgfpathlineto{\pgfqpoint{3.702297in}{2.419298in}}%
\pgfpathlineto{\pgfqpoint{3.704662in}{2.416400in}}%
\pgfpathlineto{\pgfqpoint{3.721768in}{2.399820in}}%
\pgfpathlineto{\pgfqpoint{3.737918in}{2.389839in}}%
\pgfpathlineto{\pgfqpoint{3.762857in}{2.380342in}}%
\pgfpathlineto{\pgfqpoint{3.771175in}{2.377973in}}%
\pgfpathlineto{\pgfqpoint{3.804431in}{2.371046in}}%
\pgfpathlineto{\pgfqpoint{3.837687in}{2.365455in}}%
\pgfpathlineto{\pgfqpoint{3.866741in}{2.360865in}}%
\pgfpathlineto{\pgfqpoint{3.870943in}{2.360210in}}%
\pgfpathlineto{\pgfqpoint{3.904199in}{2.354814in}}%
\pgfpathlineto{\pgfqpoint{3.937455in}{2.349153in}}%
\pgfpathlineto{\pgfqpoint{3.970712in}{2.343174in}}%
\pgfpathlineto{\pgfqpoint{3.980052in}{2.341387in}}%
\pgfpathlineto{\pgfqpoint{4.003968in}{2.336749in}}%
\pgfpathlineto{\pgfqpoint{4.037224in}{2.329930in}}%
\pgfpathlineto{\pgfqpoint{4.070480in}{2.322759in}}%
\pgfpathlineto{\pgfqpoint{4.074213in}{2.321909in}}%
\pgfpathlineto{\pgfqpoint{4.103736in}{2.315085in}}%
\pgfpathlineto{\pgfqpoint{4.136993in}{2.307047in}}%
\pgfpathlineto{\pgfqpoint{4.155250in}{2.302432in}}%
\pgfpathlineto{\pgfqpoint{4.170249in}{2.298581in}}%
\pgfpathlineto{\pgfqpoint{4.203505in}{2.289669in}}%
\pgfpathlineto{\pgfqpoint{4.227625in}{2.282954in}}%
\pgfpathlineto{\pgfqpoint{4.236761in}{2.280370in}}%
\pgfpathlineto{\pgfqpoint{4.270017in}{2.270585in}}%
\pgfpathlineto{\pgfqpoint{4.293366in}{2.263477in}}%
\pgfpathlineto{\pgfqpoint{4.303273in}{2.260412in}}%
\pgfpathlineto{\pgfqpoint{4.336530in}{2.249753in}}%
\pgfpathlineto{\pgfqpoint{4.353919in}{2.243999in}}%
\pgfpathlineto{\pgfqpoint{4.369786in}{2.238664in}}%
\pgfpathlineto{\pgfqpoint{4.403042in}{2.227134in}}%
\pgfpathlineto{\pgfqpoint{4.410345in}{2.224521in}}%
\pgfpathlineto{\pgfqpoint{4.436298in}{2.215085in}}%
\pgfpathlineto{\pgfqpoint{4.463186in}{2.205044in}}%
\pgfpathlineto{\pgfqpoint{4.469554in}{2.202626in}}%
\pgfpathlineto{\pgfqpoint{4.502811in}{2.189632in}}%
\pgfpathlineto{\pgfqpoint{4.512942in}{2.185566in}}%
\pgfpathlineto{\pgfqpoint{4.536067in}{2.176131in}}%
\pgfpathlineto{\pgfqpoint{4.560087in}{2.166088in}}%
\pgfpathlineto{\pgfqpoint{4.569323in}{2.162161in}}%
\pgfpathlineto{\pgfqpoint{4.602579in}{2.147669in}}%
\pgfpathlineto{\pgfqpoint{4.604947in}{2.146611in}}%
\pgfpathlineto{\pgfqpoint{4.635835in}{2.132567in}}%
\pgfpathlineto{\pgfqpoint{4.647522in}{2.127133in}}%
\pgfpathlineto{\pgfqpoint{4.669091in}{2.116928in}}%
\pgfpathlineto{\pgfqpoint{4.688265in}{2.107655in}}%
\pgfpathlineto{\pgfqpoint{4.702348in}{2.100723in}}%
\pgfpathlineto{\pgfqpoint{4.727292in}{2.088178in}}%
\pgfpathlineto{\pgfqpoint{4.735604in}{2.083922in}}%
\pgfpathlineto{\pgfqpoint{4.764712in}{2.068700in}}%
\pgfpathlineto{\pgfqpoint{4.768860in}{2.066491in}}%
\pgfpathlineto{\pgfqpoint{4.800622in}{2.049222in}}%
\pgfpathlineto{\pgfqpoint{4.802116in}{2.048395in}}%
\pgfpathlineto{\pgfqpoint{4.835115in}{2.029745in}}%
\pgfpathlineto{\pgfqpoint{4.835372in}{2.029596in}}%
\pgfpathlineto{\pgfqpoint{4.868272in}{2.010267in}}%
\pgfpathlineto{\pgfqpoint{4.868629in}{2.010054in}}%
\pgfpathlineto{\pgfqpoint{4.900172in}{1.990790in}}%
\pgfpathlineto{\pgfqpoint{4.901885in}{1.989723in}}%
\pgfpathlineto{\pgfqpoint{4.930886in}{1.971312in}}%
\pgfpathlineto{\pgfqpoint{4.935141in}{1.968556in}}%
\pgfpathlineto{\pgfqpoint{4.960480in}{1.951834in}}%
\pgfpathlineto{\pgfqpoint{4.968397in}{1.946502in}}%
\pgfpathlineto{\pgfqpoint{4.989015in}{1.932357in}}%
\pgfpathlineto{\pgfqpoint{5.001653in}{1.923503in}}%
\pgfpathlineto{\pgfqpoint{5.016547in}{1.912879in}}%
\pgfpathlineto{\pgfqpoint{5.034909in}{1.899499in}}%
\pgfpathlineto{\pgfqpoint{5.043131in}{1.893401in}}%
\pgfpathlineto{\pgfqpoint{5.068166in}{1.874424in}}%
\pgfpathlineto{\pgfqpoint{5.068814in}{1.873924in}}%
\pgfpathlineto{\pgfqpoint{5.093438in}{1.854446in}}%
\pgfpathlineto{\pgfqpoint{5.101422in}{1.847982in}}%
\pgfpathlineto{\pgfqpoint{5.117221in}{1.834968in}}%
\pgfpathlineto{\pgfqpoint{5.134678in}{1.820245in}}%
\pgfpathlineto{\pgfqpoint{5.140220in}{1.815491in}}%
\pgfpathlineto{\pgfqpoint{5.162335in}{1.796013in}}%
\pgfpathlineto{\pgfqpoint{5.167934in}{1.790945in}}%
\pgfpathlineto{\pgfqpoint{5.183589in}{1.776536in}}%
\pgfpathlineto{\pgfqpoint{5.201190in}{1.759912in}}%
\pgfpathlineto{\pgfqpoint{5.204162in}{1.757058in}}%
\pgfpathlineto{\pgfqpoint{5.223829in}{1.737580in}}%
\pgfpathlineto{\pgfqpoint{5.234447in}{1.726757in}}%
\pgfpathlineto{\pgfqpoint{5.242798in}{1.718103in}}%
\pgfpathlineto{\pgfqpoint{5.261015in}{1.698625in}}%
\pgfpathlineto{\pgfqpoint{5.267703in}{1.691231in}}%
\pgfpathlineto{\pgfqpoint{5.278455in}{1.679147in}}%
\pgfpathlineto{\pgfqpoint{5.295220in}{1.659670in}}%
\pgfpathlineto{\pgfqpoint{5.300959in}{1.652749in}}%
\pgfpathlineto{\pgfqpoint{5.311204in}{1.640192in}}%
\pgfpathlineto{\pgfqpoint{5.326517in}{1.620714in}}%
\pgfpathlineto{\pgfqpoint{5.334215in}{1.610519in}}%
\pgfpathlineto{\pgfqpoint{5.341112in}{1.601237in}}%
\pgfpathlineto{\pgfqpoint{5.354972in}{1.581759in}}%
\pgfpathlineto{\pgfqpoint{5.367471in}{1.563435in}}%
\pgfpathlineto{\pgfqpoint{5.368246in}{1.562281in}}%
\pgfpathlineto{\pgfqpoint{5.380653in}{1.542804in}}%
\pgfpathlineto{\pgfqpoint{5.392474in}{1.523326in}}%
\pgfpathlineto{\pgfqpoint{5.400728in}{1.508969in}}%
\pgfpathlineto{\pgfqpoint{5.403625in}{1.503849in}}%
\pgfpathlineto{\pgfqpoint{5.413975in}{1.484371in}}%
\pgfpathlineto{\pgfqpoint{5.423706in}{1.464893in}}%
\pgfpathlineto{\pgfqpoint{5.432802in}{1.445416in}}%
\pgfpathlineto{\pgfqpoint{5.433984in}{1.442670in}}%
\pgfpathlineto{\pgfqpoint{5.441072in}{1.425938in}}%
\pgfpathlineto{\pgfqpoint{5.448660in}{1.406460in}}%
\pgfpathlineto{\pgfqpoint{5.455578in}{1.386983in}}%
\pgfpathlineto{\pgfqpoint{5.461809in}{1.367505in}}%
\pgfpathlineto{\pgfqpoint{5.467240in}{1.348360in}}%
\pgfpathlineto{\pgfqpoint{5.467333in}{1.348027in}}%
\pgfpathlineto{\pgfqpoint{5.472019in}{1.328550in}}%
\pgfpathlineto{\pgfqpoint{5.475981in}{1.309072in}}%
\pgfpathlineto{\pgfqpoint{5.479201in}{1.289595in}}%
\pgfpathlineto{\pgfqpoint{5.481658in}{1.270117in}}%
\pgfpathlineto{\pgfqpoint{5.483335in}{1.250639in}}%
\pgfpathlineto{\pgfqpoint{5.484211in}{1.231162in}}%
\pgfpathlineto{\pgfqpoint{5.484264in}{1.211684in}}%
\pgfpathlineto{\pgfqpoint{5.483474in}{1.192206in}}%
\pgfpathlineto{\pgfqpoint{5.481818in}{1.172729in}}%
\pgfpathlineto{\pgfqpoint{5.479273in}{1.153251in}}%
\pgfpathlineto{\pgfqpoint{5.475814in}{1.133773in}}%
\pgfpathlineto{\pgfqpoint{5.471418in}{1.114296in}}%
\pgfpathlineto{\pgfqpoint{5.467240in}{1.099084in}}%
\pgfpathlineto{\pgfqpoint{5.466024in}{1.094818in}}%
\pgfpathlineto{\pgfqpoint{5.459495in}{1.075340in}}%
\pgfpathlineto{\pgfqpoint{5.451914in}{1.055863in}}%
\pgfpathlineto{\pgfqpoint{5.443253in}{1.036385in}}%
\pgfpathlineto{\pgfqpoint{5.433984in}{1.017905in}}%
\pgfpathlineto{\pgfqpoint{5.433463in}{1.016908in}}%
\pgfpathlineto{\pgfqpoint{5.422214in}{0.997430in}}%
\pgfpathlineto{\pgfqpoint{5.409741in}{0.977952in}}%
\pgfpathlineto{\pgfqpoint{5.400728in}{0.965123in}}%
\pgfpathlineto{\pgfqpoint{5.395858in}{0.958475in}}%
\pgfpathlineto{\pgfqpoint{5.380340in}{0.938997in}}%
\pgfpathlineto{\pgfqpoint{5.367471in}{0.924139in}}%
\pgfpathlineto{\pgfqpoint{5.363290in}{0.919519in}}%
\pgfpathlineto{\pgfqpoint{5.344309in}{0.900042in}}%
\pgfpathlineto{\pgfqpoint{5.334215in}{0.890412in}}%
\pgfpathlineto{\pgfqpoint{5.323401in}{0.880564in}}%
\pgfpathlineto{\pgfqpoint{5.300959in}{0.861513in}}%
\pgfpathlineto{\pgfqpoint{5.300432in}{0.861086in}}%
\pgfpathlineto{\pgfqpoint{5.274800in}{0.841609in}}%
\pgfpathlineto{\pgfqpoint{5.267703in}{0.836542in}}%
\pgfpathlineto{\pgfqpoint{5.246446in}{0.822131in}}%
\pgfpathlineto{\pgfqpoint{5.234447in}{0.814461in}}%
\pgfpathlineto{\pgfqpoint{5.214934in}{0.802653in}}%
\pgfpathlineto{\pgfqpoint{5.201190in}{0.794786in}}%
\pgfpathlineto{\pgfqpoint{5.179700in}{0.783176in}}%
\pgfpathlineto{\pgfqpoint{5.167934in}{0.777145in}}%
\pgfpathlineto{\pgfqpoint{5.140036in}{0.763698in}}%
\pgfpathlineto{\pgfqpoint{5.134678in}{0.761241in}}%
\pgfpathlineto{\pgfqpoint{5.101422in}{0.746926in}}%
\pgfpathlineto{\pgfqpoint{5.094705in}{0.744221in}}%
\pgfpathlineto{\pgfqpoint{5.068166in}{0.734024in}}%
\pgfpathlineto{\pgfqpoint{5.042155in}{0.724743in}}%
\pgfpathlineto{\pgfqpoint{5.034909in}{0.722271in}}%
\pgfpathlineto{\pgfqpoint{5.001653in}{0.711723in}}%
\pgfpathlineto{\pgfqpoint{4.979548in}{0.705265in}}%
\pgfpathlineto{\pgfqpoint{4.968397in}{0.702143in}}%
\pgfpathlineto{\pgfqpoint{4.935141in}{0.693574in}}%
\pgfpathlineto{\pgfqpoint{4.901888in}{0.685788in}}%
\pgfpathlineto{\pgfqpoint{4.901885in}{0.685787in}}%
\pgfpathlineto{\pgfqpoint{4.868629in}{0.678963in}}%
\pgfpathlineto{\pgfqpoint{4.835372in}{0.672850in}}%
\pgfpathlineto{\pgfqpoint{4.802116in}{0.667425in}}%
\pgfpathlineto{\pgfqpoint{4.794381in}{0.666310in}}%
\pgfpathlineto{\pgfqpoint{4.768860in}{0.662769in}}%
\pgfpathlineto{\pgfqpoint{4.735604in}{0.658773in}}%
\pgfpathlineto{\pgfqpoint{4.702348in}{0.655388in}}%
\pgfpathlineto{\pgfqpoint{4.669091in}{0.652598in}}%
\pgfpathlineto{\pgfqpoint{4.635835in}{0.650387in}}%
\pgfpathlineto{\pgfqpoint{4.602579in}{0.648738in}}%
\pgfpathlineto{\pgfqpoint{4.569323in}{0.647637in}}%
\pgfpathlineto{\pgfqpoint{4.536067in}{0.647069in}}%
\pgfpathlineto{\pgfqpoint{4.502811in}{0.647019in}}%
\pgfpathlineto{\pgfqpoint{4.469554in}{0.647475in}}%
\pgfpathlineto{\pgfqpoint{4.436298in}{0.648423in}}%
\pgfpathlineto{\pgfqpoint{4.403042in}{0.649850in}}%
\pgfpathlineto{\pgfqpoint{4.369786in}{0.651744in}}%
\pgfpathlineto{\pgfqpoint{4.336530in}{0.654094in}}%
\pgfpathlineto{\pgfqpoint{4.303273in}{0.656887in}}%
\pgfpathlineto{\pgfqpoint{4.270017in}{0.660114in}}%
\pgfpathlineto{\pgfqpoint{4.236761in}{0.663762in}}%
\pgfpathlineto{\pgfqpoint{4.215958in}{0.666310in}}%
\pgfpathlineto{\pgfqpoint{4.203505in}{0.667859in}}%
\pgfpathlineto{\pgfqpoint{4.170249in}{0.672429in}}%
\pgfpathlineto{\pgfqpoint{4.136993in}{0.677397in}}%
\pgfpathlineto{\pgfqpoint{4.103736in}{0.682756in}}%
\pgfpathlineto{\pgfqpoint{4.086237in}{0.685788in}}%
\pgfpathlineto{\pgfqpoint{4.070480in}{0.688560in}}%
\pgfpathlineto{\pgfqpoint{4.037224in}{0.694816in}}%
\pgfpathlineto{\pgfqpoint{4.003968in}{0.701440in}}%
\pgfpathlineto{\pgfqpoint{3.985817in}{0.705265in}}%
\pgfpathlineto{\pgfqpoint{3.970712in}{0.708499in}}%
\pgfpathlineto{\pgfqpoint{3.937455in}{0.716006in}}%
\pgfpathlineto{\pgfqpoint{3.904199in}{0.723856in}}%
\pgfpathlineto{\pgfqpoint{3.900620in}{0.724743in}}%
\pgfpathlineto{\pgfqpoint{3.870943in}{0.732211in}}%
\pgfpathlineto{\pgfqpoint{3.837687in}{0.740895in}}%
\pgfpathlineto{\pgfqpoint{3.825359in}{0.744221in}}%
\pgfpathlineto{\pgfqpoint{3.804431in}{0.749967in}}%
\pgfpathlineto{\pgfqpoint{3.771175in}{0.759249in}}%
\pgfpathlineto{\pgfqpoint{3.754874in}{0.763698in}}%
\pgfpathlineto{\pgfqpoint{3.737918in}{0.768462in}}%
\pgfpathlineto{\pgfqpoint{3.704662in}{0.776811in}}%
\pgfpathlineto{\pgfqpoint{3.671406in}{0.782523in}}%
\pgfpathlineto{\pgfqpoint{3.638150in}{0.782228in}}%
\pgfpathlineto{\pgfqpoint{3.604894in}{0.773251in}}%
\pgfpathlineto{\pgfqpoint{3.583214in}{0.763698in}}%
\pgfpathlineto{\pgfqpoint{3.571637in}{0.758518in}}%
\pgfpathlineto{\pgfqpoint{3.541664in}{0.744221in}}%
\pgfpathlineto{\pgfqpoint{3.538381in}{0.742706in}}%
\pgfpathlineto{\pgfqpoint{3.505125in}{0.727847in}}%
\pgfpathlineto{\pgfqpoint{3.497789in}{0.724743in}}%
\pgfpathlineto{\pgfqpoint{3.471869in}{0.714262in}}%
\pgfpathlineto{\pgfqpoint{3.448080in}{0.705265in}}%
\pgfpathlineto{\pgfqpoint{3.438613in}{0.701839in}}%
\pgfpathlineto{\pgfqpoint{3.405357in}{0.690604in}}%
\pgfpathlineto{\pgfqpoint{3.389995in}{0.685788in}}%
\pgfpathlineto{\pgfqpoint{3.372100in}{0.680408in}}%
\pgfpathlineto{\pgfqpoint{3.338844in}{0.671172in}}%
\pgfpathlineto{\pgfqpoint{3.319791in}{0.666310in}}%
\pgfpathlineto{\pgfqpoint{3.305588in}{0.662827in}}%
\pgfpathlineto{\pgfqpoint{3.272332in}{0.655360in}}%
\pgfpathlineto{\pgfqpoint{3.239076in}{0.648607in}}%
\pgfpathlineto{\pgfqpoint{3.229402in}{0.646832in}}%
\pgfpathlineto{\pgfqpoint{3.205819in}{0.642665in}}%
\pgfpathlineto{\pgfqpoint{3.172563in}{0.637423in}}%
\pgfpathlineto{\pgfqpoint{3.139307in}{0.632815in}}%
\pgfpathlineto{\pgfqpoint{3.106051in}{0.628823in}}%
\pgfpathlineto{\pgfqpoint{3.091736in}{0.627355in}}%
\pgfpathlineto{\pgfqpoint{3.072795in}{0.625479in}}%
\pgfpathlineto{\pgfqpoint{3.039539in}{0.622740in}}%
\pgfpathlineto{\pgfqpoint{3.006282in}{0.620553in}}%
\pgfpathlineto{\pgfqpoint{2.973026in}{0.618902in}}%
\pgfpathlineto{\pgfqpoint{2.939770in}{0.617776in}}%
\pgfpathlineto{\pgfqpoint{2.906514in}{0.617159in}}%
\pgfpathlineto{\pgfqpoint{2.873258in}{0.617039in}}%
\pgfpathlineto{\pgfqpoint{2.840001in}{0.617404in}}%
\pgfpathlineto{\pgfqpoint{2.806745in}{0.618242in}}%
\pgfpathlineto{\pgfqpoint{2.773489in}{0.619541in}}%
\pgfpathlineto{\pgfqpoint{2.740233in}{0.621289in}}%
\pgfpathlineto{\pgfqpoint{2.706977in}{0.623476in}}%
\pgfpathlineto{\pgfqpoint{2.673720in}{0.626091in}}%
\pgfpathclose%
\pgfusepath{stroke}%
\end{pgfscope}%
\begin{pgfscope}%
\pgfpathrectangle{\pgfqpoint{0.711606in}{0.549444in}}{\pgfqpoint{4.955171in}{2.902168in}}%
\pgfusepath{clip}%
\pgfsetbuttcap%
\pgfsetroundjoin%
\pgfsetlinewidth{1.003750pt}%
\definecolor{currentstroke}{rgb}{0.155850,0.044559,0.325338}%
\pgfsetstrokecolor{currentstroke}%
\pgfsetdash{}{0pt}%
\pgfpathmoveto{\pgfqpoint{2.806745in}{0.568642in}}%
\pgfpathlineto{\pgfqpoint{2.800937in}{0.568922in}}%
\pgfpathlineto{\pgfqpoint{2.773489in}{0.570264in}}%
\pgfpathlineto{\pgfqpoint{2.740233in}{0.572318in}}%
\pgfpathlineto{\pgfqpoint{2.706977in}{0.574785in}}%
\pgfpathlineto{\pgfqpoint{2.673720in}{0.577658in}}%
\pgfpathlineto{\pgfqpoint{2.640464in}{0.580926in}}%
\pgfpathlineto{\pgfqpoint{2.607208in}{0.584580in}}%
\pgfpathlineto{\pgfqpoint{2.575724in}{0.588399in}}%
\pgfpathlineto{\pgfqpoint{2.573952in}{0.588618in}}%
\pgfpathlineto{\pgfqpoint{2.540696in}{0.593117in}}%
\pgfpathlineto{\pgfqpoint{2.507440in}{0.597984in}}%
\pgfpathlineto{\pgfqpoint{2.474183in}{0.603210in}}%
\pgfpathlineto{\pgfqpoint{2.446392in}{0.607877in}}%
\pgfpathlineto{\pgfqpoint{2.440927in}{0.608808in}}%
\pgfpathlineto{\pgfqpoint{2.407671in}{0.614859in}}%
\pgfpathlineto{\pgfqpoint{2.374415in}{0.621252in}}%
\pgfpathlineto{\pgfqpoint{2.344261in}{0.627355in}}%
\pgfpathlineto{\pgfqpoint{2.341159in}{0.627992in}}%
\pgfpathlineto{\pgfqpoint{2.307902in}{0.635199in}}%
\pgfpathlineto{\pgfqpoint{2.274646in}{0.642729in}}%
\pgfpathlineto{\pgfqpoint{2.257321in}{0.646832in}}%
\pgfpathlineto{\pgfqpoint{2.241390in}{0.650660in}}%
\pgfpathlineto{\pgfqpoint{2.208134in}{0.658994in}}%
\pgfpathlineto{\pgfqpoint{2.180014in}{0.666310in}}%
\pgfpathlineto{\pgfqpoint{2.174878in}{0.667666in}}%
\pgfpathlineto{\pgfqpoint{2.141622in}{0.676804in}}%
\pgfpathlineto{\pgfqpoint{2.109968in}{0.685788in}}%
\pgfpathlineto{\pgfqpoint{2.108365in}{0.686249in}}%
\pgfpathlineto{\pgfqpoint{2.075109in}{0.696190in}}%
\pgfpathlineto{\pgfqpoint{2.045623in}{0.705265in}}%
\pgfpathlineto{\pgfqpoint{2.041853in}{0.706443in}}%
\pgfpathlineto{\pgfqpoint{2.008597in}{0.717184in}}%
\pgfpathlineto{\pgfqpoint{1.985837in}{0.724743in}}%
\pgfpathlineto{\pgfqpoint{1.975341in}{0.728281in}}%
\pgfpathlineto{\pgfqpoint{1.942084in}{0.739822in}}%
\pgfpathlineto{\pgfqpoint{1.929754in}{0.744221in}}%
\pgfpathlineto{\pgfqpoint{1.908828in}{0.751798in}}%
\pgfpathlineto{\pgfqpoint{1.876762in}{0.763698in}}%
\pgfpathlineto{\pgfqpoint{1.875572in}{0.764146in}}%
\pgfpathlineto{\pgfqpoint{1.842316in}{0.777030in}}%
\pgfpathlineto{\pgfqpoint{1.826819in}{0.783176in}}%
\pgfpathlineto{\pgfqpoint{1.809060in}{0.790327in}}%
\pgfpathlineto{\pgfqpoint{1.779132in}{0.802653in}}%
\pgfpathlineto{\pgfqpoint{1.775804in}{0.804046in}}%
\pgfpathlineto{\pgfqpoint{1.742547in}{0.818300in}}%
\pgfpathlineto{\pgfqpoint{1.733804in}{0.822131in}}%
\pgfpathlineto{\pgfqpoint{1.709291in}{0.833042in}}%
\pgfpathlineto{\pgfqpoint{1.690444in}{0.841609in}}%
\pgfpathlineto{\pgfqpoint{1.676035in}{0.848262in}}%
\pgfpathlineto{\pgfqpoint{1.648828in}{0.861086in}}%
\pgfpathlineto{\pgfqpoint{1.642779in}{0.863983in}}%
\pgfpathlineto{\pgfqpoint{1.609523in}{0.880239in}}%
\pgfpathlineto{\pgfqpoint{1.608871in}{0.880564in}}%
\pgfpathlineto{\pgfqpoint{1.576266in}{0.897105in}}%
\pgfpathlineto{\pgfqpoint{1.570588in}{0.900042in}}%
\pgfpathlineto{\pgfqpoint{1.543010in}{0.914539in}}%
\pgfpathlineto{\pgfqpoint{1.533712in}{0.919519in}}%
\pgfpathlineto{\pgfqpoint{1.509754in}{0.932569in}}%
\pgfpathlineto{\pgfqpoint{1.498168in}{0.938997in}}%
\pgfpathlineto{\pgfqpoint{1.476498in}{0.951226in}}%
\pgfpathlineto{\pgfqpoint{1.463883in}{0.958475in}}%
\pgfpathlineto{\pgfqpoint{1.443242in}{0.970542in}}%
\pgfpathlineto{\pgfqpoint{1.430791in}{0.977952in}}%
\pgfpathlineto{\pgfqpoint{1.409986in}{0.990554in}}%
\pgfpathlineto{\pgfqpoint{1.398829in}{0.997430in}}%
\pgfpathlineto{\pgfqpoint{1.376729in}{1.011297in}}%
\pgfpathlineto{\pgfqpoint{1.367940in}{1.016908in}}%
\pgfpathlineto{\pgfqpoint{1.343473in}{1.032813in}}%
\pgfpathlineto{\pgfqpoint{1.338070in}{1.036385in}}%
\pgfpathlineto{\pgfqpoint{1.310217in}{1.055144in}}%
\pgfpathlineto{\pgfqpoint{1.309168in}{1.055863in}}%
\pgfpathlineto{\pgfqpoint{1.281292in}{1.075340in}}%
\pgfpathlineto{\pgfqpoint{1.276961in}{1.078428in}}%
\pgfpathlineto{\pgfqpoint{1.254338in}{1.094818in}}%
\pgfpathlineto{\pgfqpoint{1.243705in}{1.102677in}}%
\pgfpathlineto{\pgfqpoint{1.228235in}{1.114296in}}%
\pgfpathlineto{\pgfqpoint{1.210448in}{1.127928in}}%
\pgfpathlineto{\pgfqpoint{1.202943in}{1.133773in}}%
\pgfpathlineto{\pgfqpoint{1.178453in}{1.153251in}}%
\pgfpathlineto{\pgfqpoint{1.177192in}{1.154278in}}%
\pgfpathlineto{\pgfqpoint{1.154894in}{1.172729in}}%
\pgfpathlineto{\pgfqpoint{1.143936in}{1.181996in}}%
\pgfpathlineto{\pgfqpoint{1.132049in}{1.192206in}}%
\pgfpathlineto{\pgfqpoint{1.110680in}{1.210974in}}%
\pgfpathlineto{\pgfqpoint{1.109884in}{1.211684in}}%
\pgfpathlineto{\pgfqpoint{1.088623in}{1.231162in}}%
\pgfpathlineto{\pgfqpoint{1.077424in}{1.241669in}}%
\pgfpathlineto{\pgfqpoint{1.068007in}{1.250639in}}%
\pgfpathlineto{\pgfqpoint{1.048078in}{1.270117in}}%
\pgfpathlineto{\pgfqpoint{1.044168in}{1.274050in}}%
\pgfpathlineto{\pgfqpoint{1.028946in}{1.289595in}}%
\pgfpathlineto{\pgfqpoint{1.010911in}{1.308490in}}%
\pgfpathlineto{\pgfqpoint{1.010364in}{1.309072in}}%
\pgfpathlineto{\pgfqpoint{0.992638in}{1.328550in}}%
\pgfpathlineto{\pgfqpoint{0.977655in}{1.345476in}}%
\pgfpathlineto{\pgfqpoint{0.975430in}{1.348027in}}%
\pgfpathlineto{\pgfqpoint{0.959024in}{1.367505in}}%
\pgfpathlineto{\pgfqpoint{0.944399in}{1.385408in}}%
\pgfpathlineto{\pgfqpoint{0.943131in}{1.386983in}}%
\pgfpathlineto{\pgfqpoint{0.928046in}{1.406460in}}%
\pgfpathlineto{\pgfqpoint{0.913461in}{1.425938in}}%
\pgfpathlineto{\pgfqpoint{0.911143in}{1.429167in}}%
\pgfpathlineto{\pgfqpoint{0.899647in}{1.445416in}}%
\pgfpathlineto{\pgfqpoint{0.886399in}{1.464893in}}%
\pgfpathlineto{\pgfqpoint{0.877887in}{1.477961in}}%
\pgfpathlineto{\pgfqpoint{0.873772in}{1.484371in}}%
\pgfpathlineto{\pgfqpoint{0.861861in}{1.503849in}}%
\pgfpathlineto{\pgfqpoint{0.850494in}{1.523326in}}%
\pgfpathlineto{\pgfqpoint{0.844630in}{1.533927in}}%
\pgfpathlineto{\pgfqpoint{0.839792in}{1.542804in}}%
\pgfpathlineto{\pgfqpoint{0.829777in}{1.562281in}}%
\pgfpathlineto{\pgfqpoint{0.820335in}{1.581759in}}%
\pgfpathlineto{\pgfqpoint{0.811478in}{1.601237in}}%
\pgfpathlineto{\pgfqpoint{0.811374in}{1.601484in}}%
\pgfpathlineto{\pgfqpoint{0.803401in}{1.620714in}}%
\pgfpathlineto{\pgfqpoint{0.795927in}{1.640192in}}%
\pgfpathlineto{\pgfqpoint{0.789069in}{1.659670in}}%
\pgfpathlineto{\pgfqpoint{0.782840in}{1.679147in}}%
\pgfpathlineto{\pgfqpoint{0.778118in}{1.695637in}}%
\pgfpathlineto{\pgfqpoint{0.777275in}{1.698625in}}%
\pgfpathlineto{\pgfqpoint{0.772460in}{1.718103in}}%
\pgfpathlineto{\pgfqpoint{0.768305in}{1.737580in}}%
\pgfpathlineto{\pgfqpoint{0.764827in}{1.757058in}}%
\pgfpathlineto{\pgfqpoint{0.762040in}{1.776536in}}%
\pgfpathlineto{\pgfqpoint{0.759961in}{1.796013in}}%
\pgfpathlineto{\pgfqpoint{0.758607in}{1.815491in}}%
\pgfpathlineto{\pgfqpoint{0.757995in}{1.834968in}}%
\pgfpathlineto{\pgfqpoint{0.758142in}{1.854446in}}%
\pgfpathlineto{\pgfqpoint{0.759066in}{1.873924in}}%
\pgfpathlineto{\pgfqpoint{0.760788in}{1.893401in}}%
\pgfpathlineto{\pgfqpoint{0.763325in}{1.912879in}}%
\pgfpathlineto{\pgfqpoint{0.766699in}{1.932357in}}%
\pgfpathlineto{\pgfqpoint{0.770931in}{1.951834in}}%
\pgfpathlineto{\pgfqpoint{0.776041in}{1.971312in}}%
\pgfpathlineto{\pgfqpoint{0.778118in}{1.978080in}}%
\pgfpathlineto{\pgfqpoint{0.782153in}{1.990790in}}%
\pgfpathlineto{\pgfqpoint{0.789269in}{2.010267in}}%
\pgfpathlineto{\pgfqpoint{0.797360in}{2.029745in}}%
\pgfpathlineto{\pgfqpoint{0.806454in}{2.049222in}}%
\pgfpathlineto{\pgfqpoint{0.811374in}{2.058735in}}%
\pgfpathlineto{\pgfqpoint{0.816718in}{2.068700in}}%
\pgfpathlineto{\pgfqpoint{0.828206in}{2.088178in}}%
\pgfpathlineto{\pgfqpoint{0.840817in}{2.107655in}}%
\pgfpathlineto{\pgfqpoint{0.844630in}{2.113089in}}%
\pgfpathlineto{\pgfqpoint{0.854868in}{2.127133in}}%
\pgfpathlineto{\pgfqpoint{0.870258in}{2.146611in}}%
\pgfpathlineto{\pgfqpoint{0.877887in}{2.155579in}}%
\pgfpathlineto{\pgfqpoint{0.887189in}{2.166088in}}%
\pgfpathlineto{\pgfqpoint{0.905702in}{2.185566in}}%
\pgfpathlineto{\pgfqpoint{0.911143in}{2.190930in}}%
\pgfpathlineto{\pgfqpoint{0.926071in}{2.205044in}}%
\pgfpathlineto{\pgfqpoint{0.944399in}{2.221279in}}%
\pgfpathlineto{\pgfqpoint{0.948222in}{2.224521in}}%
\pgfpathlineto{\pgfqpoint{0.972527in}{2.243999in}}%
\pgfpathlineto{\pgfqpoint{0.977655in}{2.247912in}}%
\pgfpathlineto{\pgfqpoint{0.998915in}{2.263477in}}%
\pgfpathlineto{\pgfqpoint{1.010911in}{2.272093in}}%
\pgfpathlineto{\pgfqpoint{1.026433in}{2.282954in}}%
\pgfpathlineto{\pgfqpoint{1.044168in}{2.296755in}}%
\pgfpathlineto{\pgfqpoint{1.051217in}{2.302432in}}%
\pgfpathlineto{\pgfqpoint{1.065327in}{2.321909in}}%
\pgfpathlineto{\pgfqpoint{1.065609in}{2.341387in}}%
\pgfpathlineto{\pgfqpoint{1.057100in}{2.360865in}}%
\pgfpathlineto{\pgfqpoint{1.044779in}{2.380342in}}%
\pgfpathlineto{\pgfqpoint{1.044168in}{2.381213in}}%
\pgfpathlineto{\pgfqpoint{1.031439in}{2.399820in}}%
\pgfpathlineto{\pgfqpoint{1.017987in}{2.419298in}}%
\pgfpathlineto{\pgfqpoint{1.010911in}{2.429798in}}%
\pgfpathlineto{\pgfqpoint{1.004967in}{2.438775in}}%
\pgfpathlineto{\pgfqpoint{0.992582in}{2.458253in}}%
\pgfpathlineto{\pgfqpoint{0.980743in}{2.477731in}}%
\pgfpathlineto{\pgfqpoint{0.977655in}{2.483105in}}%
\pgfpathlineto{\pgfqpoint{0.969683in}{2.497208in}}%
\pgfpathlineto{\pgfqpoint{0.959294in}{2.516686in}}%
\pgfpathlineto{\pgfqpoint{0.949523in}{2.536163in}}%
\pgfpathlineto{\pgfqpoint{0.944399in}{2.547124in}}%
\pgfpathlineto{\pgfqpoint{0.940481in}{2.555641in}}%
\pgfpathlineto{\pgfqpoint{0.932197in}{2.575119in}}%
\pgfpathlineto{\pgfqpoint{0.924567in}{2.594596in}}%
\pgfpathlineto{\pgfqpoint{0.917607in}{2.614074in}}%
\pgfpathlineto{\pgfqpoint{0.911333in}{2.633552in}}%
\pgfpathlineto{\pgfqpoint{0.911143in}{2.634223in}}%
\pgfpathlineto{\pgfqpoint{0.905893in}{2.653029in}}%
\pgfpathlineto{\pgfqpoint{0.901164in}{2.672507in}}%
\pgfpathlineto{\pgfqpoint{0.897159in}{2.691985in}}%
\pgfpathlineto{\pgfqpoint{0.893896in}{2.711462in}}%
\pgfpathlineto{\pgfqpoint{0.891393in}{2.730940in}}%
\pgfpathlineto{\pgfqpoint{0.889671in}{2.750418in}}%
\pgfpathlineto{\pgfqpoint{0.888749in}{2.769895in}}%
\pgfpathlineto{\pgfqpoint{0.888649in}{2.789373in}}%
\pgfpathlineto{\pgfqpoint{0.889391in}{2.808850in}}%
\pgfpathlineto{\pgfqpoint{0.890998in}{2.828328in}}%
\pgfpathlineto{\pgfqpoint{0.893494in}{2.847806in}}%
\pgfpathlineto{\pgfqpoint{0.896902in}{2.867283in}}%
\pgfpathlineto{\pgfqpoint{0.901247in}{2.886761in}}%
\pgfpathlineto{\pgfqpoint{0.906555in}{2.906239in}}%
\pgfpathlineto{\pgfqpoint{0.911143in}{2.920465in}}%
\pgfpathlineto{\pgfqpoint{0.912900in}{2.925716in}}%
\pgfpathlineto{\pgfqpoint{0.920423in}{2.945194in}}%
\pgfpathlineto{\pgfqpoint{0.929026in}{2.964672in}}%
\pgfpathlineto{\pgfqpoint{0.938741in}{2.984149in}}%
\pgfpathlineto{\pgfqpoint{0.944399in}{2.994345in}}%
\pgfpathlineto{\pgfqpoint{0.949757in}{3.003627in}}%
\pgfpathlineto{\pgfqpoint{0.962163in}{3.023104in}}%
\pgfpathlineto{\pgfqpoint{0.975829in}{3.042582in}}%
\pgfpathlineto{\pgfqpoint{0.977655in}{3.044980in}}%
\pgfpathlineto{\pgfqpoint{0.991216in}{3.062060in}}%
\pgfpathlineto{\pgfqpoint{1.008056in}{3.081537in}}%
\pgfpathlineto{\pgfqpoint{1.010911in}{3.084608in}}%
\pgfpathlineto{\pgfqpoint{1.026859in}{3.101015in}}%
\pgfpathlineto{\pgfqpoint{1.044168in}{3.117526in}}%
\pgfpathlineto{\pgfqpoint{1.047425in}{3.120493in}}%
\pgfpathlineto{\pgfqpoint{1.070283in}{3.139970in}}%
\pgfpathlineto{\pgfqpoint{1.077424in}{3.145672in}}%
\pgfpathlineto{\pgfqpoint{1.095548in}{3.159448in}}%
\pgfpathlineto{\pgfqpoint{1.110680in}{3.170252in}}%
\pgfpathlineto{\pgfqpoint{1.123474in}{3.178926in}}%
\pgfpathlineto{\pgfqpoint{1.143936in}{3.192004in}}%
\pgfpathlineto{\pgfqpoint{1.154510in}{3.198403in}}%
\pgfpathlineto{\pgfqpoint{1.177192in}{3.211387in}}%
\pgfpathlineto{\pgfqpoint{1.189212in}{3.217881in}}%
\pgfpathlineto{\pgfqpoint{1.210448in}{3.228765in}}%
\pgfpathlineto{\pgfqpoint{1.228276in}{3.237359in}}%
\pgfpathlineto{\pgfqpoint{1.243705in}{3.244433in}}%
\pgfpathlineto{\pgfqpoint{1.272587in}{3.256836in}}%
\pgfpathlineto{\pgfqpoint{1.276961in}{3.258627in}}%
\pgfpathlineto{\pgfqpoint{1.310217in}{3.271383in}}%
\pgfpathlineto{\pgfqpoint{1.324036in}{3.276314in}}%
\pgfpathlineto{\pgfqpoint{1.343473in}{3.282945in}}%
\pgfpathlineto{\pgfqpoint{1.376729in}{3.293447in}}%
\pgfpathlineto{\pgfqpoint{1.384756in}{3.295791in}}%
\pgfpathlineto{\pgfqpoint{1.409986in}{3.302855in}}%
\pgfpathlineto{\pgfqpoint{1.443242in}{3.311383in}}%
\pgfpathlineto{\pgfqpoint{1.459839in}{3.315269in}}%
\pgfpathlineto{\pgfqpoint{1.476498in}{3.319015in}}%
\pgfpathlineto{\pgfqpoint{1.509754in}{3.325801in}}%
\pgfpathlineto{\pgfqpoint{1.543010in}{3.331878in}}%
\pgfpathlineto{\pgfqpoint{1.560631in}{3.334747in}}%
\pgfpathlineto{\pgfqpoint{1.576266in}{3.337197in}}%
\pgfpathlineto{\pgfqpoint{1.609523in}{3.341789in}}%
\pgfpathlineto{\pgfqpoint{1.642779in}{3.345752in}}%
\pgfpathlineto{\pgfqpoint{1.676035in}{3.349104in}}%
\pgfpathlineto{\pgfqpoint{1.709291in}{3.351862in}}%
\pgfpathlineto{\pgfqpoint{1.742547in}{3.354042in}}%
\pgfpathlineto{\pgfqpoint{1.746259in}{3.354224in}}%
\pgfpathlineto{\pgfqpoint{1.775804in}{3.355624in}}%
\pgfpathlineto{\pgfqpoint{1.809060in}{3.356668in}}%
\pgfpathlineto{\pgfqpoint{1.842316in}{3.357194in}}%
\pgfpathlineto{\pgfqpoint{1.875572in}{3.357215in}}%
\pgfpathlineto{\pgfqpoint{1.908828in}{3.356743in}}%
\pgfpathlineto{\pgfqpoint{1.942084in}{3.355792in}}%
\pgfpathlineto{\pgfqpoint{1.975341in}{3.354373in}}%
\pgfpathlineto{\pgfqpoint{1.977955in}{3.354224in}}%
\pgfpathlineto{\pgfqpoint{2.008597in}{3.352455in}}%
\pgfpathlineto{\pgfqpoint{2.041853in}{3.350079in}}%
\pgfpathlineto{\pgfqpoint{2.075109in}{3.347259in}}%
\pgfpathlineto{\pgfqpoint{2.108365in}{3.344007in}}%
\pgfpathlineto{\pgfqpoint{2.141622in}{3.340334in}}%
\pgfpathlineto{\pgfqpoint{2.174878in}{3.336249in}}%
\pgfpathlineto{\pgfqpoint{2.185950in}{3.334747in}}%
\pgfpathlineto{\pgfqpoint{2.208134in}{3.331690in}}%
\pgfpathlineto{\pgfqpoint{2.241390in}{3.326696in}}%
\pgfpathlineto{\pgfqpoint{2.274646in}{3.321313in}}%
\pgfpathlineto{\pgfqpoint{2.307902in}{3.315549in}}%
\pgfpathlineto{\pgfqpoint{2.309408in}{3.315269in}}%
\pgfpathlineto{\pgfqpoint{2.341159in}{3.309275in}}%
\pgfpathlineto{\pgfqpoint{2.374415in}{3.302626in}}%
\pgfpathlineto{\pgfqpoint{2.406840in}{3.295791in}}%
\pgfpathlineto{\pgfqpoint{2.407671in}{3.295613in}}%
\pgfpathlineto{\pgfqpoint{2.440927in}{3.288079in}}%
\pgfpathlineto{\pgfqpoint{2.474183in}{3.280198in}}%
\pgfpathlineto{\pgfqpoint{2.489828in}{3.276314in}}%
\pgfpathlineto{\pgfqpoint{2.507440in}{3.271873in}}%
\pgfpathlineto{\pgfqpoint{2.540696in}{3.263118in}}%
\pgfpathlineto{\pgfqpoint{2.563643in}{3.256836in}}%
\pgfpathlineto{\pgfqpoint{2.573952in}{3.253969in}}%
\pgfpathlineto{\pgfqpoint{2.607208in}{3.244343in}}%
\pgfpathlineto{\pgfqpoint{2.630516in}{3.237359in}}%
\pgfpathlineto{\pgfqpoint{2.640464in}{3.234330in}}%
\pgfpathlineto{\pgfqpoint{2.673720in}{3.223833in}}%
\pgfpathlineto{\pgfqpoint{2.691982in}{3.217881in}}%
\pgfpathlineto{\pgfqpoint{2.706977in}{3.212915in}}%
\pgfpathlineto{\pgfqpoint{2.740233in}{3.201550in}}%
\pgfpathlineto{\pgfqpoint{2.749156in}{3.198403in}}%
\pgfpathlineto{\pgfqpoint{2.773489in}{3.189683in}}%
\pgfpathlineto{\pgfqpoint{2.802709in}{3.178926in}}%
\pgfpathlineto{\pgfqpoint{2.806745in}{3.177415in}}%
\pgfpathlineto{\pgfqpoint{2.840001in}{3.164594in}}%
\pgfpathlineto{\pgfqpoint{2.853001in}{3.159448in}}%
\pgfpathlineto{\pgfqpoint{2.873258in}{3.151297in}}%
\pgfpathlineto{\pgfqpoint{2.900720in}{3.139970in}}%
\pgfpathlineto{\pgfqpoint{2.906514in}{3.137541in}}%
\pgfpathlineto{\pgfqpoint{2.939770in}{3.123229in}}%
\pgfpathlineto{\pgfqpoint{2.945972in}{3.120493in}}%
\pgfpathlineto{\pgfqpoint{2.973026in}{3.108355in}}%
\pgfpathlineto{\pgfqpoint{2.989021in}{3.101015in}}%
\pgfpathlineto{\pgfqpoint{3.006282in}{3.092956in}}%
\pgfpathlineto{\pgfqpoint{3.030202in}{3.081537in}}%
\pgfpathlineto{\pgfqpoint{3.039539in}{3.077002in}}%
\pgfpathlineto{\pgfqpoint{3.069638in}{3.062060in}}%
\pgfpathlineto{\pgfqpoint{3.072795in}{3.060464in}}%
\pgfpathlineto{\pgfqpoint{3.106051in}{3.043291in}}%
\pgfpathlineto{\pgfqpoint{3.107394in}{3.042582in}}%
\pgfpathlineto{\pgfqpoint{3.139307in}{3.025440in}}%
\pgfpathlineto{\pgfqpoint{3.143569in}{3.023104in}}%
\pgfpathlineto{\pgfqpoint{3.172563in}{3.006920in}}%
\pgfpathlineto{\pgfqpoint{3.178347in}{3.003627in}}%
\pgfpathlineto{\pgfqpoint{3.205819in}{2.987692in}}%
\pgfpathlineto{\pgfqpoint{3.211810in}{2.984149in}}%
\pgfpathlineto{\pgfqpoint{3.239076in}{2.967716in}}%
\pgfpathlineto{\pgfqpoint{3.244032in}{2.964672in}}%
\pgfpathlineto{\pgfqpoint{3.272332in}{2.946948in}}%
\pgfpathlineto{\pgfqpoint{3.275081in}{2.945194in}}%
\pgfpathlineto{\pgfqpoint{3.305006in}{2.925716in}}%
\pgfpathlineto{\pgfqpoint{3.305588in}{2.925329in}}%
\pgfpathlineto{\pgfqpoint{3.333777in}{2.906239in}}%
\pgfpathlineto{\pgfqpoint{3.338844in}{2.902735in}}%
\pgfpathlineto{\pgfqpoint{3.361532in}{2.886761in}}%
\pgfpathlineto{\pgfqpoint{3.372100in}{2.879161in}}%
\pgfpathlineto{\pgfqpoint{3.388324in}{2.867283in}}%
\pgfpathlineto{\pgfqpoint{3.405357in}{2.854541in}}%
\pgfpathlineto{\pgfqpoint{3.414203in}{2.847806in}}%
\pgfpathlineto{\pgfqpoint{3.438613in}{2.828807in}}%
\pgfpathlineto{\pgfqpoint{3.439217in}{2.828328in}}%
\pgfpathlineto{\pgfqpoint{3.463189in}{2.808850in}}%
\pgfpathlineto{\pgfqpoint{3.471869in}{2.801629in}}%
\pgfpathlineto{\pgfqpoint{3.486351in}{2.789373in}}%
\pgfpathlineto{\pgfqpoint{3.505125in}{2.773099in}}%
\pgfpathlineto{\pgfqpoint{3.508760in}{2.769895in}}%
\pgfpathlineto{\pgfqpoint{3.530251in}{2.750418in}}%
\pgfpathlineto{\pgfqpoint{3.538381in}{2.742848in}}%
\pgfpathlineto{\pgfqpoint{3.550960in}{2.730940in}}%
\pgfpathlineto{\pgfqpoint{3.570999in}{2.711462in}}%
\pgfpathlineto{\pgfqpoint{3.571637in}{2.710821in}}%
\pgfpathlineto{\pgfqpoint{3.590089in}{2.691985in}}%
\pgfpathlineto{\pgfqpoint{3.604894in}{2.676442in}}%
\pgfpathlineto{\pgfqpoint{3.608582in}{2.672507in}}%
\pgfpathlineto{\pgfqpoint{3.626225in}{2.653029in}}%
\pgfpathlineto{\pgfqpoint{3.638150in}{2.639442in}}%
\pgfpathlineto{\pgfqpoint{3.643238in}{2.633552in}}%
\pgfpathlineto{\pgfqpoint{3.659459in}{2.614074in}}%
\pgfpathlineto{\pgfqpoint{3.671406in}{2.599227in}}%
\pgfpathlineto{\pgfqpoint{3.675077in}{2.594596in}}%
\pgfpathlineto{\pgfqpoint{3.689907in}{2.575119in}}%
\pgfpathlineto{\pgfqpoint{3.704231in}{2.555641in}}%
\pgfpathlineto{\pgfqpoint{3.704662in}{2.555028in}}%
\pgfpathlineto{\pgfqpoint{3.717788in}{2.536163in}}%
\pgfpathlineto{\pgfqpoint{3.730945in}{2.516686in}}%
\pgfpathlineto{\pgfqpoint{3.737918in}{2.506104in}}%
\pgfpathlineto{\pgfqpoint{3.743832in}{2.497208in}}%
\pgfpathlineto{\pgfqpoint{3.757054in}{2.477731in}}%
\pgfpathlineto{\pgfqpoint{3.771175in}{2.459248in}}%
\pgfpathlineto{\pgfqpoint{3.772037in}{2.458253in}}%
\pgfpathlineto{\pgfqpoint{3.794145in}{2.438775in}}%
\pgfpathlineto{\pgfqpoint{3.804431in}{2.432598in}}%
\pgfpathlineto{\pgfqpoint{3.837687in}{2.419500in}}%
\pgfpathlineto{\pgfqpoint{3.838414in}{2.419298in}}%
\pgfpathlineto{\pgfqpoint{3.870943in}{2.411539in}}%
\pgfpathlineto{\pgfqpoint{3.904199in}{2.404954in}}%
\pgfpathlineto{\pgfqpoint{3.931691in}{2.399820in}}%
\pgfpathlineto{\pgfqpoint{3.937455in}{2.398759in}}%
\pgfpathlineto{\pgfqpoint{3.970712in}{2.392457in}}%
\pgfpathlineto{\pgfqpoint{4.003968in}{2.385932in}}%
\pgfpathlineto{\pgfqpoint{4.031248in}{2.380342in}}%
\pgfpathlineto{\pgfqpoint{4.037224in}{2.379104in}}%
\pgfpathlineto{\pgfqpoint{4.070480in}{2.371846in}}%
\pgfpathlineto{\pgfqpoint{4.103736in}{2.364270in}}%
\pgfpathlineto{\pgfqpoint{4.118024in}{2.360865in}}%
\pgfpathlineto{\pgfqpoint{4.136993in}{2.356279in}}%
\pgfpathlineto{\pgfqpoint{4.170249in}{2.347893in}}%
\pgfpathlineto{\pgfqpoint{4.195072in}{2.341387in}}%
\pgfpathlineto{\pgfqpoint{4.203505in}{2.339144in}}%
\pgfpathlineto{\pgfqpoint{4.236761in}{2.329937in}}%
\pgfpathlineto{\pgfqpoint{4.264795in}{2.321909in}}%
\pgfpathlineto{\pgfqpoint{4.270017in}{2.320392in}}%
\pgfpathlineto{\pgfqpoint{4.303273in}{2.310360in}}%
\pgfpathlineto{\pgfqpoint{4.328764in}{2.302432in}}%
\pgfpathlineto{\pgfqpoint{4.336530in}{2.299980in}}%
\pgfpathlineto{\pgfqpoint{4.369786in}{2.289125in}}%
\pgfpathlineto{\pgfqpoint{4.388145in}{2.282954in}}%
\pgfpathlineto{\pgfqpoint{4.403042in}{2.277871in}}%
\pgfpathlineto{\pgfqpoint{4.436298in}{2.266193in}}%
\pgfpathlineto{\pgfqpoint{4.443811in}{2.263477in}}%
\pgfpathlineto{\pgfqpoint{4.469554in}{2.254026in}}%
\pgfpathlineto{\pgfqpoint{4.496192in}{2.243999in}}%
\pgfpathlineto{\pgfqpoint{4.502811in}{2.241469in}}%
\pgfpathlineto{\pgfqpoint{4.536067in}{2.228408in}}%
\pgfpathlineto{\pgfqpoint{4.545716in}{2.224521in}}%
\pgfpathlineto{\pgfqpoint{4.569323in}{2.214864in}}%
\pgfpathlineto{\pgfqpoint{4.592781in}{2.205044in}}%
\pgfpathlineto{\pgfqpoint{4.602579in}{2.200876in}}%
\pgfpathlineto{\pgfqpoint{4.635835in}{2.186402in}}%
\pgfpathlineto{\pgfqpoint{4.637711in}{2.185566in}}%
\pgfpathlineto{\pgfqpoint{4.669091in}{2.171348in}}%
\pgfpathlineto{\pgfqpoint{4.680456in}{2.166088in}}%
\pgfpathlineto{\pgfqpoint{4.702348in}{2.155790in}}%
\pgfpathlineto{\pgfqpoint{4.721457in}{2.146611in}}%
\pgfpathlineto{\pgfqpoint{4.735604in}{2.139702in}}%
\pgfpathlineto{\pgfqpoint{4.760821in}{2.127133in}}%
\pgfpathlineto{\pgfqpoint{4.768860in}{2.123058in}}%
\pgfpathlineto{\pgfqpoint{4.798644in}{2.107655in}}%
\pgfpathlineto{\pgfqpoint{4.802116in}{2.105829in}}%
\pgfpathlineto{\pgfqpoint{4.835016in}{2.088178in}}%
\pgfpathlineto{\pgfqpoint{4.835372in}{2.087983in}}%
\pgfpathlineto{\pgfqpoint{4.868629in}{2.069468in}}%
\pgfpathlineto{\pgfqpoint{4.869982in}{2.068700in}}%
\pgfpathlineto{\pgfqpoint{4.901885in}{2.050281in}}%
\pgfpathlineto{\pgfqpoint{4.903684in}{2.049222in}}%
\pgfpathlineto{\pgfqpoint{4.935141in}{2.030391in}}%
\pgfpathlineto{\pgfqpoint{4.936200in}{2.029745in}}%
\pgfpathlineto{\pgfqpoint{4.967571in}{2.010267in}}%
\pgfpathlineto{\pgfqpoint{4.968397in}{2.009744in}}%
\pgfpathlineto{\pgfqpoint{4.997820in}{1.990790in}}%
\pgfpathlineto{\pgfqpoint{5.001653in}{1.988272in}}%
\pgfpathlineto{\pgfqpoint{5.027033in}{1.971312in}}%
\pgfpathlineto{\pgfqpoint{5.034909in}{1.965945in}}%
\pgfpathlineto{\pgfqpoint{5.055265in}{1.951834in}}%
\pgfpathlineto{\pgfqpoint{5.068166in}{1.942712in}}%
\pgfpathlineto{\pgfqpoint{5.082564in}{1.932357in}}%
\pgfpathlineto{\pgfqpoint{5.101422in}{1.918517in}}%
\pgfpathlineto{\pgfqpoint{5.108978in}{1.912879in}}%
\pgfpathlineto{\pgfqpoint{5.134545in}{1.893401in}}%
\pgfpathlineto{\pgfqpoint{5.134678in}{1.893297in}}%
\pgfpathlineto{\pgfqpoint{5.159102in}{1.873924in}}%
\pgfpathlineto{\pgfqpoint{5.167934in}{1.866764in}}%
\pgfpathlineto{\pgfqpoint{5.182884in}{1.854446in}}%
\pgfpathlineto{\pgfqpoint{5.201190in}{1.839023in}}%
\pgfpathlineto{\pgfqpoint{5.205926in}{1.834968in}}%
\pgfpathlineto{\pgfqpoint{5.228115in}{1.815491in}}%
\pgfpathlineto{\pgfqpoint{5.234447in}{1.809790in}}%
\pgfpathlineto{\pgfqpoint{5.249507in}{1.796013in}}%
\pgfpathlineto{\pgfqpoint{5.267703in}{1.778962in}}%
\pgfpathlineto{\pgfqpoint{5.270251in}{1.776536in}}%
\pgfpathlineto{\pgfqpoint{5.290128in}{1.757058in}}%
\pgfpathlineto{\pgfqpoint{5.300959in}{1.746156in}}%
\pgfpathlineto{\pgfqpoint{5.309349in}{1.737580in}}%
\pgfpathlineto{\pgfqpoint{5.327856in}{1.718103in}}%
\pgfpathlineto{\pgfqpoint{5.334215in}{1.711199in}}%
\pgfpathlineto{\pgfqpoint{5.345621in}{1.698625in}}%
\pgfpathlineto{\pgfqpoint{5.362759in}{1.679147in}}%
\pgfpathlineto{\pgfqpoint{5.367471in}{1.673602in}}%
\pgfpathlineto{\pgfqpoint{5.379131in}{1.659670in}}%
\pgfpathlineto{\pgfqpoint{5.394898in}{1.640192in}}%
\pgfpathlineto{\pgfqpoint{5.400728in}{1.632714in}}%
\pgfpathlineto{\pgfqpoint{5.409940in}{1.620714in}}%
\pgfpathlineto{\pgfqpoint{5.424336in}{1.601237in}}%
\pgfpathlineto{\pgfqpoint{5.433984in}{1.587654in}}%
\pgfpathlineto{\pgfqpoint{5.438109in}{1.581759in}}%
\pgfpathlineto{\pgfqpoint{5.451133in}{1.562281in}}%
\pgfpathlineto{\pgfqpoint{5.463612in}{1.542804in}}%
\pgfpathlineto{\pgfqpoint{5.467240in}{1.536845in}}%
\pgfpathlineto{\pgfqpoint{5.475347in}{1.523326in}}%
\pgfpathlineto{\pgfqpoint{5.486438in}{1.503849in}}%
\pgfpathlineto{\pgfqpoint{5.496954in}{1.484371in}}%
\pgfpathlineto{\pgfqpoint{5.500496in}{1.477388in}}%
\pgfpathlineto{\pgfqpoint{5.506738in}{1.464893in}}%
\pgfpathlineto{\pgfqpoint{5.515851in}{1.445416in}}%
\pgfpathlineto{\pgfqpoint{5.524358in}{1.425938in}}%
\pgfpathlineto{\pgfqpoint{5.532245in}{1.406460in}}%
\pgfpathlineto{\pgfqpoint{5.533752in}{1.402388in}}%
\pgfpathlineto{\pgfqpoint{5.539367in}{1.386983in}}%
\pgfpathlineto{\pgfqpoint{5.545817in}{1.367505in}}%
\pgfpathlineto{\pgfqpoint{5.551616in}{1.348027in}}%
\pgfpathlineto{\pgfqpoint{5.556747in}{1.328550in}}%
\pgfpathlineto{\pgfqpoint{5.561194in}{1.309072in}}%
\pgfpathlineto{\pgfqpoint{5.564942in}{1.289595in}}%
\pgfpathlineto{\pgfqpoint{5.567008in}{1.276284in}}%
\pgfpathlineto{\pgfqpoint{5.567951in}{1.270117in}}%
\pgfpathlineto{\pgfqpoint{5.570194in}{1.250639in}}%
\pgfpathlineto{\pgfqpoint{5.571701in}{1.231162in}}%
\pgfpathlineto{\pgfqpoint{5.572456in}{1.211684in}}%
\pgfpathlineto{\pgfqpoint{5.572439in}{1.192206in}}%
\pgfpathlineto{\pgfqpoint{5.571633in}{1.172729in}}%
\pgfpathlineto{\pgfqpoint{5.570016in}{1.153251in}}%
\pgfpathlineto{\pgfqpoint{5.567569in}{1.133773in}}%
\pgfpathlineto{\pgfqpoint{5.567008in}{1.130440in}}%
\pgfpathlineto{\pgfqpoint{5.564201in}{1.114296in}}%
\pgfpathlineto{\pgfqpoint{5.559923in}{1.094818in}}%
\pgfpathlineto{\pgfqpoint{5.554724in}{1.075340in}}%
\pgfpathlineto{\pgfqpoint{5.548581in}{1.055863in}}%
\pgfpathlineto{\pgfqpoint{5.541468in}{1.036385in}}%
\pgfpathlineto{\pgfqpoint{5.533752in}{1.017843in}}%
\pgfpathlineto{\pgfqpoint{5.533349in}{1.016908in}}%
\pgfpathlineto{\pgfqpoint{5.523964in}{0.997430in}}%
\pgfpathlineto{\pgfqpoint{5.513494in}{0.977952in}}%
\pgfpathlineto{\pgfqpoint{5.501908in}{0.958475in}}%
\pgfpathlineto{\pgfqpoint{5.500496in}{0.956296in}}%
\pgfpathlineto{\pgfqpoint{5.488843in}{0.938997in}}%
\pgfpathlineto{\pgfqpoint{5.474512in}{0.919519in}}%
\pgfpathlineto{\pgfqpoint{5.467240in}{0.910387in}}%
\pgfpathlineto{\pgfqpoint{5.458663in}{0.900042in}}%
\pgfpathlineto{\pgfqpoint{5.441243in}{0.880564in}}%
\pgfpathlineto{\pgfqpoint{5.433984in}{0.872999in}}%
\pgfpathlineto{\pgfqpoint{5.422055in}{0.861086in}}%
\pgfpathlineto{\pgfqpoint{5.401135in}{0.841609in}}%
\pgfpathlineto{\pgfqpoint{5.400728in}{0.841252in}}%
\pgfpathlineto{\pgfqpoint{5.377908in}{0.822131in}}%
\pgfpathlineto{\pgfqpoint{5.367471in}{0.813917in}}%
\pgfpathlineto{\pgfqpoint{5.352472in}{0.802653in}}%
\pgfpathlineto{\pgfqpoint{5.334215in}{0.789729in}}%
\pgfpathlineto{\pgfqpoint{5.324490in}{0.783176in}}%
\pgfpathlineto{\pgfqpoint{5.300959in}{0.768178in}}%
\pgfpathlineto{\pgfqpoint{5.293556in}{0.763698in}}%
\pgfpathlineto{\pgfqpoint{5.267703in}{0.748857in}}%
\pgfpathlineto{\pgfqpoint{5.259173in}{0.744221in}}%
\pgfpathlineto{\pgfqpoint{5.234447in}{0.731438in}}%
\pgfpathlineto{\pgfqpoint{5.220723in}{0.724743in}}%
\pgfpathlineto{\pgfqpoint{5.201190in}{0.715658in}}%
\pgfpathlineto{\pgfqpoint{5.177432in}{0.705265in}}%
\pgfpathlineto{\pgfqpoint{5.167934in}{0.701296in}}%
\pgfpathlineto{\pgfqpoint{5.134678in}{0.688249in}}%
\pgfpathlineto{\pgfqpoint{5.127981in}{0.685788in}}%
\pgfpathlineto{\pgfqpoint{5.101422in}{0.676439in}}%
\pgfpathlineto{\pgfqpoint{5.070436in}{0.666310in}}%
\pgfpathlineto{\pgfqpoint{5.068166in}{0.665598in}}%
\pgfpathlineto{\pgfqpoint{5.034909in}{0.655873in}}%
\pgfpathlineto{\pgfqpoint{5.001653in}{0.646931in}}%
\pgfpathlineto{\pgfqpoint{5.001255in}{0.646832in}}%
\pgfpathlineto{\pgfqpoint{4.968397in}{0.638979in}}%
\pgfpathlineto{\pgfqpoint{4.935141in}{0.631743in}}%
\pgfpathlineto{\pgfqpoint{4.912915in}{0.627355in}}%
\pgfpathlineto{\pgfqpoint{4.901885in}{0.625256in}}%
\pgfpathlineto{\pgfqpoint{4.868629in}{0.619540in}}%
\pgfpathlineto{\pgfqpoint{4.835372in}{0.614458in}}%
\pgfpathlineto{\pgfqpoint{4.802116in}{0.609990in}}%
\pgfpathlineto{\pgfqpoint{4.784020in}{0.607877in}}%
\pgfpathlineto{\pgfqpoint{4.768860in}{0.606167in}}%
\pgfpathlineto{\pgfqpoint{4.735604in}{0.602965in}}%
\pgfpathlineto{\pgfqpoint{4.702348in}{0.600314in}}%
\pgfpathlineto{\pgfqpoint{4.669091in}{0.598202in}}%
\pgfpathlineto{\pgfqpoint{4.635835in}{0.596614in}}%
\pgfpathlineto{\pgfqpoint{4.602579in}{0.595537in}}%
\pgfpathlineto{\pgfqpoint{4.569323in}{0.594958in}}%
\pgfpathlineto{\pgfqpoint{4.536067in}{0.594866in}}%
\pgfpathlineto{\pgfqpoint{4.502811in}{0.595247in}}%
\pgfpathlineto{\pgfqpoint{4.469554in}{0.596091in}}%
\pgfpathlineto{\pgfqpoint{4.436298in}{0.597386in}}%
\pgfpathlineto{\pgfqpoint{4.403042in}{0.599122in}}%
\pgfpathlineto{\pgfqpoint{4.369786in}{0.601288in}}%
\pgfpathlineto{\pgfqpoint{4.336530in}{0.603874in}}%
\pgfpathlineto{\pgfqpoint{4.303273in}{0.606869in}}%
\pgfpathlineto{\pgfqpoint{4.293459in}{0.607877in}}%
\pgfpathlineto{\pgfqpoint{4.270017in}{0.610320in}}%
\pgfpathlineto{\pgfqpoint{4.236761in}{0.614194in}}%
\pgfpathlineto{\pgfqpoint{4.203505in}{0.618457in}}%
\pgfpathlineto{\pgfqpoint{4.170249in}{0.623101in}}%
\pgfpathlineto{\pgfqpoint{4.142077in}{0.627355in}}%
\pgfpathlineto{\pgfqpoint{4.136993in}{0.628134in}}%
\pgfpathlineto{\pgfqpoint{4.103736in}{0.633634in}}%
\pgfpathlineto{\pgfqpoint{4.070480in}{0.639496in}}%
\pgfpathlineto{\pgfqpoint{4.037224in}{0.645710in}}%
\pgfpathlineto{\pgfqpoint{4.031568in}{0.646832in}}%
\pgfpathlineto{\pgfqpoint{4.003968in}{0.652391in}}%
\pgfpathlineto{\pgfqpoint{3.970712in}{0.659439in}}%
\pgfpathlineto{\pgfqpoint{3.939759in}{0.666310in}}%
\pgfpathlineto{\pgfqpoint{3.937455in}{0.666829in}}%
\pgfpathlineto{\pgfqpoint{3.904199in}{0.674697in}}%
\pgfpathlineto{\pgfqpoint{3.870943in}{0.682863in}}%
\pgfpathlineto{\pgfqpoint{3.859421in}{0.685788in}}%
\pgfpathlineto{\pgfqpoint{3.837687in}{0.691397in}}%
\pgfpathlineto{\pgfqpoint{3.804431in}{0.700118in}}%
\pgfpathlineto{\pgfqpoint{3.784344in}{0.705265in}}%
\pgfpathlineto{\pgfqpoint{3.771175in}{0.708733in}}%
\pgfpathlineto{\pgfqpoint{3.737918in}{0.716476in}}%
\pgfpathlineto{\pgfqpoint{3.704662in}{0.721556in}}%
\pgfpathlineto{\pgfqpoint{3.671406in}{0.720846in}}%
\pgfpathlineto{\pgfqpoint{3.638150in}{0.712276in}}%
\pgfpathlineto{\pgfqpoint{3.620754in}{0.705265in}}%
\pgfpathlineto{\pgfqpoint{3.604894in}{0.698788in}}%
\pgfpathlineto{\pgfqpoint{3.574984in}{0.685788in}}%
\pgfpathlineto{\pgfqpoint{3.571637in}{0.684377in}}%
\pgfpathlineto{\pgfqpoint{3.538381in}{0.670794in}}%
\pgfpathlineto{\pgfqpoint{3.526785in}{0.666310in}}%
\pgfpathlineto{\pgfqpoint{3.505125in}{0.658276in}}%
\pgfpathlineto{\pgfqpoint{3.472139in}{0.646832in}}%
\pgfpathlineto{\pgfqpoint{3.471869in}{0.646742in}}%
\pgfpathlineto{\pgfqpoint{3.438613in}{0.636357in}}%
\pgfpathlineto{\pgfqpoint{3.407444in}{0.627355in}}%
\pgfpathlineto{\pgfqpoint{3.405357in}{0.626775in}}%
\pgfpathlineto{\pgfqpoint{3.372100in}{0.618186in}}%
\pgfpathlineto{\pgfqpoint{3.338844in}{0.610314in}}%
\pgfpathlineto{\pgfqpoint{3.327622in}{0.607877in}}%
\pgfpathlineto{\pgfqpoint{3.305588in}{0.603264in}}%
\pgfpathlineto{\pgfqpoint{3.272332in}{0.596935in}}%
\pgfpathlineto{\pgfqpoint{3.239076in}{0.591243in}}%
\pgfpathlineto{\pgfqpoint{3.220507in}{0.588399in}}%
\pgfpathlineto{\pgfqpoint{3.205819in}{0.586227in}}%
\pgfpathlineto{\pgfqpoint{3.172563in}{0.581870in}}%
\pgfpathlineto{\pgfqpoint{3.139307in}{0.578084in}}%
\pgfpathlineto{\pgfqpoint{3.106051in}{0.574854in}}%
\pgfpathlineto{\pgfqpoint{3.072795in}{0.572166in}}%
\pgfpathlineto{\pgfqpoint{3.039539in}{0.570006in}}%
\pgfpathlineto{\pgfqpoint{3.017663in}{0.568922in}}%
\pgfpathlineto{\pgfqpoint{3.006282in}{0.568375in}}%
\pgfpathlineto{\pgfqpoint{2.973026in}{0.567261in}}%
\pgfpathlineto{\pgfqpoint{2.939770in}{0.566624in}}%
\pgfpathlineto{\pgfqpoint{2.906514in}{0.566455in}}%
\pgfpathlineto{\pgfqpoint{2.873258in}{0.566742in}}%
\pgfpathlineto{\pgfqpoint{2.840001in}{0.567474in}}%
\pgfpathlineto{\pgfqpoint{2.806745in}{0.568642in}}%
\pgfpathclose%
\pgfusepath{stroke}%
\end{pgfscope}%
\begin{pgfscope}%
\pgfpathrectangle{\pgfqpoint{0.711606in}{0.549444in}}{\pgfqpoint{4.955171in}{2.902168in}}%
\pgfusepath{clip}%
\pgfsetbuttcap%
\pgfsetroundjoin%
\pgfsetlinewidth{1.003750pt}%
\definecolor{currentstroke}{rgb}{0.176493,0.041402,0.348111}%
\pgfsetstrokecolor{currentstroke}%
\pgfsetdash{}{0pt}%
\pgfpathmoveto{\pgfqpoint{2.529049in}{0.549444in}}%
\pgfpathlineto{\pgfqpoint{2.507440in}{0.552679in}}%
\pgfpathlineto{\pgfqpoint{2.474183in}{0.558013in}}%
\pgfpathlineto{\pgfqpoint{2.440927in}{0.563683in}}%
\pgfpathlineto{\pgfqpoint{2.411903in}{0.568922in}}%
\pgfpathlineto{\pgfqpoint{2.407671in}{0.569696in}}%
\pgfpathlineto{\pgfqpoint{2.374415in}{0.576147in}}%
\pgfpathlineto{\pgfqpoint{2.341159in}{0.582916in}}%
\pgfpathlineto{\pgfqpoint{2.315453in}{0.588399in}}%
\pgfpathlineto{\pgfqpoint{2.307902in}{0.590032in}}%
\pgfpathlineto{\pgfqpoint{2.274646in}{0.597574in}}%
\pgfpathlineto{\pgfqpoint{2.241390in}{0.605419in}}%
\pgfpathlineto{\pgfqpoint{2.231398in}{0.607877in}}%
\pgfpathlineto{\pgfqpoint{2.208134in}{0.613680in}}%
\pgfpathlineto{\pgfqpoint{2.174878in}{0.622287in}}%
\pgfpathlineto{\pgfqpoint{2.155989in}{0.627355in}}%
\pgfpathlineto{\pgfqpoint{2.141622in}{0.631263in}}%
\pgfpathlineto{\pgfqpoint{2.108365in}{0.640632in}}%
\pgfpathlineto{\pgfqpoint{2.087046in}{0.646832in}}%
\pgfpathlineto{\pgfqpoint{2.075109in}{0.650353in}}%
\pgfpathlineto{\pgfqpoint{2.041853in}{0.660482in}}%
\pgfpathlineto{\pgfqpoint{2.023274in}{0.666310in}}%
\pgfpathlineto{\pgfqpoint{2.008597in}{0.670979in}}%
\pgfpathlineto{\pgfqpoint{1.975341in}{0.681867in}}%
\pgfpathlineto{\pgfqpoint{1.963702in}{0.685788in}}%
\pgfpathlineto{\pgfqpoint{1.942084in}{0.693172in}}%
\pgfpathlineto{\pgfqpoint{1.908828in}{0.704819in}}%
\pgfpathlineto{\pgfqpoint{1.907588in}{0.705265in}}%
\pgfpathlineto{\pgfqpoint{1.875572in}{0.716964in}}%
\pgfpathlineto{\pgfqpoint{1.854772in}{0.724743in}}%
\pgfpathlineto{\pgfqpoint{1.842316in}{0.729468in}}%
\pgfpathlineto{\pgfqpoint{1.809060in}{0.742387in}}%
\pgfpathlineto{\pgfqpoint{1.804455in}{0.744221in}}%
\pgfpathlineto{\pgfqpoint{1.775804in}{0.755794in}}%
\pgfpathlineto{\pgfqpoint{1.756651in}{0.763698in}}%
\pgfpathlineto{\pgfqpoint{1.742547in}{0.769604in}}%
\pgfpathlineto{\pgfqpoint{1.710812in}{0.783176in}}%
\pgfpathlineto{\pgfqpoint{1.709291in}{0.783836in}}%
\pgfpathlineto{\pgfqpoint{1.676035in}{0.798602in}}%
\pgfpathlineto{\pgfqpoint{1.667093in}{0.802653in}}%
\pgfpathlineto{\pgfqpoint{1.642779in}{0.813836in}}%
\pgfpathlineto{\pgfqpoint{1.625092in}{0.822131in}}%
\pgfpathlineto{\pgfqpoint{1.609523in}{0.829544in}}%
\pgfpathlineto{\pgfqpoint{1.584666in}{0.841609in}}%
\pgfpathlineto{\pgfqpoint{1.576266in}{0.845749in}}%
\pgfpathlineto{\pgfqpoint{1.545729in}{0.861086in}}%
\pgfpathlineto{\pgfqpoint{1.543010in}{0.862473in}}%
\pgfpathlineto{\pgfqpoint{1.509754in}{0.879762in}}%
\pgfpathlineto{\pgfqpoint{1.508239in}{0.880564in}}%
\pgfpathlineto{\pgfqpoint{1.476498in}{0.897642in}}%
\pgfpathlineto{\pgfqpoint{1.472115in}{0.900042in}}%
\pgfpathlineto{\pgfqpoint{1.443242in}{0.916108in}}%
\pgfpathlineto{\pgfqpoint{1.437216in}{0.919519in}}%
\pgfpathlineto{\pgfqpoint{1.409986in}{0.935188in}}%
\pgfpathlineto{\pgfqpoint{1.403478in}{0.938997in}}%
\pgfpathlineto{\pgfqpoint{1.376729in}{0.954914in}}%
\pgfpathlineto{\pgfqpoint{1.370846in}{0.958475in}}%
\pgfpathlineto{\pgfqpoint{1.343473in}{0.975319in}}%
\pgfpathlineto{\pgfqpoint{1.339263in}{0.977952in}}%
\pgfpathlineto{\pgfqpoint{1.310217in}{0.996436in}}%
\pgfpathlineto{\pgfqpoint{1.308680in}{0.997430in}}%
\pgfpathlineto{\pgfqpoint{1.279101in}{1.016908in}}%
\pgfpathlineto{\pgfqpoint{1.276961in}{1.018342in}}%
\pgfpathlineto{\pgfqpoint{1.250485in}{1.036385in}}%
\pgfpathlineto{\pgfqpoint{1.243705in}{1.041091in}}%
\pgfpathlineto{\pgfqpoint{1.222753in}{1.055863in}}%
\pgfpathlineto{\pgfqpoint{1.210448in}{1.064700in}}%
\pgfpathlineto{\pgfqpoint{1.195863in}{1.075340in}}%
\pgfpathlineto{\pgfqpoint{1.177192in}{1.089219in}}%
\pgfpathlineto{\pgfqpoint{1.169776in}{1.094818in}}%
\pgfpathlineto{\pgfqpoint{1.144468in}{1.114296in}}%
\pgfpathlineto{\pgfqpoint{1.143936in}{1.114714in}}%
\pgfpathlineto{\pgfqpoint{1.120079in}{1.133773in}}%
\pgfpathlineto{\pgfqpoint{1.110680in}{1.141435in}}%
\pgfpathlineto{\pgfqpoint{1.096400in}{1.153251in}}%
\pgfpathlineto{\pgfqpoint{1.077424in}{1.169281in}}%
\pgfpathlineto{\pgfqpoint{1.073402in}{1.172729in}}%
\pgfpathlineto{\pgfqpoint{1.051207in}{1.192206in}}%
\pgfpathlineto{\pgfqpoint{1.044168in}{1.198529in}}%
\pgfpathlineto{\pgfqpoint{1.029733in}{1.211684in}}%
\pgfpathlineto{\pgfqpoint{1.010911in}{1.229222in}}%
\pgfpathlineto{\pgfqpoint{1.008860in}{1.231162in}}%
\pgfpathlineto{\pgfqpoint{0.988802in}{1.250639in}}%
\pgfpathlineto{\pgfqpoint{0.977655in}{1.261729in}}%
\pgfpathlineto{\pgfqpoint{0.969345in}{1.270117in}}%
\pgfpathlineto{\pgfqpoint{0.950551in}{1.289595in}}%
\pgfpathlineto{\pgfqpoint{0.944399in}{1.296153in}}%
\pgfpathlineto{\pgfqpoint{0.932452in}{1.309072in}}%
\pgfpathlineto{\pgfqpoint{0.914923in}{1.328550in}}%
\pgfpathlineto{\pgfqpoint{0.911143in}{1.332886in}}%
\pgfpathlineto{\pgfqpoint{0.898127in}{1.348027in}}%
\pgfpathlineto{\pgfqpoint{0.881865in}{1.367505in}}%
\pgfpathlineto{\pgfqpoint{0.877887in}{1.372437in}}%
\pgfpathlineto{\pgfqpoint{0.866318in}{1.386983in}}%
\pgfpathlineto{\pgfqpoint{0.851323in}{1.406460in}}%
\pgfpathlineto{\pgfqpoint{0.844630in}{1.415476in}}%
\pgfpathlineto{\pgfqpoint{0.836972in}{1.425938in}}%
\pgfpathlineto{\pgfqpoint{0.823245in}{1.445416in}}%
\pgfpathlineto{\pgfqpoint{0.811374in}{1.462902in}}%
\pgfpathlineto{\pgfqpoint{0.810041in}{1.464893in}}%
\pgfpathlineto{\pgfqpoint{0.797581in}{1.484371in}}%
\pgfpathlineto{\pgfqpoint{0.785630in}{1.503849in}}%
\pgfpathlineto{\pgfqpoint{0.778118in}{1.516680in}}%
\pgfpathlineto{\pgfqpoint{0.774281in}{1.523326in}}%
\pgfpathlineto{\pgfqpoint{0.763611in}{1.542804in}}%
\pgfpathlineto{\pgfqpoint{0.753477in}{1.562281in}}%
\pgfpathlineto{\pgfqpoint{0.744862in}{1.579795in}}%
\pgfpathlineto{\pgfqpoint{0.743909in}{1.581759in}}%
\pgfpathlineto{\pgfqpoint{0.735069in}{1.601237in}}%
\pgfpathlineto{\pgfqpoint{0.726790in}{1.620714in}}%
\pgfpathlineto{\pgfqpoint{0.719084in}{1.640192in}}%
\pgfpathlineto{\pgfqpoint{0.711965in}{1.659670in}}%
\pgfpathlineto{\pgfqpoint{0.711606in}{1.660749in}}%
\pgfusepath{stroke}%
\end{pgfscope}%
\begin{pgfscope}%
\pgfpathrectangle{\pgfqpoint{0.711606in}{0.549444in}}{\pgfqpoint{4.955171in}{2.902168in}}%
\pgfusepath{clip}%
\pgfsetbuttcap%
\pgfsetroundjoin%
\pgfsetlinewidth{1.003750pt}%
\definecolor{currentstroke}{rgb}{0.176493,0.041402,0.348111}%
\pgfsetstrokecolor{currentstroke}%
\pgfsetdash{}{0pt}%
\pgfpathmoveto{\pgfqpoint{4.432369in}{0.549444in}}%
\pgfpathlineto{\pgfqpoint{4.403042in}{0.551230in}}%
\pgfpathlineto{\pgfqpoint{4.369786in}{0.553663in}}%
\pgfpathlineto{\pgfqpoint{4.336530in}{0.556495in}}%
\pgfpathlineto{\pgfqpoint{4.303273in}{0.559714in}}%
\pgfpathlineto{\pgfqpoint{4.270017in}{0.563314in}}%
\pgfpathlineto{\pgfqpoint{4.236761in}{0.567285in}}%
\pgfpathlineto{\pgfqpoint{4.224264in}{0.568922in}}%
\pgfpathlineto{\pgfqpoint{4.203505in}{0.571679in}}%
\pgfpathlineto{\pgfqpoint{4.170249in}{0.576469in}}%
\pgfpathlineto{\pgfqpoint{4.136993in}{0.581614in}}%
\pgfpathlineto{\pgfqpoint{4.103736in}{0.587104in}}%
\pgfpathlineto{\pgfqpoint{4.096389in}{0.588399in}}%
\pgfpathlineto{\pgfqpoint{4.070480in}{0.593031in}}%
\pgfpathlineto{\pgfqpoint{4.037224in}{0.599321in}}%
\pgfpathlineto{\pgfqpoint{4.003968in}{0.605939in}}%
\pgfpathlineto{\pgfqpoint{3.994724in}{0.607877in}}%
\pgfpathlineto{\pgfqpoint{3.970712in}{0.612983in}}%
\pgfpathlineto{\pgfqpoint{3.937455in}{0.620379in}}%
\pgfpathlineto{\pgfqpoint{3.907290in}{0.627355in}}%
\pgfpathlineto{\pgfqpoint{3.904199in}{0.628080in}}%
\pgfpathlineto{\pgfqpoint{3.870943in}{0.636173in}}%
\pgfpathlineto{\pgfqpoint{3.837687in}{0.644397in}}%
\pgfpathlineto{\pgfqpoint{3.827599in}{0.646832in}}%
\pgfpathlineto{\pgfqpoint{3.804431in}{0.652572in}}%
\pgfpathlineto{\pgfqpoint{3.771175in}{0.659903in}}%
\pgfpathlineto{\pgfqpoint{3.737918in}{0.664757in}}%
\pgfpathlineto{\pgfqpoint{3.704662in}{0.664212in}}%
\pgfpathlineto{\pgfqpoint{3.671406in}{0.656421in}}%
\pgfpathlineto{\pgfqpoint{3.645588in}{0.646832in}}%
\pgfpathlineto{\pgfqpoint{3.638150in}{0.644038in}}%
\pgfpathlineto{\pgfqpoint{3.604894in}{0.630764in}}%
\pgfpathlineto{\pgfqpoint{3.596162in}{0.627355in}}%
\pgfpathlineto{\pgfqpoint{3.571637in}{0.618123in}}%
\pgfpathlineto{\pgfqpoint{3.542853in}{0.607877in}}%
\pgfpathlineto{\pgfqpoint{3.538381in}{0.606345in}}%
\pgfpathlineto{\pgfqpoint{3.505125in}{0.595620in}}%
\pgfpathlineto{\pgfqpoint{3.481096in}{0.588399in}}%
\pgfpathlineto{\pgfqpoint{3.471869in}{0.585728in}}%
\pgfpathlineto{\pgfqpoint{3.438613in}{0.576752in}}%
\pgfpathlineto{\pgfqpoint{3.407177in}{0.568922in}}%
\pgfpathlineto{\pgfqpoint{3.405357in}{0.568484in}}%
\pgfpathlineto{\pgfqpoint{3.372100in}{0.561081in}}%
\pgfpathlineto{\pgfqpoint{3.338844in}{0.554317in}}%
\pgfpathlineto{\pgfqpoint{3.312499in}{0.549444in}}%
\pgfusepath{stroke}%
\end{pgfscope}%
\begin{pgfscope}%
\pgfpathrectangle{\pgfqpoint{0.711606in}{0.549444in}}{\pgfqpoint{4.955171in}{2.902168in}}%
\pgfusepath{clip}%
\pgfsetbuttcap%
\pgfsetroundjoin%
\pgfsetlinewidth{1.003750pt}%
\definecolor{currentstroke}{rgb}{0.176493,0.041402,0.348111}%
\pgfsetstrokecolor{currentstroke}%
\pgfsetdash{}{0pt}%
\pgfpathmoveto{\pgfqpoint{0.711606in}{2.039760in}}%
\pgfpathlineto{\pgfqpoint{0.724286in}{2.068700in}}%
\pgfpathlineto{\pgfqpoint{0.744927in}{2.107655in}}%
\pgfpathlineto{\pgfqpoint{0.770363in}{2.146611in}}%
\pgfpathlineto{\pgfqpoint{0.784937in}{2.166088in}}%
\pgfpathlineto{\pgfqpoint{0.818329in}{2.205044in}}%
\pgfpathlineto{\pgfqpoint{0.844630in}{2.231531in}}%
\pgfpathlineto{\pgfqpoint{0.880551in}{2.263477in}}%
\pgfpathlineto{\pgfqpoint{0.952890in}{2.321909in}}%
\pgfpathlineto{\pgfqpoint{0.966150in}{2.341387in}}%
\pgfpathlineto{\pgfqpoint{0.966706in}{2.360865in}}%
\pgfpathlineto{\pgfqpoint{0.958880in}{2.380342in}}%
\pgfpathlineto{\pgfqpoint{0.934793in}{2.419298in}}%
\pgfpathlineto{\pgfqpoint{0.898198in}{2.477731in}}%
\pgfpathlineto{\pgfqpoint{0.876576in}{2.516686in}}%
\pgfpathlineto{\pgfqpoint{0.857690in}{2.555641in}}%
\pgfpathlineto{\pgfqpoint{0.841298in}{2.594596in}}%
\pgfpathlineto{\pgfqpoint{0.827685in}{2.633552in}}%
\pgfpathlineto{\pgfqpoint{0.816679in}{2.672507in}}%
\pgfpathlineto{\pgfqpoint{0.808473in}{2.711462in}}%
\pgfpathlineto{\pgfqpoint{0.803191in}{2.750418in}}%
\pgfpathlineto{\pgfqpoint{0.800848in}{2.789373in}}%
\pgfpathlineto{\pgfqpoint{0.801589in}{2.828328in}}%
\pgfpathlineto{\pgfqpoint{0.805567in}{2.867283in}}%
\pgfpathlineto{\pgfqpoint{0.812987in}{2.906239in}}%
\pgfpathlineto{\pgfqpoint{0.824232in}{2.945194in}}%
\pgfpathlineto{\pgfqpoint{0.839353in}{2.984149in}}%
\pgfpathlineto{\pgfqpoint{0.848538in}{3.003627in}}%
\pgfpathlineto{\pgfqpoint{0.870482in}{3.042582in}}%
\pgfpathlineto{\pgfqpoint{0.883314in}{3.062060in}}%
\pgfpathlineto{\pgfqpoint{0.913172in}{3.101015in}}%
\pgfpathlineto{\pgfqpoint{0.944399in}{3.134952in}}%
\pgfpathlineto{\pgfqpoint{0.970268in}{3.159448in}}%
\pgfpathlineto{\pgfqpoint{0.993174in}{3.178926in}}%
\pgfpathlineto{\pgfqpoint{1.018259in}{3.198403in}}%
\pgfpathlineto{\pgfqpoint{1.045854in}{3.217881in}}%
\pgfpathlineto{\pgfqpoint{1.077424in}{3.237978in}}%
\pgfpathlineto{\pgfqpoint{1.110680in}{3.257052in}}%
\pgfpathlineto{\pgfqpoint{1.148184in}{3.276314in}}%
\pgfpathlineto{\pgfqpoint{1.190874in}{3.295791in}}%
\pgfpathlineto{\pgfqpoint{1.243705in}{3.316918in}}%
\pgfpathlineto{\pgfqpoint{1.310217in}{3.339286in}}%
\pgfpathlineto{\pgfqpoint{1.376729in}{3.357770in}}%
\pgfpathlineto{\pgfqpoint{1.447377in}{3.373702in}}%
\pgfpathlineto{\pgfqpoint{1.509754in}{3.384926in}}%
\pgfpathlineto{\pgfqpoint{1.576266in}{3.394383in}}%
\pgfpathlineto{\pgfqpoint{1.642779in}{3.401298in}}%
\pgfpathlineto{\pgfqpoint{1.709291in}{3.406006in}}%
\pgfpathlineto{\pgfqpoint{1.775804in}{3.408618in}}%
\pgfpathlineto{\pgfqpoint{1.842316in}{3.409238in}}%
\pgfpathlineto{\pgfqpoint{1.908828in}{3.407963in}}%
\pgfpathlineto{\pgfqpoint{1.975341in}{3.404884in}}%
\pgfpathlineto{\pgfqpoint{2.041853in}{3.400086in}}%
\pgfpathlineto{\pgfqpoint{2.112436in}{3.393180in}}%
\pgfpathlineto{\pgfqpoint{2.174878in}{3.385478in}}%
\pgfpathlineto{\pgfqpoint{2.254090in}{3.373702in}}%
\pgfpathlineto{\pgfqpoint{2.341159in}{3.358178in}}%
\pgfpathlineto{\pgfqpoint{2.407671in}{3.344466in}}%
\pgfpathlineto{\pgfqpoint{2.474183in}{3.329205in}}%
\pgfpathlineto{\pgfqpoint{2.540696in}{3.312363in}}%
\pgfpathlineto{\pgfqpoint{2.607208in}{3.293905in}}%
\pgfpathlineto{\pgfqpoint{2.673720in}{3.273798in}}%
\pgfpathlineto{\pgfqpoint{2.740233in}{3.252004in}}%
\pgfpathlineto{\pgfqpoint{2.806745in}{3.228489in}}%
\pgfpathlineto{\pgfqpoint{2.885358in}{3.198403in}}%
\pgfpathlineto{\pgfqpoint{2.939770in}{3.176072in}}%
\pgfpathlineto{\pgfqpoint{3.021501in}{3.139970in}}%
\pgfpathlineto{\pgfqpoint{3.072795in}{3.115734in}}%
\pgfpathlineto{\pgfqpoint{3.140841in}{3.081537in}}%
\pgfpathlineto{\pgfqpoint{3.212739in}{3.042582in}}%
\pgfpathlineto{\pgfqpoint{3.279540in}{3.003627in}}%
\pgfpathlineto{\pgfqpoint{3.341775in}{2.964672in}}%
\pgfpathlineto{\pgfqpoint{3.405357in}{2.921810in}}%
\pgfpathlineto{\pgfqpoint{3.471869in}{2.873233in}}%
\pgfpathlineto{\pgfqpoint{3.505125in}{2.847370in}}%
\pgfpathlineto{\pgfqpoint{3.551733in}{2.808850in}}%
\pgfpathlineto{\pgfqpoint{3.604894in}{2.761506in}}%
\pgfpathlineto{\pgfqpoint{3.638150in}{2.729728in}}%
\pgfpathlineto{\pgfqpoint{3.675049in}{2.691985in}}%
\pgfpathlineto{\pgfqpoint{3.710394in}{2.653029in}}%
\pgfpathlineto{\pgfqpoint{3.743107in}{2.614074in}}%
\pgfpathlineto{\pgfqpoint{3.773403in}{2.575119in}}%
\pgfpathlineto{\pgfqpoint{3.816430in}{2.516686in}}%
\pgfpathlineto{\pgfqpoint{3.837687in}{2.492047in}}%
\pgfpathlineto{\pgfqpoint{3.855068in}{2.477731in}}%
\pgfpathlineto{\pgfqpoint{3.870943in}{2.468416in}}%
\pgfpathlineto{\pgfqpoint{3.904199in}{2.455645in}}%
\pgfpathlineto{\pgfqpoint{3.937455in}{2.447108in}}%
\pgfpathlineto{\pgfqpoint{4.149373in}{2.399820in}}%
\pgfpathlineto{\pgfqpoint{4.236761in}{2.376845in}}%
\pgfpathlineto{\pgfqpoint{4.336530in}{2.347353in}}%
\pgfpathlineto{\pgfqpoint{4.414268in}{2.321909in}}%
\pgfpathlineto{\pgfqpoint{4.502811in}{2.290192in}}%
\pgfpathlineto{\pgfqpoint{4.571623in}{2.263477in}}%
\pgfpathlineto{\pgfqpoint{4.663942in}{2.224521in}}%
\pgfpathlineto{\pgfqpoint{4.735604in}{2.191715in}}%
\pgfpathlineto{\pgfqpoint{4.802116in}{2.159132in}}%
\pgfpathlineto{\pgfqpoint{4.868629in}{2.124355in}}%
\pgfpathlineto{\pgfqpoint{4.935141in}{2.087159in}}%
\pgfpathlineto{\pgfqpoint{5.001653in}{2.047286in}}%
\pgfpathlineto{\pgfqpoint{5.068166in}{2.004443in}}%
\pgfpathlineto{\pgfqpoint{5.134678in}{1.958291in}}%
\pgfpathlineto{\pgfqpoint{5.169930in}{1.932357in}}%
\pgfpathlineto{\pgfqpoint{5.234447in}{1.881619in}}%
\pgfpathlineto{\pgfqpoint{5.267703in}{1.853770in}}%
\pgfpathlineto{\pgfqpoint{5.310741in}{1.815491in}}%
\pgfpathlineto{\pgfqpoint{5.351737in}{1.776536in}}%
\pgfpathlineto{\pgfqpoint{5.389986in}{1.737580in}}%
\pgfpathlineto{\pgfqpoint{5.425547in}{1.698625in}}%
\pgfpathlineto{\pgfqpoint{5.458476in}{1.659670in}}%
\pgfpathlineto{\pgfqpoint{5.488831in}{1.620714in}}%
\pgfpathlineto{\pgfqpoint{5.516665in}{1.581759in}}%
\pgfpathlineto{\pgfqpoint{5.542031in}{1.542804in}}%
\pgfpathlineto{\pgfqpoint{5.567008in}{1.500032in}}%
\pgfpathlineto{\pgfqpoint{5.585246in}{1.464893in}}%
\pgfpathlineto{\pgfqpoint{5.603190in}{1.425938in}}%
\pgfpathlineto{\pgfqpoint{5.618500in}{1.386983in}}%
\pgfpathlineto{\pgfqpoint{5.633521in}{1.340441in}}%
\pgfpathlineto{\pgfqpoint{5.641564in}{1.309072in}}%
\pgfpathlineto{\pgfqpoint{5.649098in}{1.270117in}}%
\pgfpathlineto{\pgfqpoint{5.653918in}{1.231162in}}%
\pgfpathlineto{\pgfqpoint{5.655896in}{1.192206in}}%
\pgfpathlineto{\pgfqpoint{5.654900in}{1.153251in}}%
\pgfpathlineto{\pgfqpoint{5.650785in}{1.114296in}}%
\pgfpathlineto{\pgfqpoint{5.643400in}{1.075340in}}%
\pgfpathlineto{\pgfqpoint{5.632560in}{1.036385in}}%
\pgfpathlineto{\pgfqpoint{5.617769in}{0.997430in}}%
\pgfpathlineto{\pgfqpoint{5.599046in}{0.958475in}}%
\pgfpathlineto{\pgfqpoint{5.575624in}{0.919519in}}%
\pgfpathlineto{\pgfqpoint{5.562115in}{0.900042in}}%
\pgfpathlineto{\pgfqpoint{5.530930in}{0.861086in}}%
\pgfpathlineto{\pgfqpoint{5.500496in}{0.828959in}}%
\pgfpathlineto{\pgfqpoint{5.472138in}{0.802653in}}%
\pgfpathlineto{\pgfqpoint{5.448718in}{0.783176in}}%
\pgfpathlineto{\pgfqpoint{5.423172in}{0.763698in}}%
\pgfpathlineto{\pgfqpoint{5.395191in}{0.744221in}}%
\pgfpathlineto{\pgfqpoint{5.364406in}{0.724743in}}%
\pgfpathlineto{\pgfqpoint{5.330369in}{0.705265in}}%
\pgfpathlineto{\pgfqpoint{5.292531in}{0.685788in}}%
\pgfpathlineto{\pgfqpoint{5.234447in}{0.659598in}}%
\pgfpathlineto{\pgfqpoint{5.201190in}{0.646307in}}%
\pgfpathlineto{\pgfqpoint{5.134678in}{0.623206in}}%
\pgfpathlineto{\pgfqpoint{5.068166in}{0.603917in}}%
\pgfpathlineto{\pgfqpoint{5.001653in}{0.587931in}}%
\pgfpathlineto{\pgfqpoint{4.935141in}{0.574978in}}%
\pgfpathlineto{\pgfqpoint{4.868629in}{0.564613in}}%
\pgfpathlineto{\pgfqpoint{4.802116in}{0.556685in}}%
\pgfpathlineto{\pgfqpoint{4.735604in}{0.550928in}}%
\pgfpathlineto{\pgfqpoint{4.712129in}{0.549444in}}%
\pgfpathlineto{\pgfqpoint{4.712129in}{0.549444in}}%
\pgfusepath{stroke}%
\end{pgfscope}%
\begin{pgfscope}%
\pgfpathrectangle{\pgfqpoint{0.711606in}{0.549444in}}{\pgfqpoint{4.955171in}{2.902168in}}%
\pgfusepath{clip}%
\pgfsetbuttcap%
\pgfsetroundjoin%
\pgfsetlinewidth{1.003750pt}%
\definecolor{currentstroke}{rgb}{0.204209,0.037632,0.373238}%
\pgfsetstrokecolor{currentstroke}%
\pgfsetdash{}{0pt}%
\pgfpathmoveto{\pgfqpoint{2.296866in}{0.549444in}}%
\pgfpathlineto{\pgfqpoint{2.274646in}{0.554468in}}%
\pgfpathlineto{\pgfqpoint{2.241390in}{0.562297in}}%
\pgfpathlineto{\pgfqpoint{2.214272in}{0.568922in}}%
\pgfpathlineto{\pgfqpoint{2.208134in}{0.570441in}}%
\pgfpathlineto{\pgfqpoint{2.174878in}{0.579002in}}%
\pgfpathlineto{\pgfqpoint{2.141622in}{0.587841in}}%
\pgfpathlineto{\pgfqpoint{2.139595in}{0.588399in}}%
\pgfpathlineto{\pgfqpoint{2.108365in}{0.597123in}}%
\pgfpathlineto{\pgfqpoint{2.075109in}{0.606685in}}%
\pgfpathlineto{\pgfqpoint{2.071099in}{0.607877in}}%
\pgfpathlineto{\pgfqpoint{2.041853in}{0.616685in}}%
\pgfpathlineto{\pgfqpoint{2.008597in}{0.626970in}}%
\pgfpathlineto{\pgfqpoint{2.007391in}{0.627355in}}%
\pgfpathlineto{\pgfqpoint{1.975341in}{0.637717in}}%
\pgfpathlineto{\pgfqpoint{1.947830in}{0.646832in}}%
\pgfpathlineto{\pgfqpoint{1.942084in}{0.648762in}}%
\pgfpathlineto{\pgfqpoint{1.908828in}{0.660245in}}%
\pgfpathlineto{\pgfqpoint{1.891687in}{0.666310in}}%
\pgfpathlineto{\pgfqpoint{1.875572in}{0.672089in}}%
\pgfpathlineto{\pgfqpoint{1.842316in}{0.684299in}}%
\pgfpathlineto{\pgfqpoint{1.838364in}{0.685788in}}%
\pgfpathlineto{\pgfqpoint{1.809060in}{0.696973in}}%
\pgfpathlineto{\pgfqpoint{1.787797in}{0.705265in}}%
\pgfpathlineto{\pgfqpoint{1.775804in}{0.710007in}}%
\pgfpathlineto{\pgfqpoint{1.742547in}{0.723445in}}%
\pgfpathlineto{\pgfqpoint{1.739408in}{0.724743in}}%
\pgfpathlineto{\pgfqpoint{1.709291in}{0.737368in}}%
\pgfpathlineto{\pgfqpoint{1.693266in}{0.744221in}}%
\pgfpathlineto{\pgfqpoint{1.676035in}{0.751694in}}%
\pgfpathlineto{\pgfqpoint{1.648891in}{0.763698in}}%
\pgfpathlineto{\pgfqpoint{1.642779in}{0.766440in}}%
\pgfpathlineto{\pgfqpoint{1.609523in}{0.781659in}}%
\pgfpathlineto{\pgfqpoint{1.606274in}{0.783176in}}%
\pgfpathlineto{\pgfqpoint{1.576266in}{0.797388in}}%
\pgfpathlineto{\pgfqpoint{1.565351in}{0.802653in}}%
\pgfpathlineto{\pgfqpoint{1.543010in}{0.813588in}}%
\pgfpathlineto{\pgfqpoint{1.525867in}{0.822131in}}%
\pgfpathlineto{\pgfqpoint{1.509754in}{0.830280in}}%
\pgfpathlineto{\pgfqpoint{1.487745in}{0.841609in}}%
\pgfpathlineto{\pgfqpoint{1.476498in}{0.847486in}}%
\pgfpathlineto{\pgfqpoint{1.450917in}{0.861086in}}%
\pgfpathlineto{\pgfqpoint{1.443242in}{0.865230in}}%
\pgfpathlineto{\pgfqpoint{1.415315in}{0.880564in}}%
\pgfpathlineto{\pgfqpoint{1.409986in}{0.883536in}}%
\pgfpathlineto{\pgfqpoint{1.380880in}{0.900042in}}%
\pgfpathlineto{\pgfqpoint{1.376729in}{0.902433in}}%
\pgfpathlineto{\pgfqpoint{1.347554in}{0.919519in}}%
\pgfpathlineto{\pgfqpoint{1.343473in}{0.921947in}}%
\pgfpathlineto{\pgfqpoint{1.315283in}{0.938997in}}%
\pgfpathlineto{\pgfqpoint{1.310217in}{0.942111in}}%
\pgfpathlineto{\pgfqpoint{1.284017in}{0.958475in}}%
\pgfpathlineto{\pgfqpoint{1.276961in}{0.962955in}}%
\pgfpathlineto{\pgfqpoint{1.253709in}{0.977952in}}%
\pgfpathlineto{\pgfqpoint{1.243705in}{0.984514in}}%
\pgfpathlineto{\pgfqpoint{1.224315in}{0.997430in}}%
\pgfpathlineto{\pgfqpoint{1.210448in}{1.006826in}}%
\pgfpathlineto{\pgfqpoint{1.195794in}{1.016908in}}%
\pgfpathlineto{\pgfqpoint{1.177192in}{1.029929in}}%
\pgfpathlineto{\pgfqpoint{1.168107in}{1.036385in}}%
\pgfpathlineto{\pgfqpoint{1.143936in}{1.053866in}}%
\pgfpathlineto{\pgfqpoint{1.141216in}{1.055863in}}%
\pgfpathlineto{\pgfqpoint{1.115186in}{1.075340in}}%
\pgfpathlineto{\pgfqpoint{1.110680in}{1.078778in}}%
\pgfpathlineto{\pgfqpoint{1.089959in}{1.094818in}}%
\pgfpathlineto{\pgfqpoint{1.077424in}{1.104706in}}%
\pgfpathlineto{\pgfqpoint{1.065441in}{1.114296in}}%
\pgfpathlineto{\pgfqpoint{1.044168in}{1.131651in}}%
\pgfpathlineto{\pgfqpoint{1.041603in}{1.133773in}}%
\pgfpathlineto{\pgfqpoint{1.018576in}{1.153251in}}%
\pgfpathlineto{\pgfqpoint{1.010911in}{1.159872in}}%
\pgfpathlineto{\pgfqpoint{0.996236in}{1.172729in}}%
\pgfpathlineto{\pgfqpoint{0.977655in}{1.189346in}}%
\pgfpathlineto{\pgfqpoint{0.974501in}{1.192206in}}%
\pgfpathlineto{\pgfqpoint{0.953534in}{1.211684in}}%
\pgfpathlineto{\pgfqpoint{0.944399in}{1.220366in}}%
\pgfpathlineto{\pgfqpoint{0.933196in}{1.231162in}}%
\pgfpathlineto{\pgfqpoint{0.913443in}{1.250639in}}%
\pgfpathlineto{\pgfqpoint{0.911143in}{1.252969in}}%
\pgfpathlineto{\pgfqpoint{0.894447in}{1.270117in}}%
\pgfpathlineto{\pgfqpoint{0.877887in}{1.287529in}}%
\pgfpathlineto{\pgfqpoint{0.875949in}{1.289595in}}%
\pgfpathlineto{\pgfqpoint{0.858202in}{1.309072in}}%
\pgfpathlineto{\pgfqpoint{0.844630in}{1.324356in}}%
\pgfpathlineto{\pgfqpoint{0.840956in}{1.328550in}}%
\pgfpathlineto{\pgfqpoint{0.824413in}{1.348027in}}%
\pgfpathlineto{\pgfqpoint{0.811374in}{1.363817in}}%
\pgfpathlineto{\pgfqpoint{0.808370in}{1.367505in}}%
\pgfpathlineto{\pgfqpoint{0.793030in}{1.386983in}}%
\pgfpathlineto{\pgfqpoint{0.778142in}{1.406460in}}%
\pgfpathlineto{\pgfqpoint{0.778118in}{1.406493in}}%
\pgfpathlineto{\pgfqpoint{0.764006in}{1.425938in}}%
\pgfpathlineto{\pgfqpoint{0.750336in}{1.445416in}}%
\pgfpathlineto{\pgfqpoint{0.744862in}{1.453529in}}%
\pgfpathlineto{\pgfqpoint{0.737296in}{1.464893in}}%
\pgfpathlineto{\pgfqpoint{0.724843in}{1.484371in}}%
\pgfpathlineto{\pgfqpoint{0.712879in}{1.503849in}}%
\pgfpathlineto{\pgfqpoint{0.711606in}{1.506026in}}%
\pgfusepath{stroke}%
\end{pgfscope}%
\begin{pgfscope}%
\pgfpathrectangle{\pgfqpoint{0.711606in}{0.549444in}}{\pgfqpoint{4.955171in}{2.902168in}}%
\pgfusepath{clip}%
\pgfsetbuttcap%
\pgfsetroundjoin%
\pgfsetlinewidth{1.003750pt}%
\definecolor{currentstroke}{rgb}{0.204209,0.037632,0.373238}%
\pgfsetstrokecolor{currentstroke}%
\pgfsetdash{}{0pt}%
\pgfpathmoveto{\pgfqpoint{4.067055in}{0.549444in}}%
\pgfpathlineto{\pgfqpoint{4.037224in}{0.555130in}}%
\pgfpathlineto{\pgfqpoint{4.003968in}{0.561787in}}%
\pgfpathlineto{\pgfqpoint{3.970712in}{0.568744in}}%
\pgfpathlineto{\pgfqpoint{3.969899in}{0.568922in}}%
\pgfpathlineto{\pgfqpoint{3.937455in}{0.576126in}}%
\pgfpathlineto{\pgfqpoint{3.904199in}{0.583754in}}%
\pgfpathlineto{\pgfqpoint{3.884342in}{0.588399in}}%
\pgfpathlineto{\pgfqpoint{3.870943in}{0.591588in}}%
\pgfpathlineto{\pgfqpoint{3.837687in}{0.599390in}}%
\pgfpathlineto{\pgfqpoint{3.804431in}{0.606456in}}%
\pgfpathlineto{\pgfqpoint{3.794259in}{0.607877in}}%
\pgfpathlineto{\pgfqpoint{3.771175in}{0.611326in}}%
\pgfpathlineto{\pgfqpoint{3.737918in}{0.611313in}}%
\pgfpathlineto{\pgfqpoint{3.720242in}{0.607877in}}%
\pgfpathlineto{\pgfqpoint{3.704662in}{0.604549in}}%
\pgfpathlineto{\pgfqpoint{3.671406in}{0.593247in}}%
\pgfpathlineto{\pgfqpoint{3.658550in}{0.588399in}}%
\pgfpathlineto{\pgfqpoint{3.638150in}{0.580872in}}%
\pgfpathlineto{\pgfqpoint{3.605029in}{0.568922in}}%
\pgfpathlineto{\pgfqpoint{3.604894in}{0.568875in}}%
\pgfpathlineto{\pgfqpoint{3.571637in}{0.557834in}}%
\pgfpathlineto{\pgfqpoint{3.544671in}{0.549444in}}%
\pgfusepath{stroke}%
\end{pgfscope}%
\begin{pgfscope}%
\pgfpathrectangle{\pgfqpoint{0.711606in}{0.549444in}}{\pgfqpoint{4.955171in}{2.902168in}}%
\pgfusepath{clip}%
\pgfsetbuttcap%
\pgfsetroundjoin%
\pgfsetlinewidth{1.003750pt}%
\definecolor{currentstroke}{rgb}{0.204209,0.037632,0.373238}%
\pgfsetstrokecolor{currentstroke}%
\pgfsetdash{}{0pt}%
\pgfpathmoveto{\pgfqpoint{5.666777in}{0.915721in}}%
\pgfpathlineto{\pgfqpoint{5.657214in}{0.900042in}}%
\pgfpathlineto{\pgfqpoint{5.644291in}{0.880564in}}%
\pgfpathlineto{\pgfqpoint{5.633521in}{0.865565in}}%
\pgfpathlineto{\pgfqpoint{5.630188in}{0.861086in}}%
\pgfpathlineto{\pgfqpoint{5.614636in}{0.841609in}}%
\pgfpathlineto{\pgfqpoint{5.600265in}{0.824869in}}%
\pgfpathlineto{\pgfqpoint{5.597824in}{0.822131in}}%
\pgfpathlineto{\pgfqpoint{5.579339in}{0.802653in}}%
\pgfpathlineto{\pgfqpoint{5.567008in}{0.790475in}}%
\pgfpathlineto{\pgfqpoint{5.559323in}{0.783176in}}%
\pgfpathlineto{\pgfqpoint{5.537544in}{0.763698in}}%
\pgfpathlineto{\pgfqpoint{5.533752in}{0.760490in}}%
\pgfpathlineto{\pgfqpoint{5.513712in}{0.744221in}}%
\pgfpathlineto{\pgfqpoint{5.500496in}{0.734074in}}%
\pgfpathlineto{\pgfqpoint{5.487809in}{0.724743in}}%
\pgfpathlineto{\pgfqpoint{5.467240in}{0.710394in}}%
\pgfpathlineto{\pgfqpoint{5.459550in}{0.705265in}}%
\pgfpathlineto{\pgfqpoint{5.433984in}{0.689049in}}%
\pgfpathlineto{\pgfqpoint{5.428595in}{0.685788in}}%
\pgfpathlineto{\pgfqpoint{5.400728in}{0.669712in}}%
\pgfpathlineto{\pgfqpoint{5.394532in}{0.666310in}}%
\pgfpathlineto{\pgfqpoint{5.367471in}{0.652114in}}%
\pgfpathlineto{\pgfqpoint{5.356867in}{0.646832in}}%
\pgfpathlineto{\pgfqpoint{5.334215in}{0.636032in}}%
\pgfpathlineto{\pgfqpoint{5.314991in}{0.627355in}}%
\pgfpathlineto{\pgfqpoint{5.300959in}{0.621281in}}%
\pgfpathlineto{\pgfqpoint{5.268149in}{0.607877in}}%
\pgfpathlineto{\pgfqpoint{5.267703in}{0.607702in}}%
\pgfpathlineto{\pgfqpoint{5.234447in}{0.595359in}}%
\pgfpathlineto{\pgfqpoint{5.214448in}{0.588399in}}%
\pgfpathlineto{\pgfqpoint{5.201190in}{0.583959in}}%
\pgfpathlineto{\pgfqpoint{5.167934in}{0.573517in}}%
\pgfpathlineto{\pgfqpoint{5.152260in}{0.568922in}}%
\pgfpathlineto{\pgfqpoint{5.134678in}{0.563952in}}%
\pgfpathlineto{\pgfqpoint{5.101422in}{0.555207in}}%
\pgfpathlineto{\pgfqpoint{5.077724in}{0.549444in}}%
\pgfusepath{stroke}%
\end{pgfscope}%
\begin{pgfscope}%
\pgfpathrectangle{\pgfqpoint{0.711606in}{0.549444in}}{\pgfqpoint{4.955171in}{2.902168in}}%
\pgfusepath{clip}%
\pgfsetbuttcap%
\pgfsetroundjoin%
\pgfsetlinewidth{1.003750pt}%
\definecolor{currentstroke}{rgb}{0.204209,0.037632,0.373238}%
\pgfsetstrokecolor{currentstroke}%
\pgfsetdash{}{0pt}%
\pgfpathmoveto{\pgfqpoint{0.711606in}{2.194599in}}%
\pgfpathlineto{\pgfqpoint{0.719906in}{2.205044in}}%
\pgfpathlineto{\pgfqpoint{0.736483in}{2.224521in}}%
\pgfpathlineto{\pgfqpoint{0.744862in}{2.233749in}}%
\pgfpathlineto{\pgfqpoint{0.754517in}{2.243999in}}%
\pgfpathlineto{\pgfqpoint{0.774040in}{2.263477in}}%
\pgfpathlineto{\pgfqpoint{0.778118in}{2.267328in}}%
\pgfpathlineto{\pgfqpoint{0.795293in}{2.282954in}}%
\pgfpathlineto{\pgfqpoint{0.811374in}{2.296914in}}%
\pgfpathlineto{\pgfqpoint{0.817954in}{2.302432in}}%
\pgfpathlineto{\pgfqpoint{0.841324in}{2.321909in}}%
\pgfpathlineto{\pgfqpoint{0.844630in}{2.324935in}}%
\pgfpathlineto{\pgfqpoint{0.862167in}{2.341387in}}%
\pgfpathlineto{\pgfqpoint{0.874000in}{2.360865in}}%
\pgfpathlineto{\pgfqpoint{0.874072in}{2.380342in}}%
\pgfpathlineto{\pgfqpoint{0.866440in}{2.399820in}}%
\pgfpathlineto{\pgfqpoint{0.855492in}{2.419298in}}%
\pgfpathlineto{\pgfqpoint{0.844630in}{2.436858in}}%
\pgfpathlineto{\pgfqpoint{0.843472in}{2.438775in}}%
\pgfpathlineto{\pgfqpoint{0.831564in}{2.458253in}}%
\pgfpathlineto{\pgfqpoint{0.819952in}{2.477731in}}%
\pgfpathlineto{\pgfqpoint{0.811374in}{2.492718in}}%
\pgfpathlineto{\pgfqpoint{0.808842in}{2.497208in}}%
\pgfpathlineto{\pgfqpoint{0.798419in}{2.516686in}}%
\pgfpathlineto{\pgfqpoint{0.788537in}{2.536163in}}%
\pgfpathlineto{\pgfqpoint{0.779213in}{2.555641in}}%
\pgfpathlineto{\pgfqpoint{0.778118in}{2.558094in}}%
\pgfpathlineto{\pgfqpoint{0.770627in}{2.575119in}}%
\pgfpathlineto{\pgfqpoint{0.762644in}{2.594596in}}%
\pgfpathlineto{\pgfqpoint{0.755251in}{2.614074in}}%
\pgfpathlineto{\pgfqpoint{0.748464in}{2.633552in}}%
\pgfpathlineto{\pgfqpoint{0.744862in}{2.644958in}}%
\pgfpathlineto{\pgfqpoint{0.742350in}{2.653029in}}%
\pgfpathlineto{\pgfqpoint{0.736934in}{2.672507in}}%
\pgfpathlineto{\pgfqpoint{0.732152in}{2.691985in}}%
\pgfpathlineto{\pgfqpoint{0.728018in}{2.711462in}}%
\pgfpathlineto{\pgfqpoint{0.724547in}{2.730940in}}%
\pgfpathlineto{\pgfqpoint{0.721754in}{2.750418in}}%
\pgfpathlineto{\pgfqpoint{0.719655in}{2.769895in}}%
\pgfpathlineto{\pgfqpoint{0.718266in}{2.789373in}}%
\pgfpathlineto{\pgfqpoint{0.717602in}{2.808850in}}%
\pgfpathlineto{\pgfqpoint{0.717682in}{2.828328in}}%
\pgfpathlineto{\pgfqpoint{0.718522in}{2.847806in}}%
\pgfpathlineto{\pgfqpoint{0.720141in}{2.867283in}}%
\pgfpathlineto{\pgfqpoint{0.722557in}{2.886761in}}%
\pgfpathlineto{\pgfqpoint{0.725790in}{2.906239in}}%
\pgfpathlineto{\pgfqpoint{0.729860in}{2.925716in}}%
\pgfpathlineto{\pgfqpoint{0.734787in}{2.945194in}}%
\pgfpathlineto{\pgfqpoint{0.740592in}{2.964672in}}%
\pgfpathlineto{\pgfqpoint{0.744862in}{2.977111in}}%
\pgfpathlineto{\pgfqpoint{0.747360in}{2.984149in}}%
\pgfpathlineto{\pgfqpoint{0.755184in}{3.003627in}}%
\pgfpathlineto{\pgfqpoint{0.763982in}{3.023104in}}%
\pgfpathlineto{\pgfqpoint{0.773780in}{3.042582in}}%
\pgfpathlineto{\pgfqpoint{0.778118in}{3.050428in}}%
\pgfpathlineto{\pgfqpoint{0.784780in}{3.062060in}}%
\pgfpathlineto{\pgfqpoint{0.796983in}{3.081537in}}%
\pgfpathlineto{\pgfqpoint{0.810304in}{3.101015in}}%
\pgfpathlineto{\pgfqpoint{0.811374in}{3.102468in}}%
\pgfpathlineto{\pgfqpoint{0.825156in}{3.120493in}}%
\pgfpathlineto{\pgfqpoint{0.841262in}{3.139970in}}%
\pgfpathlineto{\pgfqpoint{0.844630in}{3.143778in}}%
\pgfpathlineto{\pgfqpoint{0.859046in}{3.159448in}}%
\pgfpathlineto{\pgfqpoint{0.877887in}{3.178535in}}%
\pgfpathlineto{\pgfqpoint{0.878288in}{3.178926in}}%
\pgfpathlineto{\pgfqpoint{0.899556in}{3.198403in}}%
\pgfpathlineto{\pgfqpoint{0.911143in}{3.208372in}}%
\pgfpathlineto{\pgfqpoint{0.922681in}{3.217881in}}%
\pgfpathlineto{\pgfqpoint{0.944399in}{3.234753in}}%
\pgfpathlineto{\pgfqpoint{0.947907in}{3.237359in}}%
\pgfpathlineto{\pgfqpoint{0.975610in}{3.256836in}}%
\pgfpathlineto{\pgfqpoint{0.977655in}{3.258201in}}%
\pgfpathlineto{\pgfqpoint{1.006116in}{3.276314in}}%
\pgfpathlineto{\pgfqpoint{1.010911in}{3.279215in}}%
\pgfpathlineto{\pgfqpoint{1.039720in}{3.295791in}}%
\pgfpathlineto{\pgfqpoint{1.044168in}{3.298230in}}%
\pgfpathlineto{\pgfqpoint{1.076925in}{3.315269in}}%
\pgfpathlineto{\pgfqpoint{1.077424in}{3.315517in}}%
\pgfpathlineto{\pgfqpoint{1.110680in}{3.331185in}}%
\pgfpathlineto{\pgfqpoint{1.118690in}{3.334747in}}%
\pgfpathlineto{\pgfqpoint{1.143936in}{3.345495in}}%
\pgfpathlineto{\pgfqpoint{1.165741in}{3.354224in}}%
\pgfpathlineto{\pgfqpoint{1.177192in}{3.358621in}}%
\pgfpathlineto{\pgfqpoint{1.210448in}{3.370613in}}%
\pgfpathlineto{\pgfqpoint{1.219590in}{3.373702in}}%
\pgfpathlineto{\pgfqpoint{1.243705in}{3.381533in}}%
\pgfpathlineto{\pgfqpoint{1.276961in}{3.391578in}}%
\pgfpathlineto{\pgfqpoint{1.282652in}{3.393180in}}%
\pgfpathlineto{\pgfqpoint{1.310217in}{3.400650in}}%
\pgfpathlineto{\pgfqpoint{1.343473in}{3.408964in}}%
\pgfpathlineto{\pgfqpoint{1.359516in}{3.412657in}}%
\pgfpathlineto{\pgfqpoint{1.376729in}{3.416480in}}%
\pgfpathlineto{\pgfqpoint{1.409986in}{3.423245in}}%
\pgfpathlineto{\pgfqpoint{1.443242in}{3.429373in}}%
\pgfpathlineto{\pgfqpoint{1.459836in}{3.432135in}}%
\pgfpathlineto{\pgfqpoint{1.476498in}{3.434814in}}%
\pgfpathlineto{\pgfqpoint{1.509754in}{3.439600in}}%
\pgfpathlineto{\pgfqpoint{1.543010in}{3.443816in}}%
\pgfpathlineto{\pgfqpoint{1.576266in}{3.447475in}}%
\pgfpathlineto{\pgfqpoint{1.609523in}{3.450592in}}%
\pgfpathlineto{\pgfqpoint{1.622571in}{3.451613in}}%
\pgfusepath{stroke}%
\end{pgfscope}%
\begin{pgfscope}%
\pgfpathrectangle{\pgfqpoint{0.711606in}{0.549444in}}{\pgfqpoint{4.955171in}{2.902168in}}%
\pgfusepath{clip}%
\pgfsetbuttcap%
\pgfsetroundjoin%
\pgfsetlinewidth{1.003750pt}%
\definecolor{currentstroke}{rgb}{0.204209,0.037632,0.373238}%
\pgfsetstrokecolor{currentstroke}%
\pgfsetdash{}{0pt}%
\pgfpathmoveto{\pgfqpoint{1.989273in}{3.451613in}}%
\pgfpathlineto{\pgfqpoint{2.041853in}{3.447339in}}%
\pgfpathlineto{\pgfqpoint{2.108365in}{3.440478in}}%
\pgfpathlineto{\pgfqpoint{2.174878in}{3.432132in}}%
\pgfpathlineto{\pgfqpoint{2.274646in}{3.416646in}}%
\pgfpathlineto{\pgfqpoint{2.341159in}{3.404446in}}%
\pgfpathlineto{\pgfqpoint{2.407671in}{3.390795in}}%
\pgfpathlineto{\pgfqpoint{2.507440in}{3.367441in}}%
\pgfpathlineto{\pgfqpoint{2.573952in}{3.350007in}}%
\pgfpathlineto{\pgfqpoint{2.673720in}{3.320950in}}%
\pgfpathlineto{\pgfqpoint{2.751633in}{3.295791in}}%
\pgfpathlineto{\pgfqpoint{2.840001in}{3.264547in}}%
\pgfpathlineto{\pgfqpoint{2.910843in}{3.237359in}}%
\pgfpathlineto{\pgfqpoint{3.004184in}{3.198403in}}%
\pgfpathlineto{\pgfqpoint{3.072795in}{3.167328in}}%
\pgfpathlineto{\pgfqpoint{3.139307in}{3.135187in}}%
\pgfpathlineto{\pgfqpoint{3.205819in}{3.100891in}}%
\pgfpathlineto{\pgfqpoint{3.276019in}{3.062060in}}%
\pgfpathlineto{\pgfqpoint{3.341751in}{3.023104in}}%
\pgfpathlineto{\pgfqpoint{3.405357in}{2.982731in}}%
\pgfpathlineto{\pgfqpoint{3.471869in}{2.937290in}}%
\pgfpathlineto{\pgfqpoint{3.514563in}{2.906239in}}%
\pgfpathlineto{\pgfqpoint{3.571637in}{2.862042in}}%
\pgfpathlineto{\pgfqpoint{3.612430in}{2.828328in}}%
\pgfpathlineto{\pgfqpoint{3.656719in}{2.789373in}}%
\pgfpathlineto{\pgfqpoint{3.704662in}{2.744060in}}%
\pgfpathlineto{\pgfqpoint{3.737918in}{2.710401in}}%
\pgfpathlineto{\pgfqpoint{3.772945in}{2.672507in}}%
\pgfpathlineto{\pgfqpoint{3.806523in}{2.633552in}}%
\pgfpathlineto{\pgfqpoint{3.885663in}{2.536163in}}%
\pgfpathlineto{\pgfqpoint{3.906815in}{2.516686in}}%
\pgfpathlineto{\pgfqpoint{3.941858in}{2.497208in}}%
\pgfpathlineto{\pgfqpoint{3.970712in}{2.487426in}}%
\pgfpathlineto{\pgfqpoint{4.037224in}{2.470715in}}%
\pgfpathlineto{\pgfqpoint{4.171298in}{2.438775in}}%
\pgfpathlineto{\pgfqpoint{4.270017in}{2.412098in}}%
\pgfpathlineto{\pgfqpoint{4.369786in}{2.381965in}}%
\pgfpathlineto{\pgfqpoint{4.436298in}{2.359969in}}%
\pgfpathlineto{\pgfqpoint{4.536067in}{2.323995in}}%
\pgfpathlineto{\pgfqpoint{4.602579in}{2.297947in}}%
\pgfpathlineto{\pgfqpoint{4.684532in}{2.263477in}}%
\pgfpathlineto{\pgfqpoint{4.770217in}{2.224521in}}%
\pgfpathlineto{\pgfqpoint{4.849316in}{2.185566in}}%
\pgfpathlineto{\pgfqpoint{4.922983in}{2.146611in}}%
\pgfpathlineto{\pgfqpoint{4.991767in}{2.107655in}}%
\pgfpathlineto{\pgfqpoint{5.056144in}{2.068700in}}%
\pgfpathlineto{\pgfqpoint{5.116532in}{2.029745in}}%
\pgfpathlineto{\pgfqpoint{5.173295in}{1.990790in}}%
\pgfpathlineto{\pgfqpoint{5.234447in}{1.945867in}}%
\pgfpathlineto{\pgfqpoint{5.276672in}{1.912879in}}%
\pgfpathlineto{\pgfqpoint{5.334215in}{1.864898in}}%
\pgfpathlineto{\pgfqpoint{5.367995in}{1.834968in}}%
\pgfpathlineto{\pgfqpoint{5.409357in}{1.796013in}}%
\pgfpathlineto{\pgfqpoint{5.448101in}{1.757058in}}%
\pgfpathlineto{\pgfqpoint{5.484283in}{1.718103in}}%
\pgfpathlineto{\pgfqpoint{5.517954in}{1.679147in}}%
\pgfpathlineto{\pgfqpoint{5.549164in}{1.640192in}}%
\pgfpathlineto{\pgfqpoint{5.577964in}{1.601237in}}%
\pgfpathlineto{\pgfqpoint{5.604400in}{1.562281in}}%
\pgfpathlineto{\pgfqpoint{5.633521in}{1.514451in}}%
\pgfpathlineto{\pgfqpoint{5.650033in}{1.484371in}}%
\pgfpathlineto{\pgfqpoint{5.666777in}{1.450875in}}%
\pgfpathlineto{\pgfqpoint{5.666777in}{1.450875in}}%
\pgfusepath{stroke}%
\end{pgfscope}%
\begin{pgfscope}%
\pgfpathrectangle{\pgfqpoint{0.711606in}{0.549444in}}{\pgfqpoint{4.955171in}{2.902168in}}%
\pgfusepath{clip}%
\pgfsetbuttcap%
\pgfsetroundjoin%
\pgfsetlinewidth{1.003750pt}%
\definecolor{currentstroke}{rgb}{0.224763,0.036405,0.388129}%
\pgfsetstrokecolor{currentstroke}%
\pgfsetdash{}{0pt}%
\pgfpathmoveto{\pgfqpoint{2.130169in}{0.549444in}}%
\pgfpathlineto{\pgfqpoint{2.108365in}{0.555469in}}%
\pgfpathlineto{\pgfqpoint{2.075109in}{0.564940in}}%
\pgfpathlineto{\pgfqpoint{2.061547in}{0.568922in}}%
\pgfpathlineto{\pgfqpoint{2.041853in}{0.574777in}}%
\pgfpathlineto{\pgfqpoint{2.008597in}{0.584944in}}%
\pgfpathlineto{\pgfqpoint{1.997611in}{0.588399in}}%
\pgfpathlineto{\pgfqpoint{1.975341in}{0.595494in}}%
\pgfpathlineto{\pgfqpoint{1.942084in}{0.606355in}}%
\pgfpathlineto{\pgfqpoint{1.937552in}{0.607877in}}%
\pgfpathlineto{\pgfqpoint{1.908828in}{0.617645in}}%
\pgfpathlineto{\pgfqpoint{1.880913in}{0.627355in}}%
\pgfpathlineto{\pgfqpoint{1.875572in}{0.629237in}}%
\pgfpathlineto{\pgfqpoint{1.842316in}{0.641256in}}%
\pgfpathlineto{\pgfqpoint{1.827232in}{0.646832in}}%
\pgfpathlineto{\pgfqpoint{1.809060in}{0.653638in}}%
\pgfpathlineto{\pgfqpoint{1.775925in}{0.666310in}}%
\pgfpathlineto{\pgfqpoint{1.775804in}{0.666357in}}%
\pgfpathlineto{\pgfqpoint{1.742547in}{0.679556in}}%
\pgfpathlineto{\pgfqpoint{1.727157in}{0.685788in}}%
\pgfpathlineto{\pgfqpoint{1.709291in}{0.693118in}}%
\pgfpathlineto{\pgfqpoint{1.680261in}{0.705265in}}%
\pgfpathlineto{\pgfqpoint{1.676035in}{0.707057in}}%
\pgfpathlineto{\pgfqpoint{1.642779in}{0.721457in}}%
\pgfpathlineto{\pgfqpoint{1.635338in}{0.724743in}}%
\pgfpathlineto{\pgfqpoint{1.609523in}{0.736297in}}%
\pgfpathlineto{\pgfqpoint{1.592140in}{0.744221in}}%
\pgfpathlineto{\pgfqpoint{1.576266in}{0.751556in}}%
\pgfpathlineto{\pgfqpoint{1.550461in}{0.763698in}}%
\pgfpathlineto{\pgfqpoint{1.543010in}{0.767253in}}%
\pgfpathlineto{\pgfqpoint{1.510222in}{0.783176in}}%
\pgfpathlineto{\pgfqpoint{1.509754in}{0.783406in}}%
\pgfpathlineto{\pgfqpoint{1.476498in}{0.800092in}}%
\pgfpathlineto{\pgfqpoint{1.471478in}{0.802653in}}%
\pgfpathlineto{\pgfqpoint{1.443242in}{0.817270in}}%
\pgfpathlineto{\pgfqpoint{1.434005in}{0.822131in}}%
\pgfpathlineto{\pgfqpoint{1.409986in}{0.834957in}}%
\pgfpathlineto{\pgfqpoint{1.397730in}{0.841609in}}%
\pgfpathlineto{\pgfqpoint{1.376729in}{0.853176in}}%
\pgfpathlineto{\pgfqpoint{1.362596in}{0.861086in}}%
\pgfpathlineto{\pgfqpoint{1.343473in}{0.871950in}}%
\pgfpathlineto{\pgfqpoint{1.328548in}{0.880564in}}%
\pgfpathlineto{\pgfqpoint{1.310217in}{0.891305in}}%
\pgfpathlineto{\pgfqpoint{1.295537in}{0.900042in}}%
\pgfpathlineto{\pgfqpoint{1.276961in}{0.911268in}}%
\pgfpathlineto{\pgfqpoint{1.263515in}{0.919519in}}%
\pgfpathlineto{\pgfqpoint{1.243705in}{0.931867in}}%
\pgfpathlineto{\pgfqpoint{1.232437in}{0.938997in}}%
\pgfpathlineto{\pgfqpoint{1.210448in}{0.953133in}}%
\pgfpathlineto{\pgfqpoint{1.202262in}{0.958475in}}%
\pgfpathlineto{\pgfqpoint{1.177192in}{0.975098in}}%
\pgfpathlineto{\pgfqpoint{1.172950in}{0.977952in}}%
\pgfpathlineto{\pgfqpoint{1.144476in}{0.997430in}}%
\pgfpathlineto{\pgfqpoint{1.143936in}{0.997806in}}%
\pgfpathlineto{\pgfqpoint{1.116903in}{1.016908in}}%
\pgfpathlineto{\pgfqpoint{1.110680in}{1.021380in}}%
\pgfpathlineto{\pgfqpoint{1.090100in}{1.036385in}}%
\pgfpathlineto{\pgfqpoint{1.077424in}{1.045789in}}%
\pgfpathlineto{\pgfqpoint{1.064036in}{1.055863in}}%
\pgfpathlineto{\pgfqpoint{1.044168in}{1.071078in}}%
\pgfpathlineto{\pgfqpoint{1.038679in}{1.075340in}}%
\pgfpathlineto{\pgfqpoint{1.014065in}{1.094818in}}%
\pgfpathlineto{\pgfqpoint{1.010911in}{1.097365in}}%
\pgfpathlineto{\pgfqpoint{0.990229in}{1.114296in}}%
\pgfpathlineto{\pgfqpoint{0.977655in}{1.124784in}}%
\pgfpathlineto{\pgfqpoint{0.967025in}{1.133773in}}%
\pgfpathlineto{\pgfqpoint{0.944430in}{1.153251in}}%
\pgfpathlineto{\pgfqpoint{0.944399in}{1.153278in}}%
\pgfpathlineto{\pgfqpoint{0.922645in}{1.172729in}}%
\pgfpathlineto{\pgfqpoint{0.911143in}{1.183223in}}%
\pgfpathlineto{\pgfqpoint{0.901427in}{1.192206in}}%
\pgfpathlineto{\pgfqpoint{0.880809in}{1.211684in}}%
\pgfpathlineto{\pgfqpoint{0.877887in}{1.214513in}}%
\pgfpathlineto{\pgfqpoint{0.860917in}{1.231162in}}%
\pgfpathlineto{\pgfqpoint{0.844630in}{1.247490in}}%
\pgfpathlineto{\pgfqpoint{0.841530in}{1.250639in}}%
\pgfpathlineto{\pgfqpoint{0.822851in}{1.270117in}}%
\pgfpathlineto{\pgfqpoint{0.811374in}{1.282376in}}%
\pgfpathlineto{\pgfqpoint{0.804704in}{1.289595in}}%
\pgfpathlineto{\pgfqpoint{0.787183in}{1.309072in}}%
\pgfpathlineto{\pgfqpoint{0.778118in}{1.319424in}}%
\pgfpathlineto{\pgfqpoint{0.770229in}{1.328550in}}%
\pgfpathlineto{\pgfqpoint{0.753866in}{1.348027in}}%
\pgfpathlineto{\pgfqpoint{0.744862in}{1.359062in}}%
\pgfpathlineto{\pgfqpoint{0.738061in}{1.367505in}}%
\pgfpathlineto{\pgfqpoint{0.722856in}{1.386983in}}%
\pgfpathlineto{\pgfqpoint{0.711606in}{1.401844in}}%
\pgfusepath{stroke}%
\end{pgfscope}%
\begin{pgfscope}%
\pgfpathrectangle{\pgfqpoint{0.711606in}{0.549444in}}{\pgfqpoint{4.955171in}{2.902168in}}%
\pgfusepath{clip}%
\pgfsetbuttcap%
\pgfsetroundjoin%
\pgfsetlinewidth{1.003750pt}%
\definecolor{currentstroke}{rgb}{0.224763,0.036405,0.388129}%
\pgfsetstrokecolor{currentstroke}%
\pgfsetdash{}{0pt}%
\pgfpathmoveto{\pgfqpoint{3.867361in}{0.549444in}}%
\pgfpathlineto{\pgfqpoint{3.837687in}{0.555653in}}%
\pgfpathlineto{\pgfqpoint{3.804431in}{0.560676in}}%
\pgfpathlineto{\pgfqpoint{3.771175in}{0.561417in}}%
\pgfpathlineto{\pgfqpoint{3.737918in}{0.555858in}}%
\pgfpathlineto{\pgfqpoint{3.716914in}{0.549444in}}%
\pgfusepath{stroke}%
\end{pgfscope}%
\begin{pgfscope}%
\pgfpathrectangle{\pgfqpoint{0.711606in}{0.549444in}}{\pgfqpoint{4.955171in}{2.902168in}}%
\pgfusepath{clip}%
\pgfsetbuttcap%
\pgfsetroundjoin%
\pgfsetlinewidth{1.003750pt}%
\definecolor{currentstroke}{rgb}{0.224763,0.036405,0.388129}%
\pgfsetstrokecolor{currentstroke}%
\pgfsetdash{}{0pt}%
\pgfpathmoveto{\pgfqpoint{5.666777in}{0.790623in}}%
\pgfpathlineto{\pgfqpoint{5.659934in}{0.783176in}}%
\pgfpathlineto{\pgfqpoint{5.640904in}{0.763698in}}%
\pgfpathlineto{\pgfqpoint{5.633521in}{0.756572in}}%
\pgfpathlineto{\pgfqpoint{5.620234in}{0.744221in}}%
\pgfpathlineto{\pgfqpoint{5.600265in}{0.726710in}}%
\pgfpathlineto{\pgfqpoint{5.597931in}{0.724743in}}%
\pgfpathlineto{\pgfqpoint{5.573588in}{0.705265in}}%
\pgfpathlineto{\pgfqpoint{5.567008in}{0.700267in}}%
\pgfpathlineto{\pgfqpoint{5.547144in}{0.685788in}}%
\pgfpathlineto{\pgfqpoint{5.533752in}{0.676505in}}%
\pgfpathlineto{\pgfqpoint{5.518398in}{0.666310in}}%
\pgfpathlineto{\pgfqpoint{5.500496in}{0.654979in}}%
\pgfpathlineto{\pgfqpoint{5.487033in}{0.646832in}}%
\pgfpathlineto{\pgfqpoint{5.467240in}{0.635390in}}%
\pgfpathlineto{\pgfqpoint{5.452671in}{0.627355in}}%
\pgfpathlineto{\pgfqpoint{5.433984in}{0.617489in}}%
\pgfpathlineto{\pgfqpoint{5.414856in}{0.607877in}}%
\pgfpathlineto{\pgfqpoint{5.400728in}{0.601068in}}%
\pgfpathlineto{\pgfqpoint{5.373037in}{0.588399in}}%
\pgfpathlineto{\pgfqpoint{5.367471in}{0.585953in}}%
\pgfpathlineto{\pgfqpoint{5.334215in}{0.572085in}}%
\pgfpathlineto{\pgfqpoint{5.326198in}{0.568922in}}%
\pgfpathlineto{\pgfqpoint{5.300959in}{0.559337in}}%
\pgfpathlineto{\pgfqpoint{5.273279in}{0.549444in}}%
\pgfusepath{stroke}%
\end{pgfscope}%
\begin{pgfscope}%
\pgfpathrectangle{\pgfqpoint{0.711606in}{0.549444in}}{\pgfqpoint{4.955171in}{2.902168in}}%
\pgfusepath{clip}%
\pgfsetbuttcap%
\pgfsetroundjoin%
\pgfsetlinewidth{1.003750pt}%
\definecolor{currentstroke}{rgb}{0.224763,0.036405,0.388129}%
\pgfsetstrokecolor{currentstroke}%
\pgfsetdash{}{0pt}%
\pgfpathmoveto{\pgfqpoint{0.711606in}{2.298995in}}%
\pgfpathlineto{\pgfqpoint{0.715250in}{2.302432in}}%
\pgfpathlineto{\pgfqpoint{0.736736in}{2.321909in}}%
\pgfpathlineto{\pgfqpoint{0.744862in}{2.329227in}}%
\pgfpathlineto{\pgfqpoint{0.758579in}{2.341387in}}%
\pgfpathlineto{\pgfqpoint{0.777435in}{2.360865in}}%
\pgfpathlineto{\pgfqpoint{0.778118in}{2.362079in}}%
\pgfpathlineto{\pgfqpoint{0.787401in}{2.380342in}}%
\pgfpathlineto{\pgfqpoint{0.786559in}{2.399820in}}%
\pgfpathlineto{\pgfqpoint{0.778829in}{2.419298in}}%
\pgfpathlineto{\pgfqpoint{0.778118in}{2.420569in}}%
\pgfpathlineto{\pgfqpoint{0.768250in}{2.438775in}}%
\pgfpathlineto{\pgfqpoint{0.756828in}{2.458253in}}%
\pgfpathlineto{\pgfqpoint{0.745371in}{2.477731in}}%
\pgfpathlineto{\pgfqpoint{0.744862in}{2.478624in}}%
\pgfpathlineto{\pgfqpoint{0.734433in}{2.497208in}}%
\pgfpathlineto{\pgfqpoint{0.723949in}{2.516686in}}%
\pgfpathlineto{\pgfqpoint{0.713965in}{2.536163in}}%
\pgfpathlineto{\pgfqpoint{0.711606in}{2.541048in}}%
\pgfusepath{stroke}%
\end{pgfscope}%
\begin{pgfscope}%
\pgfpathrectangle{\pgfqpoint{0.711606in}{0.549444in}}{\pgfqpoint{4.955171in}{2.902168in}}%
\pgfusepath{clip}%
\pgfsetbuttcap%
\pgfsetroundjoin%
\pgfsetlinewidth{1.003750pt}%
\definecolor{currentstroke}{rgb}{0.224763,0.036405,0.388129}%
\pgfsetstrokecolor{currentstroke}%
\pgfsetdash{}{0pt}%
\pgfpathmoveto{\pgfqpoint{0.711606in}{3.094397in}}%
\pgfpathlineto{\pgfqpoint{0.715642in}{3.101015in}}%
\pgfpathlineto{\pgfqpoint{0.728516in}{3.120493in}}%
\pgfpathlineto{\pgfqpoint{0.742482in}{3.139970in}}%
\pgfpathlineto{\pgfqpoint{0.744862in}{3.143067in}}%
\pgfpathlineto{\pgfqpoint{0.757915in}{3.159448in}}%
\pgfpathlineto{\pgfqpoint{0.774605in}{3.178926in}}%
\pgfpathlineto{\pgfqpoint{0.778118in}{3.182775in}}%
\pgfpathlineto{\pgfqpoint{0.792932in}{3.198403in}}%
\pgfpathlineto{\pgfqpoint{0.811374in}{3.216619in}}%
\pgfpathlineto{\pgfqpoint{0.812703in}{3.217881in}}%
\pgfpathlineto{\pgfqpoint{0.834417in}{3.237359in}}%
\pgfpathlineto{\pgfqpoint{0.844630in}{3.245999in}}%
\pgfpathlineto{\pgfqpoint{0.857979in}{3.256836in}}%
\pgfpathlineto{\pgfqpoint{0.877887in}{3.272121in}}%
\pgfpathlineto{\pgfqpoint{0.883587in}{3.276314in}}%
\pgfpathlineto{\pgfqpoint{0.911143in}{3.295536in}}%
\pgfpathlineto{\pgfqpoint{0.911525in}{3.295791in}}%
\pgfpathlineto{\pgfqpoint{0.942208in}{3.315269in}}%
\pgfpathlineto{\pgfqpoint{0.944399in}{3.316594in}}%
\pgfpathlineto{\pgfqpoint{0.975872in}{3.334747in}}%
\pgfpathlineto{\pgfqpoint{0.977655in}{3.335729in}}%
\pgfpathlineto{\pgfqpoint{1.010911in}{3.353164in}}%
\pgfpathlineto{\pgfqpoint{1.013040in}{3.354224in}}%
\pgfpathlineto{\pgfqpoint{1.044168in}{3.369065in}}%
\pgfpathlineto{\pgfqpoint{1.054440in}{3.373702in}}%
\pgfpathlineto{\pgfqpoint{1.077424in}{3.383651in}}%
\pgfpathlineto{\pgfqpoint{1.100748in}{3.393180in}}%
\pgfpathlineto{\pgfqpoint{1.110680in}{3.397077in}}%
\pgfpathlineto{\pgfqpoint{1.143936in}{3.409383in}}%
\pgfpathlineto{\pgfqpoint{1.153344in}{3.412657in}}%
\pgfpathlineto{\pgfqpoint{1.177192in}{3.420646in}}%
\pgfpathlineto{\pgfqpoint{1.210448in}{3.431054in}}%
\pgfpathlineto{\pgfqpoint{1.214137in}{3.432135in}}%
\pgfpathlineto{\pgfqpoint{1.243705in}{3.440493in}}%
\pgfpathlineto{\pgfqpoint{1.276961in}{3.449206in}}%
\pgfpathlineto{\pgfqpoint{1.286855in}{3.451613in}}%
\pgfusepath{stroke}%
\end{pgfscope}%
\begin{pgfscope}%
\pgfpathrectangle{\pgfqpoint{0.711606in}{0.549444in}}{\pgfqpoint{4.955171in}{2.902168in}}%
\pgfusepath{clip}%
\pgfsetbuttcap%
\pgfsetroundjoin%
\pgfsetlinewidth{1.003750pt}%
\definecolor{currentstroke}{rgb}{0.224763,0.036405,0.388129}%
\pgfsetstrokecolor{currentstroke}%
\pgfsetdash{}{0pt}%
\pgfpathmoveto{\pgfqpoint{2.324960in}{3.451613in}}%
\pgfpathlineto{\pgfqpoint{2.407671in}{3.434858in}}%
\pgfpathlineto{\pgfqpoint{2.503486in}{3.412657in}}%
\pgfpathlineto{\pgfqpoint{2.578516in}{3.393180in}}%
\pgfpathlineto{\pgfqpoint{2.673720in}{3.365762in}}%
\pgfpathlineto{\pgfqpoint{2.770644in}{3.334747in}}%
\pgfpathlineto{\pgfqpoint{2.840001in}{3.310487in}}%
\pgfpathlineto{\pgfqpoint{2.930248in}{3.276314in}}%
\pgfpathlineto{\pgfqpoint{3.006282in}{3.245104in}}%
\pgfpathlineto{\pgfqpoint{3.072795in}{3.215930in}}%
\pgfpathlineto{\pgfqpoint{3.151429in}{3.178926in}}%
\pgfpathlineto{\pgfqpoint{3.228440in}{3.139970in}}%
\pgfpathlineto{\pgfqpoint{3.300226in}{3.101015in}}%
\pgfpathlineto{\pgfqpoint{3.367308in}{3.062060in}}%
\pgfpathlineto{\pgfqpoint{3.430140in}{3.023104in}}%
\pgfpathlineto{\pgfqpoint{3.489120in}{2.984149in}}%
\pgfpathlineto{\pgfqpoint{3.544597in}{2.945194in}}%
\pgfpathlineto{\pgfqpoint{3.604894in}{2.899879in}}%
\pgfpathlineto{\pgfqpoint{3.645690in}{2.867283in}}%
\pgfpathlineto{\pgfqpoint{3.704662in}{2.816893in}}%
\pgfpathlineto{\pgfqpoint{3.737918in}{2.786592in}}%
\pgfpathlineto{\pgfqpoint{3.775451in}{2.750418in}}%
\pgfpathlineto{\pgfqpoint{3.813368in}{2.711462in}}%
\pgfpathlineto{\pgfqpoint{3.848923in}{2.672507in}}%
\pgfpathlineto{\pgfqpoint{3.904199in}{2.607523in}}%
\pgfpathlineto{\pgfqpoint{3.937455in}{2.569850in}}%
\pgfpathlineto{\pgfqpoint{3.952903in}{2.555641in}}%
\pgfpathlineto{\pgfqpoint{3.970712in}{2.542694in}}%
\pgfpathlineto{\pgfqpoint{3.982276in}{2.536163in}}%
\pgfpathlineto{\pgfqpoint{4.003968in}{2.526564in}}%
\pgfpathlineto{\pgfqpoint{4.037224in}{2.515543in}}%
\pgfpathlineto{\pgfqpoint{4.108001in}{2.497208in}}%
\pgfpathlineto{\pgfqpoint{4.203505in}{2.472985in}}%
\pgfpathlineto{\pgfqpoint{4.303273in}{2.445246in}}%
\pgfpathlineto{\pgfqpoint{4.387922in}{2.419298in}}%
\pgfpathlineto{\pgfqpoint{4.469554in}{2.392056in}}%
\pgfpathlineto{\pgfqpoint{4.555614in}{2.360865in}}%
\pgfpathlineto{\pgfqpoint{4.635835in}{2.329412in}}%
\pgfpathlineto{\pgfqpoint{4.702348in}{2.301547in}}%
\pgfpathlineto{\pgfqpoint{4.786968in}{2.263477in}}%
\pgfpathlineto{\pgfqpoint{4.868629in}{2.223968in}}%
\pgfpathlineto{\pgfqpoint{4.942458in}{2.185566in}}%
\pgfpathlineto{\pgfqpoint{5.012598in}{2.146611in}}%
\pgfpathlineto{\pgfqpoint{5.078421in}{2.107655in}}%
\pgfpathlineto{\pgfqpoint{5.140319in}{2.068700in}}%
\pgfpathlineto{\pgfqpoint{5.201190in}{2.027941in}}%
\pgfpathlineto{\pgfqpoint{5.267703in}{1.980226in}}%
\pgfpathlineto{\pgfqpoint{5.329844in}{1.932357in}}%
\pgfpathlineto{\pgfqpoint{5.377089in}{1.893401in}}%
\pgfpathlineto{\pgfqpoint{5.433984in}{1.843153in}}%
\pgfpathlineto{\pgfqpoint{5.467240in}{1.811849in}}%
\pgfpathlineto{\pgfqpoint{5.502770in}{1.776536in}}%
\pgfpathlineto{\pgfqpoint{5.539537in}{1.737580in}}%
\pgfpathlineto{\pgfqpoint{5.573902in}{1.698625in}}%
\pgfpathlineto{\pgfqpoint{5.605912in}{1.659670in}}%
\pgfpathlineto{\pgfqpoint{5.635611in}{1.620714in}}%
\pgfpathlineto{\pgfqpoint{5.666777in}{1.576018in}}%
\pgfpathlineto{\pgfqpoint{5.666777in}{1.576018in}}%
\pgfusepath{stroke}%
\end{pgfscope}%
\begin{pgfscope}%
\pgfpathrectangle{\pgfqpoint{0.711606in}{0.549444in}}{\pgfqpoint{4.955171in}{2.902168in}}%
\pgfusepath{clip}%
\pgfsetbuttcap%
\pgfsetroundjoin%
\pgfsetlinewidth{1.003750pt}%
\definecolor{currentstroke}{rgb}{0.251620,0.037705,0.403378}%
\pgfsetstrokecolor{currentstroke}%
\pgfsetdash{}{0pt}%
\pgfpathmoveto{\pgfqpoint{1.992965in}{0.549444in}}%
\pgfpathlineto{\pgfqpoint{1.975341in}{0.554981in}}%
\pgfpathlineto{\pgfqpoint{1.942084in}{0.565698in}}%
\pgfpathlineto{\pgfqpoint{1.932337in}{0.568922in}}%
\pgfpathlineto{\pgfqpoint{1.908828in}{0.576793in}}%
\pgfpathlineto{\pgfqpoint{1.875572in}{0.588179in}}%
\pgfpathlineto{\pgfqpoint{1.874945in}{0.588399in}}%
\pgfpathlineto{\pgfqpoint{1.842316in}{0.600007in}}%
\pgfpathlineto{\pgfqpoint{1.820650in}{0.607877in}}%
\pgfpathlineto{\pgfqpoint{1.809060in}{0.612141in}}%
\pgfpathlineto{\pgfqpoint{1.775804in}{0.624646in}}%
\pgfpathlineto{\pgfqpoint{1.768758in}{0.627355in}}%
\pgfpathlineto{\pgfqpoint{1.742547in}{0.637560in}}%
\pgfpathlineto{\pgfqpoint{1.719193in}{0.646832in}}%
\pgfpathlineto{\pgfqpoint{1.709291in}{0.650814in}}%
\pgfpathlineto{\pgfqpoint{1.676035in}{0.664459in}}%
\pgfpathlineto{\pgfqpoint{1.671615in}{0.666310in}}%
\pgfpathlineto{\pgfqpoint{1.642779in}{0.678544in}}%
\pgfpathlineto{\pgfqpoint{1.626013in}{0.685788in}}%
\pgfpathlineto{\pgfqpoint{1.609523in}{0.693006in}}%
\pgfpathlineto{\pgfqpoint{1.582016in}{0.705265in}}%
\pgfpathlineto{\pgfqpoint{1.576266in}{0.707862in}}%
\pgfpathlineto{\pgfqpoint{1.543010in}{0.723158in}}%
\pgfpathlineto{\pgfqpoint{1.539627in}{0.724743in}}%
\pgfpathlineto{\pgfqpoint{1.509754in}{0.738925in}}%
\pgfpathlineto{\pgfqpoint{1.498787in}{0.744221in}}%
\pgfpathlineto{\pgfqpoint{1.476498in}{0.755128in}}%
\pgfpathlineto{\pgfqpoint{1.459274in}{0.763698in}}%
\pgfpathlineto{\pgfqpoint{1.443242in}{0.771785in}}%
\pgfpathlineto{\pgfqpoint{1.421025in}{0.783176in}}%
\pgfpathlineto{\pgfqpoint{1.409986in}{0.788915in}}%
\pgfpathlineto{\pgfqpoint{1.383980in}{0.802653in}}%
\pgfpathlineto{\pgfqpoint{1.376729in}{0.806538in}}%
\pgfpathlineto{\pgfqpoint{1.348080in}{0.822131in}}%
\pgfpathlineto{\pgfqpoint{1.343473in}{0.824675in}}%
\pgfpathlineto{\pgfqpoint{1.313274in}{0.841609in}}%
\pgfpathlineto{\pgfqpoint{1.310217in}{0.843348in}}%
\pgfpathlineto{\pgfqpoint{1.279512in}{0.861086in}}%
\pgfpathlineto{\pgfqpoint{1.276961in}{0.862582in}}%
\pgfpathlineto{\pgfqpoint{1.246746in}{0.880564in}}%
\pgfpathlineto{\pgfqpoint{1.243705in}{0.882401in}}%
\pgfpathlineto{\pgfqpoint{1.214932in}{0.900042in}}%
\pgfpathlineto{\pgfqpoint{1.210448in}{0.902832in}}%
\pgfpathlineto{\pgfqpoint{1.184029in}{0.919519in}}%
\pgfpathlineto{\pgfqpoint{1.177192in}{0.923904in}}%
\pgfpathlineto{\pgfqpoint{1.153997in}{0.938997in}}%
\pgfpathlineto{\pgfqpoint{1.143936in}{0.945646in}}%
\pgfpathlineto{\pgfqpoint{1.124800in}{0.958475in}}%
\pgfpathlineto{\pgfqpoint{1.110680in}{0.968091in}}%
\pgfpathlineto{\pgfqpoint{1.096402in}{0.977952in}}%
\pgfpathlineto{\pgfqpoint{1.077424in}{0.991273in}}%
\pgfpathlineto{\pgfqpoint{1.068772in}{0.997430in}}%
\pgfpathlineto{\pgfqpoint{1.044168in}{1.015227in}}%
\pgfpathlineto{\pgfqpoint{1.041876in}{1.016908in}}%
\pgfpathlineto{\pgfqpoint{1.015785in}{1.036385in}}%
\pgfpathlineto{\pgfqpoint{1.010911in}{1.040088in}}%
\pgfpathlineto{\pgfqpoint{0.990430in}{1.055863in}}%
\pgfpathlineto{\pgfqpoint{0.977655in}{1.065873in}}%
\pgfpathlineto{\pgfqpoint{0.965735in}{1.075340in}}%
\pgfpathlineto{\pgfqpoint{0.944399in}{1.092586in}}%
\pgfpathlineto{\pgfqpoint{0.941674in}{1.094818in}}%
\pgfpathlineto{\pgfqpoint{0.918364in}{1.114296in}}%
\pgfpathlineto{\pgfqpoint{0.911143in}{1.120447in}}%
\pgfpathlineto{\pgfqpoint{0.895702in}{1.133773in}}%
\pgfpathlineto{\pgfqpoint{0.877887in}{1.149440in}}%
\pgfpathlineto{\pgfqpoint{0.873609in}{1.153251in}}%
\pgfpathlineto{\pgfqpoint{0.852210in}{1.172729in}}%
\pgfpathlineto{\pgfqpoint{0.844630in}{1.179775in}}%
\pgfpathlineto{\pgfqpoint{0.831428in}{1.192206in}}%
\pgfpathlineto{\pgfqpoint{0.811374in}{1.211472in}}%
\pgfpathlineto{\pgfqpoint{0.811157in}{1.211684in}}%
\pgfpathlineto{\pgfqpoint{0.791633in}{1.231162in}}%
\pgfpathlineto{\pgfqpoint{0.778118in}{1.244934in}}%
\pgfpathlineto{\pgfqpoint{0.772590in}{1.250639in}}%
\pgfpathlineto{\pgfqpoint{0.754180in}{1.270117in}}%
\pgfpathlineto{\pgfqpoint{0.744862in}{1.280219in}}%
\pgfpathlineto{\pgfqpoint{0.736321in}{1.289595in}}%
\pgfpathlineto{\pgfqpoint{0.719026in}{1.309072in}}%
\pgfpathlineto{\pgfqpoint{0.711606in}{1.317657in}}%
\pgfusepath{stroke}%
\end{pgfscope}%
\begin{pgfscope}%
\pgfpathrectangle{\pgfqpoint{0.711606in}{0.549444in}}{\pgfqpoint{4.955171in}{2.902168in}}%
\pgfusepath{clip}%
\pgfsetbuttcap%
\pgfsetroundjoin%
\pgfsetlinewidth{1.003750pt}%
\definecolor{currentstroke}{rgb}{0.251620,0.037705,0.403378}%
\pgfsetstrokecolor{currentstroke}%
\pgfsetdash{}{0pt}%
\pgfpathmoveto{\pgfqpoint{5.666777in}{0.696661in}}%
\pgfpathlineto{\pgfqpoint{5.653688in}{0.685788in}}%
\pgfpathlineto{\pgfqpoint{5.633521in}{0.669888in}}%
\pgfpathlineto{\pgfqpoint{5.628800in}{0.666310in}}%
\pgfpathlineto{\pgfqpoint{5.601811in}{0.646832in}}%
\pgfpathlineto{\pgfqpoint{5.600265in}{0.645766in}}%
\pgfpathlineto{\pgfqpoint{5.572423in}{0.627355in}}%
\pgfpathlineto{\pgfqpoint{5.567008in}{0.623934in}}%
\pgfpathlineto{\pgfqpoint{5.540473in}{0.607877in}}%
\pgfpathlineto{\pgfqpoint{5.533752in}{0.603984in}}%
\pgfpathlineto{\pgfqpoint{5.505608in}{0.588399in}}%
\pgfpathlineto{\pgfqpoint{5.500496in}{0.585685in}}%
\pgfpathlineto{\pgfqpoint{5.467408in}{0.568922in}}%
\pgfpathlineto{\pgfqpoint{5.467240in}{0.568840in}}%
\pgfpathlineto{\pgfqpoint{5.433984in}{0.553397in}}%
\pgfpathlineto{\pgfqpoint{5.425025in}{0.549444in}}%
\pgfusepath{stroke}%
\end{pgfscope}%
\begin{pgfscope}%
\pgfpathrectangle{\pgfqpoint{0.711606in}{0.549444in}}{\pgfqpoint{4.955171in}{2.902168in}}%
\pgfusepath{clip}%
\pgfsetbuttcap%
\pgfsetroundjoin%
\pgfsetlinewidth{1.003750pt}%
\definecolor{currentstroke}{rgb}{0.251620,0.037705,0.403378}%
\pgfsetstrokecolor{currentstroke}%
\pgfsetdash{}{0pt}%
\pgfpathmoveto{\pgfqpoint{0.711606in}{3.216782in}}%
\pgfpathlineto{\pgfqpoint{0.712615in}{3.217881in}}%
\pgfpathlineto{\pgfqpoint{0.731591in}{3.237359in}}%
\pgfpathlineto{\pgfqpoint{0.744862in}{3.250169in}}%
\pgfpathlineto{\pgfqpoint{0.752034in}{3.256836in}}%
\pgfpathlineto{\pgfqpoint{0.774209in}{3.276314in}}%
\pgfpathlineto{\pgfqpoint{0.778118in}{3.279571in}}%
\pgfpathlineto{\pgfqpoint{0.798371in}{3.295791in}}%
\pgfpathlineto{\pgfqpoint{0.811374in}{3.305669in}}%
\pgfpathlineto{\pgfqpoint{0.824545in}{3.315269in}}%
\pgfpathlineto{\pgfqpoint{0.844630in}{3.329192in}}%
\pgfpathlineto{\pgfqpoint{0.852996in}{3.334747in}}%
\pgfpathlineto{\pgfqpoint{0.877887in}{3.350501in}}%
\pgfpathlineto{\pgfqpoint{0.884040in}{3.354224in}}%
\pgfpathlineto{\pgfqpoint{0.911143in}{3.369893in}}%
\pgfpathlineto{\pgfqpoint{0.918049in}{3.373702in}}%
\pgfpathlineto{\pgfqpoint{0.944399in}{3.387613in}}%
\pgfpathlineto{\pgfqpoint{0.955476in}{3.393180in}}%
\pgfpathlineto{\pgfqpoint{0.977655in}{3.403868in}}%
\pgfpathlineto{\pgfqpoint{0.996866in}{3.412657in}}%
\pgfpathlineto{\pgfqpoint{1.010911in}{3.418830in}}%
\pgfpathlineto{\pgfqpoint{1.042890in}{3.432135in}}%
\pgfpathlineto{\pgfqpoint{1.044168in}{3.432646in}}%
\pgfpathlineto{\pgfqpoint{1.077424in}{3.445270in}}%
\pgfpathlineto{\pgfqpoint{1.095169in}{3.451613in}}%
\pgfusepath{stroke}%
\end{pgfscope}%
\begin{pgfscope}%
\pgfpathrectangle{\pgfqpoint{0.711606in}{0.549444in}}{\pgfqpoint{4.955171in}{2.902168in}}%
\pgfusepath{clip}%
\pgfsetbuttcap%
\pgfsetroundjoin%
\pgfsetlinewidth{1.003750pt}%
\definecolor{currentstroke}{rgb}{0.251620,0.037705,0.403378}%
\pgfsetstrokecolor{currentstroke}%
\pgfsetdash{}{0pt}%
\pgfpathmoveto{\pgfqpoint{2.516726in}{3.451613in}}%
\pgfpathlineto{\pgfqpoint{2.607208in}{3.427758in}}%
\pgfpathlineto{\pgfqpoint{2.706977in}{3.398424in}}%
\pgfpathlineto{\pgfqpoint{2.783564in}{3.373702in}}%
\pgfpathlineto{\pgfqpoint{2.873258in}{3.342237in}}%
\pgfpathlineto{\pgfqpoint{2.944495in}{3.315269in}}%
\pgfpathlineto{\pgfqpoint{3.039708in}{3.276314in}}%
\pgfpathlineto{\pgfqpoint{3.127173in}{3.237359in}}%
\pgfpathlineto{\pgfqpoint{3.208415in}{3.198403in}}%
\pgfpathlineto{\pgfqpoint{3.283974in}{3.159448in}}%
\pgfpathlineto{\pgfqpoint{3.354701in}{3.120493in}}%
\pgfpathlineto{\pgfqpoint{3.421051in}{3.081537in}}%
\pgfpathlineto{\pgfqpoint{3.483425in}{3.042582in}}%
\pgfpathlineto{\pgfqpoint{3.542174in}{3.003627in}}%
\pgfpathlineto{\pgfqpoint{3.604894in}{2.959241in}}%
\pgfpathlineto{\pgfqpoint{3.671406in}{2.908706in}}%
\pgfpathlineto{\pgfqpoint{3.722156in}{2.867283in}}%
\pgfpathlineto{\pgfqpoint{3.771175in}{2.824672in}}%
\pgfpathlineto{\pgfqpoint{3.809366in}{2.789373in}}%
\pgfpathlineto{\pgfqpoint{3.849081in}{2.750418in}}%
\pgfpathlineto{\pgfqpoint{3.886493in}{2.711462in}}%
\pgfpathlineto{\pgfqpoint{3.939176in}{2.653029in}}%
\pgfpathlineto{\pgfqpoint{3.974071in}{2.614074in}}%
\pgfpathlineto{\pgfqpoint{4.003968in}{2.585971in}}%
\pgfpathlineto{\pgfqpoint{4.019187in}{2.575119in}}%
\pgfpathlineto{\pgfqpoint{4.037224in}{2.564938in}}%
\pgfpathlineto{\pgfqpoint{4.070480in}{2.551243in}}%
\pgfpathlineto{\pgfqpoint{4.120389in}{2.536163in}}%
\pgfpathlineto{\pgfqpoint{4.332536in}{2.477731in}}%
\pgfpathlineto{\pgfqpoint{4.403042in}{2.455912in}}%
\pgfpathlineto{\pgfqpoint{4.502811in}{2.422362in}}%
\pgfpathlineto{\pgfqpoint{4.569323in}{2.398194in}}%
\pgfpathlineto{\pgfqpoint{4.669091in}{2.359073in}}%
\pgfpathlineto{\pgfqpoint{4.756277in}{2.321909in}}%
\pgfpathlineto{\pgfqpoint{4.835372in}{2.285717in}}%
\pgfpathlineto{\pgfqpoint{4.901885in}{2.253255in}}%
\pgfpathlineto{\pgfqpoint{4.968397in}{2.218894in}}%
\pgfpathlineto{\pgfqpoint{5.034909in}{2.182467in}}%
\pgfpathlineto{\pgfqpoint{5.101422in}{2.143786in}}%
\pgfpathlineto{\pgfqpoint{5.167934in}{2.102638in}}%
\pgfpathlineto{\pgfqpoint{5.234447in}{2.058784in}}%
\pgfpathlineto{\pgfqpoint{5.303270in}{2.010267in}}%
\pgfpathlineto{\pgfqpoint{5.367471in}{1.961576in}}%
\pgfpathlineto{\pgfqpoint{5.427333in}{1.912879in}}%
\pgfpathlineto{\pgfqpoint{5.472185in}{1.873924in}}%
\pgfpathlineto{\pgfqpoint{5.514417in}{1.834968in}}%
\pgfpathlineto{\pgfqpoint{5.567008in}{1.782959in}}%
\pgfpathlineto{\pgfqpoint{5.609418in}{1.737580in}}%
\pgfpathlineto{\pgfqpoint{5.643307in}{1.698625in}}%
\pgfpathlineto{\pgfqpoint{5.666777in}{1.669966in}}%
\pgfpathlineto{\pgfqpoint{5.666777in}{1.669966in}}%
\pgfusepath{stroke}%
\end{pgfscope}%
\begin{pgfscope}%
\pgfpathrectangle{\pgfqpoint{0.711606in}{0.549444in}}{\pgfqpoint{4.955171in}{2.902168in}}%
\pgfusepath{clip}%
\pgfsetbuttcap%
\pgfsetroundjoin%
\pgfsetlinewidth{1.003750pt}%
\definecolor{currentstroke}{rgb}{0.277850,0.042353,0.414392}%
\pgfsetstrokecolor{currentstroke}%
\pgfsetdash{}{0pt}%
\pgfpathmoveto{\pgfqpoint{1.873475in}{0.549444in}}%
\pgfpathlineto{\pgfqpoint{1.842316in}{0.560353in}}%
\pgfpathlineto{\pgfqpoint{1.818337in}{0.568922in}}%
\pgfpathlineto{\pgfqpoint{1.809060in}{0.572277in}}%
\pgfpathlineto{\pgfqpoint{1.775804in}{0.584575in}}%
\pgfpathlineto{\pgfqpoint{1.765679in}{0.588399in}}%
\pgfpathlineto{\pgfqpoint{1.742547in}{0.597244in}}%
\pgfpathlineto{\pgfqpoint{1.715265in}{0.607877in}}%
\pgfpathlineto{\pgfqpoint{1.709291in}{0.610234in}}%
\pgfpathlineto{\pgfqpoint{1.676035in}{0.623629in}}%
\pgfpathlineto{\pgfqpoint{1.666964in}{0.627355in}}%
\pgfpathlineto{\pgfqpoint{1.642779in}{0.637411in}}%
\pgfpathlineto{\pgfqpoint{1.620523in}{0.646832in}}%
\pgfpathlineto{\pgfqpoint{1.609523in}{0.651548in}}%
\pgfpathlineto{\pgfqpoint{1.576266in}{0.666059in}}%
\pgfpathlineto{\pgfqpoint{1.575702in}{0.666310in}}%
\pgfpathlineto{\pgfqpoint{1.543010in}{0.681038in}}%
\pgfpathlineto{\pgfqpoint{1.532643in}{0.685788in}}%
\pgfpathlineto{\pgfqpoint{1.509754in}{0.696409in}}%
\pgfpathlineto{\pgfqpoint{1.490984in}{0.705265in}}%
\pgfpathlineto{\pgfqpoint{1.476498in}{0.712189in}}%
\pgfpathlineto{\pgfqpoint{1.450658in}{0.724743in}}%
\pgfpathlineto{\pgfqpoint{1.443242in}{0.728393in}}%
\pgfpathlineto{\pgfqpoint{1.411601in}{0.744221in}}%
\pgfpathlineto{\pgfqpoint{1.409986in}{0.745039in}}%
\pgfpathlineto{\pgfqpoint{1.376729in}{0.762176in}}%
\pgfpathlineto{\pgfqpoint{1.373822in}{0.763698in}}%
\pgfpathlineto{\pgfqpoint{1.343473in}{0.779799in}}%
\pgfpathlineto{\pgfqpoint{1.337205in}{0.783176in}}%
\pgfpathlineto{\pgfqpoint{1.310217in}{0.797911in}}%
\pgfpathlineto{\pgfqpoint{1.301661in}{0.802653in}}%
\pgfpathlineto{\pgfqpoint{1.276961in}{0.816533in}}%
\pgfpathlineto{\pgfqpoint{1.267144in}{0.822131in}}%
\pgfpathlineto{\pgfqpoint{1.243705in}{0.835685in}}%
\pgfpathlineto{\pgfqpoint{1.233608in}{0.841609in}}%
\pgfpathlineto{\pgfqpoint{1.210448in}{0.855390in}}%
\pgfpathlineto{\pgfqpoint{1.201013in}{0.861086in}}%
\pgfpathlineto{\pgfqpoint{1.177192in}{0.875673in}}%
\pgfpathlineto{\pgfqpoint{1.169317in}{0.880564in}}%
\pgfpathlineto{\pgfqpoint{1.143936in}{0.896558in}}%
\pgfpathlineto{\pgfqpoint{1.138485in}{0.900042in}}%
\pgfpathlineto{\pgfqpoint{1.110680in}{0.918073in}}%
\pgfpathlineto{\pgfqpoint{1.108480in}{0.919519in}}%
\pgfpathlineto{\pgfqpoint{1.079308in}{0.938997in}}%
\pgfpathlineto{\pgfqpoint{1.077424in}{0.940275in}}%
\pgfpathlineto{\pgfqpoint{1.050958in}{0.958475in}}%
\pgfpathlineto{\pgfqpoint{1.044168in}{0.963217in}}%
\pgfpathlineto{\pgfqpoint{1.023353in}{0.977952in}}%
\pgfpathlineto{\pgfqpoint{1.010911in}{0.986900in}}%
\pgfpathlineto{\pgfqpoint{0.996462in}{0.997430in}}%
\pgfpathlineto{\pgfqpoint{0.977655in}{1.011358in}}%
\pgfpathlineto{\pgfqpoint{0.970259in}{1.016908in}}%
\pgfpathlineto{\pgfqpoint{0.944723in}{1.036385in}}%
\pgfpathlineto{\pgfqpoint{0.944399in}{1.036637in}}%
\pgfpathlineto{\pgfqpoint{0.919979in}{1.055863in}}%
\pgfpathlineto{\pgfqpoint{0.911143in}{1.062939in}}%
\pgfpathlineto{\pgfqpoint{0.895854in}{1.075340in}}%
\pgfpathlineto{\pgfqpoint{0.877887in}{1.090169in}}%
\pgfpathlineto{\pgfqpoint{0.872325in}{1.094818in}}%
\pgfpathlineto{\pgfqpoint{0.849460in}{1.114296in}}%
\pgfpathlineto{\pgfqpoint{0.844630in}{1.118492in}}%
\pgfpathlineto{\pgfqpoint{0.827261in}{1.133773in}}%
\pgfpathlineto{\pgfqpoint{0.811374in}{1.148011in}}%
\pgfpathlineto{\pgfqpoint{0.805600in}{1.153251in}}%
\pgfpathlineto{\pgfqpoint{0.784574in}{1.172729in}}%
\pgfpathlineto{\pgfqpoint{0.778118in}{1.178837in}}%
\pgfpathlineto{\pgfqpoint{0.764161in}{1.192206in}}%
\pgfpathlineto{\pgfqpoint{0.744862in}{1.211061in}}%
\pgfpathlineto{\pgfqpoint{0.744232in}{1.211684in}}%
\pgfpathlineto{\pgfqpoint{0.725012in}{1.231162in}}%
\pgfpathlineto{\pgfqpoint{0.711606in}{1.245035in}}%
\pgfusepath{stroke}%
\end{pgfscope}%
\begin{pgfscope}%
\pgfpathrectangle{\pgfqpoint{0.711606in}{0.549444in}}{\pgfqpoint{4.955171in}{2.902168in}}%
\pgfusepath{clip}%
\pgfsetbuttcap%
\pgfsetroundjoin%
\pgfsetlinewidth{1.003750pt}%
\definecolor{currentstroke}{rgb}{0.277850,0.042353,0.414392}%
\pgfsetstrokecolor{currentstroke}%
\pgfsetdash{}{0pt}%
\pgfpathmoveto{\pgfqpoint{5.666777in}{0.618106in}}%
\pgfpathlineto{\pgfqpoint{5.651920in}{0.607877in}}%
\pgfpathlineto{\pgfqpoint{5.633521in}{0.595770in}}%
\pgfpathlineto{\pgfqpoint{5.621852in}{0.588399in}}%
\pgfpathlineto{\pgfqpoint{5.600265in}{0.575341in}}%
\pgfpathlineto{\pgfqpoint{5.589191in}{0.568922in}}%
\pgfpathlineto{\pgfqpoint{5.567008in}{0.556586in}}%
\pgfpathlineto{\pgfqpoint{5.553583in}{0.549444in}}%
\pgfusepath{stroke}%
\end{pgfscope}%
\begin{pgfscope}%
\pgfpathrectangle{\pgfqpoint{0.711606in}{0.549444in}}{\pgfqpoint{4.955171in}{2.902168in}}%
\pgfusepath{clip}%
\pgfsetbuttcap%
\pgfsetroundjoin%
\pgfsetlinewidth{1.003750pt}%
\definecolor{currentstroke}{rgb}{0.277850,0.042353,0.414392}%
\pgfsetstrokecolor{currentstroke}%
\pgfsetdash{}{0pt}%
\pgfpathmoveto{\pgfqpoint{0.711606in}{3.309477in}}%
\pgfpathlineto{\pgfqpoint{0.718578in}{3.315269in}}%
\pgfpathlineto{\pgfqpoint{0.743269in}{3.334747in}}%
\pgfpathlineto{\pgfqpoint{0.744862in}{3.335945in}}%
\pgfpathlineto{\pgfqpoint{0.770145in}{3.354224in}}%
\pgfpathlineto{\pgfqpoint{0.778118in}{3.359719in}}%
\pgfpathlineto{\pgfqpoint{0.799263in}{3.373702in}}%
\pgfpathlineto{\pgfqpoint{0.811374in}{3.381353in}}%
\pgfpathlineto{\pgfqpoint{0.830918in}{3.393180in}}%
\pgfpathlineto{\pgfqpoint{0.844630in}{3.401122in}}%
\pgfpathlineto{\pgfqpoint{0.865461in}{3.412657in}}%
\pgfpathlineto{\pgfqpoint{0.877887in}{3.419256in}}%
\pgfpathlineto{\pgfqpoint{0.903309in}{3.432135in}}%
\pgfpathlineto{\pgfqpoint{0.911143in}{3.435947in}}%
\pgfpathlineto{\pgfqpoint{0.944399in}{3.451353in}}%
\pgfpathlineto{\pgfqpoint{0.944987in}{3.451613in}}%
\pgfusepath{stroke}%
\end{pgfscope}%
\begin{pgfscope}%
\pgfpathrectangle{\pgfqpoint{0.711606in}{0.549444in}}{\pgfqpoint{4.955171in}{2.902168in}}%
\pgfusepath{clip}%
\pgfsetbuttcap%
\pgfsetroundjoin%
\pgfsetlinewidth{1.003750pt}%
\definecolor{currentstroke}{rgb}{0.277850,0.042353,0.414392}%
\pgfsetstrokecolor{currentstroke}%
\pgfsetdash{}{0pt}%
\pgfpathmoveto{\pgfqpoint{2.666777in}{3.451613in}}%
\pgfpathlineto{\pgfqpoint{2.740233in}{3.429238in}}%
\pgfpathlineto{\pgfqpoint{2.840001in}{3.396096in}}%
\pgfpathlineto{\pgfqpoint{2.906514in}{3.372204in}}%
\pgfpathlineto{\pgfqpoint{3.006282in}{3.333523in}}%
\pgfpathlineto{\pgfqpoint{3.106051in}{3.291242in}}%
\pgfpathlineto{\pgfqpoint{3.181328in}{3.256836in}}%
\pgfpathlineto{\pgfqpoint{3.261029in}{3.217881in}}%
\pgfpathlineto{\pgfqpoint{3.338844in}{3.177190in}}%
\pgfpathlineto{\pgfqpoint{3.405606in}{3.139970in}}%
\pgfpathlineto{\pgfqpoint{3.471869in}{3.100727in}}%
\pgfpathlineto{\pgfqpoint{3.538381in}{3.058800in}}%
\pgfpathlineto{\pgfqpoint{3.604894in}{3.014086in}}%
\pgfpathlineto{\pgfqpoint{3.673598in}{2.964672in}}%
\pgfpathlineto{\pgfqpoint{3.737918in}{2.914839in}}%
\pgfpathlineto{\pgfqpoint{3.795134in}{2.867283in}}%
\pgfpathlineto{\pgfqpoint{3.839132in}{2.828328in}}%
\pgfpathlineto{\pgfqpoint{3.900440in}{2.769895in}}%
\pgfpathlineto{\pgfqpoint{3.938584in}{2.730940in}}%
\pgfpathlineto{\pgfqpoint{4.037224in}{2.627531in}}%
\pgfpathlineto{\pgfqpoint{4.053597in}{2.614074in}}%
\pgfpathlineto{\pgfqpoint{4.070480in}{2.602539in}}%
\pgfpathlineto{\pgfqpoint{4.084746in}{2.594596in}}%
\pgfpathlineto{\pgfqpoint{4.103736in}{2.585889in}}%
\pgfpathlineto{\pgfqpoint{4.136993in}{2.573695in}}%
\pgfpathlineto{\pgfqpoint{4.203505in}{2.554087in}}%
\pgfpathlineto{\pgfqpoint{4.336530in}{2.516198in}}%
\pgfpathlineto{\pgfqpoint{4.436298in}{2.485062in}}%
\pgfpathlineto{\pgfqpoint{4.515524in}{2.458253in}}%
\pgfpathlineto{\pgfqpoint{4.602579in}{2.426507in}}%
\pgfpathlineto{\pgfqpoint{4.671029in}{2.399820in}}%
\pgfpathlineto{\pgfqpoint{4.768860in}{2.358806in}}%
\pgfpathlineto{\pgfqpoint{4.850422in}{2.321909in}}%
\pgfpathlineto{\pgfqpoint{4.935141in}{2.280892in}}%
\pgfpathlineto{\pgfqpoint{5.006455in}{2.243999in}}%
\pgfpathlineto{\pgfqpoint{5.077326in}{2.205044in}}%
\pgfpathlineto{\pgfqpoint{5.144124in}{2.166088in}}%
\pgfpathlineto{\pgfqpoint{5.207195in}{2.127133in}}%
\pgfpathlineto{\pgfqpoint{5.267703in}{2.087589in}}%
\pgfpathlineto{\pgfqpoint{5.334215in}{2.041300in}}%
\pgfpathlineto{\pgfqpoint{5.402136in}{1.990790in}}%
\pgfpathlineto{\pgfqpoint{5.467240in}{1.938633in}}%
\pgfpathlineto{\pgfqpoint{5.519940in}{1.893401in}}%
\pgfpathlineto{\pgfqpoint{5.567008in}{1.850352in}}%
\pgfpathlineto{\pgfqpoint{5.622202in}{1.796013in}}%
\pgfpathlineto{\pgfqpoint{5.666777in}{1.748523in}}%
\pgfpathlineto{\pgfqpoint{5.666777in}{1.748523in}}%
\pgfusepath{stroke}%
\end{pgfscope}%
\begin{pgfscope}%
\pgfpathrectangle{\pgfqpoint{0.711606in}{0.549444in}}{\pgfqpoint{4.955171in}{2.902168in}}%
\pgfusepath{clip}%
\pgfsetbuttcap%
\pgfsetroundjoin%
\pgfsetlinewidth{1.003750pt}%
\definecolor{currentstroke}{rgb}{0.297178,0.047470,0.420491}%
\pgfsetstrokecolor{currentstroke}%
\pgfsetdash{}{0pt}%
\pgfpathmoveto{\pgfqpoint{1.766446in}{0.549444in}}%
\pgfpathlineto{\pgfqpoint{1.742547in}{0.558427in}}%
\pgfpathlineto{\pgfqpoint{1.715145in}{0.568922in}}%
\pgfpathlineto{\pgfqpoint{1.709291in}{0.571191in}}%
\pgfpathlineto{\pgfqpoint{1.676035in}{0.584345in}}%
\pgfpathlineto{\pgfqpoint{1.665977in}{0.588399in}}%
\pgfpathlineto{\pgfqpoint{1.642779in}{0.597864in}}%
\pgfpathlineto{\pgfqpoint{1.618663in}{0.607877in}}%
\pgfpathlineto{\pgfqpoint{1.609523in}{0.611718in}}%
\pgfpathlineto{\pgfqpoint{1.576266in}{0.625948in}}%
\pgfpathlineto{\pgfqpoint{1.573039in}{0.627355in}}%
\pgfpathlineto{\pgfqpoint{1.543010in}{0.640605in}}%
\pgfpathlineto{\pgfqpoint{1.529128in}{0.646832in}}%
\pgfpathlineto{\pgfqpoint{1.509754in}{0.655631in}}%
\pgfpathlineto{\pgfqpoint{1.486622in}{0.666310in}}%
\pgfpathlineto{\pgfqpoint{1.476498in}{0.671042in}}%
\pgfpathlineto{\pgfqpoint{1.445454in}{0.685788in}}%
\pgfpathlineto{\pgfqpoint{1.443242in}{0.686852in}}%
\pgfpathlineto{\pgfqpoint{1.409986in}{0.703117in}}%
\pgfpathlineto{\pgfqpoint{1.405663in}{0.705265in}}%
\pgfpathlineto{\pgfqpoint{1.376729in}{0.719827in}}%
\pgfpathlineto{\pgfqpoint{1.367108in}{0.724743in}}%
\pgfpathlineto{\pgfqpoint{1.343473in}{0.736976in}}%
\pgfpathlineto{\pgfqpoint{1.329684in}{0.744221in}}%
\pgfpathlineto{\pgfqpoint{1.310217in}{0.754582in}}%
\pgfpathlineto{\pgfqpoint{1.293342in}{0.763698in}}%
\pgfpathlineto{\pgfqpoint{1.276961in}{0.772664in}}%
\pgfpathlineto{\pgfqpoint{1.258033in}{0.783176in}}%
\pgfpathlineto{\pgfqpoint{1.243705in}{0.791240in}}%
\pgfpathlineto{\pgfqpoint{1.223715in}{0.802653in}}%
\pgfpathlineto{\pgfqpoint{1.210448in}{0.810331in}}%
\pgfpathlineto{\pgfqpoint{1.190345in}{0.822131in}}%
\pgfpathlineto{\pgfqpoint{1.177192in}{0.829958in}}%
\pgfpathlineto{\pgfqpoint{1.157884in}{0.841609in}}%
\pgfpathlineto{\pgfqpoint{1.143936in}{0.850143in}}%
\pgfpathlineto{\pgfqpoint{1.126295in}{0.861086in}}%
\pgfpathlineto{\pgfqpoint{1.110680in}{0.870911in}}%
\pgfpathlineto{\pgfqpoint{1.095543in}{0.880564in}}%
\pgfpathlineto{\pgfqpoint{1.077424in}{0.892286in}}%
\pgfpathlineto{\pgfqpoint{1.065594in}{0.900042in}}%
\pgfpathlineto{\pgfqpoint{1.044168in}{0.914295in}}%
\pgfpathlineto{\pgfqpoint{1.036416in}{0.919519in}}%
\pgfpathlineto{\pgfqpoint{1.010911in}{0.936966in}}%
\pgfpathlineto{\pgfqpoint{1.007981in}{0.938997in}}%
\pgfpathlineto{\pgfqpoint{0.980309in}{0.958475in}}%
\pgfpathlineto{\pgfqpoint{0.977655in}{0.960373in}}%
\pgfpathlineto{\pgfqpoint{0.953393in}{0.977952in}}%
\pgfpathlineto{\pgfqpoint{0.944399in}{0.984571in}}%
\pgfpathlineto{\pgfqpoint{0.927147in}{0.997430in}}%
\pgfpathlineto{\pgfqpoint{0.911143in}{1.009549in}}%
\pgfpathlineto{\pgfqpoint{0.901547in}{1.016908in}}%
\pgfpathlineto{\pgfqpoint{0.877887in}{1.035344in}}%
\pgfpathlineto{\pgfqpoint{0.876567in}{1.036385in}}%
\pgfpathlineto{\pgfqpoint{0.852327in}{1.055863in}}%
\pgfpathlineto{\pgfqpoint{0.844630in}{1.062154in}}%
\pgfpathlineto{\pgfqpoint{0.828696in}{1.075340in}}%
\pgfpathlineto{\pgfqpoint{0.811374in}{1.089921in}}%
\pgfpathlineto{\pgfqpoint{0.805627in}{1.094818in}}%
\pgfpathlineto{\pgfqpoint{0.783191in}{1.114296in}}%
\pgfpathlineto{\pgfqpoint{0.778118in}{1.118786in}}%
\pgfpathlineto{\pgfqpoint{0.761389in}{1.133773in}}%
\pgfpathlineto{\pgfqpoint{0.744862in}{1.148851in}}%
\pgfpathlineto{\pgfqpoint{0.740096in}{1.153251in}}%
\pgfpathlineto{\pgfqpoint{0.719433in}{1.172729in}}%
\pgfpathlineto{\pgfqpoint{0.711606in}{1.180258in}}%
\pgfusepath{stroke}%
\end{pgfscope}%
\begin{pgfscope}%
\pgfpathrectangle{\pgfqpoint{0.711606in}{0.549444in}}{\pgfqpoint{4.955171in}{2.902168in}}%
\pgfusepath{clip}%
\pgfsetbuttcap%
\pgfsetroundjoin%
\pgfsetlinewidth{1.003750pt}%
\definecolor{currentstroke}{rgb}{0.297178,0.047470,0.420491}%
\pgfsetstrokecolor{currentstroke}%
\pgfsetdash{}{0pt}%
\pgfpathmoveto{\pgfqpoint{0.711606in}{3.387332in}}%
\pgfpathlineto{\pgfqpoint{0.720099in}{3.393180in}}%
\pgfpathlineto{\pgfqpoint{0.744862in}{3.409474in}}%
\pgfpathlineto{\pgfqpoint{0.749901in}{3.412657in}}%
\pgfpathlineto{\pgfqpoint{0.778118in}{3.429726in}}%
\pgfpathlineto{\pgfqpoint{0.782274in}{3.432135in}}%
\pgfpathlineto{\pgfqpoint{0.811374in}{3.448318in}}%
\pgfpathlineto{\pgfqpoint{0.817568in}{3.451613in}}%
\pgfusepath{stroke}%
\end{pgfscope}%
\begin{pgfscope}%
\pgfpathrectangle{\pgfqpoint{0.711606in}{0.549444in}}{\pgfqpoint{4.955171in}{2.902168in}}%
\pgfusepath{clip}%
\pgfsetbuttcap%
\pgfsetroundjoin%
\pgfsetlinewidth{1.003750pt}%
\definecolor{currentstroke}{rgb}{0.297178,0.047470,0.420491}%
\pgfsetstrokecolor{currentstroke}%
\pgfsetdash{}{0pt}%
\pgfpathmoveto{\pgfqpoint{2.794244in}{3.451613in}}%
\pgfpathlineto{\pgfqpoint{2.873258in}{3.424685in}}%
\pgfpathlineto{\pgfqpoint{2.959070in}{3.393180in}}%
\pgfpathlineto{\pgfqpoint{3.039539in}{3.361420in}}%
\pgfpathlineto{\pgfqpoint{3.106051in}{3.333504in}}%
\pgfpathlineto{\pgfqpoint{3.190207in}{3.295791in}}%
\pgfpathlineto{\pgfqpoint{3.272332in}{3.256431in}}%
\pgfpathlineto{\pgfqpoint{3.347495in}{3.217881in}}%
\pgfpathlineto{\pgfqpoint{3.418931in}{3.178926in}}%
\pgfpathlineto{\pgfqpoint{3.486241in}{3.139970in}}%
\pgfpathlineto{\pgfqpoint{3.549777in}{3.101015in}}%
\pgfpathlineto{\pgfqpoint{3.609852in}{3.062060in}}%
\pgfpathlineto{\pgfqpoint{3.671406in}{3.019739in}}%
\pgfpathlineto{\pgfqpoint{3.737918in}{2.970961in}}%
\pgfpathlineto{\pgfqpoint{3.795688in}{2.925716in}}%
\pgfpathlineto{\pgfqpoint{3.842569in}{2.886761in}}%
\pgfpathlineto{\pgfqpoint{3.904199in}{2.832000in}}%
\pgfpathlineto{\pgfqpoint{3.948992in}{2.789373in}}%
\pgfpathlineto{\pgfqpoint{4.006785in}{2.730940in}}%
\pgfpathlineto{\pgfqpoint{4.063900in}{2.672507in}}%
\pgfpathlineto{\pgfqpoint{4.085513in}{2.653029in}}%
\pgfpathlineto{\pgfqpoint{4.111794in}{2.633552in}}%
\pgfpathlineto{\pgfqpoint{4.148713in}{2.614074in}}%
\pgfpathlineto{\pgfqpoint{4.170249in}{2.605485in}}%
\pgfpathlineto{\pgfqpoint{4.203505in}{2.594077in}}%
\pgfpathlineto{\pgfqpoint{4.332853in}{2.555641in}}%
\pgfpathlineto{\pgfqpoint{4.403042in}{2.534224in}}%
\pgfpathlineto{\pgfqpoint{4.502811in}{2.501549in}}%
\pgfpathlineto{\pgfqpoint{4.570552in}{2.477731in}}%
\pgfpathlineto{\pgfqpoint{4.673403in}{2.438775in}}%
\pgfpathlineto{\pgfqpoint{4.768860in}{2.399543in}}%
\pgfpathlineto{\pgfqpoint{4.868629in}{2.355083in}}%
\pgfpathlineto{\pgfqpoint{4.938262in}{2.321909in}}%
\pgfpathlineto{\pgfqpoint{5.015207in}{2.282954in}}%
\pgfpathlineto{\pgfqpoint{5.087659in}{2.243999in}}%
\pgfpathlineto{\pgfqpoint{5.156007in}{2.205044in}}%
\pgfpathlineto{\pgfqpoint{5.220594in}{2.166088in}}%
\pgfpathlineto{\pgfqpoint{5.281728in}{2.127133in}}%
\pgfpathlineto{\pgfqpoint{5.339681in}{2.088178in}}%
\pgfpathlineto{\pgfqpoint{5.400728in}{2.044724in}}%
\pgfpathlineto{\pgfqpoint{5.471667in}{1.990790in}}%
\pgfpathlineto{\pgfqpoint{5.533752in}{1.940035in}}%
\pgfpathlineto{\pgfqpoint{5.587099in}{1.893401in}}%
\pgfpathlineto{\pgfqpoint{5.633521in}{1.850178in}}%
\pgfpathlineto{\pgfqpoint{5.666777in}{1.817466in}}%
\pgfpathlineto{\pgfqpoint{5.666777in}{1.817466in}}%
\pgfusepath{stroke}%
\end{pgfscope}%
\begin{pgfscope}%
\pgfpathrectangle{\pgfqpoint{0.711606in}{0.549444in}}{\pgfqpoint{4.955171in}{2.902168in}}%
\pgfusepath{clip}%
\pgfsetbuttcap%
\pgfsetroundjoin%
\pgfsetlinewidth{1.003750pt}%
\definecolor{currentstroke}{rgb}{0.322610,0.055634,0.426377}%
\pgfsetstrokecolor{currentstroke}%
\pgfsetdash{}{0pt}%
\pgfpathmoveto{\pgfqpoint{1.668458in}{0.549444in}}%
\pgfpathlineto{\pgfqpoint{1.642779in}{0.559734in}}%
\pgfpathlineto{\pgfqpoint{1.620242in}{0.568922in}}%
\pgfpathlineto{\pgfqpoint{1.609523in}{0.573343in}}%
\pgfpathlineto{\pgfqpoint{1.576266in}{0.587302in}}%
\pgfpathlineto{\pgfqpoint{1.573701in}{0.588399in}}%
\pgfpathlineto{\pgfqpoint{1.543010in}{0.601677in}}%
\pgfpathlineto{\pgfqpoint{1.528912in}{0.607877in}}%
\pgfpathlineto{\pgfqpoint{1.509754in}{0.616402in}}%
\pgfpathlineto{\pgfqpoint{1.485533in}{0.627355in}}%
\pgfpathlineto{\pgfqpoint{1.476498in}{0.631490in}}%
\pgfpathlineto{\pgfqpoint{1.443499in}{0.646832in}}%
\pgfpathlineto{\pgfqpoint{1.443242in}{0.646953in}}%
\pgfpathlineto{\pgfqpoint{1.409986in}{0.662872in}}%
\pgfpathlineto{\pgfqpoint{1.402911in}{0.666310in}}%
\pgfpathlineto{\pgfqpoint{1.376729in}{0.679191in}}%
\pgfpathlineto{\pgfqpoint{1.363520in}{0.685788in}}%
\pgfpathlineto{\pgfqpoint{1.343473in}{0.695924in}}%
\pgfpathlineto{\pgfqpoint{1.325268in}{0.705265in}}%
\pgfpathlineto{\pgfqpoint{1.310217in}{0.713085in}}%
\pgfpathlineto{\pgfqpoint{1.288104in}{0.724743in}}%
\pgfpathlineto{\pgfqpoint{1.276961in}{0.730693in}}%
\pgfpathlineto{\pgfqpoint{1.251984in}{0.744221in}}%
\pgfpathlineto{\pgfqpoint{1.243705in}{0.748763in}}%
\pgfpathlineto{\pgfqpoint{1.216862in}{0.763698in}}%
\pgfpathlineto{\pgfqpoint{1.210448in}{0.767314in}}%
\pgfpathlineto{\pgfqpoint{1.182698in}{0.783176in}}%
\pgfpathlineto{\pgfqpoint{1.177192in}{0.786364in}}%
\pgfpathlineto{\pgfqpoint{1.149451in}{0.802653in}}%
\pgfpathlineto{\pgfqpoint{1.143936in}{0.805935in}}%
\pgfpathlineto{\pgfqpoint{1.117086in}{0.822131in}}%
\pgfpathlineto{\pgfqpoint{1.110680in}{0.826048in}}%
\pgfpathlineto{\pgfqpoint{1.085567in}{0.841609in}}%
\pgfpathlineto{\pgfqpoint{1.077424in}{0.846724in}}%
\pgfpathlineto{\pgfqpoint{1.054860in}{0.861086in}}%
\pgfpathlineto{\pgfqpoint{1.044168in}{0.867988in}}%
\pgfpathlineto{\pgfqpoint{1.024935in}{0.880564in}}%
\pgfpathlineto{\pgfqpoint{1.010911in}{0.889864in}}%
\pgfpathlineto{\pgfqpoint{0.995760in}{0.900042in}}%
\pgfpathlineto{\pgfqpoint{0.977655in}{0.912379in}}%
\pgfpathlineto{\pgfqpoint{0.967308in}{0.919519in}}%
\pgfpathlineto{\pgfqpoint{0.944399in}{0.935561in}}%
\pgfpathlineto{\pgfqpoint{0.939552in}{0.938997in}}%
\pgfpathlineto{\pgfqpoint{0.912491in}{0.958475in}}%
\pgfpathlineto{\pgfqpoint{0.911143in}{0.959461in}}%
\pgfpathlineto{\pgfqpoint{0.886178in}{0.977952in}}%
\pgfpathlineto{\pgfqpoint{0.877887in}{0.984188in}}%
\pgfpathlineto{\pgfqpoint{0.860495in}{0.997430in}}%
\pgfpathlineto{\pgfqpoint{0.844630in}{1.009698in}}%
\pgfpathlineto{\pgfqpoint{0.835421in}{1.016908in}}%
\pgfpathlineto{\pgfqpoint{0.811374in}{1.036029in}}%
\pgfpathlineto{\pgfqpoint{0.810932in}{1.036385in}}%
\pgfpathlineto{\pgfqpoint{0.787170in}{1.055863in}}%
\pgfpathlineto{\pgfqpoint{0.778118in}{1.063406in}}%
\pgfpathlineto{\pgfqpoint{0.763967in}{1.075340in}}%
\pgfpathlineto{\pgfqpoint{0.744862in}{1.091724in}}%
\pgfpathlineto{\pgfqpoint{0.741296in}{1.094818in}}%
\pgfpathlineto{\pgfqpoint{0.719273in}{1.114296in}}%
\pgfpathlineto{\pgfqpoint{0.711606in}{1.121203in}}%
\pgfusepath{stroke}%
\end{pgfscope}%
\begin{pgfscope}%
\pgfpathrectangle{\pgfqpoint{0.711606in}{0.549444in}}{\pgfqpoint{4.955171in}{2.902168in}}%
\pgfusepath{clip}%
\pgfsetbuttcap%
\pgfsetroundjoin%
\pgfsetlinewidth{1.003750pt}%
\definecolor{currentstroke}{rgb}{0.322610,0.055634,0.426377}%
\pgfsetstrokecolor{currentstroke}%
\pgfsetdash{}{0pt}%
\pgfpathmoveto{\pgfqpoint{2.907197in}{3.451613in}}%
\pgfpathlineto{\pgfqpoint{3.011124in}{3.412657in}}%
\pgfpathlineto{\pgfqpoint{3.106836in}{3.373702in}}%
\pgfpathlineto{\pgfqpoint{3.205819in}{3.330035in}}%
\pgfpathlineto{\pgfqpoint{3.278323in}{3.295791in}}%
\pgfpathlineto{\pgfqpoint{3.355902in}{3.256836in}}%
\pgfpathlineto{\pgfqpoint{3.438613in}{3.212539in}}%
\pgfpathlineto{\pgfqpoint{3.505125in}{3.174647in}}%
\pgfpathlineto{\pgfqpoint{3.571637in}{3.134546in}}%
\pgfpathlineto{\pgfqpoint{3.638150in}{3.092038in}}%
\pgfpathlineto{\pgfqpoint{3.710823in}{3.042582in}}%
\pgfpathlineto{\pgfqpoint{3.771175in}{2.998783in}}%
\pgfpathlineto{\pgfqpoint{3.840337in}{2.945194in}}%
\pgfpathlineto{\pgfqpoint{3.904199in}{2.891917in}}%
\pgfpathlineto{\pgfqpoint{3.953700in}{2.847806in}}%
\pgfpathlineto{\pgfqpoint{4.003968in}{2.800283in}}%
\pgfpathlineto{\pgfqpoint{4.114674in}{2.691985in}}%
\pgfpathlineto{\pgfqpoint{4.138389in}{2.672507in}}%
\pgfpathlineto{\pgfqpoint{4.170249in}{2.651939in}}%
\pgfpathlineto{\pgfqpoint{4.209707in}{2.633552in}}%
\pgfpathlineto{\pgfqpoint{4.264921in}{2.614074in}}%
\pgfpathlineto{\pgfqpoint{4.369786in}{2.581770in}}%
\pgfpathlineto{\pgfqpoint{4.469554in}{2.550145in}}%
\pgfpathlineto{\pgfqpoint{4.569323in}{2.516014in}}%
\pgfpathlineto{\pgfqpoint{4.672155in}{2.477731in}}%
\pgfpathlineto{\pgfqpoint{4.768860in}{2.438724in}}%
\pgfpathlineto{\pgfqpoint{4.868629in}{2.395175in}}%
\pgfpathlineto{\pgfqpoint{4.942136in}{2.360865in}}%
\pgfpathlineto{\pgfqpoint{5.034909in}{2.314633in}}%
\pgfpathlineto{\pgfqpoint{5.101422in}{2.279371in}}%
\pgfpathlineto{\pgfqpoint{5.167934in}{2.242180in}}%
\pgfpathlineto{\pgfqpoint{5.234447in}{2.202900in}}%
\pgfpathlineto{\pgfqpoint{5.300959in}{2.161353in}}%
\pgfpathlineto{\pgfqpoint{5.367471in}{2.117342in}}%
\pgfpathlineto{\pgfqpoint{5.436664in}{2.068700in}}%
\pgfpathlineto{\pgfqpoint{5.500496in}{2.020793in}}%
\pgfpathlineto{\pgfqpoint{5.562400in}{1.971312in}}%
\pgfpathlineto{\pgfqpoint{5.608292in}{1.932357in}}%
\pgfpathlineto{\pgfqpoint{5.666777in}{1.879544in}}%
\pgfpathlineto{\pgfqpoint{5.666777in}{1.879544in}}%
\pgfusepath{stroke}%
\end{pgfscope}%
\begin{pgfscope}%
\pgfpathrectangle{\pgfqpoint{0.711606in}{0.549444in}}{\pgfqpoint{4.955171in}{2.902168in}}%
\pgfusepath{clip}%
\pgfsetbuttcap%
\pgfsetroundjoin%
\pgfsetlinewidth{1.003750pt}%
\definecolor{currentstroke}{rgb}{0.341500,0.062325,0.429425}%
\pgfsetstrokecolor{currentstroke}%
\pgfsetdash{}{0pt}%
\pgfpathmoveto{\pgfqpoint{1.577553in}{0.549444in}}%
\pgfpathlineto{\pgfqpoint{1.576266in}{0.549980in}}%
\pgfpathlineto{\pgfqpoint{1.543010in}{0.564099in}}%
\pgfpathlineto{\pgfqpoint{1.531833in}{0.568922in}}%
\pgfpathlineto{\pgfqpoint{1.509754in}{0.578559in}}%
\pgfpathlineto{\pgfqpoint{1.487561in}{0.588399in}}%
\pgfpathlineto{\pgfqpoint{1.476498in}{0.593362in}}%
\pgfpathlineto{\pgfqpoint{1.444639in}{0.607877in}}%
\pgfpathlineto{\pgfqpoint{1.443242in}{0.608521in}}%
\pgfpathlineto{\pgfqpoint{1.409986in}{0.624108in}}%
\pgfpathlineto{\pgfqpoint{1.403163in}{0.627355in}}%
\pgfpathlineto{\pgfqpoint{1.376729in}{0.640083in}}%
\pgfpathlineto{\pgfqpoint{1.362917in}{0.646832in}}%
\pgfpathlineto{\pgfqpoint{1.343473in}{0.656448in}}%
\pgfpathlineto{\pgfqpoint{1.323817in}{0.666310in}}%
\pgfpathlineto{\pgfqpoint{1.310217in}{0.673217in}}%
\pgfpathlineto{\pgfqpoint{1.285815in}{0.685788in}}%
\pgfpathlineto{\pgfqpoint{1.276961in}{0.690405in}}%
\pgfpathlineto{\pgfqpoint{1.248865in}{0.705265in}}%
\pgfpathlineto{\pgfqpoint{1.243705in}{0.708028in}}%
\pgfpathlineto{\pgfqpoint{1.212921in}{0.724743in}}%
\pgfpathlineto{\pgfqpoint{1.210448in}{0.726103in}}%
\pgfpathlineto{\pgfqpoint{1.177945in}{0.744221in}}%
\pgfpathlineto{\pgfqpoint{1.177192in}{0.744645in}}%
\pgfpathlineto{\pgfqpoint{1.143936in}{0.763675in}}%
\pgfpathlineto{\pgfqpoint{1.143896in}{0.763698in}}%
\pgfpathlineto{\pgfqpoint{1.110737in}{0.783176in}}%
\pgfpathlineto{\pgfqpoint{1.110680in}{0.783210in}}%
\pgfpathlineto{\pgfqpoint{1.078433in}{0.802653in}}%
\pgfpathlineto{\pgfqpoint{1.077424in}{0.803270in}}%
\pgfpathlineto{\pgfqpoint{1.046952in}{0.822131in}}%
\pgfpathlineto{\pgfqpoint{1.044168in}{0.823878in}}%
\pgfpathlineto{\pgfqpoint{1.016262in}{0.841609in}}%
\pgfpathlineto{\pgfqpoint{1.010911in}{0.845055in}}%
\pgfpathlineto{\pgfqpoint{0.986332in}{0.861086in}}%
\pgfpathlineto{\pgfqpoint{0.977655in}{0.866824in}}%
\pgfpathlineto{\pgfqpoint{0.957135in}{0.880564in}}%
\pgfpathlineto{\pgfqpoint{0.944399in}{0.889211in}}%
\pgfpathlineto{\pgfqpoint{0.928643in}{0.900042in}}%
\pgfpathlineto{\pgfqpoint{0.911143in}{0.912242in}}%
\pgfpathlineto{\pgfqpoint{0.900831in}{0.919519in}}%
\pgfpathlineto{\pgfqpoint{0.877887in}{0.935944in}}%
\pgfpathlineto{\pgfqpoint{0.873674in}{0.938997in}}%
\pgfpathlineto{\pgfqpoint{0.847194in}{0.958475in}}%
\pgfpathlineto{\pgfqpoint{0.844630in}{0.960390in}}%
\pgfpathlineto{\pgfqpoint{0.821410in}{0.977952in}}%
\pgfpathlineto{\pgfqpoint{0.811374in}{0.985658in}}%
\pgfpathlineto{\pgfqpoint{0.796222in}{0.997430in}}%
\pgfpathlineto{\pgfqpoint{0.778118in}{1.011712in}}%
\pgfpathlineto{\pgfqpoint{0.771609in}{1.016908in}}%
\pgfpathlineto{\pgfqpoint{0.747598in}{1.036385in}}%
\pgfpathlineto{\pgfqpoint{0.744862in}{1.038644in}}%
\pgfpathlineto{\pgfqpoint{0.724245in}{1.055863in}}%
\pgfpathlineto{\pgfqpoint{0.711606in}{1.066591in}}%
\pgfusepath{stroke}%
\end{pgfscope}%
\begin{pgfscope}%
\pgfpathrectangle{\pgfqpoint{0.711606in}{0.549444in}}{\pgfqpoint{4.955171in}{2.902168in}}%
\pgfusepath{clip}%
\pgfsetbuttcap%
\pgfsetroundjoin%
\pgfsetlinewidth{1.003750pt}%
\definecolor{currentstroke}{rgb}{0.341500,0.062325,0.429425}%
\pgfsetstrokecolor{currentstroke}%
\pgfsetdash{}{0pt}%
\pgfpathmoveto{\pgfqpoint{3.009466in}{3.451613in}}%
\pgfpathlineto{\pgfqpoint{3.106964in}{3.412657in}}%
\pgfpathlineto{\pgfqpoint{3.205819in}{3.369924in}}%
\pgfpathlineto{\pgfqpoint{3.281838in}{3.334747in}}%
\pgfpathlineto{\pgfqpoint{3.372100in}{3.290187in}}%
\pgfpathlineto{\pgfqpoint{3.438613in}{3.255296in}}%
\pgfpathlineto{\pgfqpoint{3.506200in}{3.217881in}}%
\pgfpathlineto{\pgfqpoint{3.572829in}{3.178926in}}%
\pgfpathlineto{\pgfqpoint{3.638150in}{3.138576in}}%
\pgfpathlineto{\pgfqpoint{3.704662in}{3.095070in}}%
\pgfpathlineto{\pgfqpoint{3.780003in}{3.042582in}}%
\pgfpathlineto{\pgfqpoint{3.837687in}{2.999793in}}%
\pgfpathlineto{\pgfqpoint{3.906739in}{2.945194in}}%
\pgfpathlineto{\pgfqpoint{3.975320in}{2.886761in}}%
\pgfpathlineto{\pgfqpoint{4.039082in}{2.828328in}}%
\pgfpathlineto{\pgfqpoint{4.140898in}{2.730940in}}%
\pgfpathlineto{\pgfqpoint{4.170249in}{2.706200in}}%
\pgfpathlineto{\pgfqpoint{4.203505in}{2.683357in}}%
\pgfpathlineto{\pgfqpoint{4.236761in}{2.665827in}}%
\pgfpathlineto{\pgfqpoint{4.270017in}{2.651832in}}%
\pgfpathlineto{\pgfqpoint{4.336530in}{2.628898in}}%
\pgfpathlineto{\pgfqpoint{4.536067in}{2.564144in}}%
\pgfpathlineto{\pgfqpoint{4.635835in}{2.528534in}}%
\pgfpathlineto{\pgfqpoint{4.735604in}{2.490008in}}%
\pgfpathlineto{\pgfqpoint{4.835372in}{2.448374in}}%
\pgfpathlineto{\pgfqpoint{4.901885in}{2.418841in}}%
\pgfpathlineto{\pgfqpoint{5.001653in}{2.371545in}}%
\pgfpathlineto{\pgfqpoint{5.068166in}{2.338011in}}%
\pgfpathlineto{\pgfqpoint{5.135223in}{2.302432in}}%
\pgfpathlineto{\pgfqpoint{5.204782in}{2.263477in}}%
\pgfpathlineto{\pgfqpoint{5.270740in}{2.224521in}}%
\pgfpathlineto{\pgfqpoint{5.334215in}{2.185018in}}%
\pgfpathlineto{\pgfqpoint{5.400728in}{2.141254in}}%
\pgfpathlineto{\pgfqpoint{5.476564in}{2.088178in}}%
\pgfpathlineto{\pgfqpoint{5.533752in}{2.045632in}}%
\pgfpathlineto{\pgfqpoint{5.603009in}{1.990790in}}%
\pgfpathlineto{\pgfqpoint{5.666777in}{1.936560in}}%
\pgfpathlineto{\pgfqpoint{5.666777in}{1.936560in}}%
\pgfusepath{stroke}%
\end{pgfscope}%
\begin{pgfscope}%
\pgfpathrectangle{\pgfqpoint{0.711606in}{0.549444in}}{\pgfqpoint{4.955171in}{2.902168in}}%
\pgfusepath{clip}%
\pgfsetbuttcap%
\pgfsetroundjoin%
\pgfsetlinewidth{1.003750pt}%
\definecolor{currentstroke}{rgb}{0.366529,0.071579,0.431994}%
\pgfsetstrokecolor{currentstroke}%
\pgfsetdash{}{0pt}%
\pgfpathmoveto{\pgfqpoint{1.492558in}{0.549444in}}%
\pgfpathlineto{\pgfqpoint{1.476498in}{0.556513in}}%
\pgfpathlineto{\pgfqpoint{1.448731in}{0.568922in}}%
\pgfpathlineto{\pgfqpoint{1.443242in}{0.571403in}}%
\pgfpathlineto{\pgfqpoint{1.409986in}{0.586673in}}%
\pgfpathlineto{\pgfqpoint{1.406285in}{0.588399in}}%
\pgfpathlineto{\pgfqpoint{1.376729in}{0.602343in}}%
\pgfpathlineto{\pgfqpoint{1.365167in}{0.607877in}}%
\pgfpathlineto{\pgfqpoint{1.343473in}{0.618381in}}%
\pgfpathlineto{\pgfqpoint{1.325205in}{0.627355in}}%
\pgfpathlineto{\pgfqpoint{1.310217in}{0.634802in}}%
\pgfpathlineto{\pgfqpoint{1.286348in}{0.646832in}}%
\pgfpathlineto{\pgfqpoint{1.276961in}{0.651619in}}%
\pgfpathlineto{\pgfqpoint{1.248551in}{0.666310in}}%
\pgfpathlineto{\pgfqpoint{1.243705in}{0.668846in}}%
\pgfpathlineto{\pgfqpoint{1.211771in}{0.685788in}}%
\pgfpathlineto{\pgfqpoint{1.210448in}{0.686498in}}%
\pgfpathlineto{\pgfqpoint{1.177192in}{0.704603in}}%
\pgfpathlineto{\pgfqpoint{1.175992in}{0.705265in}}%
\pgfpathlineto{\pgfqpoint{1.143936in}{0.723170in}}%
\pgfpathlineto{\pgfqpoint{1.141156in}{0.724743in}}%
\pgfpathlineto{\pgfqpoint{1.110680in}{0.742202in}}%
\pgfpathlineto{\pgfqpoint{1.107202in}{0.744221in}}%
\pgfpathlineto{\pgfqpoint{1.077424in}{0.761717in}}%
\pgfpathlineto{\pgfqpoint{1.074095in}{0.763698in}}%
\pgfpathlineto{\pgfqpoint{1.044168in}{0.781734in}}%
\pgfpathlineto{\pgfqpoint{1.041806in}{0.783176in}}%
\pgfpathlineto{\pgfqpoint{1.010911in}{0.802272in}}%
\pgfpathlineto{\pgfqpoint{1.010302in}{0.802653in}}%
\pgfpathlineto{\pgfqpoint{0.979591in}{0.822131in}}%
\pgfpathlineto{\pgfqpoint{0.977655in}{0.823375in}}%
\pgfpathlineto{\pgfqpoint{0.949634in}{0.841609in}}%
\pgfpathlineto{\pgfqpoint{0.944399in}{0.845061in}}%
\pgfpathlineto{\pgfqpoint{0.920392in}{0.861086in}}%
\pgfpathlineto{\pgfqpoint{0.911143in}{0.867344in}}%
\pgfpathlineto{\pgfqpoint{0.891839in}{0.880564in}}%
\pgfpathlineto{\pgfqpoint{0.877887in}{0.890251in}}%
\pgfpathlineto{\pgfqpoint{0.863951in}{0.900042in}}%
\pgfpathlineto{\pgfqpoint{0.844630in}{0.913806in}}%
\pgfpathlineto{\pgfqpoint{0.836705in}{0.919519in}}%
\pgfpathlineto{\pgfqpoint{0.811374in}{0.938038in}}%
\pgfpathlineto{\pgfqpoint{0.810077in}{0.938997in}}%
\pgfpathlineto{\pgfqpoint{0.784152in}{0.958475in}}%
\pgfpathlineto{\pgfqpoint{0.778118in}{0.963076in}}%
\pgfpathlineto{\pgfqpoint{0.758834in}{0.977952in}}%
\pgfpathlineto{\pgfqpoint{0.744862in}{0.988892in}}%
\pgfpathlineto{\pgfqpoint{0.734081in}{0.997430in}}%
\pgfpathlineto{\pgfqpoint{0.711606in}{1.015499in}}%
\pgfusepath{stroke}%
\end{pgfscope}%
\begin{pgfscope}%
\pgfpathrectangle{\pgfqpoint{0.711606in}{0.549444in}}{\pgfqpoint{4.955171in}{2.902168in}}%
\pgfusepath{clip}%
\pgfsetbuttcap%
\pgfsetroundjoin%
\pgfsetlinewidth{1.003750pt}%
\definecolor{currentstroke}{rgb}{0.366529,0.071579,0.431994}%
\pgfsetstrokecolor{currentstroke}%
\pgfsetdash{}{0pt}%
\pgfpathmoveto{\pgfqpoint{3.103740in}{3.451613in}}%
\pgfpathlineto{\pgfqpoint{3.172563in}{3.422800in}}%
\pgfpathlineto{\pgfqpoint{3.239890in}{3.393180in}}%
\pgfpathlineto{\pgfqpoint{3.338844in}{3.346708in}}%
\pgfpathlineto{\pgfqpoint{3.405357in}{3.313509in}}%
\pgfpathlineto{\pgfqpoint{3.476007in}{3.276314in}}%
\pgfpathlineto{\pgfqpoint{3.546073in}{3.237359in}}%
\pgfpathlineto{\pgfqpoint{3.612497in}{3.198403in}}%
\pgfpathlineto{\pgfqpoint{3.675559in}{3.159448in}}%
\pgfpathlineto{\pgfqpoint{3.737918in}{3.118854in}}%
\pgfpathlineto{\pgfqpoint{3.804431in}{3.073025in}}%
\pgfpathlineto{\pgfqpoint{3.872684in}{3.023104in}}%
\pgfpathlineto{\pgfqpoint{3.947136in}{2.964672in}}%
\pgfpathlineto{\pgfqpoint{4.003968in}{2.916984in}}%
\pgfpathlineto{\pgfqpoint{4.059829in}{2.867283in}}%
\pgfpathlineto{\pgfqpoint{4.186612in}{2.750418in}}%
\pgfpathlineto{\pgfqpoint{4.211070in}{2.730940in}}%
\pgfpathlineto{\pgfqpoint{4.239474in}{2.711462in}}%
\pgfpathlineto{\pgfqpoint{4.274867in}{2.691985in}}%
\pgfpathlineto{\pgfqpoint{4.320063in}{2.672507in}}%
\pgfpathlineto{\pgfqpoint{4.374469in}{2.653029in}}%
\pgfpathlineto{\pgfqpoint{4.606090in}{2.575119in}}%
\pgfpathlineto{\pgfqpoint{4.710559in}{2.536163in}}%
\pgfpathlineto{\pgfqpoint{4.807414in}{2.497208in}}%
\pgfpathlineto{\pgfqpoint{4.901885in}{2.456394in}}%
\pgfpathlineto{\pgfqpoint{5.001653in}{2.410042in}}%
\pgfpathlineto{\pgfqpoint{5.068166in}{2.377232in}}%
\pgfpathlineto{\pgfqpoint{5.137245in}{2.341387in}}%
\pgfpathlineto{\pgfqpoint{5.208422in}{2.302432in}}%
\pgfpathlineto{\pgfqpoint{5.275963in}{2.263477in}}%
\pgfpathlineto{\pgfqpoint{5.340143in}{2.224521in}}%
\pgfpathlineto{\pgfqpoint{5.401212in}{2.185566in}}%
\pgfpathlineto{\pgfqpoint{5.467240in}{2.141096in}}%
\pgfpathlineto{\pgfqpoint{5.541240in}{2.088178in}}%
\pgfpathlineto{\pgfqpoint{5.600265in}{2.043304in}}%
\pgfpathlineto{\pgfqpoint{5.666777in}{1.989571in}}%
\pgfpathlineto{\pgfqpoint{5.666777in}{1.989571in}}%
\pgfusepath{stroke}%
\end{pgfscope}%
\begin{pgfscope}%
\pgfpathrectangle{\pgfqpoint{0.711606in}{0.549444in}}{\pgfqpoint{4.955171in}{2.902168in}}%
\pgfusepath{clip}%
\pgfsetbuttcap%
\pgfsetroundjoin%
\pgfsetlinewidth{1.003750pt}%
\definecolor{currentstroke}{rgb}{0.391453,0.080927,0.433109}%
\pgfsetstrokecolor{currentstroke}%
\pgfsetdash{}{0pt}%
\pgfpathmoveto{\pgfqpoint{1.412198in}{0.549444in}}%
\pgfpathlineto{\pgfqpoint{1.409986in}{0.550452in}}%
\pgfpathlineto{\pgfqpoint{1.376729in}{0.565833in}}%
\pgfpathlineto{\pgfqpoint{1.370148in}{0.568922in}}%
\pgfpathlineto{\pgfqpoint{1.343473in}{0.581581in}}%
\pgfpathlineto{\pgfqpoint{1.329308in}{0.588399in}}%
\pgfpathlineto{\pgfqpoint{1.310217in}{0.597692in}}%
\pgfpathlineto{\pgfqpoint{1.289583in}{0.607877in}}%
\pgfpathlineto{\pgfqpoint{1.276961in}{0.614178in}}%
\pgfpathlineto{\pgfqpoint{1.250927in}{0.627355in}}%
\pgfpathlineto{\pgfqpoint{1.243705in}{0.631052in}}%
\pgfpathlineto{\pgfqpoint{1.213297in}{0.646832in}}%
\pgfpathlineto{\pgfqpoint{1.210448in}{0.648328in}}%
\pgfpathlineto{\pgfqpoint{1.177192in}{0.666025in}}%
\pgfpathlineto{\pgfqpoint{1.176663in}{0.666310in}}%
\pgfpathlineto{\pgfqpoint{1.143936in}{0.684171in}}%
\pgfpathlineto{\pgfqpoint{1.141012in}{0.685788in}}%
\pgfpathlineto{\pgfqpoint{1.110680in}{0.702756in}}%
\pgfpathlineto{\pgfqpoint{1.106252in}{0.705265in}}%
\pgfpathlineto{\pgfqpoint{1.077424in}{0.721796in}}%
\pgfpathlineto{\pgfqpoint{1.072349in}{0.724743in}}%
\pgfpathlineto{\pgfqpoint{1.044168in}{0.741307in}}%
\pgfpathlineto{\pgfqpoint{1.039272in}{0.744221in}}%
\pgfpathlineto{\pgfqpoint{1.010911in}{0.761307in}}%
\pgfpathlineto{\pgfqpoint{1.006991in}{0.763698in}}%
\pgfpathlineto{\pgfqpoint{0.977655in}{0.781814in}}%
\pgfpathlineto{\pgfqpoint{0.975476in}{0.783176in}}%
\pgfpathlineto{\pgfqpoint{0.944706in}{0.802653in}}%
\pgfpathlineto{\pgfqpoint{0.944399in}{0.802850in}}%
\pgfpathlineto{\pgfqpoint{0.914701in}{0.822131in}}%
\pgfpathlineto{\pgfqpoint{0.911143in}{0.824471in}}%
\pgfpathlineto{\pgfqpoint{0.885395in}{0.841609in}}%
\pgfpathlineto{\pgfqpoint{0.877887in}{0.846673in}}%
\pgfpathlineto{\pgfqpoint{0.856764in}{0.861086in}}%
\pgfpathlineto{\pgfqpoint{0.844630in}{0.869477in}}%
\pgfpathlineto{\pgfqpoint{0.828784in}{0.880564in}}%
\pgfpathlineto{\pgfqpoint{0.811374in}{0.892910in}}%
\pgfpathlineto{\pgfqpoint{0.801433in}{0.900042in}}%
\pgfpathlineto{\pgfqpoint{0.778118in}{0.916997in}}%
\pgfpathlineto{\pgfqpoint{0.774689in}{0.919519in}}%
\pgfpathlineto{\pgfqpoint{0.748594in}{0.938997in}}%
\pgfpathlineto{\pgfqpoint{0.744862in}{0.941825in}}%
\pgfpathlineto{\pgfqpoint{0.723134in}{0.958475in}}%
\pgfpathlineto{\pgfqpoint{0.711606in}{0.967437in}}%
\pgfusepath{stroke}%
\end{pgfscope}%
\begin{pgfscope}%
\pgfpathrectangle{\pgfqpoint{0.711606in}{0.549444in}}{\pgfqpoint{4.955171in}{2.902168in}}%
\pgfusepath{clip}%
\pgfsetbuttcap%
\pgfsetroundjoin%
\pgfsetlinewidth{1.003750pt}%
\definecolor{currentstroke}{rgb}{0.391453,0.080927,0.433109}%
\pgfsetstrokecolor{currentstroke}%
\pgfsetdash{}{0pt}%
\pgfpathmoveto{\pgfqpoint{3.191533in}{3.451613in}}%
\pgfpathlineto{\pgfqpoint{3.279650in}{3.412657in}}%
\pgfpathlineto{\pgfqpoint{3.372100in}{3.368958in}}%
\pgfpathlineto{\pgfqpoint{3.440371in}{3.334747in}}%
\pgfpathlineto{\pgfqpoint{3.514019in}{3.295791in}}%
\pgfpathlineto{\pgfqpoint{3.583837in}{3.256836in}}%
\pgfpathlineto{\pgfqpoint{3.650119in}{3.217881in}}%
\pgfpathlineto{\pgfqpoint{3.713133in}{3.178926in}}%
\pgfpathlineto{\pgfqpoint{3.773117in}{3.139970in}}%
\pgfpathlineto{\pgfqpoint{3.837687in}{3.095724in}}%
\pgfpathlineto{\pgfqpoint{3.910738in}{3.042582in}}%
\pgfpathlineto{\pgfqpoint{3.970712in}{2.996153in}}%
\pgfpathlineto{\pgfqpoint{4.037224in}{2.941407in}}%
\pgfpathlineto{\pgfqpoint{4.103736in}{2.882958in}}%
\pgfpathlineto{\pgfqpoint{4.206991in}{2.789373in}}%
\pgfpathlineto{\pgfqpoint{4.236761in}{2.765100in}}%
\pgfpathlineto{\pgfqpoint{4.270017in}{2.741558in}}%
\pgfpathlineto{\pgfqpoint{4.303273in}{2.721979in}}%
\pgfpathlineto{\pgfqpoint{4.336530in}{2.705723in}}%
\pgfpathlineto{\pgfqpoint{4.369786in}{2.691781in}}%
\pgfpathlineto{\pgfqpoint{4.436298in}{2.667763in}}%
\pgfpathlineto{\pgfqpoint{4.635835in}{2.599081in}}%
\pgfpathlineto{\pgfqpoint{4.735604in}{2.561594in}}%
\pgfpathlineto{\pgfqpoint{4.835372in}{2.521276in}}%
\pgfpathlineto{\pgfqpoint{4.901885in}{2.492722in}}%
\pgfpathlineto{\pgfqpoint{5.001653in}{2.447230in}}%
\pgfpathlineto{\pgfqpoint{5.068166in}{2.415061in}}%
\pgfpathlineto{\pgfqpoint{5.136561in}{2.380342in}}%
\pgfpathlineto{\pgfqpoint{5.209385in}{2.341387in}}%
\pgfpathlineto{\pgfqpoint{5.278535in}{2.302432in}}%
\pgfpathlineto{\pgfqpoint{5.344288in}{2.263477in}}%
\pgfpathlineto{\pgfqpoint{5.406890in}{2.224521in}}%
\pgfpathlineto{\pgfqpoint{5.467240in}{2.185107in}}%
\pgfpathlineto{\pgfqpoint{5.550759in}{2.127133in}}%
\pgfpathlineto{\pgfqpoint{5.603752in}{2.088178in}}%
\pgfpathlineto{\pgfqpoint{5.666777in}{2.039239in}}%
\pgfpathlineto{\pgfqpoint{5.666777in}{2.039239in}}%
\pgfusepath{stroke}%
\end{pgfscope}%
\begin{pgfscope}%
\pgfpathrectangle{\pgfqpoint{0.711606in}{0.549444in}}{\pgfqpoint{4.955171in}{2.902168in}}%
\pgfusepath{clip}%
\pgfsetbuttcap%
\pgfsetroundjoin%
\pgfsetlinewidth{1.003750pt}%
\definecolor{currentstroke}{rgb}{0.410113,0.087896,0.433098}%
\pgfsetstrokecolor{currentstroke}%
\pgfsetdash{}{0pt}%
\pgfpathmoveto{\pgfqpoint{1.336012in}{0.549444in}}%
\pgfpathlineto{\pgfqpoint{1.310217in}{0.561756in}}%
\pgfpathlineto{\pgfqpoint{1.295408in}{0.568922in}}%
\pgfpathlineto{\pgfqpoint{1.276961in}{0.577947in}}%
\pgfpathlineto{\pgfqpoint{1.255882in}{0.588399in}}%
\pgfpathlineto{\pgfqpoint{1.243705in}{0.594505in}}%
\pgfpathlineto{\pgfqpoint{1.217391in}{0.607877in}}%
\pgfpathlineto{\pgfqpoint{1.210448in}{0.611445in}}%
\pgfpathlineto{\pgfqpoint{1.179894in}{0.627355in}}%
\pgfpathlineto{\pgfqpoint{1.177192in}{0.628778in}}%
\pgfpathlineto{\pgfqpoint{1.143936in}{0.646523in}}%
\pgfpathlineto{\pgfqpoint{1.143364in}{0.646832in}}%
\pgfpathlineto{\pgfqpoint{1.110680in}{0.664708in}}%
\pgfpathlineto{\pgfqpoint{1.107788in}{0.666310in}}%
\pgfpathlineto{\pgfqpoint{1.077424in}{0.683322in}}%
\pgfpathlineto{\pgfqpoint{1.073078in}{0.685788in}}%
\pgfpathlineto{\pgfqpoint{1.044168in}{0.702381in}}%
\pgfpathlineto{\pgfqpoint{1.039203in}{0.705265in}}%
\pgfpathlineto{\pgfqpoint{1.010911in}{0.721899in}}%
\pgfpathlineto{\pgfqpoint{1.006133in}{0.724743in}}%
\pgfpathlineto{\pgfqpoint{0.977655in}{0.741895in}}%
\pgfpathlineto{\pgfqpoint{0.973840in}{0.744221in}}%
\pgfpathlineto{\pgfqpoint{0.944399in}{0.762384in}}%
\pgfpathlineto{\pgfqpoint{0.942295in}{0.763698in}}%
\pgfpathlineto{\pgfqpoint{0.911478in}{0.783176in}}%
\pgfpathlineto{\pgfqpoint{0.911143in}{0.783391in}}%
\pgfpathlineto{\pgfqpoint{0.881409in}{0.802653in}}%
\pgfpathlineto{\pgfqpoint{0.877887in}{0.804964in}}%
\pgfpathlineto{\pgfqpoint{0.852023in}{0.822131in}}%
\pgfpathlineto{\pgfqpoint{0.844630in}{0.827101in}}%
\pgfpathlineto{\pgfqpoint{0.823299in}{0.841609in}}%
\pgfpathlineto{\pgfqpoint{0.811374in}{0.849824in}}%
\pgfpathlineto{\pgfqpoint{0.795212in}{0.861086in}}%
\pgfpathlineto{\pgfqpoint{0.778118in}{0.873156in}}%
\pgfpathlineto{\pgfqpoint{0.767743in}{0.880564in}}%
\pgfpathlineto{\pgfqpoint{0.744862in}{0.897121in}}%
\pgfpathlineto{\pgfqpoint{0.740870in}{0.900042in}}%
\pgfpathlineto{\pgfqpoint{0.714623in}{0.919519in}}%
\pgfpathlineto{\pgfqpoint{0.711606in}{0.921792in}}%
\pgfusepath{stroke}%
\end{pgfscope}%
\begin{pgfscope}%
\pgfpathrectangle{\pgfqpoint{0.711606in}{0.549444in}}{\pgfqpoint{4.955171in}{2.902168in}}%
\pgfusepath{clip}%
\pgfsetbuttcap%
\pgfsetroundjoin%
\pgfsetlinewidth{1.003750pt}%
\definecolor{currentstroke}{rgb}{0.410113,0.087896,0.433098}%
\pgfsetstrokecolor{currentstroke}%
\pgfsetdash{}{0pt}%
\pgfpathmoveto{\pgfqpoint{3.274226in}{3.451613in}}%
\pgfpathlineto{\pgfqpoint{3.372100in}{3.406270in}}%
\pgfpathlineto{\pgfqpoint{3.438613in}{3.373648in}}%
\pgfpathlineto{\pgfqpoint{3.538381in}{3.321589in}}%
\pgfpathlineto{\pgfqpoint{3.619654in}{3.276314in}}%
\pgfpathlineto{\pgfqpoint{3.704662in}{3.225928in}}%
\pgfpathlineto{\pgfqpoint{3.779250in}{3.178926in}}%
\pgfpathlineto{\pgfqpoint{3.837908in}{3.139970in}}%
\pgfpathlineto{\pgfqpoint{3.920690in}{3.081537in}}%
\pgfpathlineto{\pgfqpoint{3.972818in}{3.042582in}}%
\pgfpathlineto{\pgfqpoint{4.046473in}{2.984149in}}%
\pgfpathlineto{\pgfqpoint{4.115381in}{2.925716in}}%
\pgfpathlineto{\pgfqpoint{4.248251in}{2.808850in}}%
\pgfpathlineto{\pgfqpoint{4.273207in}{2.789373in}}%
\pgfpathlineto{\pgfqpoint{4.303273in}{2.768359in}}%
\pgfpathlineto{\pgfqpoint{4.336530in}{2.748294in}}%
\pgfpathlineto{\pgfqpoint{4.370209in}{2.730940in}}%
\pgfpathlineto{\pgfqpoint{4.415072in}{2.711462in}}%
\pgfpathlineto{\pgfqpoint{4.469554in}{2.691038in}}%
\pgfpathlineto{\pgfqpoint{4.702348in}{2.608467in}}%
\pgfpathlineto{\pgfqpoint{4.802116in}{2.569635in}}%
\pgfpathlineto{\pgfqpoint{4.901885in}{2.527946in}}%
\pgfpathlineto{\pgfqpoint{5.001653in}{2.483237in}}%
\pgfpathlineto{\pgfqpoint{5.094406in}{2.438775in}}%
\pgfpathlineto{\pgfqpoint{5.171010in}{2.399820in}}%
\pgfpathlineto{\pgfqpoint{5.267703in}{2.347491in}}%
\pgfpathlineto{\pgfqpoint{5.345907in}{2.302432in}}%
\pgfpathlineto{\pgfqpoint{5.410067in}{2.263477in}}%
\pgfpathlineto{\pgfqpoint{5.471263in}{2.224521in}}%
\pgfpathlineto{\pgfqpoint{5.533752in}{2.182742in}}%
\pgfpathlineto{\pgfqpoint{5.612228in}{2.127133in}}%
\pgfpathlineto{\pgfqpoint{5.666777in}{2.086253in}}%
\pgfpathlineto{\pgfqpoint{5.666777in}{2.086253in}}%
\pgfusepath{stroke}%
\end{pgfscope}%
\begin{pgfscope}%
\pgfpathrectangle{\pgfqpoint{0.711606in}{0.549444in}}{\pgfqpoint{4.955171in}{2.902168in}}%
\pgfusepath{clip}%
\pgfsetbuttcap%
\pgfsetroundjoin%
\pgfsetlinewidth{1.003750pt}%
\definecolor{currentstroke}{rgb}{0.434987,0.097069,0.432039}%
\pgfsetstrokecolor{currentstroke}%
\pgfsetdash{}{0pt}%
\pgfpathmoveto{\pgfqpoint{1.263310in}{0.549444in}}%
\pgfpathlineto{\pgfqpoint{1.243705in}{0.559082in}}%
\pgfpathlineto{\pgfqpoint{1.223948in}{0.568922in}}%
\pgfpathlineto{\pgfqpoint{1.210448in}{0.575719in}}%
\pgfpathlineto{\pgfqpoint{1.185591in}{0.588399in}}%
\pgfpathlineto{\pgfqpoint{1.177192in}{0.592731in}}%
\pgfpathlineto{\pgfqpoint{1.148198in}{0.607877in}}%
\pgfpathlineto{\pgfqpoint{1.143936in}{0.610128in}}%
\pgfpathlineto{\pgfqpoint{1.111735in}{0.627355in}}%
\pgfpathlineto{\pgfqpoint{1.110680in}{0.627925in}}%
\pgfpathlineto{\pgfqpoint{1.077424in}{0.646147in}}%
\pgfpathlineto{\pgfqpoint{1.076189in}{0.646832in}}%
\pgfpathlineto{\pgfqpoint{1.044168in}{0.664799in}}%
\pgfpathlineto{\pgfqpoint{1.041507in}{0.666310in}}%
\pgfpathlineto{\pgfqpoint{1.010911in}{0.683886in}}%
\pgfpathlineto{\pgfqpoint{1.007640in}{0.685788in}}%
\pgfpathlineto{\pgfqpoint{0.977655in}{0.703422in}}%
\pgfpathlineto{\pgfqpoint{0.974558in}{0.705265in}}%
\pgfpathlineto{\pgfqpoint{0.944399in}{0.723425in}}%
\pgfpathlineto{\pgfqpoint{0.942235in}{0.724743in}}%
\pgfpathlineto{\pgfqpoint{0.911143in}{0.743909in}}%
\pgfpathlineto{\pgfqpoint{0.910643in}{0.744221in}}%
\pgfpathlineto{\pgfqpoint{0.879792in}{0.763698in}}%
\pgfpathlineto{\pgfqpoint{0.877887in}{0.764916in}}%
\pgfpathlineto{\pgfqpoint{0.849642in}{0.783176in}}%
\pgfpathlineto{\pgfqpoint{0.844630in}{0.786456in}}%
\pgfpathlineto{\pgfqpoint{0.820164in}{0.802653in}}%
\pgfpathlineto{\pgfqpoint{0.811374in}{0.808545in}}%
\pgfpathlineto{\pgfqpoint{0.791333in}{0.822131in}}%
\pgfpathlineto{\pgfqpoint{0.778118in}{0.831203in}}%
\pgfpathlineto{\pgfqpoint{0.763129in}{0.841609in}}%
\pgfpathlineto{\pgfqpoint{0.744862in}{0.854453in}}%
\pgfpathlineto{\pgfqpoint{0.735530in}{0.861086in}}%
\pgfpathlineto{\pgfqpoint{0.711606in}{0.878316in}}%
\pgfusepath{stroke}%
\end{pgfscope}%
\begin{pgfscope}%
\pgfpathrectangle{\pgfqpoint{0.711606in}{0.549444in}}{\pgfqpoint{4.955171in}{2.902168in}}%
\pgfusepath{clip}%
\pgfsetbuttcap%
\pgfsetroundjoin%
\pgfsetlinewidth{1.003750pt}%
\definecolor{currentstroke}{rgb}{0.434987,0.097069,0.432039}%
\pgfsetstrokecolor{currentstroke}%
\pgfsetdash{}{0pt}%
\pgfpathmoveto{\pgfqpoint{3.352368in}{3.451613in}}%
\pgfpathlineto{\pgfqpoint{3.438613in}{3.410358in}}%
\pgfpathlineto{\pgfqpoint{3.538381in}{3.359375in}}%
\pgfpathlineto{\pgfqpoint{3.619355in}{3.315269in}}%
\pgfpathlineto{\pgfqpoint{3.704662in}{3.265939in}}%
\pgfpathlineto{\pgfqpoint{3.782890in}{3.217881in}}%
\pgfpathlineto{\pgfqpoint{3.843056in}{3.178926in}}%
\pgfpathlineto{\pgfqpoint{3.904199in}{3.137371in}}%
\pgfpathlineto{\pgfqpoint{3.981723in}{3.081537in}}%
\pgfpathlineto{\pgfqpoint{4.037224in}{3.039227in}}%
\pgfpathlineto{\pgfqpoint{4.105539in}{2.984149in}}%
\pgfpathlineto{\pgfqpoint{4.173862in}{2.925716in}}%
\pgfpathlineto{\pgfqpoint{4.270017in}{2.842528in}}%
\pgfpathlineto{\pgfqpoint{4.314355in}{2.808850in}}%
\pgfpathlineto{\pgfqpoint{4.343418in}{2.789373in}}%
\pgfpathlineto{\pgfqpoint{4.376161in}{2.769895in}}%
\pgfpathlineto{\pgfqpoint{4.413827in}{2.750418in}}%
\pgfpathlineto{\pgfqpoint{4.457795in}{2.730940in}}%
\pgfpathlineto{\pgfqpoint{4.508118in}{2.711462in}}%
\pgfpathlineto{\pgfqpoint{4.774732in}{2.614074in}}%
\pgfpathlineto{\pgfqpoint{4.871243in}{2.575119in}}%
\pgfpathlineto{\pgfqpoint{4.968397in}{2.533178in}}%
\pgfpathlineto{\pgfqpoint{5.068166in}{2.487094in}}%
\pgfpathlineto{\pgfqpoint{5.134678in}{2.454610in}}%
\pgfpathlineto{\pgfqpoint{5.203693in}{2.419298in}}%
\pgfpathlineto{\pgfqpoint{5.300959in}{2.366535in}}%
\pgfpathlineto{\pgfqpoint{5.378352in}{2.321909in}}%
\pgfpathlineto{\pgfqpoint{5.442562in}{2.282954in}}%
\pgfpathlineto{\pgfqpoint{5.503874in}{2.243999in}}%
\pgfpathlineto{\pgfqpoint{5.567008in}{2.201903in}}%
\pgfpathlineto{\pgfqpoint{5.645369in}{2.146611in}}%
\pgfpathlineto{\pgfqpoint{5.666777in}{2.130891in}}%
\pgfpathlineto{\pgfqpoint{5.666777in}{2.130891in}}%
\pgfusepath{stroke}%
\end{pgfscope}%
\begin{pgfscope}%
\pgfpathrectangle{\pgfqpoint{0.711606in}{0.549444in}}{\pgfqpoint{4.955171in}{2.902168in}}%
\pgfusepath{clip}%
\pgfsetbuttcap%
\pgfsetroundjoin%
\pgfsetlinewidth{1.003750pt}%
\definecolor{currentstroke}{rgb}{0.453651,0.103848,0.430498}%
\pgfsetstrokecolor{currentstroke}%
\pgfsetdash{}{0pt}%
\pgfpathmoveto{\pgfqpoint{1.193647in}{0.549444in}}%
\pgfpathlineto{\pgfqpoint{1.177192in}{0.557762in}}%
\pgfpathlineto{\pgfqpoint{1.155395in}{0.568922in}}%
\pgfpathlineto{\pgfqpoint{1.143936in}{0.574851in}}%
\pgfpathlineto{\pgfqpoint{1.118081in}{0.588399in}}%
\pgfpathlineto{\pgfqpoint{1.110680in}{0.592320in}}%
\pgfpathlineto{\pgfqpoint{1.081672in}{0.607877in}}%
\pgfpathlineto{\pgfqpoint{1.077424in}{0.610180in}}%
\pgfpathlineto{\pgfqpoint{1.046134in}{0.627355in}}%
\pgfpathlineto{\pgfqpoint{1.044168in}{0.628446in}}%
\pgfpathlineto{\pgfqpoint{1.011435in}{0.646832in}}%
\pgfpathlineto{\pgfqpoint{1.010911in}{0.647130in}}%
\pgfpathlineto{\pgfqpoint{0.977655in}{0.666248in}}%
\pgfpathlineto{\pgfqpoint{0.977548in}{0.666310in}}%
\pgfpathlineto{\pgfqpoint{0.944439in}{0.685788in}}%
\pgfpathlineto{\pgfqpoint{0.944399in}{0.685811in}}%
\pgfpathlineto{\pgfqpoint{0.912086in}{0.705265in}}%
\pgfpathlineto{\pgfqpoint{0.911143in}{0.705840in}}%
\pgfpathlineto{\pgfqpoint{0.880461in}{0.724743in}}%
\pgfpathlineto{\pgfqpoint{0.877887in}{0.726348in}}%
\pgfpathlineto{\pgfqpoint{0.849540in}{0.744221in}}%
\pgfpathlineto{\pgfqpoint{0.844630in}{0.747353in}}%
\pgfpathlineto{\pgfqpoint{0.819300in}{0.763698in}}%
\pgfpathlineto{\pgfqpoint{0.811374in}{0.768874in}}%
\pgfpathlineto{\pgfqpoint{0.789716in}{0.783176in}}%
\pgfpathlineto{\pgfqpoint{0.778118in}{0.790928in}}%
\pgfpathlineto{\pgfqpoint{0.760768in}{0.802653in}}%
\pgfpathlineto{\pgfqpoint{0.744862in}{0.813537in}}%
\pgfpathlineto{\pgfqpoint{0.732436in}{0.822131in}}%
\pgfpathlineto{\pgfqpoint{0.711606in}{0.836719in}}%
\pgfusepath{stroke}%
\end{pgfscope}%
\begin{pgfscope}%
\pgfpathrectangle{\pgfqpoint{0.711606in}{0.549444in}}{\pgfqpoint{4.955171in}{2.902168in}}%
\pgfusepath{clip}%
\pgfsetbuttcap%
\pgfsetroundjoin%
\pgfsetlinewidth{1.003750pt}%
\definecolor{currentstroke}{rgb}{0.453651,0.103848,0.430498}%
\pgfsetstrokecolor{currentstroke}%
\pgfsetdash{}{0pt}%
\pgfpathmoveto{\pgfqpoint{3.426816in}{3.451613in}}%
\pgfpathlineto{\pgfqpoint{3.505791in}{3.412657in}}%
\pgfpathlineto{\pgfqpoint{3.604894in}{3.360611in}}%
\pgfpathlineto{\pgfqpoint{3.686028in}{3.315269in}}%
\pgfpathlineto{\pgfqpoint{3.771175in}{3.264792in}}%
\pgfpathlineto{\pgfqpoint{3.845821in}{3.217881in}}%
\pgfpathlineto{\pgfqpoint{3.904786in}{3.178926in}}%
\pgfpathlineto{\pgfqpoint{3.988253in}{3.120493in}}%
\pgfpathlineto{\pgfqpoint{4.040947in}{3.081537in}}%
\pgfpathlineto{\pgfqpoint{4.115669in}{3.023104in}}%
\pgfpathlineto{\pgfqpoint{4.185983in}{2.964672in}}%
\pgfpathlineto{\pgfqpoint{4.303273in}{2.865368in}}%
\pgfpathlineto{\pgfqpoint{4.336530in}{2.840366in}}%
\pgfpathlineto{\pgfqpoint{4.369786in}{2.817823in}}%
\pgfpathlineto{\pgfqpoint{4.403042in}{2.797532in}}%
\pgfpathlineto{\pgfqpoint{4.436298in}{2.779296in}}%
\pgfpathlineto{\pgfqpoint{4.469554in}{2.762989in}}%
\pgfpathlineto{\pgfqpoint{4.502811in}{2.748342in}}%
\pgfpathlineto{\pgfqpoint{4.569323in}{2.722478in}}%
\pgfpathlineto{\pgfqpoint{4.768860in}{2.648975in}}%
\pgfpathlineto{\pgfqpoint{4.868629in}{2.609295in}}%
\pgfpathlineto{\pgfqpoint{4.968397in}{2.566868in}}%
\pgfpathlineto{\pgfqpoint{5.068166in}{2.521530in}}%
\pgfpathlineto{\pgfqpoint{5.158601in}{2.477731in}}%
\pgfpathlineto{\pgfqpoint{5.234745in}{2.438775in}}%
\pgfpathlineto{\pgfqpoint{5.334215in}{2.384697in}}%
\pgfpathlineto{\pgfqpoint{5.409272in}{2.341387in}}%
\pgfpathlineto{\pgfqpoint{5.473564in}{2.302432in}}%
\pgfpathlineto{\pgfqpoint{5.535021in}{2.263477in}}%
\pgfpathlineto{\pgfqpoint{5.622108in}{2.205044in}}%
\pgfpathlineto{\pgfqpoint{5.666777in}{2.173517in}}%
\pgfpathlineto{\pgfqpoint{5.666777in}{2.173517in}}%
\pgfusepath{stroke}%
\end{pgfscope}%
\begin{pgfscope}%
\pgfpathrectangle{\pgfqpoint{0.711606in}{0.549444in}}{\pgfqpoint{4.955171in}{2.902168in}}%
\pgfusepath{clip}%
\pgfsetbuttcap%
\pgfsetroundjoin%
\pgfsetlinewidth{1.003750pt}%
\definecolor{currentstroke}{rgb}{0.478558,0.112764,0.427475}%
\pgfsetstrokecolor{currentstroke}%
\pgfsetdash{}{0pt}%
\pgfpathmoveto{\pgfqpoint{1.126683in}{0.549444in}}%
\pgfpathlineto{\pgfqpoint{1.110680in}{0.557753in}}%
\pgfpathlineto{\pgfqpoint{1.089427in}{0.568922in}}%
\pgfpathlineto{\pgfqpoint{1.077424in}{0.575297in}}%
\pgfpathlineto{\pgfqpoint{1.053052in}{0.588399in}}%
\pgfpathlineto{\pgfqpoint{1.044168in}{0.593227in}}%
\pgfpathlineto{\pgfqpoint{1.017526in}{0.607877in}}%
\pgfpathlineto{\pgfqpoint{1.010911in}{0.611554in}}%
\pgfpathlineto{\pgfqpoint{0.982819in}{0.627355in}}%
\pgfpathlineto{\pgfqpoint{0.977655in}{0.630291in}}%
\pgfpathlineto{\pgfqpoint{0.948904in}{0.646832in}}%
\pgfpathlineto{\pgfqpoint{0.944399in}{0.649453in}}%
\pgfpathlineto{\pgfqpoint{0.915752in}{0.666310in}}%
\pgfpathlineto{\pgfqpoint{0.911143in}{0.669053in}}%
\pgfpathlineto{\pgfqpoint{0.883339in}{0.685788in}}%
\pgfpathlineto{\pgfqpoint{0.877887in}{0.689107in}}%
\pgfpathlineto{\pgfqpoint{0.851639in}{0.705265in}}%
\pgfpathlineto{\pgfqpoint{0.844630in}{0.709630in}}%
\pgfpathlineto{\pgfqpoint{0.820629in}{0.724743in}}%
\pgfpathlineto{\pgfqpoint{0.811374in}{0.730638in}}%
\pgfpathlineto{\pgfqpoint{0.790285in}{0.744221in}}%
\pgfpathlineto{\pgfqpoint{0.778118in}{0.752149in}}%
\pgfpathlineto{\pgfqpoint{0.760587in}{0.763698in}}%
\pgfpathlineto{\pgfqpoint{0.744862in}{0.774181in}}%
\pgfpathlineto{\pgfqpoint{0.731514in}{0.783176in}}%
\pgfpathlineto{\pgfqpoint{0.711606in}{0.796753in}}%
\pgfusepath{stroke}%
\end{pgfscope}%
\begin{pgfscope}%
\pgfpathrectangle{\pgfqpoint{0.711606in}{0.549444in}}{\pgfqpoint{4.955171in}{2.902168in}}%
\pgfusepath{clip}%
\pgfsetbuttcap%
\pgfsetroundjoin%
\pgfsetlinewidth{1.003750pt}%
\definecolor{currentstroke}{rgb}{0.478558,0.112764,0.427475}%
\pgfsetstrokecolor{currentstroke}%
\pgfsetdash{}{0pt}%
\pgfpathmoveto{\pgfqpoint{3.498006in}{3.451613in}}%
\pgfpathlineto{\pgfqpoint{3.574640in}{3.412657in}}%
\pgfpathlineto{\pgfqpoint{3.671406in}{3.360462in}}%
\pgfpathlineto{\pgfqpoint{3.750325in}{3.315269in}}%
\pgfpathlineto{\pgfqpoint{3.837687in}{3.262237in}}%
\pgfpathlineto{\pgfqpoint{3.906758in}{3.217881in}}%
\pgfpathlineto{\pgfqpoint{3.992487in}{3.159448in}}%
\pgfpathlineto{\pgfqpoint{4.046615in}{3.120493in}}%
\pgfpathlineto{\pgfqpoint{4.123520in}{3.062060in}}%
\pgfpathlineto{\pgfqpoint{4.195855in}{3.003627in}}%
\pgfpathlineto{\pgfqpoint{4.336783in}{2.886761in}}%
\pgfpathlineto{\pgfqpoint{4.369786in}{2.862713in}}%
\pgfpathlineto{\pgfqpoint{4.403042in}{2.840647in}}%
\pgfpathlineto{\pgfqpoint{4.456658in}{2.808850in}}%
\pgfpathlineto{\pgfqpoint{4.502811in}{2.784960in}}%
\pgfpathlineto{\pgfqpoint{4.536067in}{2.769657in}}%
\pgfpathlineto{\pgfqpoint{4.602579in}{2.742813in}}%
\pgfpathlineto{\pgfqpoint{4.840141in}{2.653029in}}%
\pgfpathlineto{\pgfqpoint{4.935141in}{2.613981in}}%
\pgfpathlineto{\pgfqpoint{5.034909in}{2.570228in}}%
\pgfpathlineto{\pgfqpoint{5.134678in}{2.523552in}}%
\pgfpathlineto{\pgfqpoint{5.226705in}{2.477731in}}%
\pgfpathlineto{\pgfqpoint{5.300959in}{2.438707in}}%
\pgfpathlineto{\pgfqpoint{5.400728in}{2.383081in}}%
\pgfpathlineto{\pgfqpoint{5.471333in}{2.341387in}}%
\pgfpathlineto{\pgfqpoint{5.534324in}{2.302432in}}%
\pgfpathlineto{\pgfqpoint{5.623669in}{2.243999in}}%
\pgfpathlineto{\pgfqpoint{5.666777in}{2.214394in}}%
\pgfpathlineto{\pgfqpoint{5.666777in}{2.214394in}}%
\pgfusepath{stroke}%
\end{pgfscope}%
\begin{pgfscope}%
\pgfpathrectangle{\pgfqpoint{0.711606in}{0.549444in}}{\pgfqpoint{4.955171in}{2.902168in}}%
\pgfusepath{clip}%
\pgfsetbuttcap%
\pgfsetroundjoin%
\pgfsetlinewidth{1.003750pt}%
\definecolor{currentstroke}{rgb}{0.503493,0.121575,0.423356}%
\pgfsetstrokecolor{currentstroke}%
\pgfsetdash{}{0pt}%
\pgfpathmoveto{\pgfqpoint{1.062131in}{0.549444in}}%
\pgfpathlineto{\pgfqpoint{1.044168in}{0.559010in}}%
\pgfpathlineto{\pgfqpoint{1.025772in}{0.568922in}}%
\pgfpathlineto{\pgfqpoint{1.010911in}{0.577013in}}%
\pgfpathlineto{\pgfqpoint{0.990242in}{0.588399in}}%
\pgfpathlineto{\pgfqpoint{0.977655in}{0.595407in}}%
\pgfpathlineto{\pgfqpoint{0.955513in}{0.607877in}}%
\pgfpathlineto{\pgfqpoint{0.944399in}{0.614204in}}%
\pgfpathlineto{\pgfqpoint{0.921557in}{0.627355in}}%
\pgfpathlineto{\pgfqpoint{0.911143in}{0.633416in}}%
\pgfpathlineto{\pgfqpoint{0.888349in}{0.646832in}}%
\pgfpathlineto{\pgfqpoint{0.877887in}{0.653059in}}%
\pgfpathlineto{\pgfqpoint{0.855864in}{0.666310in}}%
\pgfpathlineto{\pgfqpoint{0.844630in}{0.673145in}}%
\pgfpathlineto{\pgfqpoint{0.824078in}{0.685788in}}%
\pgfpathlineto{\pgfqpoint{0.811374in}{0.693690in}}%
\pgfpathlineto{\pgfqpoint{0.792969in}{0.705265in}}%
\pgfpathlineto{\pgfqpoint{0.778118in}{0.714711in}}%
\pgfpathlineto{\pgfqpoint{0.762514in}{0.724743in}}%
\pgfpathlineto{\pgfqpoint{0.744862in}{0.736222in}}%
\pgfpathlineto{\pgfqpoint{0.732693in}{0.744221in}}%
\pgfpathlineto{\pgfqpoint{0.711606in}{0.758242in}}%
\pgfusepath{stroke}%
\end{pgfscope}%
\begin{pgfscope}%
\pgfpathrectangle{\pgfqpoint{0.711606in}{0.549444in}}{\pgfqpoint{4.955171in}{2.902168in}}%
\pgfusepath{clip}%
\pgfsetbuttcap%
\pgfsetroundjoin%
\pgfsetlinewidth{1.003750pt}%
\definecolor{currentstroke}{rgb}{0.503493,0.121575,0.423356}%
\pgfsetstrokecolor{currentstroke}%
\pgfsetdash{}{0pt}%
\pgfpathmoveto{\pgfqpoint{3.566312in}{3.451613in}}%
\pgfpathlineto{\pgfqpoint{3.571637in}{3.448896in}}%
\pgfpathlineto{\pgfqpoint{3.604087in}{3.432135in}}%
\pgfpathlineto{\pgfqpoint{3.604894in}{3.431714in}}%
\pgfpathlineto{\pgfqpoint{3.638150in}{3.414128in}}%
\pgfpathlineto{\pgfqpoint{3.640896in}{3.412657in}}%
\pgfpathlineto{\pgfqpoint{3.671406in}{3.396148in}}%
\pgfpathlineto{\pgfqpoint{3.676826in}{3.393180in}}%
\pgfpathlineto{\pgfqpoint{3.704662in}{3.377769in}}%
\pgfpathlineto{\pgfqpoint{3.711921in}{3.373702in}}%
\pgfpathlineto{\pgfqpoint{3.737918in}{3.358977in}}%
\pgfpathlineto{\pgfqpoint{3.746212in}{3.354224in}}%
\pgfpathlineto{\pgfqpoint{3.771175in}{3.339760in}}%
\pgfpathlineto{\pgfqpoint{3.779726in}{3.334747in}}%
\pgfpathlineto{\pgfqpoint{3.804431in}{3.320101in}}%
\pgfpathlineto{\pgfqpoint{3.812489in}{3.315269in}}%
\pgfpathlineto{\pgfqpoint{3.837687in}{3.299987in}}%
\pgfpathlineto{\pgfqpoint{3.844527in}{3.295791in}}%
\pgfpathlineto{\pgfqpoint{3.870943in}{3.279402in}}%
\pgfpathlineto{\pgfqpoint{3.875865in}{3.276314in}}%
\pgfpathlineto{\pgfqpoint{3.904199in}{3.258328in}}%
\pgfpathlineto{\pgfqpoint{3.906524in}{3.256836in}}%
\pgfpathlineto{\pgfqpoint{3.936512in}{3.237359in}}%
\pgfpathlineto{\pgfqpoint{3.937455in}{3.236738in}}%
\pgfpathlineto{\pgfqpoint{3.965818in}{3.217881in}}%
\pgfpathlineto{\pgfqpoint{3.970712in}{3.214587in}}%
\pgfpathlineto{\pgfqpoint{3.994501in}{3.198403in}}%
\pgfpathlineto{\pgfqpoint{4.003968in}{3.191883in}}%
\pgfpathlineto{\pgfqpoint{4.022582in}{3.178926in}}%
\pgfpathlineto{\pgfqpoint{4.037224in}{3.168606in}}%
\pgfpathlineto{\pgfqpoint{4.050081in}{3.159448in}}%
\pgfpathlineto{\pgfqpoint{4.070480in}{3.144735in}}%
\pgfpathlineto{\pgfqpoint{4.077018in}{3.139970in}}%
\pgfpathlineto{\pgfqpoint{4.103407in}{3.120493in}}%
\pgfpathlineto{\pgfqpoint{4.103736in}{3.120246in}}%
\pgfpathlineto{\pgfqpoint{4.129166in}{3.101015in}}%
\pgfpathlineto{\pgfqpoint{4.136993in}{3.095020in}}%
\pgfpathlineto{\pgfqpoint{4.154423in}{3.081537in}}%
\pgfpathlineto{\pgfqpoint{4.170249in}{3.069141in}}%
\pgfpathlineto{\pgfqpoint{4.179208in}{3.062060in}}%
\pgfpathlineto{\pgfqpoint{4.203505in}{3.042623in}}%
\pgfpathlineto{\pgfqpoint{4.203556in}{3.042582in}}%
\pgfpathlineto{\pgfqpoint{4.227399in}{3.023104in}}%
\pgfpathlineto{\pgfqpoint{4.236761in}{3.015378in}}%
\pgfpathlineto{\pgfqpoint{4.250951in}{3.003627in}}%
\pgfpathlineto{\pgfqpoint{4.270017in}{2.987731in}}%
\pgfpathlineto{\pgfqpoint{4.274325in}{2.984149in}}%
\pgfpathlineto{\pgfqpoint{4.297673in}{2.964672in}}%
\pgfpathlineto{\pgfqpoint{4.303273in}{2.960002in}}%
\pgfpathlineto{\pgfqpoint{4.321356in}{2.945194in}}%
\pgfpathlineto{\pgfqpoint{4.336530in}{2.932939in}}%
\pgfpathlineto{\pgfqpoint{4.345779in}{2.925716in}}%
\pgfpathlineto{\pgfqpoint{4.369786in}{2.907430in}}%
\pgfpathlineto{\pgfqpoint{4.371426in}{2.906239in}}%
\pgfpathlineto{\pgfqpoint{4.398938in}{2.886761in}}%
\pgfpathlineto{\pgfqpoint{4.403042in}{2.883941in}}%
\pgfpathlineto{\pgfqpoint{4.428524in}{2.867283in}}%
\pgfpathlineto{\pgfqpoint{4.436298in}{2.862348in}}%
\pgfpathlineto{\pgfqpoint{4.460167in}{2.847806in}}%
\pgfpathlineto{\pgfqpoint{4.469554in}{2.842256in}}%
\pgfpathlineto{\pgfqpoint{4.493953in}{2.828328in}}%
\pgfpathlineto{\pgfqpoint{4.502811in}{2.823471in}}%
\pgfpathlineto{\pgfqpoint{4.530515in}{2.808850in}}%
\pgfpathlineto{\pgfqpoint{4.536067in}{2.806080in}}%
\pgfpathlineto{\pgfqpoint{4.569323in}{2.790213in}}%
\pgfpathlineto{\pgfqpoint{4.571157in}{2.789373in}}%
\pgfpathlineto{\pgfqpoint{4.602579in}{2.775813in}}%
\pgfpathlineto{\pgfqpoint{4.616883in}{2.769895in}}%
\pgfpathlineto{\pgfqpoint{4.635835in}{2.762388in}}%
\pgfpathlineto{\pgfqpoint{4.666838in}{2.750418in}}%
\pgfpathlineto{\pgfqpoint{4.669091in}{2.749567in}}%
\pgfpathlineto{\pgfqpoint{4.702348in}{2.737038in}}%
\pgfpathlineto{\pgfqpoint{4.718565in}{2.730940in}}%
\pgfpathlineto{\pgfqpoint{4.735604in}{2.724554in}}%
\pgfpathlineto{\pgfqpoint{4.768860in}{2.712006in}}%
\pgfpathlineto{\pgfqpoint{4.770283in}{2.711462in}}%
\pgfpathlineto{\pgfqpoint{4.802116in}{2.699238in}}%
\pgfpathlineto{\pgfqpoint{4.820782in}{2.691985in}}%
\pgfpathlineto{\pgfqpoint{4.835372in}{2.686272in}}%
\pgfpathlineto{\pgfqpoint{4.868629in}{2.673073in}}%
\pgfpathlineto{\pgfqpoint{4.870033in}{2.672507in}}%
\pgfpathlineto{\pgfqpoint{4.901885in}{2.659549in}}%
\pgfpathlineto{\pgfqpoint{4.917690in}{2.653029in}}%
\pgfpathlineto{\pgfqpoint{4.935141in}{2.645764in}}%
\pgfpathlineto{\pgfqpoint{4.964067in}{2.633552in}}%
\pgfpathlineto{\pgfqpoint{4.968397in}{2.631706in}}%
\pgfpathlineto{\pgfqpoint{5.001653in}{2.617317in}}%
\pgfpathlineto{\pgfqpoint{5.009044in}{2.614074in}}%
\pgfpathlineto{\pgfqpoint{5.034909in}{2.602612in}}%
\pgfpathlineto{\pgfqpoint{5.052757in}{2.594596in}}%
\pgfpathlineto{\pgfqpoint{5.068166in}{2.587607in}}%
\pgfpathlineto{\pgfqpoint{5.095339in}{2.575119in}}%
\pgfpathlineto{\pgfqpoint{5.101422in}{2.572295in}}%
\pgfpathlineto{\pgfqpoint{5.134678in}{2.556649in}}%
\pgfpathlineto{\pgfqpoint{5.136791in}{2.555641in}}%
\pgfpathlineto{\pgfqpoint{5.167934in}{2.540637in}}%
\pgfpathlineto{\pgfqpoint{5.177102in}{2.536163in}}%
\pgfpathlineto{\pgfqpoint{5.201190in}{2.524290in}}%
\pgfpathlineto{\pgfqpoint{5.216428in}{2.516686in}}%
\pgfpathlineto{\pgfqpoint{5.234447in}{2.507600in}}%
\pgfpathlineto{\pgfqpoint{5.254804in}{2.497208in}}%
\pgfpathlineto{\pgfqpoint{5.267703in}{2.490555in}}%
\pgfpathlineto{\pgfqpoint{5.292266in}{2.477731in}}%
\pgfpathlineto{\pgfqpoint{5.300959in}{2.473144in}}%
\pgfpathlineto{\pgfqpoint{5.328845in}{2.458253in}}%
\pgfpathlineto{\pgfqpoint{5.334215in}{2.455355in}}%
\pgfpathlineto{\pgfqpoint{5.364574in}{2.438775in}}%
\pgfpathlineto{\pgfqpoint{5.367471in}{2.437176in}}%
\pgfpathlineto{\pgfqpoint{5.399482in}{2.419298in}}%
\pgfpathlineto{\pgfqpoint{5.400728in}{2.418594in}}%
\pgfpathlineto{\pgfqpoint{5.433597in}{2.399820in}}%
\pgfpathlineto{\pgfqpoint{5.433984in}{2.399597in}}%
\pgfpathlineto{\pgfqpoint{5.466946in}{2.380342in}}%
\pgfpathlineto{\pgfqpoint{5.467240in}{2.380169in}}%
\pgfpathlineto{\pgfqpoint{5.499556in}{2.360865in}}%
\pgfpathlineto{\pgfqpoint{5.500496in}{2.360297in}}%
\pgfpathlineto{\pgfqpoint{5.531451in}{2.341387in}}%
\pgfpathlineto{\pgfqpoint{5.533752in}{2.339965in}}%
\pgfpathlineto{\pgfqpoint{5.562654in}{2.321909in}}%
\pgfpathlineto{\pgfqpoint{5.567008in}{2.319158in}}%
\pgfpathlineto{\pgfqpoint{5.593189in}{2.302432in}}%
\pgfpathlineto{\pgfqpoint{5.600265in}{2.297859in}}%
\pgfpathlineto{\pgfqpoint{5.623077in}{2.282954in}}%
\pgfpathlineto{\pgfqpoint{5.633521in}{2.276050in}}%
\pgfpathlineto{\pgfqpoint{5.652338in}{2.263477in}}%
\pgfpathlineto{\pgfqpoint{5.666777in}{2.253714in}}%
\pgfusepath{stroke}%
\end{pgfscope}%
\begin{pgfscope}%
\pgfpathrectangle{\pgfqpoint{0.711606in}{0.549444in}}{\pgfqpoint{4.955171in}{2.902168in}}%
\pgfusepath{clip}%
\pgfsetbuttcap%
\pgfsetroundjoin%
\pgfsetlinewidth{1.003750pt}%
\definecolor{currentstroke}{rgb}{0.522206,0.128150,0.419549}%
\pgfsetstrokecolor{currentstroke}%
\pgfsetdash{}{0pt}%
\pgfpathmoveto{\pgfqpoint{0.999740in}{0.549444in}}%
\pgfpathlineto{\pgfqpoint{0.977655in}{0.561493in}}%
\pgfpathlineto{\pgfqpoint{0.964192in}{0.568922in}}%
\pgfpathlineto{\pgfqpoint{0.944399in}{0.579958in}}%
\pgfpathlineto{\pgfqpoint{0.929427in}{0.588399in}}%
\pgfpathlineto{\pgfqpoint{0.911143in}{0.598818in}}%
\pgfpathlineto{\pgfqpoint{0.895420in}{0.607877in}}%
\pgfpathlineto{\pgfqpoint{0.877887in}{0.618087in}}%
\pgfpathlineto{\pgfqpoint{0.862145in}{0.627355in}}%
\pgfpathlineto{\pgfqpoint{0.844630in}{0.637777in}}%
\pgfpathlineto{\pgfqpoint{0.829578in}{0.646832in}}%
\pgfpathlineto{\pgfqpoint{0.811374in}{0.657903in}}%
\pgfpathlineto{\pgfqpoint{0.797697in}{0.666310in}}%
\pgfpathlineto{\pgfqpoint{0.778118in}{0.678478in}}%
\pgfpathlineto{\pgfqpoint{0.766480in}{0.685788in}}%
\pgfpathlineto{\pgfqpoint{0.744862in}{0.699517in}}%
\pgfpathlineto{\pgfqpoint{0.735906in}{0.705265in}}%
\pgfpathlineto{\pgfqpoint{0.711606in}{0.721037in}}%
\pgfusepath{stroke}%
\end{pgfscope}%
\begin{pgfscope}%
\pgfpathrectangle{\pgfqpoint{0.711606in}{0.549444in}}{\pgfqpoint{4.955171in}{2.902168in}}%
\pgfusepath{clip}%
\pgfsetbuttcap%
\pgfsetroundjoin%
\pgfsetlinewidth{1.003750pt}%
\definecolor{currentstroke}{rgb}{0.522206,0.128150,0.419549}%
\pgfsetstrokecolor{currentstroke}%
\pgfsetdash{}{0pt}%
\pgfpathmoveto{\pgfqpoint{3.632050in}{3.451613in}}%
\pgfpathlineto{\pgfqpoint{3.638150in}{3.448418in}}%
\pgfpathlineto{\pgfqpoint{3.668873in}{3.432135in}}%
\pgfpathlineto{\pgfqpoint{3.671406in}{3.430778in}}%
\pgfpathlineto{\pgfqpoint{3.704662in}{3.412751in}}%
\pgfpathlineto{\pgfqpoint{3.704834in}{3.412657in}}%
\pgfpathlineto{\pgfqpoint{3.737918in}{3.394311in}}%
\pgfpathlineto{\pgfqpoint{3.739935in}{3.393180in}}%
\pgfpathlineto{\pgfqpoint{3.771175in}{3.375466in}}%
\pgfpathlineto{\pgfqpoint{3.774249in}{3.373702in}}%
\pgfpathlineto{\pgfqpoint{3.804431in}{3.356202in}}%
\pgfpathlineto{\pgfqpoint{3.807804in}{3.354224in}}%
\pgfpathlineto{\pgfqpoint{3.837687in}{3.336508in}}%
\pgfpathlineto{\pgfqpoint{3.840624in}{3.334747in}}%
\pgfpathlineto{\pgfqpoint{3.870943in}{3.316367in}}%
\pgfpathlineto{\pgfqpoint{3.872734in}{3.315269in}}%
\pgfpathlineto{\pgfqpoint{3.904156in}{3.295791in}}%
\pgfpathlineto{\pgfqpoint{3.904199in}{3.295764in}}%
\pgfpathlineto{\pgfqpoint{3.934872in}{3.276314in}}%
\pgfpathlineto{\pgfqpoint{3.937455in}{3.274657in}}%
\pgfpathlineto{\pgfqpoint{3.964936in}{3.256836in}}%
\pgfpathlineto{\pgfqpoint{3.970712in}{3.253047in}}%
\pgfpathlineto{\pgfqpoint{3.994367in}{3.237359in}}%
\pgfpathlineto{\pgfqpoint{4.003968in}{3.230916in}}%
\pgfpathlineto{\pgfqpoint{4.023187in}{3.217881in}}%
\pgfpathlineto{\pgfqpoint{4.037224in}{3.208246in}}%
\pgfpathlineto{\pgfqpoint{4.051414in}{3.198403in}}%
\pgfpathlineto{\pgfqpoint{4.070480in}{3.185018in}}%
\pgfpathlineto{\pgfqpoint{4.079070in}{3.178926in}}%
\pgfpathlineto{\pgfqpoint{4.103736in}{3.161214in}}%
\pgfpathlineto{\pgfqpoint{4.106171in}{3.159448in}}%
\pgfpathlineto{\pgfqpoint{4.132674in}{3.139970in}}%
\pgfpathlineto{\pgfqpoint{4.136993in}{3.136755in}}%
\pgfpathlineto{\pgfqpoint{4.158621in}{3.120493in}}%
\pgfpathlineto{\pgfqpoint{4.170249in}{3.111641in}}%
\pgfpathlineto{\pgfqpoint{4.184076in}{3.101015in}}%
\pgfpathlineto{\pgfqpoint{4.203505in}{3.085902in}}%
\pgfpathlineto{\pgfqpoint{4.209069in}{3.081537in}}%
\pgfpathlineto{\pgfqpoint{4.233596in}{3.062060in}}%
\pgfpathlineto{\pgfqpoint{4.236761in}{3.059515in}}%
\pgfpathlineto{\pgfqpoint{4.257692in}{3.042582in}}%
\pgfpathlineto{\pgfqpoint{4.270017in}{3.032520in}}%
\pgfpathlineto{\pgfqpoint{4.281526in}{3.023104in}}%
\pgfpathlineto{\pgfqpoint{4.303273in}{3.005223in}}%
\pgfpathlineto{\pgfqpoint{4.305225in}{3.003627in}}%
\pgfpathlineto{\pgfqpoint{4.328981in}{2.984149in}}%
\pgfpathlineto{\pgfqpoint{4.336530in}{2.977987in}}%
\pgfpathlineto{\pgfqpoint{4.353200in}{2.964672in}}%
\pgfpathlineto{\pgfqpoint{4.369786in}{2.951640in}}%
\pgfpathlineto{\pgfqpoint{4.378301in}{2.945194in}}%
\pgfpathlineto{\pgfqpoint{4.403042in}{2.926944in}}%
\pgfpathlineto{\pgfqpoint{4.404788in}{2.925716in}}%
\pgfpathlineto{\pgfqpoint{4.433195in}{2.906239in}}%
\pgfpathlineto{\pgfqpoint{4.436298in}{2.904166in}}%
\pgfpathlineto{\pgfqpoint{4.463556in}{2.886761in}}%
\pgfpathlineto{\pgfqpoint{4.469554in}{2.883022in}}%
\pgfpathlineto{\pgfqpoint{4.495663in}{2.867283in}}%
\pgfpathlineto{\pgfqpoint{4.502811in}{2.863075in}}%
\pgfpathlineto{\pgfqpoint{4.529463in}{2.847806in}}%
\pgfpathlineto{\pgfqpoint{4.536067in}{2.844144in}}%
\pgfpathlineto{\pgfqpoint{4.565488in}{2.828328in}}%
\pgfpathlineto{\pgfqpoint{4.569323in}{2.826366in}}%
\pgfpathlineto{\pgfqpoint{4.602579in}{2.809991in}}%
\pgfpathlineto{\pgfqpoint{4.604987in}{2.808850in}}%
\pgfpathlineto{\pgfqpoint{4.635835in}{2.795078in}}%
\pgfpathlineto{\pgfqpoint{4.649167in}{2.789373in}}%
\pgfpathlineto{\pgfqpoint{4.669091in}{2.781235in}}%
\pgfpathlineto{\pgfqpoint{4.697661in}{2.769895in}}%
\pgfpathlineto{\pgfqpoint{4.702348in}{2.768084in}}%
\pgfpathlineto{\pgfqpoint{4.735604in}{2.755302in}}%
\pgfpathlineto{\pgfqpoint{4.748357in}{2.750418in}}%
\pgfpathlineto{\pgfqpoint{4.768860in}{2.742608in}}%
\pgfpathlineto{\pgfqpoint{4.799345in}{2.730940in}}%
\pgfpathlineto{\pgfqpoint{4.802116in}{2.729877in}}%
\pgfpathlineto{\pgfqpoint{4.835372in}{2.716957in}}%
\pgfpathlineto{\pgfqpoint{4.849351in}{2.711462in}}%
\pgfpathlineto{\pgfqpoint{4.868629in}{2.703833in}}%
\pgfpathlineto{\pgfqpoint{4.898187in}{2.691985in}}%
\pgfpathlineto{\pgfqpoint{4.901885in}{2.690490in}}%
\pgfpathlineto{\pgfqpoint{4.935141in}{2.676843in}}%
\pgfpathlineto{\pgfqpoint{4.945562in}{2.672507in}}%
\pgfpathlineto{\pgfqpoint{4.968397in}{2.662918in}}%
\pgfpathlineto{\pgfqpoint{4.991633in}{2.653029in}}%
\pgfpathlineto{\pgfqpoint{5.001653in}{2.648724in}}%
\pgfpathlineto{\pgfqpoint{5.034909in}{2.634241in}}%
\pgfpathlineto{\pgfqpoint{5.036470in}{2.633552in}}%
\pgfpathlineto{\pgfqpoint{5.068166in}{2.619410in}}%
\pgfpathlineto{\pgfqpoint{5.079971in}{2.614074in}}%
\pgfpathlineto{\pgfqpoint{5.101422in}{2.604283in}}%
\pgfpathlineto{\pgfqpoint{5.122375in}{2.594596in}}%
\pgfpathlineto{\pgfqpoint{5.134678in}{2.588853in}}%
\pgfpathlineto{\pgfqpoint{5.163724in}{2.575119in}}%
\pgfpathlineto{\pgfqpoint{5.167934in}{2.573108in}}%
\pgfpathlineto{\pgfqpoint{5.201190in}{2.557019in}}%
\pgfpathlineto{\pgfqpoint{5.204003in}{2.555641in}}%
\pgfpathlineto{\pgfqpoint{5.234447in}{2.540572in}}%
\pgfpathlineto{\pgfqpoint{5.243244in}{2.536163in}}%
\pgfpathlineto{\pgfqpoint{5.267703in}{2.523785in}}%
\pgfpathlineto{\pgfqpoint{5.281562in}{2.516686in}}%
\pgfpathlineto{\pgfqpoint{5.300959in}{2.506649in}}%
\pgfpathlineto{\pgfqpoint{5.318987in}{2.497208in}}%
\pgfpathlineto{\pgfqpoint{5.334215in}{2.489152in}}%
\pgfpathlineto{\pgfqpoint{5.355553in}{2.477731in}}%
\pgfpathlineto{\pgfqpoint{5.367471in}{2.471284in}}%
\pgfpathlineto{\pgfqpoint{5.391287in}{2.458253in}}%
\pgfpathlineto{\pgfqpoint{5.400728in}{2.453033in}}%
\pgfpathlineto{\pgfqpoint{5.426219in}{2.438775in}}%
\pgfpathlineto{\pgfqpoint{5.433984in}{2.434386in}}%
\pgfpathlineto{\pgfqpoint{5.460376in}{2.419298in}}%
\pgfpathlineto{\pgfqpoint{5.467240in}{2.415331in}}%
\pgfpathlineto{\pgfqpoint{5.493784in}{2.399820in}}%
\pgfpathlineto{\pgfqpoint{5.500496in}{2.395855in}}%
\pgfpathlineto{\pgfqpoint{5.526467in}{2.380342in}}%
\pgfpathlineto{\pgfqpoint{5.533752in}{2.375943in}}%
\pgfpathlineto{\pgfqpoint{5.558449in}{2.360865in}}%
\pgfpathlineto{\pgfqpoint{5.567008in}{2.355581in}}%
\pgfpathlineto{\pgfqpoint{5.589753in}{2.341387in}}%
\pgfpathlineto{\pgfqpoint{5.600265in}{2.334754in}}%
\pgfpathlineto{\pgfqpoint{5.620401in}{2.321909in}}%
\pgfpathlineto{\pgfqpoint{5.633521in}{2.313446in}}%
\pgfpathlineto{\pgfqpoint{5.650413in}{2.302432in}}%
\pgfpathlineto{\pgfqpoint{5.666777in}{2.291640in}}%
\pgfusepath{stroke}%
\end{pgfscope}%
\begin{pgfscope}%
\pgfpathrectangle{\pgfqpoint{0.711606in}{0.549444in}}{\pgfqpoint{4.955171in}{2.902168in}}%
\pgfusepath{clip}%
\pgfsetbuttcap%
\pgfsetroundjoin%
\pgfsetlinewidth{1.003750pt}%
\definecolor{currentstroke}{rgb}{0.547157,0.136929,0.413511}%
\pgfsetstrokecolor{currentstroke}%
\pgfsetdash{}{0pt}%
\pgfpathmoveto{\pgfqpoint{0.939293in}{0.549444in}}%
\pgfpathlineto{\pgfqpoint{0.911143in}{0.565162in}}%
\pgfpathlineto{\pgfqpoint{0.904482in}{0.568922in}}%
\pgfpathlineto{\pgfqpoint{0.877887in}{0.584090in}}%
\pgfpathlineto{\pgfqpoint{0.870412in}{0.588399in}}%
\pgfpathlineto{\pgfqpoint{0.844630in}{0.603419in}}%
\pgfpathlineto{\pgfqpoint{0.837061in}{0.607877in}}%
\pgfpathlineto{\pgfqpoint{0.811374in}{0.623163in}}%
\pgfpathlineto{\pgfqpoint{0.804404in}{0.627355in}}%
\pgfpathlineto{\pgfqpoint{0.778118in}{0.643332in}}%
\pgfpathlineto{\pgfqpoint{0.772420in}{0.646832in}}%
\pgfpathlineto{\pgfqpoint{0.744862in}{0.663943in}}%
\pgfpathlineto{\pgfqpoint{0.741089in}{0.666310in}}%
\pgfpathlineto{\pgfqpoint{0.711606in}{0.685008in}}%
\pgfusepath{stroke}%
\end{pgfscope}%
\begin{pgfscope}%
\pgfpathrectangle{\pgfqpoint{0.711606in}{0.549444in}}{\pgfqpoint{4.955171in}{2.902168in}}%
\pgfusepath{clip}%
\pgfsetbuttcap%
\pgfsetroundjoin%
\pgfsetlinewidth{1.003750pt}%
\definecolor{currentstroke}{rgb}{0.547157,0.136929,0.413511}%
\pgfsetstrokecolor{currentstroke}%
\pgfsetdash{}{0pt}%
\pgfpathmoveto{\pgfqpoint{3.695492in}{3.451613in}}%
\pgfpathlineto{\pgfqpoint{3.704662in}{3.446689in}}%
\pgfpathlineto{\pgfqpoint{3.731457in}{3.432135in}}%
\pgfpathlineto{\pgfqpoint{3.737918in}{3.428588in}}%
\pgfpathlineto{\pgfqpoint{3.766611in}{3.412657in}}%
\pgfpathlineto{\pgfqpoint{3.771175in}{3.410096in}}%
\pgfpathlineto{\pgfqpoint{3.800982in}{3.393180in}}%
\pgfpathlineto{\pgfqpoint{3.804431in}{3.391201in}}%
\pgfpathlineto{\pgfqpoint{3.834596in}{3.373702in}}%
\pgfpathlineto{\pgfqpoint{3.837687in}{3.371889in}}%
\pgfpathlineto{\pgfqpoint{3.867478in}{3.354224in}}%
\pgfpathlineto{\pgfqpoint{3.870943in}{3.352147in}}%
\pgfpathlineto{\pgfqpoint{3.899652in}{3.334747in}}%
\pgfpathlineto{\pgfqpoint{3.904199in}{3.331960in}}%
\pgfpathlineto{\pgfqpoint{3.931142in}{3.315269in}}%
\pgfpathlineto{\pgfqpoint{3.937455in}{3.311314in}}%
\pgfpathlineto{\pgfqpoint{3.961970in}{3.295791in}}%
\pgfpathlineto{\pgfqpoint{3.970712in}{3.290193in}}%
\pgfpathlineto{\pgfqpoint{3.992155in}{3.276314in}}%
\pgfpathlineto{\pgfqpoint{4.003968in}{3.268581in}}%
\pgfpathlineto{\pgfqpoint{4.021720in}{3.256836in}}%
\pgfpathlineto{\pgfqpoint{4.037224in}{3.246460in}}%
\pgfpathlineto{\pgfqpoint{4.050683in}{3.237359in}}%
\pgfpathlineto{\pgfqpoint{4.070480in}{3.223814in}}%
\pgfpathlineto{\pgfqpoint{4.079063in}{3.217881in}}%
\pgfpathlineto{\pgfqpoint{4.103736in}{3.200624in}}%
\pgfpathlineto{\pgfqpoint{4.106880in}{3.198403in}}%
\pgfpathlineto{\pgfqpoint{4.134108in}{3.178926in}}%
\pgfpathlineto{\pgfqpoint{4.136993in}{3.176836in}}%
\pgfpathlineto{\pgfqpoint{4.160756in}{3.159448in}}%
\pgfpathlineto{\pgfqpoint{4.170249in}{3.152418in}}%
\pgfpathlineto{\pgfqpoint{4.186896in}{3.139970in}}%
\pgfpathlineto{\pgfqpoint{4.203505in}{3.127402in}}%
\pgfpathlineto{\pgfqpoint{4.212553in}{3.120493in}}%
\pgfpathlineto{\pgfqpoint{4.236761in}{3.101790in}}%
\pgfpathlineto{\pgfqpoint{4.237757in}{3.101015in}}%
\pgfpathlineto{\pgfqpoint{4.262454in}{3.081537in}}%
\pgfpathlineto{\pgfqpoint{4.270017in}{3.075509in}}%
\pgfpathlineto{\pgfqpoint{4.286805in}{3.062060in}}%
\pgfpathlineto{\pgfqpoint{4.303273in}{3.048761in}}%
\pgfpathlineto{\pgfqpoint{4.310922in}{3.042582in}}%
\pgfpathlineto{\pgfqpoint{4.334935in}{3.023104in}}%
\pgfpathlineto{\pgfqpoint{4.336530in}{3.021806in}}%
\pgfpathlineto{\pgfqpoint{4.359111in}{3.003627in}}%
\pgfpathlineto{\pgfqpoint{4.369786in}{2.995103in}}%
\pgfpathlineto{\pgfqpoint{4.383858in}{2.984149in}}%
\pgfpathlineto{\pgfqpoint{4.403042in}{2.969497in}}%
\pgfpathlineto{\pgfqpoint{4.409618in}{2.964672in}}%
\pgfpathlineto{\pgfqpoint{4.436298in}{2.945602in}}%
\pgfpathlineto{\pgfqpoint{4.436897in}{2.945194in}}%
\pgfpathlineto{\pgfqpoint{4.466122in}{2.925716in}}%
\pgfpathlineto{\pgfqpoint{4.469554in}{2.923483in}}%
\pgfpathlineto{\pgfqpoint{4.497120in}{2.906239in}}%
\pgfpathlineto{\pgfqpoint{4.502811in}{2.902745in}}%
\pgfpathlineto{\pgfqpoint{4.529563in}{2.886761in}}%
\pgfpathlineto{\pgfqpoint{4.536067in}{2.882944in}}%
\pgfpathlineto{\pgfqpoint{4.563312in}{2.867283in}}%
\pgfpathlineto{\pgfqpoint{4.569323in}{2.863917in}}%
\pgfpathlineto{\pgfqpoint{4.598808in}{2.847806in}}%
\pgfpathlineto{\pgfqpoint{4.602579in}{2.845831in}}%
\pgfpathlineto{\pgfqpoint{4.635835in}{2.829001in}}%
\pgfpathlineto{\pgfqpoint{4.637213in}{2.828328in}}%
\pgfpathlineto{\pgfqpoint{4.669091in}{2.813620in}}%
\pgfpathlineto{\pgfqpoint{4.679870in}{2.808850in}}%
\pgfpathlineto{\pgfqpoint{4.702348in}{2.799384in}}%
\pgfpathlineto{\pgfqpoint{4.726867in}{2.789373in}}%
\pgfpathlineto{\pgfqpoint{4.735604in}{2.785912in}}%
\pgfpathlineto{\pgfqpoint{4.768860in}{2.772880in}}%
\pgfpathlineto{\pgfqpoint{4.776510in}{2.769895in}}%
\pgfpathlineto{\pgfqpoint{4.802116in}{2.759980in}}%
\pgfpathlineto{\pgfqpoint{4.826743in}{2.750418in}}%
\pgfpathlineto{\pgfqpoint{4.835372in}{2.747064in}}%
\pgfpathlineto{\pgfqpoint{4.868629in}{2.734003in}}%
\pgfpathlineto{\pgfqpoint{4.876336in}{2.730940in}}%
\pgfpathlineto{\pgfqpoint{4.901885in}{2.720722in}}%
\pgfpathlineto{\pgfqpoint{4.924761in}{2.711462in}}%
\pgfpathlineto{\pgfqpoint{4.935141in}{2.707227in}}%
\pgfpathlineto{\pgfqpoint{4.968397in}{2.693473in}}%
\pgfpathlineto{\pgfqpoint{4.971945in}{2.691985in}}%
\pgfpathlineto{\pgfqpoint{5.001653in}{2.679408in}}%
\pgfpathlineto{\pgfqpoint{5.017745in}{2.672507in}}%
\pgfpathlineto{\pgfqpoint{5.034909in}{2.665077in}}%
\pgfpathlineto{\pgfqpoint{5.062384in}{2.653029in}}%
\pgfpathlineto{\pgfqpoint{5.068166in}{2.650470in}}%
\pgfpathlineto{\pgfqpoint{5.101422in}{2.635548in}}%
\pgfpathlineto{\pgfqpoint{5.105812in}{2.633552in}}%
\pgfpathlineto{\pgfqpoint{5.134678in}{2.620298in}}%
\pgfpathlineto{\pgfqpoint{5.148065in}{2.614074in}}%
\pgfpathlineto{\pgfqpoint{5.167934in}{2.604747in}}%
\pgfpathlineto{\pgfqpoint{5.189293in}{2.594596in}}%
\pgfpathlineto{\pgfqpoint{5.201190in}{2.588887in}}%
\pgfpathlineto{\pgfqpoint{5.229531in}{2.575119in}}%
\pgfpathlineto{\pgfqpoint{5.234447in}{2.572707in}}%
\pgfpathlineto{\pgfqpoint{5.267703in}{2.556191in}}%
\pgfpathlineto{\pgfqpoint{5.268797in}{2.555641in}}%
\pgfpathlineto{\pgfqpoint{5.300959in}{2.539305in}}%
\pgfpathlineto{\pgfqpoint{5.307072in}{2.536163in}}%
\pgfpathlineto{\pgfqpoint{5.334215in}{2.522075in}}%
\pgfpathlineto{\pgfqpoint{5.344478in}{2.516686in}}%
\pgfpathlineto{\pgfqpoint{5.367471in}{2.504490in}}%
\pgfpathlineto{\pgfqpoint{5.381043in}{2.497208in}}%
\pgfpathlineto{\pgfqpoint{5.400728in}{2.486539in}}%
\pgfpathlineto{\pgfqpoint{5.416797in}{2.477731in}}%
\pgfpathlineto{\pgfqpoint{5.433984in}{2.468212in}}%
\pgfpathlineto{\pgfqpoint{5.451766in}{2.458253in}}%
\pgfpathlineto{\pgfqpoint{5.467240in}{2.449496in}}%
\pgfpathlineto{\pgfqpoint{5.485976in}{2.438775in}}%
\pgfpathlineto{\pgfqpoint{5.500496in}{2.430379in}}%
\pgfpathlineto{\pgfqpoint{5.519452in}{2.419298in}}%
\pgfpathlineto{\pgfqpoint{5.533752in}{2.410849in}}%
\pgfpathlineto{\pgfqpoint{5.552218in}{2.399820in}}%
\pgfpathlineto{\pgfqpoint{5.567008in}{2.390891in}}%
\pgfpathlineto{\pgfqpoint{5.584297in}{2.380342in}}%
\pgfpathlineto{\pgfqpoint{5.600265in}{2.370493in}}%
\pgfpathlineto{\pgfqpoint{5.615709in}{2.360865in}}%
\pgfpathlineto{\pgfqpoint{5.633521in}{2.349639in}}%
\pgfpathlineto{\pgfqpoint{5.646477in}{2.341387in}}%
\pgfpathlineto{\pgfqpoint{5.666777in}{2.328314in}}%
\pgfusepath{stroke}%
\end{pgfscope}%
\begin{pgfscope}%
\pgfpathrectangle{\pgfqpoint{0.711606in}{0.549444in}}{\pgfqpoint{4.955171in}{2.902168in}}%
\pgfusepath{clip}%
\pgfsetbuttcap%
\pgfsetroundjoin%
\pgfsetlinewidth{1.003750pt}%
\definecolor{currentstroke}{rgb}{0.565854,0.143567,0.408258}%
\pgfsetstrokecolor{currentstroke}%
\pgfsetdash{}{0pt}%
\pgfpathmoveto{\pgfqpoint{0.880647in}{0.549444in}}%
\pgfpathlineto{\pgfqpoint{0.877887in}{0.551003in}}%
\pgfpathlineto{\pgfqpoint{0.846492in}{0.568922in}}%
\pgfpathlineto{\pgfqpoint{0.844630in}{0.569995in}}%
\pgfpathlineto{\pgfqpoint{0.813053in}{0.588399in}}%
\pgfpathlineto{\pgfqpoint{0.811374in}{0.589388in}}%
\pgfpathlineto{\pgfqpoint{0.780307in}{0.607877in}}%
\pgfpathlineto{\pgfqpoint{0.778118in}{0.609193in}}%
\pgfpathlineto{\pgfqpoint{0.748232in}{0.627355in}}%
\pgfpathlineto{\pgfqpoint{0.744862in}{0.629424in}}%
\pgfpathlineto{\pgfqpoint{0.716808in}{0.646832in}}%
\pgfpathlineto{\pgfqpoint{0.711606in}{0.650095in}}%
\pgfusepath{stroke}%
\end{pgfscope}%
\begin{pgfscope}%
\pgfpathrectangle{\pgfqpoint{0.711606in}{0.549444in}}{\pgfqpoint{4.955171in}{2.902168in}}%
\pgfusepath{clip}%
\pgfsetbuttcap%
\pgfsetroundjoin%
\pgfsetlinewidth{1.003750pt}%
\definecolor{currentstroke}{rgb}{0.565854,0.143567,0.408258}%
\pgfsetstrokecolor{currentstroke}%
\pgfsetdash{}{0pt}%
\pgfpathmoveto{\pgfqpoint{3.756873in}{3.451613in}}%
\pgfpathlineto{\pgfqpoint{3.771175in}{3.443749in}}%
\pgfpathlineto{\pgfqpoint{3.792060in}{3.432135in}}%
\pgfpathlineto{\pgfqpoint{3.804431in}{3.425185in}}%
\pgfpathlineto{\pgfqpoint{3.826481in}{3.412657in}}%
\pgfpathlineto{\pgfqpoint{3.837687in}{3.406224in}}%
\pgfpathlineto{\pgfqpoint{3.860160in}{3.393180in}}%
\pgfpathlineto{\pgfqpoint{3.870943in}{3.386854in}}%
\pgfpathlineto{\pgfqpoint{3.893122in}{3.373702in}}%
\pgfpathlineto{\pgfqpoint{3.904199in}{3.367063in}}%
\pgfpathlineto{\pgfqpoint{3.925390in}{3.354224in}}%
\pgfpathlineto{\pgfqpoint{3.937455in}{3.346836in}}%
\pgfpathlineto{\pgfqpoint{3.956986in}{3.334747in}}%
\pgfpathlineto{\pgfqpoint{3.970712in}{3.326158in}}%
\pgfpathlineto{\pgfqpoint{3.987932in}{3.315269in}}%
\pgfpathlineto{\pgfqpoint{4.003968in}{3.305016in}}%
\pgfpathlineto{\pgfqpoint{4.018246in}{3.295791in}}%
\pgfpathlineto{\pgfqpoint{4.037224in}{3.283393in}}%
\pgfpathlineto{\pgfqpoint{4.047950in}{3.276314in}}%
\pgfpathlineto{\pgfqpoint{4.070480in}{3.261274in}}%
\pgfpathlineto{\pgfqpoint{4.077061in}{3.256836in}}%
\pgfpathlineto{\pgfqpoint{4.103736in}{3.238642in}}%
\pgfpathlineto{\pgfqpoint{4.105600in}{3.237359in}}%
\pgfpathlineto{\pgfqpoint{4.133531in}{3.217881in}}%
\pgfpathlineto{\pgfqpoint{4.136993in}{3.215438in}}%
\pgfpathlineto{\pgfqpoint{4.160890in}{3.198403in}}%
\pgfpathlineto{\pgfqpoint{4.170249in}{3.191653in}}%
\pgfpathlineto{\pgfqpoint{4.187726in}{3.178926in}}%
\pgfpathlineto{\pgfqpoint{4.203505in}{3.167299in}}%
\pgfpathlineto{\pgfqpoint{4.214062in}{3.159448in}}%
\pgfpathlineto{\pgfqpoint{4.236761in}{3.142369in}}%
\pgfpathlineto{\pgfqpoint{4.239922in}{3.139970in}}%
\pgfpathlineto{\pgfqpoint{4.265276in}{3.120493in}}%
\pgfpathlineto{\pgfqpoint{4.270017in}{3.116807in}}%
\pgfpathlineto{\pgfqpoint{4.290196in}{3.101015in}}%
\pgfpathlineto{\pgfqpoint{4.303273in}{3.090682in}}%
\pgfpathlineto{\pgfqpoint{4.314802in}{3.081537in}}%
\pgfpathlineto{\pgfqpoint{4.336530in}{3.064186in}}%
\pgfpathlineto{\pgfqpoint{4.339196in}{3.062060in}}%
\pgfpathlineto{\pgfqpoint{4.363518in}{3.042582in}}%
\pgfpathlineto{\pgfqpoint{4.369786in}{3.037563in}}%
\pgfpathlineto{\pgfqpoint{4.388114in}{3.023104in}}%
\pgfpathlineto{\pgfqpoint{4.403042in}{3.011455in}}%
\pgfpathlineto{\pgfqpoint{4.413372in}{3.003627in}}%
\pgfpathlineto{\pgfqpoint{4.436298in}{2.986611in}}%
\pgfpathlineto{\pgfqpoint{4.439759in}{2.984149in}}%
\pgfpathlineto{\pgfqpoint{4.467808in}{2.964672in}}%
\pgfpathlineto{\pgfqpoint{4.469554in}{2.963486in}}%
\pgfpathlineto{\pgfqpoint{4.497737in}{2.945194in}}%
\pgfpathlineto{\pgfqpoint{4.502811in}{2.941969in}}%
\pgfpathlineto{\pgfqpoint{4.529252in}{2.925716in}}%
\pgfpathlineto{\pgfqpoint{4.536067in}{2.921588in}}%
\pgfpathlineto{\pgfqpoint{4.561943in}{2.906239in}}%
\pgfpathlineto{\pgfqpoint{4.569323in}{2.901918in}}%
\pgfpathlineto{\pgfqpoint{4.595611in}{2.886761in}}%
\pgfpathlineto{\pgfqpoint{4.602579in}{2.882825in}}%
\pgfpathlineto{\pgfqpoint{4.630616in}{2.867283in}}%
\pgfpathlineto{\pgfqpoint{4.635835in}{2.864494in}}%
\pgfpathlineto{\pgfqpoint{4.668009in}{2.847806in}}%
\pgfpathlineto{\pgfqpoint{4.669091in}{2.847273in}}%
\pgfpathlineto{\pgfqpoint{4.702348in}{2.831455in}}%
\pgfpathlineto{\pgfqpoint{4.709186in}{2.828328in}}%
\pgfpathlineto{\pgfqpoint{4.735604in}{2.816855in}}%
\pgfpathlineto{\pgfqpoint{4.754654in}{2.808850in}}%
\pgfpathlineto{\pgfqpoint{4.768860in}{2.803082in}}%
\pgfpathlineto{\pgfqpoint{4.802116in}{2.789802in}}%
\pgfpathlineto{\pgfqpoint{4.803195in}{2.789373in}}%
\pgfpathlineto{\pgfqpoint{4.835372in}{2.776706in}}%
\pgfpathlineto{\pgfqpoint{4.852660in}{2.769895in}}%
\pgfpathlineto{\pgfqpoint{4.868629in}{2.763607in}}%
\pgfpathlineto{\pgfqpoint{4.901872in}{2.750418in}}%
\pgfpathlineto{\pgfqpoint{4.901885in}{2.750412in}}%
\pgfpathlineto{\pgfqpoint{4.935141in}{2.736977in}}%
\pgfpathlineto{\pgfqpoint{4.949915in}{2.730940in}}%
\pgfpathlineto{\pgfqpoint{4.968397in}{2.723330in}}%
\pgfpathlineto{\pgfqpoint{4.996866in}{2.711462in}}%
\pgfpathlineto{\pgfqpoint{5.001653in}{2.709449in}}%
\pgfpathlineto{\pgfqpoint{5.034909in}{2.695271in}}%
\pgfpathlineto{\pgfqpoint{5.042516in}{2.691985in}}%
\pgfpathlineto{\pgfqpoint{5.068166in}{2.680802in}}%
\pgfpathlineto{\pgfqpoint{5.086954in}{2.672507in}}%
\pgfpathlineto{\pgfqpoint{5.101422in}{2.666060in}}%
\pgfpathlineto{\pgfqpoint{5.130302in}{2.653029in}}%
\pgfpathlineto{\pgfqpoint{5.134678in}{2.651036in}}%
\pgfpathlineto{\pgfqpoint{5.167934in}{2.635689in}}%
\pgfpathlineto{\pgfqpoint{5.172508in}{2.633552in}}%
\pgfpathlineto{\pgfqpoint{5.201190in}{2.620017in}}%
\pgfpathlineto{\pgfqpoint{5.213635in}{2.614074in}}%
\pgfpathlineto{\pgfqpoint{5.234447in}{2.604039in}}%
\pgfpathlineto{\pgfqpoint{5.253799in}{2.594596in}}%
\pgfpathlineto{\pgfqpoint{5.267703in}{2.587747in}}%
\pgfpathlineto{\pgfqpoint{5.293034in}{2.575119in}}%
\pgfpathlineto{\pgfqpoint{5.300959in}{2.571130in}}%
\pgfpathlineto{\pgfqpoint{5.331373in}{2.555641in}}%
\pgfpathlineto{\pgfqpoint{5.334215in}{2.554179in}}%
\pgfpathlineto{\pgfqpoint{5.367471in}{2.536874in}}%
\pgfpathlineto{\pgfqpoint{5.368821in}{2.536163in}}%
\pgfpathlineto{\pgfqpoint{5.400728in}{2.519198in}}%
\pgfpathlineto{\pgfqpoint{5.405400in}{2.516686in}}%
\pgfpathlineto{\pgfqpoint{5.433984in}{2.501162in}}%
\pgfpathlineto{\pgfqpoint{5.441184in}{2.497208in}}%
\pgfpathlineto{\pgfqpoint{5.467240in}{2.482755in}}%
\pgfpathlineto{\pgfqpoint{5.476200in}{2.477731in}}%
\pgfpathlineto{\pgfqpoint{5.500496in}{2.463967in}}%
\pgfpathlineto{\pgfqpoint{5.510473in}{2.458253in}}%
\pgfpathlineto{\pgfqpoint{5.533752in}{2.444784in}}%
\pgfpathlineto{\pgfqpoint{5.544027in}{2.438775in}}%
\pgfpathlineto{\pgfqpoint{5.567008in}{2.425195in}}%
\pgfpathlineto{\pgfqpoint{5.576884in}{2.419298in}}%
\pgfpathlineto{\pgfqpoint{5.600265in}{2.405187in}}%
\pgfpathlineto{\pgfqpoint{5.609065in}{2.399820in}}%
\pgfpathlineto{\pgfqpoint{5.633521in}{2.384747in}}%
\pgfpathlineto{\pgfqpoint{5.640593in}{2.380342in}}%
\pgfpathlineto{\pgfqpoint{5.666777in}{2.363860in}}%
\pgfusepath{stroke}%
\end{pgfscope}%
\begin{pgfscope}%
\pgfpathrectangle{\pgfqpoint{0.711606in}{0.549444in}}{\pgfqpoint{4.955171in}{2.902168in}}%
\pgfusepath{clip}%
\pgfsetbuttcap%
\pgfsetroundjoin%
\pgfsetlinewidth{1.003750pt}%
\definecolor{currentstroke}{rgb}{0.590734,0.152563,0.400290}%
\pgfsetstrokecolor{currentstroke}%
\pgfsetdash{}{0pt}%
\pgfpathmoveto{\pgfqpoint{0.823697in}{0.549444in}}%
\pgfpathlineto{\pgfqpoint{0.811374in}{0.556554in}}%
\pgfpathlineto{\pgfqpoint{0.790163in}{0.568922in}}%
\pgfpathlineto{\pgfqpoint{0.778118in}{0.576016in}}%
\pgfpathlineto{\pgfqpoint{0.757310in}{0.588399in}}%
\pgfpathlineto{\pgfqpoint{0.744862in}{0.595884in}}%
\pgfpathlineto{\pgfqpoint{0.725117in}{0.607877in}}%
\pgfpathlineto{\pgfqpoint{0.711606in}{0.616169in}}%
\pgfusepath{stroke}%
\end{pgfscope}%
\begin{pgfscope}%
\pgfpathrectangle{\pgfqpoint{0.711606in}{0.549444in}}{\pgfqpoint{4.955171in}{2.902168in}}%
\pgfusepath{clip}%
\pgfsetbuttcap%
\pgfsetroundjoin%
\pgfsetlinewidth{1.003750pt}%
\definecolor{currentstroke}{rgb}{0.590734,0.152563,0.400290}%
\pgfsetstrokecolor{currentstroke}%
\pgfsetdash{}{0pt}%
\pgfpathmoveto{\pgfqpoint{3.816398in}{3.451613in}}%
\pgfpathlineto{\pgfqpoint{3.837687in}{3.439637in}}%
\pgfpathlineto{\pgfqpoint{3.850878in}{3.432135in}}%
\pgfpathlineto{\pgfqpoint{3.870943in}{3.420607in}}%
\pgfpathlineto{\pgfqpoint{3.884632in}{3.412657in}}%
\pgfpathlineto{\pgfqpoint{3.904199in}{3.401176in}}%
\pgfpathlineto{\pgfqpoint{3.917682in}{3.393180in}}%
\pgfpathlineto{\pgfqpoint{3.937455in}{3.381330in}}%
\pgfpathlineto{\pgfqpoint{3.950051in}{3.373702in}}%
\pgfpathlineto{\pgfqpoint{3.970712in}{3.361057in}}%
\pgfpathlineto{\pgfqpoint{3.981760in}{3.354224in}}%
\pgfpathlineto{\pgfqpoint{4.003968in}{3.340343in}}%
\pgfpathlineto{\pgfqpoint{4.012829in}{3.334747in}}%
\pgfpathlineto{\pgfqpoint{4.037224in}{3.319173in}}%
\pgfpathlineto{\pgfqpoint{4.043278in}{3.315269in}}%
\pgfpathlineto{\pgfqpoint{4.070480in}{3.297534in}}%
\pgfpathlineto{\pgfqpoint{4.073126in}{3.295791in}}%
\pgfpathlineto{\pgfqpoint{4.102370in}{3.276314in}}%
\pgfpathlineto{\pgfqpoint{4.103736in}{3.275393in}}%
\pgfpathlineto{\pgfqpoint{4.131002in}{3.256836in}}%
\pgfpathlineto{\pgfqpoint{4.136993in}{3.252712in}}%
\pgfpathlineto{\pgfqpoint{4.159079in}{3.237359in}}%
\pgfpathlineto{\pgfqpoint{4.170249in}{3.229505in}}%
\pgfpathlineto{\pgfqpoint{4.186621in}{3.217881in}}%
\pgfpathlineto{\pgfqpoint{4.203505in}{3.205756in}}%
\pgfpathlineto{\pgfqpoint{4.213648in}{3.198403in}}%
\pgfpathlineto{\pgfqpoint{4.236761in}{3.181456in}}%
\pgfpathlineto{\pgfqpoint{4.240181in}{3.178926in}}%
\pgfpathlineto{\pgfqpoint{4.266194in}{3.159448in}}%
\pgfpathlineto{\pgfqpoint{4.270017in}{3.156551in}}%
\pgfpathlineto{\pgfqpoint{4.291728in}{3.139970in}}%
\pgfpathlineto{\pgfqpoint{4.303273in}{3.131061in}}%
\pgfpathlineto{\pgfqpoint{4.316885in}{3.120493in}}%
\pgfpathlineto{\pgfqpoint{4.336530in}{3.105107in}}%
\pgfpathlineto{\pgfqpoint{4.341740in}{3.101015in}}%
\pgfpathlineto{\pgfqpoint{4.366382in}{3.081537in}}%
\pgfpathlineto{\pgfqpoint{4.369786in}{3.078832in}}%
\pgfpathlineto{\pgfqpoint{4.391032in}{3.062060in}}%
\pgfpathlineto{\pgfqpoint{4.403042in}{3.052613in}}%
\pgfpathlineto{\pgfqpoint{4.416032in}{3.042582in}}%
\pgfpathlineto{\pgfqpoint{4.436298in}{3.027140in}}%
\pgfpathlineto{\pgfqpoint{4.441772in}{3.023104in}}%
\pgfpathlineto{\pgfqpoint{4.468765in}{3.003627in}}%
\pgfpathlineto{\pgfqpoint{4.469554in}{3.003068in}}%
\pgfpathlineto{\pgfqpoint{4.497539in}{2.984149in}}%
\pgfpathlineto{\pgfqpoint{4.502811in}{2.980670in}}%
\pgfpathlineto{\pgfqpoint{4.528060in}{2.964672in}}%
\pgfpathlineto{\pgfqpoint{4.536067in}{2.959686in}}%
\pgfpathlineto{\pgfqpoint{4.559998in}{2.945194in}}%
\pgfpathlineto{\pgfqpoint{4.569323in}{2.939607in}}%
\pgfpathlineto{\pgfqpoint{4.592877in}{2.925716in}}%
\pgfpathlineto{\pgfqpoint{4.602579in}{2.920046in}}%
\pgfpathlineto{\pgfqpoint{4.626461in}{2.906239in}}%
\pgfpathlineto{\pgfqpoint{4.635835in}{2.900902in}}%
\pgfpathlineto{\pgfqpoint{4.661032in}{2.886761in}}%
\pgfpathlineto{\pgfqpoint{4.669091in}{2.882374in}}%
\pgfpathlineto{\pgfqpoint{4.697515in}{2.867283in}}%
\pgfpathlineto{\pgfqpoint{4.702348in}{2.864839in}}%
\pgfpathlineto{\pgfqpoint{4.735604in}{2.848599in}}%
\pgfpathlineto{\pgfqpoint{4.737284in}{2.847806in}}%
\pgfpathlineto{\pgfqpoint{4.768860in}{2.833666in}}%
\pgfpathlineto{\pgfqpoint{4.781201in}{2.828328in}}%
\pgfpathlineto{\pgfqpoint{4.802116in}{2.819616in}}%
\pgfpathlineto{\pgfqpoint{4.828492in}{2.808850in}}%
\pgfpathlineto{\pgfqpoint{4.835372in}{2.806096in}}%
\pgfpathlineto{\pgfqpoint{4.868629in}{2.792815in}}%
\pgfpathlineto{\pgfqpoint{4.877239in}{2.789373in}}%
\pgfpathlineto{\pgfqpoint{4.901885in}{2.779541in}}%
\pgfpathlineto{\pgfqpoint{4.925915in}{2.769895in}}%
\pgfpathlineto{\pgfqpoint{4.935141in}{2.766178in}}%
\pgfpathlineto{\pgfqpoint{4.968397in}{2.752631in}}%
\pgfpathlineto{\pgfqpoint{4.973766in}{2.750418in}}%
\pgfpathlineto{\pgfqpoint{5.001653in}{2.738834in}}%
\pgfpathlineto{\pgfqpoint{5.020435in}{2.730940in}}%
\pgfpathlineto{\pgfqpoint{5.034909in}{2.724806in}}%
\pgfpathlineto{\pgfqpoint{5.066009in}{2.711462in}}%
\pgfpathlineto{\pgfqpoint{5.068166in}{2.710529in}}%
\pgfpathlineto{\pgfqpoint{5.101422in}{2.695935in}}%
\pgfpathlineto{\pgfqpoint{5.110311in}{2.691985in}}%
\pgfpathlineto{\pgfqpoint{5.134678in}{2.681058in}}%
\pgfpathlineto{\pgfqpoint{5.153515in}{2.672507in}}%
\pgfpathlineto{\pgfqpoint{5.167934in}{2.665901in}}%
\pgfpathlineto{\pgfqpoint{5.195694in}{2.653029in}}%
\pgfpathlineto{\pgfqpoint{5.201190in}{2.650457in}}%
\pgfpathlineto{\pgfqpoint{5.234447in}{2.634699in}}%
\pgfpathlineto{\pgfqpoint{5.236837in}{2.633552in}}%
\pgfpathlineto{\pgfqpoint{5.267703in}{2.618602in}}%
\pgfpathlineto{\pgfqpoint{5.276944in}{2.614074in}}%
\pgfpathlineto{\pgfqpoint{5.300959in}{2.602195in}}%
\pgfpathlineto{\pgfqpoint{5.316145in}{2.594596in}}%
\pgfpathlineto{\pgfqpoint{5.334215in}{2.585468in}}%
\pgfpathlineto{\pgfqpoint{5.354469in}{2.575119in}}%
\pgfpathlineto{\pgfqpoint{5.367471in}{2.568411in}}%
\pgfpathlineto{\pgfqpoint{5.391948in}{2.555641in}}%
\pgfpathlineto{\pgfqpoint{5.400728in}{2.551016in}}%
\pgfpathlineto{\pgfqpoint{5.428608in}{2.536163in}}%
\pgfpathlineto{\pgfqpoint{5.433984in}{2.533272in}}%
\pgfpathlineto{\pgfqpoint{5.464477in}{2.516686in}}%
\pgfpathlineto{\pgfqpoint{5.467240in}{2.515168in}}%
\pgfpathlineto{\pgfqpoint{5.499580in}{2.497208in}}%
\pgfpathlineto{\pgfqpoint{5.500496in}{2.496694in}}%
\pgfpathlineto{\pgfqpoint{5.533752in}{2.477837in}}%
\pgfpathlineto{\pgfqpoint{5.533939in}{2.477731in}}%
\pgfpathlineto{\pgfqpoint{5.567008in}{2.458586in}}%
\pgfpathlineto{\pgfqpoint{5.567577in}{2.458253in}}%
\pgfpathlineto{\pgfqpoint{5.600265in}{2.438935in}}%
\pgfpathlineto{\pgfqpoint{5.600531in}{2.438775in}}%
\pgfpathlineto{\pgfqpoint{5.632812in}{2.419298in}}%
\pgfpathlineto{\pgfqpoint{5.633521in}{2.418865in}}%
\pgfpathlineto{\pgfqpoint{5.664436in}{2.399820in}}%
\pgfpathlineto{\pgfqpoint{5.666777in}{2.398362in}}%
\pgfusepath{stroke}%
\end{pgfscope}%
\begin{pgfscope}%
\pgfpathrectangle{\pgfqpoint{0.711606in}{0.549444in}}{\pgfqpoint{4.955171in}{2.902168in}}%
\pgfusepath{clip}%
\pgfsetbuttcap%
\pgfsetroundjoin%
\pgfsetlinewidth{1.003750pt}%
\definecolor{currentstroke}{rgb}{0.615513,0.161817,0.391219}%
\pgfsetstrokecolor{currentstroke}%
\pgfsetdash{}{0pt}%
\pgfpathmoveto{\pgfqpoint{0.768206in}{0.549444in}}%
\pgfpathlineto{\pgfqpoint{0.744862in}{0.563196in}}%
\pgfpathlineto{\pgfqpoint{0.735241in}{0.568922in}}%
\pgfpathlineto{\pgfqpoint{0.711606in}{0.583129in}}%
\pgfusepath{stroke}%
\end{pgfscope}%
\begin{pgfscope}%
\pgfpathrectangle{\pgfqpoint{0.711606in}{0.549444in}}{\pgfqpoint{4.955171in}{2.902168in}}%
\pgfusepath{clip}%
\pgfsetbuttcap%
\pgfsetroundjoin%
\pgfsetlinewidth{1.003750pt}%
\definecolor{currentstroke}{rgb}{0.615513,0.161817,0.391219}%
\pgfsetstrokecolor{currentstroke}%
\pgfsetdash{}{0pt}%
\pgfpathmoveto{\pgfqpoint{3.874244in}{3.451613in}}%
\pgfpathlineto{\pgfqpoint{3.904199in}{3.434390in}}%
\pgfpathlineto{\pgfqpoint{3.908080in}{3.432135in}}%
\pgfpathlineto{\pgfqpoint{3.937455in}{3.414893in}}%
\pgfpathlineto{\pgfqpoint{3.941225in}{3.412657in}}%
\pgfpathlineto{\pgfqpoint{3.970712in}{3.394989in}}%
\pgfpathlineto{\pgfqpoint{3.973701in}{3.393180in}}%
\pgfpathlineto{\pgfqpoint{4.003968in}{3.374666in}}%
\pgfpathlineto{\pgfqpoint{4.005528in}{3.373702in}}%
\pgfpathlineto{\pgfqpoint{4.036719in}{3.354224in}}%
\pgfpathlineto{\pgfqpoint{4.037224in}{3.353905in}}%
\pgfpathlineto{\pgfqpoint{4.067266in}{3.334747in}}%
\pgfpathlineto{\pgfqpoint{4.070480in}{3.332675in}}%
\pgfpathlineto{\pgfqpoint{4.097214in}{3.315269in}}%
\pgfpathlineto{\pgfqpoint{4.103736in}{3.310976in}}%
\pgfpathlineto{\pgfqpoint{4.126579in}{3.295791in}}%
\pgfpathlineto{\pgfqpoint{4.136993in}{3.288793in}}%
\pgfpathlineto{\pgfqpoint{4.155381in}{3.276314in}}%
\pgfpathlineto{\pgfqpoint{4.170249in}{3.266112in}}%
\pgfpathlineto{\pgfqpoint{4.183636in}{3.256836in}}%
\pgfpathlineto{\pgfqpoint{4.203505in}{3.242917in}}%
\pgfpathlineto{\pgfqpoint{4.211364in}{3.237359in}}%
\pgfpathlineto{\pgfqpoint{4.236761in}{3.219195in}}%
\pgfpathlineto{\pgfqpoint{4.238582in}{3.217881in}}%
\pgfpathlineto{\pgfqpoint{4.265248in}{3.198403in}}%
\pgfpathlineto{\pgfqpoint{4.270017in}{3.194880in}}%
\pgfpathlineto{\pgfqpoint{4.291427in}{3.178926in}}%
\pgfpathlineto{\pgfqpoint{4.303273in}{3.170003in}}%
\pgfpathlineto{\pgfqpoint{4.317184in}{3.159448in}}%
\pgfpathlineto{\pgfqpoint{4.336530in}{3.144625in}}%
\pgfpathlineto{\pgfqpoint{4.342573in}{3.139970in}}%
\pgfpathlineto{\pgfqpoint{4.367652in}{3.120493in}}%
\pgfpathlineto{\pgfqpoint{4.369786in}{3.118821in}}%
\pgfpathlineto{\pgfqpoint{4.392539in}{3.101015in}}%
\pgfpathlineto{\pgfqpoint{4.403042in}{3.092781in}}%
\pgfpathlineto{\pgfqpoint{4.417520in}{3.081537in}}%
\pgfpathlineto{\pgfqpoint{4.436298in}{3.067041in}}%
\pgfpathlineto{\pgfqpoint{4.442895in}{3.062060in}}%
\pgfpathlineto{\pgfqpoint{4.469090in}{3.042582in}}%
\pgfpathlineto{\pgfqpoint{4.469554in}{3.042242in}}%
\pgfpathlineto{\pgfqpoint{4.496734in}{3.023104in}}%
\pgfpathlineto{\pgfqpoint{4.502811in}{3.018929in}}%
\pgfpathlineto{\pgfqpoint{4.526091in}{3.003627in}}%
\pgfpathlineto{\pgfqpoint{4.536067in}{2.997210in}}%
\pgfpathlineto{\pgfqpoint{4.557123in}{2.984149in}}%
\pgfpathlineto{\pgfqpoint{4.569323in}{2.976686in}}%
\pgfpathlineto{\pgfqpoint{4.589412in}{2.964672in}}%
\pgfpathlineto{\pgfqpoint{4.602579in}{2.956854in}}%
\pgfpathlineto{\pgfqpoint{4.622443in}{2.945194in}}%
\pgfpathlineto{\pgfqpoint{4.635835in}{2.937377in}}%
\pgfpathlineto{\pgfqpoint{4.655954in}{2.925716in}}%
\pgfpathlineto{\pgfqpoint{4.669091in}{2.918188in}}%
\pgfpathlineto{\pgfqpoint{4.690156in}{2.906239in}}%
\pgfpathlineto{\pgfqpoint{4.702348in}{2.899494in}}%
\pgfpathlineto{\pgfqpoint{4.725841in}{2.886761in}}%
\pgfpathlineto{\pgfqpoint{4.735604in}{2.881694in}}%
\pgfpathlineto{\pgfqpoint{4.764307in}{2.867283in}}%
\pgfpathlineto{\pgfqpoint{4.768860in}{2.865118in}}%
\pgfpathlineto{\pgfqpoint{4.802116in}{2.849833in}}%
\pgfpathlineto{\pgfqpoint{4.806667in}{2.847806in}}%
\pgfpathlineto{\pgfqpoint{4.835372in}{2.835535in}}%
\pgfpathlineto{\pgfqpoint{4.852626in}{2.828328in}}%
\pgfpathlineto{\pgfqpoint{4.868629in}{2.821794in}}%
\pgfpathlineto{\pgfqpoint{4.900590in}{2.808850in}}%
\pgfpathlineto{\pgfqpoint{4.901885in}{2.808330in}}%
\pgfpathlineto{\pgfqpoint{4.935141in}{2.794895in}}%
\pgfpathlineto{\pgfqpoint{4.948738in}{2.789373in}}%
\pgfpathlineto{\pgfqpoint{4.968397in}{2.781368in}}%
\pgfpathlineto{\pgfqpoint{4.996318in}{2.769895in}}%
\pgfpathlineto{\pgfqpoint{5.001653in}{2.767690in}}%
\pgfpathlineto{\pgfqpoint{5.034909in}{2.753771in}}%
\pgfpathlineto{\pgfqpoint{5.042824in}{2.750418in}}%
\pgfpathlineto{\pgfqpoint{5.068166in}{2.739596in}}%
\pgfpathlineto{\pgfqpoint{5.088194in}{2.730940in}}%
\pgfpathlineto{\pgfqpoint{5.101422in}{2.725174in}}%
\pgfpathlineto{\pgfqpoint{5.132503in}{2.711462in}}%
\pgfpathlineto{\pgfqpoint{5.134678in}{2.710494in}}%
\pgfpathlineto{\pgfqpoint{5.167934in}{2.695495in}}%
\pgfpathlineto{\pgfqpoint{5.175623in}{2.691985in}}%
\pgfpathlineto{\pgfqpoint{5.201190in}{2.680207in}}%
\pgfpathlineto{\pgfqpoint{5.217708in}{2.672507in}}%
\pgfpathlineto{\pgfqpoint{5.234447in}{2.664633in}}%
\pgfpathlineto{\pgfqpoint{5.258826in}{2.653029in}}%
\pgfpathlineto{\pgfqpoint{5.267703in}{2.648765in}}%
\pgfpathlineto{\pgfqpoint{5.299010in}{2.633552in}}%
\pgfpathlineto{\pgfqpoint{5.300959in}{2.632596in}}%
\pgfpathlineto{\pgfqpoint{5.334215in}{2.616086in}}%
\pgfpathlineto{\pgfqpoint{5.338222in}{2.614074in}}%
\pgfpathlineto{\pgfqpoint{5.367471in}{2.599248in}}%
\pgfpathlineto{\pgfqpoint{5.376546in}{2.594596in}}%
\pgfpathlineto{\pgfqpoint{5.400728in}{2.582084in}}%
\pgfpathlineto{\pgfqpoint{5.414041in}{2.575119in}}%
\pgfpathlineto{\pgfqpoint{5.433984in}{2.564586in}}%
\pgfpathlineto{\pgfqpoint{5.450737in}{2.555641in}}%
\pgfpathlineto{\pgfqpoint{5.467240in}{2.546744in}}%
\pgfpathlineto{\pgfqpoint{5.486657in}{2.536163in}}%
\pgfpathlineto{\pgfqpoint{5.500496in}{2.528548in}}%
\pgfpathlineto{\pgfqpoint{5.521826in}{2.516686in}}%
\pgfpathlineto{\pgfqpoint{5.533752in}{2.509988in}}%
\pgfpathlineto{\pgfqpoint{5.556269in}{2.497208in}}%
\pgfpathlineto{\pgfqpoint{5.567008in}{2.491052in}}%
\pgfpathlineto{\pgfqpoint{5.590008in}{2.477731in}}%
\pgfpathlineto{\pgfqpoint{5.600265in}{2.471730in}}%
\pgfpathlineto{\pgfqpoint{5.623064in}{2.458253in}}%
\pgfpathlineto{\pgfqpoint{5.633521in}{2.452009in}}%
\pgfpathlineto{\pgfqpoint{5.655459in}{2.438775in}}%
\pgfpathlineto{\pgfqpoint{5.666777in}{2.431878in}}%
\pgfusepath{stroke}%
\end{pgfscope}%
\begin{pgfscope}%
\pgfpathrectangle{\pgfqpoint{0.711606in}{0.549444in}}{\pgfqpoint{4.955171in}{2.902168in}}%
\pgfusepath{clip}%
\pgfsetbuttcap%
\pgfsetroundjoin%
\pgfsetlinewidth{1.003750pt}%
\definecolor{currentstroke}{rgb}{0.633998,0.168992,0.383704}%
\pgfsetstrokecolor{currentstroke}%
\pgfsetdash{}{0pt}%
\pgfpathmoveto{\pgfqpoint{0.714080in}{0.549444in}}%
\pgfpathlineto{\pgfqpoint{0.711606in}{0.550917in}}%
\pgfusepath{stroke}%
\end{pgfscope}%
\begin{pgfscope}%
\pgfpathrectangle{\pgfqpoint{0.711606in}{0.549444in}}{\pgfqpoint{4.955171in}{2.902168in}}%
\pgfusepath{clip}%
\pgfsetbuttcap%
\pgfsetroundjoin%
\pgfsetlinewidth{1.003750pt}%
\definecolor{currentstroke}{rgb}{0.633998,0.168992,0.383704}%
\pgfsetstrokecolor{currentstroke}%
\pgfsetdash{}{0pt}%
\pgfpathmoveto{\pgfqpoint{3.930459in}{3.451613in}}%
\pgfpathlineto{\pgfqpoint{3.937455in}{3.447547in}}%
\pgfpathlineto{\pgfqpoint{3.963706in}{3.432135in}}%
\pgfpathlineto{\pgfqpoint{3.970712in}{3.427980in}}%
\pgfpathlineto{\pgfqpoint{3.996285in}{3.412657in}}%
\pgfpathlineto{\pgfqpoint{4.003968in}{3.408007in}}%
\pgfpathlineto{\pgfqpoint{4.028217in}{3.393180in}}%
\pgfpathlineto{\pgfqpoint{4.037224in}{3.387615in}}%
\pgfpathlineto{\pgfqpoint{4.059521in}{3.373702in}}%
\pgfpathlineto{\pgfqpoint{4.070480in}{3.366792in}}%
\pgfpathlineto{\pgfqpoint{4.090215in}{3.354224in}}%
\pgfpathlineto{\pgfqpoint{4.103736in}{3.345523in}}%
\pgfpathlineto{\pgfqpoint{4.120319in}{3.334747in}}%
\pgfpathlineto{\pgfqpoint{4.136993in}{3.323796in}}%
\pgfpathlineto{\pgfqpoint{4.149849in}{3.315269in}}%
\pgfpathlineto{\pgfqpoint{4.170249in}{3.301595in}}%
\pgfpathlineto{\pgfqpoint{4.178823in}{3.295791in}}%
\pgfpathlineto{\pgfqpoint{4.203505in}{3.278906in}}%
\pgfpathlineto{\pgfqpoint{4.207259in}{3.276314in}}%
\pgfpathlineto{\pgfqpoint{4.235150in}{3.256836in}}%
\pgfpathlineto{\pgfqpoint{4.236761in}{3.255698in}}%
\pgfpathlineto{\pgfqpoint{4.262480in}{3.237359in}}%
\pgfpathlineto{\pgfqpoint{4.270017in}{3.231925in}}%
\pgfpathlineto{\pgfqpoint{4.289327in}{3.217881in}}%
\pgfpathlineto{\pgfqpoint{4.303273in}{3.207628in}}%
\pgfpathlineto{\pgfqpoint{4.315720in}{3.198403in}}%
\pgfpathlineto{\pgfqpoint{4.336530in}{3.182821in}}%
\pgfpathlineto{\pgfqpoint{4.341697in}{3.178926in}}%
\pgfpathlineto{\pgfqpoint{4.367286in}{3.159448in}}%
\pgfpathlineto{\pgfqpoint{4.369786in}{3.157526in}}%
\pgfpathlineto{\pgfqpoint{4.392558in}{3.139970in}}%
\pgfpathlineto{\pgfqpoint{4.403042in}{3.131845in}}%
\pgfpathlineto{\pgfqpoint{4.417729in}{3.120493in}}%
\pgfpathlineto{\pgfqpoint{4.436298in}{3.106142in}}%
\pgfpathlineto{\pgfqpoint{4.443015in}{3.101015in}}%
\pgfpathlineto{\pgfqpoint{4.468737in}{3.081537in}}%
\pgfpathlineto{\pgfqpoint{4.469554in}{3.080924in}}%
\pgfpathlineto{\pgfqpoint{4.495455in}{3.062060in}}%
\pgfpathlineto{\pgfqpoint{4.502811in}{3.056812in}}%
\pgfpathlineto{\pgfqpoint{4.523613in}{3.042582in}}%
\pgfpathlineto{\pgfqpoint{4.536067in}{3.034261in}}%
\pgfpathlineto{\pgfqpoint{4.553488in}{3.023104in}}%
\pgfpathlineto{\pgfqpoint{4.569323in}{3.013151in}}%
\pgfpathlineto{\pgfqpoint{4.584964in}{3.003627in}}%
\pgfpathlineto{\pgfqpoint{4.602579in}{2.993012in}}%
\pgfpathlineto{\pgfqpoint{4.617553in}{2.984149in}}%
\pgfpathlineto{\pgfqpoint{4.635835in}{2.973374in}}%
\pgfpathlineto{\pgfqpoint{4.650716in}{2.964672in}}%
\pgfpathlineto{\pgfqpoint{4.669091in}{2.953956in}}%
\pgfpathlineto{\pgfqpoint{4.684175in}{2.945194in}}%
\pgfpathlineto{\pgfqpoint{4.702348in}{2.934720in}}%
\pgfpathlineto{\pgfqpoint{4.718075in}{2.925716in}}%
\pgfpathlineto{\pgfqpoint{4.735604in}{2.915883in}}%
\pgfpathlineto{\pgfqpoint{4.753077in}{2.906239in}}%
\pgfpathlineto{\pgfqpoint{4.768860in}{2.897855in}}%
\pgfpathlineto{\pgfqpoint{4.790366in}{2.886761in}}%
\pgfpathlineto{\pgfqpoint{4.802116in}{2.881007in}}%
\pgfpathlineto{\pgfqpoint{4.831190in}{2.867283in}}%
\pgfpathlineto{\pgfqpoint{4.835372in}{2.865405in}}%
\pgfpathlineto{\pgfqpoint{4.868629in}{2.850854in}}%
\pgfpathlineto{\pgfqpoint{4.875754in}{2.847806in}}%
\pgfpathlineto{\pgfqpoint{4.901885in}{2.836918in}}%
\pgfpathlineto{\pgfqpoint{4.922726in}{2.828328in}}%
\pgfpathlineto{\pgfqpoint{4.935141in}{2.823266in}}%
\pgfpathlineto{\pgfqpoint{4.968397in}{2.809694in}}%
\pgfpathlineto{\pgfqpoint{4.970449in}{2.808850in}}%
\pgfpathlineto{\pgfqpoint{5.001653in}{2.796012in}}%
\pgfpathlineto{\pgfqpoint{5.017656in}{2.789373in}}%
\pgfpathlineto{\pgfqpoint{5.034909in}{2.782177in}}%
\pgfpathlineto{\pgfqpoint{5.064050in}{2.769895in}}%
\pgfpathlineto{\pgfqpoint{5.068166in}{2.768148in}}%
\pgfpathlineto{\pgfqpoint{5.101422in}{2.753850in}}%
\pgfpathlineto{\pgfqpoint{5.109308in}{2.750418in}}%
\pgfpathlineto{\pgfqpoint{5.134678in}{2.739285in}}%
\pgfpathlineto{\pgfqpoint{5.153473in}{2.730940in}}%
\pgfpathlineto{\pgfqpoint{5.167934in}{2.724464in}}%
\pgfpathlineto{\pgfqpoint{5.196625in}{2.711462in}}%
\pgfpathlineto{\pgfqpoint{5.201190in}{2.709375in}}%
\pgfpathlineto{\pgfqpoint{5.234447in}{2.693982in}}%
\pgfpathlineto{\pgfqpoint{5.238710in}{2.691985in}}%
\pgfpathlineto{\pgfqpoint{5.267703in}{2.678279in}}%
\pgfpathlineto{\pgfqpoint{5.279774in}{2.672507in}}%
\pgfpathlineto{\pgfqpoint{5.300959in}{2.662286in}}%
\pgfpathlineto{\pgfqpoint{5.319927in}{2.653029in}}%
\pgfpathlineto{\pgfqpoint{5.334215in}{2.645993in}}%
\pgfpathlineto{\pgfqpoint{5.359199in}{2.633552in}}%
\pgfpathlineto{\pgfqpoint{5.367471in}{2.629394in}}%
\pgfpathlineto{\pgfqpoint{5.397617in}{2.614074in}}%
\pgfpathlineto{\pgfqpoint{5.400728in}{2.612479in}}%
\pgfpathlineto{\pgfqpoint{5.433984in}{2.595230in}}%
\pgfpathlineto{\pgfqpoint{5.435191in}{2.594596in}}%
\pgfpathlineto{\pgfqpoint{5.467240in}{2.577628in}}%
\pgfpathlineto{\pgfqpoint{5.471929in}{2.575119in}}%
\pgfpathlineto{\pgfqpoint{5.500496in}{2.559688in}}%
\pgfpathlineto{\pgfqpoint{5.507908in}{2.555641in}}%
\pgfpathlineto{\pgfqpoint{5.533752in}{2.541398in}}%
\pgfpathlineto{\pgfqpoint{5.543152in}{2.536163in}}%
\pgfpathlineto{\pgfqpoint{5.567008in}{2.522749in}}%
\pgfpathlineto{\pgfqpoint{5.577682in}{2.516686in}}%
\pgfpathlineto{\pgfqpoint{5.600265in}{2.503731in}}%
\pgfpathlineto{\pgfqpoint{5.611520in}{2.497208in}}%
\pgfpathlineto{\pgfqpoint{5.633521in}{2.484332in}}%
\pgfpathlineto{\pgfqpoint{5.644688in}{2.477731in}}%
\pgfpathlineto{\pgfqpoint{5.666777in}{2.464542in}}%
\pgfusepath{stroke}%
\end{pgfscope}%
\begin{pgfscope}%
\pgfpathrectangle{\pgfqpoint{0.711606in}{0.549444in}}{\pgfqpoint{4.955171in}{2.902168in}}%
\pgfusepath{clip}%
\pgfsetbuttcap%
\pgfsetroundjoin%
\pgfsetlinewidth{1.003750pt}%
\definecolor{currentstroke}{rgb}{0.658463,0.178962,0.372748}%
\pgfsetstrokecolor{currentstroke}%
\pgfsetdash{}{0pt}%
\pgfpathmoveto{\pgfqpoint{3.985224in}{3.451613in}}%
\pgfpathlineto{\pgfqpoint{4.003968in}{3.440496in}}%
\pgfpathlineto{\pgfqpoint{4.017923in}{3.432135in}}%
\pgfpathlineto{\pgfqpoint{4.037224in}{3.420457in}}%
\pgfpathlineto{\pgfqpoint{4.049986in}{3.412657in}}%
\pgfpathlineto{\pgfqpoint{4.070480in}{3.400006in}}%
\pgfpathlineto{\pgfqpoint{4.081431in}{3.393180in}}%
\pgfpathlineto{\pgfqpoint{4.103736in}{3.379132in}}%
\pgfpathlineto{\pgfqpoint{4.112275in}{3.373702in}}%
\pgfpathlineto{\pgfqpoint{4.136993in}{3.357822in}}%
\pgfpathlineto{\pgfqpoint{4.142538in}{3.354224in}}%
\pgfpathlineto{\pgfqpoint{4.170249in}{3.336061in}}%
\pgfpathlineto{\pgfqpoint{4.172235in}{3.334747in}}%
\pgfpathlineto{\pgfqpoint{4.201352in}{3.315269in}}%
\pgfpathlineto{\pgfqpoint{4.203505in}{3.313813in}}%
\pgfpathlineto{\pgfqpoint{4.229903in}{3.295791in}}%
\pgfpathlineto{\pgfqpoint{4.236761in}{3.291059in}}%
\pgfpathlineto{\pgfqpoint{4.257934in}{3.276314in}}%
\pgfpathlineto{\pgfqpoint{4.270017in}{3.267808in}}%
\pgfpathlineto{\pgfqpoint{4.285464in}{3.256836in}}%
\pgfpathlineto{\pgfqpoint{4.303273in}{3.244051in}}%
\pgfpathlineto{\pgfqpoint{4.312518in}{3.237359in}}%
\pgfpathlineto{\pgfqpoint{4.336530in}{3.219792in}}%
\pgfpathlineto{\pgfqpoint{4.339122in}{3.217881in}}%
\pgfpathlineto{\pgfqpoint{4.365262in}{3.198403in}}%
\pgfpathlineto{\pgfqpoint{4.369786in}{3.194998in}}%
\pgfpathlineto{\pgfqpoint{4.391028in}{3.178926in}}%
\pgfpathlineto{\pgfqpoint{4.403042in}{3.169766in}}%
\pgfpathlineto{\pgfqpoint{4.416554in}{3.159448in}}%
\pgfpathlineto{\pgfqpoint{4.436298in}{3.144310in}}%
\pgfpathlineto{\pgfqpoint{4.441986in}{3.139970in}}%
\pgfpathlineto{\pgfqpoint{4.467549in}{3.120493in}}%
\pgfpathlineto{\pgfqpoint{4.469554in}{3.118969in}}%
\pgfpathlineto{\pgfqpoint{4.493663in}{3.101015in}}%
\pgfpathlineto{\pgfqpoint{4.502811in}{3.094300in}}%
\pgfpathlineto{\pgfqpoint{4.520802in}{3.081537in}}%
\pgfpathlineto{\pgfqpoint{4.536067in}{3.070942in}}%
\pgfpathlineto{\pgfqpoint{4.549419in}{3.062060in}}%
\pgfpathlineto{\pgfqpoint{4.569323in}{3.049109in}}%
\pgfpathlineto{\pgfqpoint{4.579756in}{3.042582in}}%
\pgfpathlineto{\pgfqpoint{4.602579in}{3.028528in}}%
\pgfpathlineto{\pgfqpoint{4.611627in}{3.023104in}}%
\pgfpathlineto{\pgfqpoint{4.635835in}{3.008702in}}%
\pgfpathlineto{\pgfqpoint{4.644482in}{3.003627in}}%
\pgfpathlineto{\pgfqpoint{4.669091in}{2.989210in}}%
\pgfpathlineto{\pgfqpoint{4.677769in}{2.984149in}}%
\pgfpathlineto{\pgfqpoint{4.702348in}{2.969824in}}%
\pgfpathlineto{\pgfqpoint{4.711201in}{2.964672in}}%
\pgfpathlineto{\pgfqpoint{4.735604in}{2.950537in}}%
\pgfpathlineto{\pgfqpoint{4.744865in}{2.945194in}}%
\pgfpathlineto{\pgfqpoint{4.768860in}{2.931569in}}%
\pgfpathlineto{\pgfqpoint{4.779298in}{2.925716in}}%
\pgfpathlineto{\pgfqpoint{4.802116in}{2.913340in}}%
\pgfpathlineto{\pgfqpoint{4.815545in}{2.906239in}}%
\pgfpathlineto{\pgfqpoint{4.835372in}{2.896255in}}%
\pgfpathlineto{\pgfqpoint{4.854900in}{2.886761in}}%
\pgfpathlineto{\pgfqpoint{4.868629in}{2.880417in}}%
\pgfpathlineto{\pgfqpoint{4.897989in}{2.867283in}}%
\pgfpathlineto{\pgfqpoint{4.901885in}{2.865604in}}%
\pgfpathlineto{\pgfqpoint{4.935141in}{2.851484in}}%
\pgfpathlineto{\pgfqpoint{4.943905in}{2.847806in}}%
\pgfpathlineto{\pgfqpoint{4.968397in}{2.837665in}}%
\pgfpathlineto{\pgfqpoint{4.990984in}{2.828328in}}%
\pgfpathlineto{\pgfqpoint{5.001653in}{2.823930in}}%
\pgfpathlineto{\pgfqpoint{5.034909in}{2.810134in}}%
\pgfpathlineto{\pgfqpoint{5.037973in}{2.808850in}}%
\pgfpathlineto{\pgfqpoint{5.068166in}{2.796148in}}%
\pgfpathlineto{\pgfqpoint{5.084112in}{2.789373in}}%
\pgfpathlineto{\pgfqpoint{5.101422in}{2.781968in}}%
\pgfpathlineto{\pgfqpoint{5.129335in}{2.769895in}}%
\pgfpathlineto{\pgfqpoint{5.134678in}{2.767566in}}%
\pgfpathlineto{\pgfqpoint{5.167934in}{2.752889in}}%
\pgfpathlineto{\pgfqpoint{5.173466in}{2.750418in}}%
\pgfpathlineto{\pgfqpoint{5.201190in}{2.737927in}}%
\pgfpathlineto{\pgfqpoint{5.216522in}{2.730940in}}%
\pgfpathlineto{\pgfqpoint{5.234447in}{2.722701in}}%
\pgfpathlineto{\pgfqpoint{5.258616in}{2.711462in}}%
\pgfpathlineto{\pgfqpoint{5.267703in}{2.707200in}}%
\pgfpathlineto{\pgfqpoint{5.299774in}{2.691985in}}%
\pgfpathlineto{\pgfqpoint{5.300959in}{2.691417in}}%
\pgfpathlineto{\pgfqpoint{5.334215in}{2.675306in}}%
\pgfpathlineto{\pgfqpoint{5.339927in}{2.672507in}}%
\pgfpathlineto{\pgfqpoint{5.367471in}{2.658891in}}%
\pgfpathlineto{\pgfqpoint{5.379199in}{2.653029in}}%
\pgfpathlineto{\pgfqpoint{5.400728in}{2.642172in}}%
\pgfpathlineto{\pgfqpoint{5.417636in}{2.633552in}}%
\pgfpathlineto{\pgfqpoint{5.433984in}{2.625141in}}%
\pgfpathlineto{\pgfqpoint{5.455266in}{2.614074in}}%
\pgfpathlineto{\pgfqpoint{5.467240in}{2.607790in}}%
\pgfpathlineto{\pgfqpoint{5.492114in}{2.594596in}}%
\pgfpathlineto{\pgfqpoint{5.500496in}{2.590109in}}%
\pgfpathlineto{\pgfqpoint{5.528204in}{2.575119in}}%
\pgfpathlineto{\pgfqpoint{5.533752in}{2.572089in}}%
\pgfpathlineto{\pgfqpoint{5.563559in}{2.555641in}}%
\pgfpathlineto{\pgfqpoint{5.567008in}{2.553720in}}%
\pgfpathlineto{\pgfqpoint{5.598203in}{2.536163in}}%
\pgfpathlineto{\pgfqpoint{5.600265in}{2.534992in}}%
\pgfpathlineto{\pgfqpoint{5.632157in}{2.516686in}}%
\pgfpathlineto{\pgfqpoint{5.633521in}{2.515895in}}%
\pgfpathlineto{\pgfqpoint{5.665441in}{2.497208in}}%
\pgfpathlineto{\pgfqpoint{5.666777in}{2.496418in}}%
\pgfusepath{stroke}%
\end{pgfscope}%
\begin{pgfscope}%
\pgfpathrectangle{\pgfqpoint{0.711606in}{0.549444in}}{\pgfqpoint{4.955171in}{2.902168in}}%
\pgfusepath{clip}%
\pgfsetbuttcap%
\pgfsetroundjoin%
\pgfsetlinewidth{1.003750pt}%
\definecolor{currentstroke}{rgb}{0.676638,0.186807,0.363849}%
\pgfsetstrokecolor{currentstroke}%
\pgfsetdash{}{0pt}%
\pgfpathmoveto{\pgfqpoint{4.038716in}{3.451613in}}%
\pgfpathlineto{\pgfqpoint{4.070480in}{3.432399in}}%
\pgfpathlineto{\pgfqpoint{4.070912in}{3.432135in}}%
\pgfpathlineto{\pgfqpoint{4.102482in}{3.412657in}}%
\pgfpathlineto{\pgfqpoint{4.103736in}{3.411876in}}%
\pgfpathlineto{\pgfqpoint{4.133447in}{3.393180in}}%
\pgfpathlineto{\pgfqpoint{4.136993in}{3.390926in}}%
\pgfpathlineto{\pgfqpoint{4.163829in}{3.373702in}}%
\pgfpathlineto{\pgfqpoint{4.170249in}{3.369540in}}%
\pgfpathlineto{\pgfqpoint{4.193646in}{3.354224in}}%
\pgfpathlineto{\pgfqpoint{4.203505in}{3.347705in}}%
\pgfpathlineto{\pgfqpoint{4.222916in}{3.334747in}}%
\pgfpathlineto{\pgfqpoint{4.236761in}{3.325408in}}%
\pgfpathlineto{\pgfqpoint{4.251654in}{3.315269in}}%
\pgfpathlineto{\pgfqpoint{4.270017in}{3.302637in}}%
\pgfpathlineto{\pgfqpoint{4.279879in}{3.295791in}}%
\pgfpathlineto{\pgfqpoint{4.303273in}{3.279382in}}%
\pgfpathlineto{\pgfqpoint{4.307609in}{3.276314in}}%
\pgfpathlineto{\pgfqpoint{4.334845in}{3.256836in}}%
\pgfpathlineto{\pgfqpoint{4.336530in}{3.255618in}}%
\pgfpathlineto{\pgfqpoint{4.361578in}{3.237359in}}%
\pgfpathlineto{\pgfqpoint{4.369786in}{3.231316in}}%
\pgfpathlineto{\pgfqpoint{4.387912in}{3.217881in}}%
\pgfpathlineto{\pgfqpoint{4.403042in}{3.206567in}}%
\pgfpathlineto{\pgfqpoint{4.413912in}{3.198403in}}%
\pgfpathlineto{\pgfqpoint{4.436298in}{3.181478in}}%
\pgfpathlineto{\pgfqpoint{4.439673in}{3.178926in}}%
\pgfpathlineto{\pgfqpoint{4.465329in}{3.159448in}}%
\pgfpathlineto{\pgfqpoint{4.469554in}{3.156235in}}%
\pgfpathlineto{\pgfqpoint{4.491186in}{3.139970in}}%
\pgfpathlineto{\pgfqpoint{4.502811in}{3.131295in}}%
\pgfpathlineto{\pgfqpoint{4.517638in}{3.120493in}}%
\pgfpathlineto{\pgfqpoint{4.536067in}{3.107285in}}%
\pgfpathlineto{\pgfqpoint{4.545144in}{3.101015in}}%
\pgfpathlineto{\pgfqpoint{4.569323in}{3.084678in}}%
\pgfpathlineto{\pgfqpoint{4.574170in}{3.081537in}}%
\pgfpathlineto{\pgfqpoint{4.602579in}{3.063502in}}%
\pgfpathlineto{\pgfqpoint{4.604933in}{3.062060in}}%
\pgfpathlineto{\pgfqpoint{4.635835in}{3.043368in}}%
\pgfpathlineto{\pgfqpoint{4.637164in}{3.042582in}}%
\pgfpathlineto{\pgfqpoint{4.669091in}{3.023788in}}%
\pgfpathlineto{\pgfqpoint{4.670264in}{3.023104in}}%
\pgfpathlineto{\pgfqpoint{4.702348in}{3.004398in}}%
\pgfpathlineto{\pgfqpoint{4.703673in}{3.003627in}}%
\pgfpathlineto{\pgfqpoint{4.735604in}{2.985023in}}%
\pgfpathlineto{\pgfqpoint{4.737103in}{2.984149in}}%
\pgfpathlineto{\pgfqpoint{4.768860in}{2.965678in}}%
\pgfpathlineto{\pgfqpoint{4.770593in}{2.964672in}}%
\pgfpathlineto{\pgfqpoint{4.802116in}{2.946586in}}%
\pgfpathlineto{\pgfqpoint{4.804565in}{2.945194in}}%
\pgfpathlineto{\pgfqpoint{4.835372in}{2.928171in}}%
\pgfpathlineto{\pgfqpoint{4.839911in}{2.925716in}}%
\pgfpathlineto{\pgfqpoint{4.868629in}{2.910875in}}%
\pgfpathlineto{\pgfqpoint{4.877899in}{2.906239in}}%
\pgfpathlineto{\pgfqpoint{4.901885in}{2.894838in}}%
\pgfpathlineto{\pgfqpoint{4.919454in}{2.886761in}}%
\pgfpathlineto{\pgfqpoint{4.935141in}{2.879834in}}%
\pgfpathlineto{\pgfqpoint{4.964203in}{2.867283in}}%
\pgfpathlineto{\pgfqpoint{4.968397in}{2.865513in}}%
\pgfpathlineto{\pgfqpoint{5.001653in}{2.851545in}}%
\pgfpathlineto{\pgfqpoint{5.010578in}{2.847806in}}%
\pgfpathlineto{\pgfqpoint{5.034909in}{2.837655in}}%
\pgfpathlineto{\pgfqpoint{5.057174in}{2.828328in}}%
\pgfpathlineto{\pgfqpoint{5.068166in}{2.823713in}}%
\pgfpathlineto{\pgfqpoint{5.101422in}{2.809629in}}%
\pgfpathlineto{\pgfqpoint{5.103239in}{2.808850in}}%
\pgfpathlineto{\pgfqpoint{5.134678in}{2.795301in}}%
\pgfpathlineto{\pgfqpoint{5.148290in}{2.789373in}}%
\pgfpathlineto{\pgfqpoint{5.167934in}{2.780752in}}%
\pgfpathlineto{\pgfqpoint{5.192399in}{2.769895in}}%
\pgfpathlineto{\pgfqpoint{5.201190in}{2.765962in}}%
\pgfpathlineto{\pgfqpoint{5.234447in}{2.750911in}}%
\pgfpathlineto{\pgfqpoint{5.235524in}{2.750418in}}%
\pgfpathlineto{\pgfqpoint{5.267703in}{2.735548in}}%
\pgfpathlineto{\pgfqpoint{5.277564in}{2.730940in}}%
\pgfpathlineto{\pgfqpoint{5.300959in}{2.719913in}}%
\pgfpathlineto{\pgfqpoint{5.318690in}{2.711462in}}%
\pgfpathlineto{\pgfqpoint{5.334215in}{2.703998in}}%
\pgfpathlineto{\pgfqpoint{5.358930in}{2.691985in}}%
\pgfpathlineto{\pgfqpoint{5.367471in}{2.687796in}}%
\pgfpathlineto{\pgfqpoint{5.398311in}{2.672507in}}%
\pgfpathlineto{\pgfqpoint{5.400728in}{2.671298in}}%
\pgfpathlineto{\pgfqpoint{5.433984in}{2.654477in}}%
\pgfpathlineto{\pgfqpoint{5.436815in}{2.653029in}}%
\pgfpathlineto{\pgfqpoint{5.467240in}{2.637331in}}%
\pgfpathlineto{\pgfqpoint{5.474488in}{2.633552in}}%
\pgfpathlineto{\pgfqpoint{5.500496in}{2.619868in}}%
\pgfpathlineto{\pgfqpoint{5.511395in}{2.614074in}}%
\pgfpathlineto{\pgfqpoint{5.533752in}{2.602080in}}%
\pgfpathlineto{\pgfqpoint{5.547558in}{2.594596in}}%
\pgfpathlineto{\pgfqpoint{5.567008in}{2.583957in}}%
\pgfpathlineto{\pgfqpoint{5.583001in}{2.575119in}}%
\pgfpathlineto{\pgfqpoint{5.600265in}{2.565490in}}%
\pgfpathlineto{\pgfqpoint{5.617744in}{2.555641in}}%
\pgfpathlineto{\pgfqpoint{5.633521in}{2.546669in}}%
\pgfpathlineto{\pgfqpoint{5.651809in}{2.536163in}}%
\pgfpathlineto{\pgfqpoint{5.666777in}{2.527484in}}%
\pgfusepath{stroke}%
\end{pgfscope}%
\begin{pgfscope}%
\pgfpathrectangle{\pgfqpoint{0.711606in}{0.549444in}}{\pgfqpoint{4.955171in}{2.902168in}}%
\pgfusepath{clip}%
\pgfsetbuttcap%
\pgfsetroundjoin%
\pgfsetlinewidth{1.003750pt}%
\definecolor{currentstroke}{rgb}{0.700576,0.197851,0.351113}%
\pgfsetstrokecolor{currentstroke}%
\pgfsetdash{}{0pt}%
\pgfpathmoveto{\pgfqpoint{4.090859in}{3.451613in}}%
\pgfpathlineto{\pgfqpoint{4.103736in}{3.443749in}}%
\pgfpathlineto{\pgfqpoint{4.122573in}{3.432135in}}%
\pgfpathlineto{\pgfqpoint{4.136993in}{3.423157in}}%
\pgfpathlineto{\pgfqpoint{4.153696in}{3.412657in}}%
\pgfpathlineto{\pgfqpoint{4.170249in}{3.402149in}}%
\pgfpathlineto{\pgfqpoint{4.184245in}{3.393180in}}%
\pgfpathlineto{\pgfqpoint{4.203505in}{3.380713in}}%
\pgfpathlineto{\pgfqpoint{4.214236in}{3.373702in}}%
\pgfpathlineto{\pgfqpoint{4.236761in}{3.358837in}}%
\pgfpathlineto{\pgfqpoint{4.243687in}{3.354224in}}%
\pgfpathlineto{\pgfqpoint{4.270017in}{3.336509in}}%
\pgfpathlineto{\pgfqpoint{4.272613in}{3.334747in}}%
\pgfpathlineto{\pgfqpoint{4.301000in}{3.315269in}}%
\pgfpathlineto{\pgfqpoint{4.303273in}{3.313692in}}%
\pgfpathlineto{\pgfqpoint{4.328858in}{3.295791in}}%
\pgfpathlineto{\pgfqpoint{4.336530in}{3.290369in}}%
\pgfpathlineto{\pgfqpoint{4.356247in}{3.276314in}}%
\pgfpathlineto{\pgfqpoint{4.369786in}{3.266566in}}%
\pgfpathlineto{\pgfqpoint{4.383198in}{3.256836in}}%
\pgfpathlineto{\pgfqpoint{4.403042in}{3.242303in}}%
\pgfpathlineto{\pgfqpoint{4.409753in}{3.237359in}}%
\pgfpathlineto{\pgfqpoint{4.435970in}{3.217881in}}%
\pgfpathlineto{\pgfqpoint{4.436298in}{3.217635in}}%
\pgfpathlineto{\pgfqpoint{4.461902in}{3.198403in}}%
\pgfpathlineto{\pgfqpoint{4.469554in}{3.192631in}}%
\pgfpathlineto{\pgfqpoint{4.487803in}{3.178926in}}%
\pgfpathlineto{\pgfqpoint{4.502811in}{3.167672in}}%
\pgfpathlineto{\pgfqpoint{4.513936in}{3.159448in}}%
\pgfpathlineto{\pgfqpoint{4.536067in}{3.143249in}}%
\pgfpathlineto{\pgfqpoint{4.540671in}{3.139970in}}%
\pgfpathlineto{\pgfqpoint{4.568507in}{3.120493in}}%
\pgfpathlineto{\pgfqpoint{4.569323in}{3.119931in}}%
\pgfpathlineto{\pgfqpoint{4.597974in}{3.101015in}}%
\pgfpathlineto{\pgfqpoint{4.602579in}{3.098039in}}%
\pgfpathlineto{\pgfqpoint{4.629126in}{3.081537in}}%
\pgfpathlineto{\pgfqpoint{4.635835in}{3.077431in}}%
\pgfpathlineto{\pgfqpoint{4.661635in}{3.062060in}}%
\pgfpathlineto{\pgfqpoint{4.669091in}{3.057650in}}%
\pgfpathlineto{\pgfqpoint{4.694913in}{3.042582in}}%
\pgfpathlineto{\pgfqpoint{4.702348in}{3.038248in}}%
\pgfpathlineto{\pgfqpoint{4.728420in}{3.023104in}}%
\pgfpathlineto{\pgfqpoint{4.735604in}{3.018924in}}%
\pgfpathlineto{\pgfqpoint{4.761864in}{3.003627in}}%
\pgfpathlineto{\pgfqpoint{4.768860in}{2.999548in}}%
\pgfpathlineto{\pgfqpoint{4.795241in}{2.984149in}}%
\pgfpathlineto{\pgfqpoint{4.802116in}{2.980157in}}%
\pgfpathlineto{\pgfqpoint{4.828875in}{2.964672in}}%
\pgfpathlineto{\pgfqpoint{4.835372in}{2.960976in}}%
\pgfpathlineto{\pgfqpoint{4.863498in}{2.945194in}}%
\pgfpathlineto{\pgfqpoint{4.868629in}{2.942414in}}%
\pgfpathlineto{\pgfqpoint{4.900261in}{2.925716in}}%
\pgfpathlineto{\pgfqpoint{4.901885in}{2.924900in}}%
\pgfpathlineto{\pgfqpoint{4.935141in}{2.908675in}}%
\pgfpathlineto{\pgfqpoint{4.940298in}{2.906239in}}%
\pgfpathlineto{\pgfqpoint{4.968397in}{2.893521in}}%
\pgfpathlineto{\pgfqpoint{4.983706in}{2.886761in}}%
\pgfpathlineto{\pgfqpoint{5.001653in}{2.879041in}}%
\pgfpathlineto{\pgfqpoint{5.029288in}{2.867283in}}%
\pgfpathlineto{\pgfqpoint{5.034909in}{2.864918in}}%
\pgfpathlineto{\pgfqpoint{5.068166in}{2.850898in}}%
\pgfpathlineto{\pgfqpoint{5.075470in}{2.847806in}}%
\pgfpathlineto{\pgfqpoint{5.101422in}{2.836807in}}%
\pgfpathlineto{\pgfqpoint{5.121279in}{2.828328in}}%
\pgfpathlineto{\pgfqpoint{5.134678in}{2.822578in}}%
\pgfpathlineto{\pgfqpoint{5.166359in}{2.808850in}}%
\pgfpathlineto{\pgfqpoint{5.167934in}{2.808163in}}%
\pgfpathlineto{\pgfqpoint{5.201190in}{2.793477in}}%
\pgfpathlineto{\pgfqpoint{5.210383in}{2.789373in}}%
\pgfpathlineto{\pgfqpoint{5.234447in}{2.778544in}}%
\pgfpathlineto{\pgfqpoint{5.253453in}{2.769895in}}%
\pgfpathlineto{\pgfqpoint{5.267703in}{2.763358in}}%
\pgfpathlineto{\pgfqpoint{5.295597in}{2.750418in}}%
\pgfpathlineto{\pgfqpoint{5.300959in}{2.747909in}}%
\pgfpathlineto{\pgfqpoint{5.334215in}{2.732173in}}%
\pgfpathlineto{\pgfqpoint{5.336791in}{2.730940in}}%
\pgfpathlineto{\pgfqpoint{5.367471in}{2.716127in}}%
\pgfpathlineto{\pgfqpoint{5.377031in}{2.711462in}}%
\pgfpathlineto{\pgfqpoint{5.400728in}{2.699797in}}%
\pgfpathlineto{\pgfqpoint{5.416430in}{2.691985in}}%
\pgfpathlineto{\pgfqpoint{5.433984in}{2.683174in}}%
\pgfpathlineto{\pgfqpoint{5.455015in}{2.672507in}}%
\pgfpathlineto{\pgfqpoint{5.467240in}{2.666251in}}%
\pgfpathlineto{\pgfqpoint{5.492810in}{2.653029in}}%
\pgfpathlineto{\pgfqpoint{5.500496in}{2.649019in}}%
\pgfpathlineto{\pgfqpoint{5.529840in}{2.633552in}}%
\pgfpathlineto{\pgfqpoint{5.533752in}{2.631471in}}%
\pgfpathlineto{\pgfqpoint{5.566128in}{2.614074in}}%
\pgfpathlineto{\pgfqpoint{5.567008in}{2.613597in}}%
\pgfpathlineto{\pgfqpoint{5.600265in}{2.595377in}}%
\pgfpathlineto{\pgfqpoint{5.601675in}{2.594596in}}%
\pgfpathlineto{\pgfqpoint{5.633521in}{2.576811in}}%
\pgfpathlineto{\pgfqpoint{5.636522in}{2.575119in}}%
\pgfpathlineto{\pgfqpoint{5.666777in}{2.557897in}}%
\pgfusepath{stroke}%
\end{pgfscope}%
\begin{pgfscope}%
\pgfpathrectangle{\pgfqpoint{0.711606in}{0.549444in}}{\pgfqpoint{4.955171in}{2.902168in}}%
\pgfusepath{clip}%
\pgfsetbuttcap%
\pgfsetroundjoin%
\pgfsetlinewidth{1.003750pt}%
\definecolor{currentstroke}{rgb}{0.724103,0.209670,0.337424}%
\pgfsetstrokecolor{currentstroke}%
\pgfsetdash{}{0pt}%
\pgfpathmoveto{\pgfqpoint{4.141912in}{3.451613in}}%
\pgfpathlineto{\pgfqpoint{4.170249in}{3.433985in}}%
\pgfpathlineto{\pgfqpoint{4.173194in}{3.432135in}}%
\pgfpathlineto{\pgfqpoint{4.203505in}{3.412917in}}%
\pgfpathlineto{\pgfqpoint{4.203911in}{3.412657in}}%
\pgfpathlineto{\pgfqpoint{4.234039in}{3.393180in}}%
\pgfpathlineto{\pgfqpoint{4.236761in}{3.391402in}}%
\pgfpathlineto{\pgfqpoint{4.263621in}{3.373702in}}%
\pgfpathlineto{\pgfqpoint{4.270017in}{3.369445in}}%
\pgfpathlineto{\pgfqpoint{4.292679in}{3.354224in}}%
\pgfpathlineto{\pgfqpoint{4.303273in}{3.347037in}}%
\pgfpathlineto{\pgfqpoint{4.321230in}{3.334747in}}%
\pgfpathlineto{\pgfqpoint{4.336530in}{3.324169in}}%
\pgfpathlineto{\pgfqpoint{4.349293in}{3.315269in}}%
\pgfpathlineto{\pgfqpoint{4.369786in}{3.300836in}}%
\pgfpathlineto{\pgfqpoint{4.376891in}{3.295791in}}%
\pgfpathlineto{\pgfqpoint{4.403042in}{3.277044in}}%
\pgfpathlineto{\pgfqpoint{4.404054in}{3.276314in}}%
\pgfpathlineto{\pgfqpoint{4.430760in}{3.256836in}}%
\pgfpathlineto{\pgfqpoint{4.436298in}{3.252762in}}%
\pgfpathlineto{\pgfqpoint{4.457138in}{3.237359in}}%
\pgfpathlineto{\pgfqpoint{4.469554in}{3.228121in}}%
\pgfpathlineto{\pgfqpoint{4.483311in}{3.217881in}}%
\pgfpathlineto{\pgfqpoint{4.502811in}{3.203326in}}%
\pgfpathlineto{\pgfqpoint{4.509450in}{3.198403in}}%
\pgfpathlineto{\pgfqpoint{4.535811in}{3.178926in}}%
\pgfpathlineto{\pgfqpoint{4.536067in}{3.178737in}}%
\pgfpathlineto{\pgfqpoint{4.562855in}{3.159448in}}%
\pgfpathlineto{\pgfqpoint{4.569323in}{3.154861in}}%
\pgfpathlineto{\pgfqpoint{4.591071in}{3.139970in}}%
\pgfpathlineto{\pgfqpoint{4.602579in}{3.132255in}}%
\pgfpathlineto{\pgfqpoint{4.620858in}{3.120493in}}%
\pgfpathlineto{\pgfqpoint{4.635835in}{3.111043in}}%
\pgfpathlineto{\pgfqpoint{4.652303in}{3.101015in}}%
\pgfpathlineto{\pgfqpoint{4.669091in}{3.090921in}}%
\pgfpathlineto{\pgfqpoint{4.685053in}{3.081537in}}%
\pgfpathlineto{\pgfqpoint{4.702348in}{3.071419in}}%
\pgfpathlineto{\pgfqpoint{4.718500in}{3.062060in}}%
\pgfpathlineto{\pgfqpoint{4.735604in}{3.052142in}}%
\pgfpathlineto{\pgfqpoint{4.752111in}{3.042582in}}%
\pgfpathlineto{\pgfqpoint{4.768860in}{3.032851in}}%
\pgfpathlineto{\pgfqpoint{4.785591in}{3.023104in}}%
\pgfpathlineto{\pgfqpoint{4.802116in}{3.013455in}}%
\pgfpathlineto{\pgfqpoint{4.818903in}{3.003627in}}%
\pgfpathlineto{\pgfqpoint{4.835372in}{2.994010in}}%
\pgfpathlineto{\pgfqpoint{4.852284in}{2.984149in}}%
\pgfpathlineto{\pgfqpoint{4.868629in}{2.974747in}}%
\pgfpathlineto{\pgfqpoint{4.886325in}{2.964672in}}%
\pgfpathlineto{\pgfqpoint{4.901885in}{2.956076in}}%
\pgfpathlineto{\pgfqpoint{4.922036in}{2.945194in}}%
\pgfpathlineto{\pgfqpoint{4.935141in}{2.938435in}}%
\pgfpathlineto{\pgfqpoint{4.960612in}{2.925716in}}%
\pgfpathlineto{\pgfqpoint{4.968397in}{2.922015in}}%
\pgfpathlineto{\pgfqpoint{5.001653in}{2.906671in}}%
\pgfpathlineto{\pgfqpoint{5.002609in}{2.906239in}}%
\pgfpathlineto{\pgfqpoint{5.034909in}{2.892071in}}%
\pgfpathlineto{\pgfqpoint{5.047180in}{2.886761in}}%
\pgfpathlineto{\pgfqpoint{5.068166in}{2.877804in}}%
\pgfpathlineto{\pgfqpoint{5.092867in}{2.867283in}}%
\pgfpathlineto{\pgfqpoint{5.101422in}{2.863651in}}%
\pgfpathlineto{\pgfqpoint{5.134678in}{2.849448in}}%
\pgfpathlineto{\pgfqpoint{5.138492in}{2.847806in}}%
\pgfpathlineto{\pgfqpoint{5.167934in}{2.835074in}}%
\pgfpathlineto{\pgfqpoint{5.183390in}{2.828328in}}%
\pgfpathlineto{\pgfqpoint{5.201190in}{2.820509in}}%
\pgfpathlineto{\pgfqpoint{5.227461in}{2.808850in}}%
\pgfpathlineto{\pgfqpoint{5.234447in}{2.805727in}}%
\pgfpathlineto{\pgfqpoint{5.267703in}{2.790690in}}%
\pgfpathlineto{\pgfqpoint{5.270582in}{2.789373in}}%
\pgfpathlineto{\pgfqpoint{5.300959in}{2.775364in}}%
\pgfpathlineto{\pgfqpoint{5.312690in}{2.769895in}}%
\pgfpathlineto{\pgfqpoint{5.334215in}{2.759777in}}%
\pgfpathlineto{\pgfqpoint{5.353913in}{2.750418in}}%
\pgfpathlineto{\pgfqpoint{5.367471in}{2.743921in}}%
\pgfpathlineto{\pgfqpoint{5.394274in}{2.730940in}}%
\pgfpathlineto{\pgfqpoint{5.400728in}{2.727788in}}%
\pgfpathlineto{\pgfqpoint{5.433798in}{2.711462in}}%
\pgfpathlineto{\pgfqpoint{5.433984in}{2.711370in}}%
\pgfpathlineto{\pgfqpoint{5.467240in}{2.694624in}}%
\pgfpathlineto{\pgfqpoint{5.472427in}{2.691985in}}%
\pgfpathlineto{\pgfqpoint{5.500496in}{2.677579in}}%
\pgfpathlineto{\pgfqpoint{5.510279in}{2.672507in}}%
\pgfpathlineto{\pgfqpoint{5.533752in}{2.660230in}}%
\pgfpathlineto{\pgfqpoint{5.547381in}{2.653029in}}%
\pgfpathlineto{\pgfqpoint{5.567008in}{2.642567in}}%
\pgfpathlineto{\pgfqpoint{5.583754in}{2.633552in}}%
\pgfpathlineto{\pgfqpoint{5.600265in}{2.624583in}}%
\pgfpathlineto{\pgfqpoint{5.619420in}{2.614074in}}%
\pgfpathlineto{\pgfqpoint{5.633521in}{2.606268in}}%
\pgfpathlineto{\pgfqpoint{5.654400in}{2.594596in}}%
\pgfpathlineto{\pgfqpoint{5.666777in}{2.587614in}}%
\pgfusepath{stroke}%
\end{pgfscope}%
\begin{pgfscope}%
\pgfpathrectangle{\pgfqpoint{0.711606in}{0.549444in}}{\pgfqpoint{4.955171in}{2.902168in}}%
\pgfusepath{clip}%
\pgfsetbuttcap%
\pgfsetroundjoin%
\pgfsetlinewidth{1.003750pt}%
\definecolor{currentstroke}{rgb}{0.741423,0.219112,0.326576}%
\pgfsetstrokecolor{currentstroke}%
\pgfsetdash{}{0pt}%
\pgfpathmoveto{\pgfqpoint{4.191824in}{3.451613in}}%
\pgfpathlineto{\pgfqpoint{4.203505in}{3.444280in}}%
\pgfpathlineto{\pgfqpoint{4.222676in}{3.432135in}}%
\pgfpathlineto{\pgfqpoint{4.236761in}{3.423126in}}%
\pgfpathlineto{\pgfqpoint{4.252979in}{3.412657in}}%
\pgfpathlineto{\pgfqpoint{4.270017in}{3.401552in}}%
\pgfpathlineto{\pgfqpoint{4.282747in}{3.393180in}}%
\pgfpathlineto{\pgfqpoint{4.303273in}{3.379547in}}%
\pgfpathlineto{\pgfqpoint{4.311997in}{3.373702in}}%
\pgfpathlineto{\pgfqpoint{4.336530in}{3.357102in}}%
\pgfpathlineto{\pgfqpoint{4.340745in}{3.354224in}}%
\pgfpathlineto{\pgfqpoint{4.369000in}{3.334747in}}%
\pgfpathlineto{\pgfqpoint{4.369786in}{3.334200in}}%
\pgfpathlineto{\pgfqpoint{4.396736in}{3.315269in}}%
\pgfpathlineto{\pgfqpoint{4.403042in}{3.310796in}}%
\pgfpathlineto{\pgfqpoint{4.424040in}{3.295791in}}%
\pgfpathlineto{\pgfqpoint{4.436298in}{3.286951in}}%
\pgfpathlineto{\pgfqpoint{4.450958in}{3.276314in}}%
\pgfpathlineto{\pgfqpoint{4.469554in}{3.262713in}}%
\pgfpathlineto{\pgfqpoint{4.477561in}{3.256836in}}%
\pgfpathlineto{\pgfqpoint{4.502811in}{3.238199in}}%
\pgfpathlineto{\pgfqpoint{4.503950in}{3.237359in}}%
\pgfpathlineto{\pgfqpoint{4.530281in}{3.217881in}}%
\pgfpathlineto{\pgfqpoint{4.536067in}{3.213604in}}%
\pgfpathlineto{\pgfqpoint{4.556900in}{3.198403in}}%
\pgfpathlineto{\pgfqpoint{4.569323in}{3.189419in}}%
\pgfpathlineto{\pgfqpoint{4.584212in}{3.178926in}}%
\pgfpathlineto{\pgfqpoint{4.602579in}{3.166199in}}%
\pgfpathlineto{\pgfqpoint{4.612692in}{3.159448in}}%
\pgfpathlineto{\pgfqpoint{4.635835in}{3.144318in}}%
\pgfpathlineto{\pgfqpoint{4.642755in}{3.139970in}}%
\pgfpathlineto{\pgfqpoint{4.669091in}{3.123716in}}%
\pgfpathlineto{\pgfqpoint{4.674481in}{3.120493in}}%
\pgfpathlineto{\pgfqpoint{4.702348in}{3.103994in}}%
\pgfpathlineto{\pgfqpoint{4.707472in}{3.101015in}}%
\pgfpathlineto{\pgfqpoint{4.735604in}{3.084701in}}%
\pgfpathlineto{\pgfqpoint{4.741095in}{3.081537in}}%
\pgfpathlineto{\pgfqpoint{4.768860in}{3.065505in}}%
\pgfpathlineto{\pgfqpoint{4.774822in}{3.062060in}}%
\pgfpathlineto{\pgfqpoint{4.802116in}{3.046220in}}%
\pgfpathlineto{\pgfqpoint{4.808360in}{3.042582in}}%
\pgfpathlineto{\pgfqpoint{4.835372in}{3.026787in}}%
\pgfpathlineto{\pgfqpoint{4.841647in}{3.023104in}}%
\pgfpathlineto{\pgfqpoint{4.868629in}{3.007278in}}%
\pgfpathlineto{\pgfqpoint{4.874851in}{3.003627in}}%
\pgfpathlineto{\pgfqpoint{4.901885in}{2.987925in}}%
\pgfpathlineto{\pgfqpoint{4.908434in}{2.984149in}}%
\pgfpathlineto{\pgfqpoint{4.935141in}{2.969142in}}%
\pgfpathlineto{\pgfqpoint{4.943251in}{2.964672in}}%
\pgfpathlineto{\pgfqpoint{4.968397in}{2.951385in}}%
\pgfpathlineto{\pgfqpoint{4.980474in}{2.945194in}}%
\pgfpathlineto{\pgfqpoint{5.001653in}{2.934855in}}%
\pgfpathlineto{\pgfqpoint{5.020970in}{2.925716in}}%
\pgfpathlineto{\pgfqpoint{5.034909in}{2.919378in}}%
\pgfpathlineto{\pgfqpoint{5.064436in}{2.906239in}}%
\pgfpathlineto{\pgfqpoint{5.068166in}{2.904617in}}%
\pgfpathlineto{\pgfqpoint{5.101422in}{2.890232in}}%
\pgfpathlineto{\pgfqpoint{5.109469in}{2.886761in}}%
\pgfpathlineto{\pgfqpoint{5.134678in}{2.875940in}}%
\pgfpathlineto{\pgfqpoint{5.154774in}{2.867283in}}%
\pgfpathlineto{\pgfqpoint{5.167934in}{2.861603in}}%
\pgfpathlineto{\pgfqpoint{5.199659in}{2.847806in}}%
\pgfpathlineto{\pgfqpoint{5.201190in}{2.847136in}}%
\pgfpathlineto{\pgfqpoint{5.234447in}{2.832435in}}%
\pgfpathlineto{\pgfqpoint{5.243645in}{2.828328in}}%
\pgfpathlineto{\pgfqpoint{5.267703in}{2.817508in}}%
\pgfpathlineto{\pgfqpoint{5.286752in}{2.808850in}}%
\pgfpathlineto{\pgfqpoint{5.300959in}{2.802343in}}%
\pgfpathlineto{\pgfqpoint{5.328976in}{2.789373in}}%
\pgfpathlineto{\pgfqpoint{5.334215in}{2.786927in}}%
\pgfpathlineto{\pgfqpoint{5.367471in}{2.771234in}}%
\pgfpathlineto{\pgfqpoint{5.370277in}{2.769895in}}%
\pgfpathlineto{\pgfqpoint{5.400728in}{2.755244in}}%
\pgfpathlineto{\pgfqpoint{5.410653in}{2.750418in}}%
\pgfpathlineto{\pgfqpoint{5.433984in}{2.738978in}}%
\pgfpathlineto{\pgfqpoint{5.450209in}{2.730940in}}%
\pgfpathlineto{\pgfqpoint{5.467240in}{2.722430in}}%
\pgfpathlineto{\pgfqpoint{5.488968in}{2.711462in}}%
\pgfpathlineto{\pgfqpoint{5.500496in}{2.705593in}}%
\pgfpathlineto{\pgfqpoint{5.526955in}{2.691985in}}%
\pgfpathlineto{\pgfqpoint{5.533752in}{2.688458in}}%
\pgfpathlineto{\pgfqpoint{5.564192in}{2.672507in}}%
\pgfpathlineto{\pgfqpoint{5.567008in}{2.671018in}}%
\pgfpathlineto{\pgfqpoint{5.600265in}{2.653262in}}%
\pgfpathlineto{\pgfqpoint{5.600695in}{2.653029in}}%
\pgfpathlineto{\pgfqpoint{5.633521in}{2.635168in}}%
\pgfpathlineto{\pgfqpoint{5.636462in}{2.633552in}}%
\pgfpathlineto{\pgfqpoint{5.666777in}{2.616747in}}%
\pgfusepath{stroke}%
\end{pgfscope}%
\begin{pgfscope}%
\pgfpathrectangle{\pgfqpoint{0.711606in}{0.549444in}}{\pgfqpoint{4.955171in}{2.902168in}}%
\pgfusepath{clip}%
\pgfsetbuttcap%
\pgfsetroundjoin%
\pgfsetlinewidth{1.003750pt}%
\definecolor{currentstroke}{rgb}{0.764010,0.232554,0.311399}%
\pgfsetstrokecolor{currentstroke}%
\pgfsetdash{}{0pt}%
\pgfpathmoveto{\pgfqpoint{4.240763in}{3.451613in}}%
\pgfpathlineto{\pgfqpoint{4.270017in}{3.432926in}}%
\pgfpathlineto{\pgfqpoint{4.271245in}{3.432135in}}%
\pgfpathlineto{\pgfqpoint{4.301170in}{3.412657in}}%
\pgfpathlineto{\pgfqpoint{4.303273in}{3.411275in}}%
\pgfpathlineto{\pgfqpoint{4.330561in}{3.393180in}}%
\pgfpathlineto{\pgfqpoint{4.336530in}{3.389183in}}%
\pgfpathlineto{\pgfqpoint{4.359450in}{3.373702in}}%
\pgfpathlineto{\pgfqpoint{4.369786in}{3.366653in}}%
\pgfpathlineto{\pgfqpoint{4.387857in}{3.354224in}}%
\pgfpathlineto{\pgfqpoint{4.403042in}{3.343680in}}%
\pgfpathlineto{\pgfqpoint{4.415804in}{3.334747in}}%
\pgfpathlineto{\pgfqpoint{4.436298in}{3.320268in}}%
\pgfpathlineto{\pgfqpoint{4.443323in}{3.315269in}}%
\pgfpathlineto{\pgfqpoint{4.469554in}{3.296442in}}%
\pgfpathlineto{\pgfqpoint{4.470456in}{3.295791in}}%
\pgfpathlineto{\pgfqpoint{4.497216in}{3.276314in}}%
\pgfpathlineto{\pgfqpoint{4.502811in}{3.272214in}}%
\pgfpathlineto{\pgfqpoint{4.523779in}{3.256836in}}%
\pgfpathlineto{\pgfqpoint{4.536067in}{3.247797in}}%
\pgfpathlineto{\pgfqpoint{4.550340in}{3.237359in}}%
\pgfpathlineto{\pgfqpoint{4.569323in}{3.223516in}}%
\pgfpathlineto{\pgfqpoint{4.577178in}{3.217881in}}%
\pgfpathlineto{\pgfqpoint{4.602579in}{3.199857in}}%
\pgfpathlineto{\pgfqpoint{4.604689in}{3.198403in}}%
\pgfpathlineto{\pgfqpoint{4.633421in}{3.178926in}}%
\pgfpathlineto{\pgfqpoint{4.635835in}{3.177319in}}%
\pgfpathlineto{\pgfqpoint{4.663783in}{3.159448in}}%
\pgfpathlineto{\pgfqpoint{4.669091in}{3.156118in}}%
\pgfpathlineto{\pgfqpoint{4.695761in}{3.139970in}}%
\pgfpathlineto{\pgfqpoint{4.702348in}{3.136033in}}%
\pgfpathlineto{\pgfqpoint{4.728942in}{3.120493in}}%
\pgfpathlineto{\pgfqpoint{4.735604in}{3.116619in}}%
\pgfpathlineto{\pgfqpoint{4.762710in}{3.101015in}}%
\pgfpathlineto{\pgfqpoint{4.768860in}{3.097472in}}%
\pgfpathlineto{\pgfqpoint{4.796549in}{3.081537in}}%
\pgfpathlineto{\pgfqpoint{4.802116in}{3.078321in}}%
\pgfpathlineto{\pgfqpoint{4.830161in}{3.062060in}}%
\pgfpathlineto{\pgfqpoint{4.835372in}{3.059023in}}%
\pgfpathlineto{\pgfqpoint{4.863458in}{3.042582in}}%
\pgfpathlineto{\pgfqpoint{4.868629in}{3.039547in}}%
\pgfpathlineto{\pgfqpoint{4.896554in}{3.023104in}}%
\pgfpathlineto{\pgfqpoint{4.901885in}{3.019977in}}%
\pgfpathlineto{\pgfqpoint{4.929805in}{3.003627in}}%
\pgfpathlineto{\pgfqpoint{4.935141in}{3.000550in}}%
\pgfpathlineto{\pgfqpoint{4.963908in}{2.984149in}}%
\pgfpathlineto{\pgfqpoint{4.968397in}{2.981671in}}%
\pgfpathlineto{\pgfqpoint{4.999931in}{2.964672in}}%
\pgfpathlineto{\pgfqpoint{5.001653in}{2.963785in}}%
\pgfpathlineto{\pgfqpoint{5.034909in}{2.947150in}}%
\pgfpathlineto{\pgfqpoint{5.038940in}{2.945194in}}%
\pgfpathlineto{\pgfqpoint{5.068166in}{2.931593in}}%
\pgfpathlineto{\pgfqpoint{5.081097in}{2.925716in}}%
\pgfpathlineto{\pgfqpoint{5.101422in}{2.916719in}}%
\pgfpathlineto{\pgfqpoint{5.125357in}{2.906239in}}%
\pgfpathlineto{\pgfqpoint{5.134678in}{2.902204in}}%
\pgfpathlineto{\pgfqpoint{5.167934in}{2.887806in}}%
\pgfpathlineto{\pgfqpoint{5.170336in}{2.886761in}}%
\pgfpathlineto{\pgfqpoint{5.201190in}{2.873331in}}%
\pgfpathlineto{\pgfqpoint{5.214990in}{2.867283in}}%
\pgfpathlineto{\pgfqpoint{5.234447in}{2.858717in}}%
\pgfpathlineto{\pgfqpoint{5.259005in}{2.847806in}}%
\pgfpathlineto{\pgfqpoint{5.267703in}{2.843916in}}%
\pgfpathlineto{\pgfqpoint{5.300959in}{2.828891in}}%
\pgfpathlineto{\pgfqpoint{5.302192in}{2.828328in}}%
\pgfpathlineto{\pgfqpoint{5.334215in}{2.813584in}}%
\pgfpathlineto{\pgfqpoint{5.344391in}{2.808850in}}%
\pgfpathlineto{\pgfqpoint{5.367471in}{2.798028in}}%
\pgfpathlineto{\pgfqpoint{5.385739in}{2.789373in}}%
\pgfpathlineto{\pgfqpoint{5.400728in}{2.782213in}}%
\pgfpathlineto{\pgfqpoint{5.426248in}{2.769895in}}%
\pgfpathlineto{\pgfqpoint{5.433984in}{2.766131in}}%
\pgfpathlineto{\pgfqpoint{5.465941in}{2.750418in}}%
\pgfpathlineto{\pgfqpoint{5.467240in}{2.749773in}}%
\pgfpathlineto{\pgfqpoint{5.500496in}{2.733105in}}%
\pgfpathlineto{\pgfqpoint{5.504773in}{2.730940in}}%
\pgfpathlineto{\pgfqpoint{5.533752in}{2.716143in}}%
\pgfpathlineto{\pgfqpoint{5.542827in}{2.711462in}}%
\pgfpathlineto{\pgfqpoint{5.567008in}{2.698885in}}%
\pgfpathlineto{\pgfqpoint{5.580146in}{2.691985in}}%
\pgfpathlineto{\pgfqpoint{5.600265in}{2.681326in}}%
\pgfpathlineto{\pgfqpoint{5.616750in}{2.672507in}}%
\pgfpathlineto{\pgfqpoint{5.633521in}{2.663457in}}%
\pgfpathlineto{\pgfqpoint{5.652659in}{2.653029in}}%
\pgfpathlineto{\pgfqpoint{5.666777in}{2.645270in}}%
\pgfusepath{stroke}%
\end{pgfscope}%
\begin{pgfscope}%
\pgfpathrectangle{\pgfqpoint{0.711606in}{0.549444in}}{\pgfqpoint{4.955171in}{2.902168in}}%
\pgfusepath{clip}%
\pgfsetbuttcap%
\pgfsetroundjoin%
\pgfsetlinewidth{1.003750pt}%
\definecolor{currentstroke}{rgb}{0.785929,0.247056,0.295477}%
\pgfsetstrokecolor{currentstroke}%
\pgfsetdash{}{0pt}%
\pgfpathmoveto{\pgfqpoint{4.288681in}{3.451613in}}%
\pgfpathlineto{\pgfqpoint{4.303273in}{3.442212in}}%
\pgfpathlineto{\pgfqpoint{4.318780in}{3.432135in}}%
\pgfpathlineto{\pgfqpoint{4.336530in}{3.420490in}}%
\pgfpathlineto{\pgfqpoint{4.348365in}{3.412657in}}%
\pgfpathlineto{\pgfqpoint{4.369786in}{3.398347in}}%
\pgfpathlineto{\pgfqpoint{4.377455in}{3.393180in}}%
\pgfpathlineto{\pgfqpoint{4.403042in}{3.375776in}}%
\pgfpathlineto{\pgfqpoint{4.406067in}{3.373702in}}%
\pgfpathlineto{\pgfqpoint{4.434196in}{3.354224in}}%
\pgfpathlineto{\pgfqpoint{4.436298in}{3.352754in}}%
\pgfpathlineto{\pgfqpoint{4.461860in}{3.334747in}}%
\pgfpathlineto{\pgfqpoint{4.469554in}{3.329278in}}%
\pgfpathlineto{\pgfqpoint{4.489143in}{3.315269in}}%
\pgfpathlineto{\pgfqpoint{4.502811in}{3.305419in}}%
\pgfpathlineto{\pgfqpoint{4.516118in}{3.295791in}}%
\pgfpathlineto{\pgfqpoint{4.536067in}{3.281279in}}%
\pgfpathlineto{\pgfqpoint{4.542897in}{3.276314in}}%
\pgfpathlineto{\pgfqpoint{4.569323in}{3.257073in}}%
\pgfpathlineto{\pgfqpoint{4.569650in}{3.256836in}}%
\pgfpathlineto{\pgfqpoint{4.596682in}{3.237359in}}%
\pgfpathlineto{\pgfqpoint{4.602579in}{3.233142in}}%
\pgfpathlineto{\pgfqpoint{4.624448in}{3.217881in}}%
\pgfpathlineto{\pgfqpoint{4.635835in}{3.210059in}}%
\pgfpathlineto{\pgfqpoint{4.653423in}{3.198403in}}%
\pgfpathlineto{\pgfqpoint{4.669091in}{3.188226in}}%
\pgfpathlineto{\pgfqpoint{4.683981in}{3.178926in}}%
\pgfpathlineto{\pgfqpoint{4.702348in}{3.167652in}}%
\pgfpathlineto{\pgfqpoint{4.716142in}{3.159448in}}%
\pgfpathlineto{\pgfqpoint{4.735604in}{3.147990in}}%
\pgfpathlineto{\pgfqpoint{4.749482in}{3.139970in}}%
\pgfpathlineto{\pgfqpoint{4.768860in}{3.128803in}}%
\pgfpathlineto{\pgfqpoint{4.783381in}{3.120493in}}%
\pgfpathlineto{\pgfqpoint{4.802116in}{3.109745in}}%
\pgfpathlineto{\pgfqpoint{4.817325in}{3.101015in}}%
\pgfpathlineto{\pgfqpoint{4.835372in}{3.090603in}}%
\pgfpathlineto{\pgfqpoint{4.851016in}{3.081537in}}%
\pgfpathlineto{\pgfqpoint{4.868629in}{3.071274in}}%
\pgfpathlineto{\pgfqpoint{4.884353in}{3.062060in}}%
\pgfpathlineto{\pgfqpoint{4.901885in}{3.051748in}}%
\pgfpathlineto{\pgfqpoint{4.917402in}{3.042582in}}%
\pgfpathlineto{\pgfqpoint{4.935141in}{3.032120in}}%
\pgfpathlineto{\pgfqpoint{4.950429in}{3.023104in}}%
\pgfpathlineto{\pgfqpoint{4.968397in}{3.012634in}}%
\pgfpathlineto{\pgfqpoint{4.983988in}{3.003627in}}%
\pgfpathlineto{\pgfqpoint{5.001653in}{2.993701in}}%
\pgfpathlineto{\pgfqpoint{5.019003in}{2.984149in}}%
\pgfpathlineto{\pgfqpoint{5.034909in}{2.975764in}}%
\pgfpathlineto{\pgfqpoint{5.056592in}{2.964672in}}%
\pgfpathlineto{\pgfqpoint{5.068166in}{2.959024in}}%
\pgfpathlineto{\pgfqpoint{5.097352in}{2.945194in}}%
\pgfpathlineto{\pgfqpoint{5.101422in}{2.943334in}}%
\pgfpathlineto{\pgfqpoint{5.134678in}{2.928370in}}%
\pgfpathlineto{\pgfqpoint{5.140644in}{2.925716in}}%
\pgfpathlineto{\pgfqpoint{5.167934in}{2.913744in}}%
\pgfpathlineto{\pgfqpoint{5.185087in}{2.906239in}}%
\pgfpathlineto{\pgfqpoint{5.201190in}{2.899216in}}%
\pgfpathlineto{\pgfqpoint{5.229625in}{2.886761in}}%
\pgfpathlineto{\pgfqpoint{5.234447in}{2.884644in}}%
\pgfpathlineto{\pgfqpoint{5.267703in}{2.869911in}}%
\pgfpathlineto{\pgfqpoint{5.273580in}{2.867283in}}%
\pgfpathlineto{\pgfqpoint{5.300959in}{2.854966in}}%
\pgfpathlineto{\pgfqpoint{5.316723in}{2.847806in}}%
\pgfpathlineto{\pgfqpoint{5.334215in}{2.839803in}}%
\pgfpathlineto{\pgfqpoint{5.359044in}{2.828328in}}%
\pgfpathlineto{\pgfqpoint{5.367471in}{2.824403in}}%
\pgfpathlineto{\pgfqpoint{5.400525in}{2.808850in}}%
\pgfpathlineto{\pgfqpoint{5.400728in}{2.808754in}}%
\pgfpathlineto{\pgfqpoint{5.433984in}{2.792803in}}%
\pgfpathlineto{\pgfqpoint{5.441063in}{2.789373in}}%
\pgfpathlineto{\pgfqpoint{5.467240in}{2.776586in}}%
\pgfpathlineto{\pgfqpoint{5.480798in}{2.769895in}}%
\pgfpathlineto{\pgfqpoint{5.500496in}{2.760095in}}%
\pgfpathlineto{\pgfqpoint{5.519755in}{2.750418in}}%
\pgfpathlineto{\pgfqpoint{5.533752in}{2.743325in}}%
\pgfpathlineto{\pgfqpoint{5.557955in}{2.730940in}}%
\pgfpathlineto{\pgfqpoint{5.567008in}{2.726268in}}%
\pgfpathlineto{\pgfqpoint{5.595420in}{2.711462in}}%
\pgfpathlineto{\pgfqpoint{5.600265in}{2.708916in}}%
\pgfpathlineto{\pgfqpoint{5.632171in}{2.691985in}}%
\pgfpathlineto{\pgfqpoint{5.633521in}{2.691262in}}%
\pgfpathlineto{\pgfqpoint{5.666777in}{2.673287in}}%
\pgfusepath{stroke}%
\end{pgfscope}%
\begin{pgfscope}%
\pgfpathrectangle{\pgfqpoint{0.711606in}{0.549444in}}{\pgfqpoint{4.955171in}{2.902168in}}%
\pgfusepath{clip}%
\pgfsetbuttcap%
\pgfsetroundjoin%
\pgfsetlinewidth{1.003750pt}%
\definecolor{currentstroke}{rgb}{0.801871,0.258674,0.283099}%
\pgfsetstrokecolor{currentstroke}%
\pgfsetdash{}{0pt}%
\pgfpathmoveto{\pgfqpoint{4.335773in}{3.451613in}}%
\pgfpathlineto{\pgfqpoint{4.336530in}{3.451121in}}%
\pgfpathlineto{\pgfqpoint{4.365507in}{3.432135in}}%
\pgfpathlineto{\pgfqpoint{4.369786in}{3.429305in}}%
\pgfpathlineto{\pgfqpoint{4.394744in}{3.412657in}}%
\pgfpathlineto{\pgfqpoint{4.403042in}{3.407070in}}%
\pgfpathlineto{\pgfqpoint{4.423504in}{3.393180in}}%
\pgfpathlineto{\pgfqpoint{4.436298in}{3.384413in}}%
\pgfpathlineto{\pgfqpoint{4.451808in}{3.373702in}}%
\pgfpathlineto{\pgfqpoint{4.469554in}{3.361335in}}%
\pgfpathlineto{\pgfqpoint{4.479687in}{3.354224in}}%
\pgfpathlineto{\pgfqpoint{4.502811in}{3.337860in}}%
\pgfpathlineto{\pgfqpoint{4.507185in}{3.334747in}}%
\pgfpathlineto{\pgfqpoint{4.534351in}{3.315269in}}%
\pgfpathlineto{\pgfqpoint{4.536067in}{3.314030in}}%
\pgfpathlineto{\pgfqpoint{4.561285in}{3.295791in}}%
\pgfpathlineto{\pgfqpoint{4.569323in}{3.289960in}}%
\pgfpathlineto{\pgfqpoint{4.588228in}{3.276314in}}%
\pgfpathlineto{\pgfqpoint{4.602579in}{3.265978in}}%
\pgfpathlineto{\pgfqpoint{4.615466in}{3.256836in}}%
\pgfpathlineto{\pgfqpoint{4.635835in}{3.242530in}}%
\pgfpathlineto{\pgfqpoint{4.643404in}{3.237359in}}%
\pgfpathlineto{\pgfqpoint{4.669091in}{3.220108in}}%
\pgfpathlineto{\pgfqpoint{4.672533in}{3.217881in}}%
\pgfpathlineto{\pgfqpoint{4.702348in}{3.198964in}}%
\pgfpathlineto{\pgfqpoint{4.703265in}{3.198403in}}%
\pgfpathlineto{\pgfqpoint{4.735604in}{3.178937in}}%
\pgfpathlineto{\pgfqpoint{4.735623in}{3.178926in}}%
\pgfpathlineto{\pgfqpoint{4.768860in}{3.159613in}}%
\pgfpathlineto{\pgfqpoint{4.769148in}{3.159448in}}%
\pgfpathlineto{\pgfqpoint{4.802116in}{3.140587in}}%
\pgfpathlineto{\pgfqpoint{4.803199in}{3.139970in}}%
\pgfpathlineto{\pgfqpoint{4.835372in}{3.121578in}}%
\pgfpathlineto{\pgfqpoint{4.837266in}{3.120493in}}%
\pgfpathlineto{\pgfqpoint{4.868629in}{3.102421in}}%
\pgfpathlineto{\pgfqpoint{4.871055in}{3.101015in}}%
\pgfpathlineto{\pgfqpoint{4.901885in}{3.083045in}}%
\pgfpathlineto{\pgfqpoint{4.904455in}{3.081537in}}%
\pgfpathlineto{\pgfqpoint{4.935141in}{3.063456in}}%
\pgfpathlineto{\pgfqpoint{4.937499in}{3.062060in}}%
\pgfpathlineto{\pgfqpoint{4.968397in}{3.043757in}}%
\pgfpathlineto{\pgfqpoint{4.970379in}{3.042582in}}%
\pgfpathlineto{\pgfqpoint{5.001653in}{3.024195in}}%
\pgfpathlineto{\pgfqpoint{5.003520in}{3.023104in}}%
\pgfpathlineto{\pgfqpoint{5.034909in}{3.005188in}}%
\pgfpathlineto{\pgfqpoint{5.037692in}{3.003627in}}%
\pgfpathlineto{\pgfqpoint{5.068166in}{2.987198in}}%
\pgfpathlineto{\pgfqpoint{5.073981in}{2.984149in}}%
\pgfpathlineto{\pgfqpoint{5.101422in}{2.970424in}}%
\pgfpathlineto{\pgfqpoint{5.113270in}{2.964672in}}%
\pgfpathlineto{\pgfqpoint{5.134678in}{2.954675in}}%
\pgfpathlineto{\pgfqpoint{5.155413in}{2.945194in}}%
\pgfpathlineto{\pgfqpoint{5.167934in}{2.939598in}}%
\pgfpathlineto{\pgfqpoint{5.199269in}{2.925716in}}%
\pgfpathlineto{\pgfqpoint{5.201190in}{2.924873in}}%
\pgfpathlineto{\pgfqpoint{5.234447in}{2.910231in}}%
\pgfpathlineto{\pgfqpoint{5.243483in}{2.906239in}}%
\pgfpathlineto{\pgfqpoint{5.267703in}{2.895522in}}%
\pgfpathlineto{\pgfqpoint{5.287361in}{2.886761in}}%
\pgfpathlineto{\pgfqpoint{5.300959in}{2.880672in}}%
\pgfpathlineto{\pgfqpoint{5.330588in}{2.867283in}}%
\pgfpathlineto{\pgfqpoint{5.334215in}{2.865634in}}%
\pgfpathlineto{\pgfqpoint{5.367471in}{2.850347in}}%
\pgfpathlineto{\pgfqpoint{5.372944in}{2.847806in}}%
\pgfpathlineto{\pgfqpoint{5.400728in}{2.834806in}}%
\pgfpathlineto{\pgfqpoint{5.414433in}{2.828328in}}%
\pgfpathlineto{\pgfqpoint{5.433984in}{2.819016in}}%
\pgfpathlineto{\pgfqpoint{5.455112in}{2.808850in}}%
\pgfpathlineto{\pgfqpoint{5.467240in}{2.802968in}}%
\pgfpathlineto{\pgfqpoint{5.494994in}{2.789373in}}%
\pgfpathlineto{\pgfqpoint{5.500496in}{2.786656in}}%
\pgfpathlineto{\pgfqpoint{5.533752in}{2.770068in}}%
\pgfpathlineto{\pgfqpoint{5.534094in}{2.769895in}}%
\pgfpathlineto{\pgfqpoint{5.567008in}{2.753169in}}%
\pgfpathlineto{\pgfqpoint{5.572370in}{2.750418in}}%
\pgfpathlineto{\pgfqpoint{5.600265in}{2.735986in}}%
\pgfpathlineto{\pgfqpoint{5.609924in}{2.730940in}}%
\pgfpathlineto{\pgfqpoint{5.633521in}{2.718511in}}%
\pgfpathlineto{\pgfqpoint{5.646775in}{2.711462in}}%
\pgfpathlineto{\pgfqpoint{5.666777in}{2.700737in}}%
\pgfusepath{stroke}%
\end{pgfscope}%
\begin{pgfscope}%
\pgfpathrectangle{\pgfqpoint{0.711606in}{0.549444in}}{\pgfqpoint{4.955171in}{2.902168in}}%
\pgfusepath{clip}%
\pgfsetbuttcap%
\pgfsetroundjoin%
\pgfsetlinewidth{1.003750pt}%
\definecolor{currentstroke}{rgb}{0.822386,0.275197,0.266085}%
\pgfsetstrokecolor{currentstroke}%
\pgfsetdash{}{0pt}%
\pgfpathmoveto{\pgfqpoint{4.381914in}{3.451613in}}%
\pgfpathlineto{\pgfqpoint{4.403042in}{3.437667in}}%
\pgfpathlineto{\pgfqpoint{4.411353in}{3.432135in}}%
\pgfpathlineto{\pgfqpoint{4.436298in}{3.415379in}}%
\pgfpathlineto{\pgfqpoint{4.440318in}{3.412657in}}%
\pgfpathlineto{\pgfqpoint{4.468822in}{3.393180in}}%
\pgfpathlineto{\pgfqpoint{4.469554in}{3.392674in}}%
\pgfpathlineto{\pgfqpoint{4.496851in}{3.373702in}}%
\pgfpathlineto{\pgfqpoint{4.502811in}{3.369524in}}%
\pgfpathlineto{\pgfqpoint{4.524500in}{3.354224in}}%
\pgfpathlineto{\pgfqpoint{4.536067in}{3.346003in}}%
\pgfpathlineto{\pgfqpoint{4.551841in}{3.334747in}}%
\pgfpathlineto{\pgfqpoint{4.569323in}{3.322203in}}%
\pgfpathlineto{\pgfqpoint{4.578988in}{3.315269in}}%
\pgfpathlineto{\pgfqpoint{4.602579in}{3.298314in}}%
\pgfpathlineto{\pgfqpoint{4.606115in}{3.295791in}}%
\pgfpathlineto{\pgfqpoint{4.633505in}{3.276314in}}%
\pgfpathlineto{\pgfqpoint{4.635835in}{3.274666in}}%
\pgfpathlineto{\pgfqpoint{4.661626in}{3.256836in}}%
\pgfpathlineto{\pgfqpoint{4.669091in}{3.251751in}}%
\pgfpathlineto{\pgfqpoint{4.690959in}{3.237359in}}%
\pgfpathlineto{\pgfqpoint{4.702348in}{3.230006in}}%
\pgfpathlineto{\pgfqpoint{4.721863in}{3.217881in}}%
\pgfpathlineto{\pgfqpoint{4.735604in}{3.209489in}}%
\pgfpathlineto{\pgfqpoint{4.754340in}{3.198403in}}%
\pgfpathlineto{\pgfqpoint{4.768860in}{3.189901in}}%
\pgfpathlineto{\pgfqpoint{4.787962in}{3.178926in}}%
\pgfpathlineto{\pgfqpoint{4.802116in}{3.170818in}}%
\pgfpathlineto{\pgfqpoint{4.822113in}{3.159448in}}%
\pgfpathlineto{\pgfqpoint{4.835372in}{3.151892in}}%
\pgfpathlineto{\pgfqpoint{4.856286in}{3.139970in}}%
\pgfpathlineto{\pgfqpoint{4.868629in}{3.132897in}}%
\pgfpathlineto{\pgfqpoint{4.890176in}{3.120493in}}%
\pgfpathlineto{\pgfqpoint{4.901885in}{3.113709in}}%
\pgfpathlineto{\pgfqpoint{4.923657in}{3.101015in}}%
\pgfpathlineto{\pgfqpoint{4.935141in}{3.094280in}}%
\pgfpathlineto{\pgfqpoint{4.956733in}{3.081537in}}%
\pgfpathlineto{\pgfqpoint{4.968397in}{3.074631in}}%
\pgfpathlineto{\pgfqpoint{4.989542in}{3.062060in}}%
\pgfpathlineto{\pgfqpoint{5.001653in}{3.054877in}}%
\pgfpathlineto{\pgfqpoint{5.022409in}{3.042582in}}%
\pgfpathlineto{\pgfqpoint{5.034909in}{3.035278in}}%
\pgfpathlineto{\pgfqpoint{5.055951in}{3.023104in}}%
\pgfpathlineto{\pgfqpoint{5.068166in}{3.016246in}}%
\pgfpathlineto{\pgfqpoint{5.091130in}{3.003627in}}%
\pgfpathlineto{\pgfqpoint{5.101422in}{2.998215in}}%
\pgfpathlineto{\pgfqpoint{5.128975in}{2.984149in}}%
\pgfpathlineto{\pgfqpoint{5.134678in}{2.981367in}}%
\pgfpathlineto{\pgfqpoint{5.167934in}{2.965554in}}%
\pgfpathlineto{\pgfqpoint{5.169825in}{2.964672in}}%
\pgfpathlineto{\pgfqpoint{5.201190in}{2.950420in}}%
\pgfpathlineto{\pgfqpoint{5.212818in}{2.945194in}}%
\pgfpathlineto{\pgfqpoint{5.234447in}{2.935586in}}%
\pgfpathlineto{\pgfqpoint{5.256688in}{2.925716in}}%
\pgfpathlineto{\pgfqpoint{5.267703in}{2.920838in}}%
\pgfpathlineto{\pgfqpoint{5.300508in}{2.906239in}}%
\pgfpathlineto{\pgfqpoint{5.300959in}{2.906038in}}%
\pgfpathlineto{\pgfqpoint{5.334215in}{2.891052in}}%
\pgfpathlineto{\pgfqpoint{5.343656in}{2.886761in}}%
\pgfpathlineto{\pgfqpoint{5.367471in}{2.875870in}}%
\pgfpathlineto{\pgfqpoint{5.386071in}{2.867283in}}%
\pgfpathlineto{\pgfqpoint{5.400728in}{2.860469in}}%
\pgfpathlineto{\pgfqpoint{5.427698in}{2.847806in}}%
\pgfpathlineto{\pgfqpoint{5.433984in}{2.844832in}}%
\pgfpathlineto{\pgfqpoint{5.467240in}{2.828939in}}%
\pgfpathlineto{\pgfqpoint{5.468504in}{2.828328in}}%
\pgfpathlineto{\pgfqpoint{5.500496in}{2.812754in}}%
\pgfpathlineto{\pgfqpoint{5.508438in}{2.808850in}}%
\pgfpathlineto{\pgfqpoint{5.533752in}{2.796307in}}%
\pgfpathlineto{\pgfqpoint{5.547610in}{2.789373in}}%
\pgfpathlineto{\pgfqpoint{5.567008in}{2.779588in}}%
\pgfpathlineto{\pgfqpoint{5.586040in}{2.769895in}}%
\pgfpathlineto{\pgfqpoint{5.600265in}{2.762592in}}%
\pgfpathlineto{\pgfqpoint{5.623749in}{2.750418in}}%
\pgfpathlineto{\pgfqpoint{5.633521in}{2.745310in}}%
\pgfpathlineto{\pgfqpoint{5.660757in}{2.730940in}}%
\pgfpathlineto{\pgfqpoint{5.666777in}{2.727737in}}%
\pgfusepath{stroke}%
\end{pgfscope}%
\begin{pgfscope}%
\pgfpathrectangle{\pgfqpoint{0.711606in}{0.549444in}}{\pgfqpoint{4.955171in}{2.902168in}}%
\pgfusepath{clip}%
\pgfsetbuttcap%
\pgfsetroundjoin%
\pgfsetlinewidth{1.003750pt}%
\definecolor{currentstroke}{rgb}{0.837165,0.288385,0.252988}%
\pgfsetstrokecolor{currentstroke}%
\pgfsetdash{}{0pt}%
\pgfpathmoveto{\pgfqpoint{4.427299in}{3.451613in}}%
\pgfpathlineto{\pgfqpoint{4.436298in}{3.445627in}}%
\pgfpathlineto{\pgfqpoint{4.456419in}{3.432135in}}%
\pgfpathlineto{\pgfqpoint{4.469554in}{3.423247in}}%
\pgfpathlineto{\pgfqpoint{4.485086in}{3.412657in}}%
\pgfpathlineto{\pgfqpoint{4.502811in}{3.400465in}}%
\pgfpathlineto{\pgfqpoint{4.513329in}{3.393180in}}%
\pgfpathlineto{\pgfqpoint{4.536067in}{3.377300in}}%
\pgfpathlineto{\pgfqpoint{4.541191in}{3.373702in}}%
\pgfpathlineto{\pgfqpoint{4.568730in}{3.354224in}}%
\pgfpathlineto{\pgfqpoint{4.569323in}{3.353802in}}%
\pgfpathlineto{\pgfqpoint{4.596023in}{3.334747in}}%
\pgfpathlineto{\pgfqpoint{4.602579in}{3.330052in}}%
\pgfpathlineto{\pgfqpoint{4.623315in}{3.315269in}}%
\pgfpathlineto{\pgfqpoint{4.635835in}{3.306358in}}%
\pgfpathlineto{\pgfqpoint{4.650891in}{3.295791in}}%
\pgfpathlineto{\pgfqpoint{4.669091in}{3.283133in}}%
\pgfpathlineto{\pgfqpoint{4.679154in}{3.276314in}}%
\pgfpathlineto{\pgfqpoint{4.702348in}{3.260850in}}%
\pgfpathlineto{\pgfqpoint{4.708589in}{3.256836in}}%
\pgfpathlineto{\pgfqpoint{4.735604in}{3.239788in}}%
\pgfpathlineto{\pgfqpoint{4.739597in}{3.237359in}}%
\pgfpathlineto{\pgfqpoint{4.768860in}{3.219829in}}%
\pgfpathlineto{\pgfqpoint{4.772204in}{3.217881in}}%
\pgfpathlineto{\pgfqpoint{4.802116in}{3.200592in}}%
\pgfpathlineto{\pgfqpoint{4.805961in}{3.198403in}}%
\pgfpathlineto{\pgfqpoint{4.835372in}{3.181680in}}%
\pgfpathlineto{\pgfqpoint{4.840239in}{3.178926in}}%
\pgfpathlineto{\pgfqpoint{4.868629in}{3.162805in}}%
\pgfpathlineto{\pgfqpoint{4.874529in}{3.159448in}}%
\pgfpathlineto{\pgfqpoint{4.901885in}{3.143793in}}%
\pgfpathlineto{\pgfqpoint{4.908529in}{3.139970in}}%
\pgfpathlineto{\pgfqpoint{4.935141in}{3.124555in}}%
\pgfpathlineto{\pgfqpoint{4.942106in}{3.120493in}}%
\pgfpathlineto{\pgfqpoint{4.968397in}{3.105061in}}%
\pgfpathlineto{\pgfqpoint{4.975244in}{3.101015in}}%
\pgfpathlineto{\pgfqpoint{5.001653in}{3.085344in}}%
\pgfpathlineto{\pgfqpoint{5.008037in}{3.081537in}}%
\pgfpathlineto{\pgfqpoint{5.034909in}{3.065526in}}%
\pgfpathlineto{\pgfqpoint{5.040724in}{3.062060in}}%
\pgfpathlineto{\pgfqpoint{5.068166in}{3.045872in}}%
\pgfpathlineto{\pgfqpoint{5.073784in}{3.042582in}}%
\pgfpathlineto{\pgfqpoint{5.101422in}{3.026808in}}%
\pgfpathlineto{\pgfqpoint{5.108032in}{3.023104in}}%
\pgfpathlineto{\pgfqpoint{5.134678in}{3.008775in}}%
\pgfpathlineto{\pgfqpoint{5.144523in}{3.003627in}}%
\pgfpathlineto{\pgfqpoint{5.167934in}{2.991931in}}%
\pgfpathlineto{\pgfqpoint{5.183952in}{2.984149in}}%
\pgfpathlineto{\pgfqpoint{5.201190in}{2.976069in}}%
\pgfpathlineto{\pgfqpoint{5.225951in}{2.964672in}}%
\pgfpathlineto{\pgfqpoint{5.234447in}{2.960839in}}%
\pgfpathlineto{\pgfqpoint{5.267703in}{2.945924in}}%
\pgfpathlineto{\pgfqpoint{5.269331in}{2.945194in}}%
\pgfpathlineto{\pgfqpoint{5.300959in}{2.931062in}}%
\pgfpathlineto{\pgfqpoint{5.312877in}{2.925716in}}%
\pgfpathlineto{\pgfqpoint{5.334215in}{2.916125in}}%
\pgfpathlineto{\pgfqpoint{5.356050in}{2.906239in}}%
\pgfpathlineto{\pgfqpoint{5.367471in}{2.901040in}}%
\pgfpathlineto{\pgfqpoint{5.398570in}{2.886761in}}%
\pgfpathlineto{\pgfqpoint{5.400728in}{2.885764in}}%
\pgfpathlineto{\pgfqpoint{5.433984in}{2.870233in}}%
\pgfpathlineto{\pgfqpoint{5.440239in}{2.867283in}}%
\pgfpathlineto{\pgfqpoint{5.467240in}{2.854455in}}%
\pgfpathlineto{\pgfqpoint{5.481100in}{2.847806in}}%
\pgfpathlineto{\pgfqpoint{5.500496in}{2.838429in}}%
\pgfpathlineto{\pgfqpoint{5.521188in}{2.828328in}}%
\pgfpathlineto{\pgfqpoint{5.533752in}{2.822147in}}%
\pgfpathlineto{\pgfqpoint{5.560517in}{2.808850in}}%
\pgfpathlineto{\pgfqpoint{5.567008in}{2.805600in}}%
\pgfpathlineto{\pgfqpoint{5.599107in}{2.789373in}}%
\pgfpathlineto{\pgfqpoint{5.600265in}{2.788783in}}%
\pgfpathlineto{\pgfqpoint{5.633521in}{2.771665in}}%
\pgfpathlineto{\pgfqpoint{5.636927in}{2.769895in}}%
\pgfpathlineto{\pgfqpoint{5.666777in}{2.754258in}}%
\pgfusepath{stroke}%
\end{pgfscope}%
\begin{pgfscope}%
\pgfpathrectangle{\pgfqpoint{0.711606in}{0.549444in}}{\pgfqpoint{4.955171in}{2.902168in}}%
\pgfusepath{clip}%
\pgfsetbuttcap%
\pgfsetroundjoin%
\pgfsetlinewidth{1.003750pt}%
\definecolor{currentstroke}{rgb}{0.855992,0.307038,0.235133}%
\pgfsetstrokecolor{currentstroke}%
\pgfsetdash{}{0pt}%
\pgfpathmoveto{\pgfqpoint{4.471942in}{3.451613in}}%
\pgfpathlineto{\pgfqpoint{4.500793in}{3.432135in}}%
\pgfpathlineto{\pgfqpoint{4.502811in}{3.430760in}}%
\pgfpathlineto{\pgfqpoint{4.529189in}{3.412657in}}%
\pgfpathlineto{\pgfqpoint{4.536067in}{3.407898in}}%
\pgfpathlineto{\pgfqpoint{4.557206in}{3.393180in}}%
\pgfpathlineto{\pgfqpoint{4.569323in}{3.384679in}}%
\pgfpathlineto{\pgfqpoint{4.584909in}{3.373702in}}%
\pgfpathlineto{\pgfqpoint{4.602579in}{3.361186in}}%
\pgfpathlineto{\pgfqpoint{4.612407in}{3.354224in}}%
\pgfpathlineto{\pgfqpoint{4.635835in}{3.337593in}}%
\pgfpathlineto{\pgfqpoint{4.639871in}{3.334747in}}%
\pgfpathlineto{\pgfqpoint{4.667573in}{3.315269in}}%
\pgfpathlineto{\pgfqpoint{4.669091in}{3.314206in}}%
\pgfpathlineto{\pgfqpoint{4.695969in}{3.295791in}}%
\pgfpathlineto{\pgfqpoint{4.702348in}{3.291481in}}%
\pgfpathlineto{\pgfqpoint{4.725542in}{3.276314in}}%
\pgfpathlineto{\pgfqpoint{4.735604in}{3.269856in}}%
\pgfpathlineto{\pgfqpoint{4.756663in}{3.256836in}}%
\pgfpathlineto{\pgfqpoint{4.768860in}{3.249423in}}%
\pgfpathlineto{\pgfqpoint{4.789347in}{3.237359in}}%
\pgfpathlineto{\pgfqpoint{4.802116in}{3.229918in}}%
\pgfpathlineto{\pgfqpoint{4.823181in}{3.217881in}}%
\pgfpathlineto{\pgfqpoint{4.835372in}{3.210938in}}%
\pgfpathlineto{\pgfqpoint{4.857555in}{3.198403in}}%
\pgfpathlineto{\pgfqpoint{4.868629in}{3.192134in}}%
\pgfpathlineto{\pgfqpoint{4.891958in}{3.178926in}}%
\pgfpathlineto{\pgfqpoint{4.901885in}{3.173276in}}%
\pgfpathlineto{\pgfqpoint{4.926076in}{3.159448in}}%
\pgfpathlineto{\pgfqpoint{4.935141in}{3.154232in}}%
\pgfpathlineto{\pgfqpoint{4.959763in}{3.139970in}}%
\pgfpathlineto{\pgfqpoint{4.968397in}{3.134934in}}%
\pgfpathlineto{\pgfqpoint{4.992984in}{3.120493in}}%
\pgfpathlineto{\pgfqpoint{5.001653in}{3.115370in}}%
\pgfpathlineto{\pgfqpoint{5.025800in}{3.101015in}}%
\pgfpathlineto{\pgfqpoint{5.034909in}{3.095584in}}%
\pgfpathlineto{\pgfqpoint{5.058389in}{3.081537in}}%
\pgfpathlineto{\pgfqpoint{5.068166in}{3.075710in}}%
\pgfpathlineto{\pgfqpoint{5.091117in}{3.062060in}}%
\pgfpathlineto{\pgfqpoint{5.101422in}{3.056026in}}%
\pgfpathlineto{\pgfqpoint{5.124646in}{3.042582in}}%
\pgfpathlineto{\pgfqpoint{5.134678in}{3.036956in}}%
\pgfpathlineto{\pgfqpoint{5.159945in}{3.023104in}}%
\pgfpathlineto{\pgfqpoint{5.167934in}{3.018914in}}%
\pgfpathlineto{\pgfqpoint{5.197939in}{3.003627in}}%
\pgfpathlineto{\pgfqpoint{5.201190in}{3.002040in}}%
\pgfpathlineto{\pgfqpoint{5.234447in}{2.986163in}}%
\pgfpathlineto{\pgfqpoint{5.238742in}{2.984149in}}%
\pgfpathlineto{\pgfqpoint{5.267703in}{2.970884in}}%
\pgfpathlineto{\pgfqpoint{5.281385in}{2.964672in}}%
\pgfpathlineto{\pgfqpoint{5.300959in}{2.955868in}}%
\pgfpathlineto{\pgfqpoint{5.324687in}{2.945194in}}%
\pgfpathlineto{\pgfqpoint{5.334215in}{2.940912in}}%
\pgfpathlineto{\pgfqpoint{5.367471in}{2.925884in}}%
\pgfpathlineto{\pgfqpoint{5.367839in}{2.925716in}}%
\pgfpathlineto{\pgfqpoint{5.400728in}{2.910663in}}%
\pgfpathlineto{\pgfqpoint{5.410312in}{2.906239in}}%
\pgfpathlineto{\pgfqpoint{5.433984in}{2.895244in}}%
\pgfpathlineto{\pgfqpoint{5.452080in}{2.886761in}}%
\pgfpathlineto{\pgfqpoint{5.467240in}{2.879604in}}%
\pgfpathlineto{\pgfqpoint{5.493090in}{2.867283in}}%
\pgfpathlineto{\pgfqpoint{5.500496in}{2.863727in}}%
\pgfpathlineto{\pgfqpoint{5.533336in}{2.847806in}}%
\pgfpathlineto{\pgfqpoint{5.533752in}{2.847602in}}%
\pgfpathlineto{\pgfqpoint{5.567008in}{2.831186in}}%
\pgfpathlineto{\pgfqpoint{5.572743in}{2.828328in}}%
\pgfpathlineto{\pgfqpoint{5.600265in}{2.814505in}}%
\pgfpathlineto{\pgfqpoint{5.611418in}{2.808850in}}%
\pgfpathlineto{\pgfqpoint{5.633521in}{2.797556in}}%
\pgfpathlineto{\pgfqpoint{5.649385in}{2.789373in}}%
\pgfpathlineto{\pgfqpoint{5.666777in}{2.780330in}}%
\pgfusepath{stroke}%
\end{pgfscope}%
\begin{pgfscope}%
\pgfpathrectangle{\pgfqpoint{0.711606in}{0.549444in}}{\pgfqpoint{4.955171in}{2.902168in}}%
\pgfusepath{clip}%
\pgfsetbuttcap%
\pgfsetroundjoin%
\pgfsetlinewidth{1.003750pt}%
\definecolor{currentstroke}{rgb}{0.873741,0.326906,0.216886}%
\pgfsetstrokecolor{currentstroke}%
\pgfsetdash{}{0pt}%
\pgfpathmoveto{\pgfqpoint{4.515806in}{3.451613in}}%
\pgfpathlineto{\pgfqpoint{4.536067in}{3.437847in}}%
\pgfpathlineto{\pgfqpoint{4.544416in}{3.432135in}}%
\pgfpathlineto{\pgfqpoint{4.569323in}{3.414959in}}%
\pgfpathlineto{\pgfqpoint{4.572642in}{3.412657in}}%
\pgfpathlineto{\pgfqpoint{4.600525in}{3.393180in}}%
\pgfpathlineto{\pgfqpoint{4.602579in}{3.391735in}}%
\pgfpathlineto{\pgfqpoint{4.628167in}{3.373702in}}%
\pgfpathlineto{\pgfqpoint{4.635835in}{3.368279in}}%
\pgfpathlineto{\pgfqpoint{4.655783in}{3.354224in}}%
\pgfpathlineto{\pgfqpoint{4.669091in}{3.344859in}}%
\pgfpathlineto{\pgfqpoint{4.683647in}{3.334747in}}%
\pgfpathlineto{\pgfqpoint{4.702348in}{3.321861in}}%
\pgfpathlineto{\pgfqpoint{4.712152in}{3.315269in}}%
\pgfpathlineto{\pgfqpoint{4.735604in}{3.299742in}}%
\pgfpathlineto{\pgfqpoint{4.741781in}{3.295791in}}%
\pgfpathlineto{\pgfqpoint{4.768860in}{3.278789in}}%
\pgfpathlineto{\pgfqpoint{4.772948in}{3.276314in}}%
\pgfpathlineto{\pgfqpoint{4.802116in}{3.258919in}}%
\pgfpathlineto{\pgfqpoint{4.805708in}{3.256836in}}%
\pgfpathlineto{\pgfqpoint{4.835372in}{3.239776in}}%
\pgfpathlineto{\pgfqpoint{4.839642in}{3.237359in}}%
\pgfpathlineto{\pgfqpoint{4.868629in}{3.220972in}}%
\pgfpathlineto{\pgfqpoint{4.874125in}{3.217881in}}%
\pgfpathlineto{\pgfqpoint{4.901885in}{3.202220in}}%
\pgfpathlineto{\pgfqpoint{4.908640in}{3.198403in}}%
\pgfpathlineto{\pgfqpoint{4.935141in}{3.183344in}}%
\pgfpathlineto{\pgfqpoint{4.942875in}{3.178926in}}%
\pgfpathlineto{\pgfqpoint{4.968397in}{3.164244in}}%
\pgfpathlineto{\pgfqpoint{4.976676in}{3.159448in}}%
\pgfpathlineto{\pgfqpoint{5.001653in}{3.144876in}}%
\pgfpathlineto{\pgfqpoint{5.009999in}{3.139970in}}%
\pgfpathlineto{\pgfqpoint{5.034909in}{3.125237in}}%
\pgfpathlineto{\pgfqpoint{5.042879in}{3.120493in}}%
\pgfpathlineto{\pgfqpoint{5.068166in}{3.105381in}}%
\pgfpathlineto{\pgfqpoint{5.075441in}{3.101015in}}%
\pgfpathlineto{\pgfqpoint{5.101422in}{3.085451in}}%
\pgfpathlineto{\pgfqpoint{5.107957in}{3.081537in}}%
\pgfpathlineto{\pgfqpoint{5.134678in}{3.065732in}}%
\pgfpathlineto{\pgfqpoint{5.140942in}{3.062060in}}%
\pgfpathlineto{\pgfqpoint{5.167934in}{3.046665in}}%
\pgfpathlineto{\pgfqpoint{5.175237in}{3.042582in}}%
\pgfpathlineto{\pgfqpoint{5.201190in}{3.028663in}}%
\pgfpathlineto{\pgfqpoint{5.211848in}{3.023104in}}%
\pgfpathlineto{\pgfqpoint{5.234447in}{3.011824in}}%
\pgfpathlineto{\pgfqpoint{5.251301in}{3.003627in}}%
\pgfpathlineto{\pgfqpoint{5.267703in}{2.995905in}}%
\pgfpathlineto{\pgfqpoint{5.293066in}{2.984149in}}%
\pgfpathlineto{\pgfqpoint{5.300959in}{2.980553in}}%
\pgfpathlineto{\pgfqpoint{5.334215in}{2.965465in}}%
\pgfpathlineto{\pgfqpoint{5.335963in}{2.964672in}}%
\pgfpathlineto{\pgfqpoint{5.367471in}{2.950397in}}%
\pgfpathlineto{\pgfqpoint{5.378905in}{2.945194in}}%
\pgfpathlineto{\pgfqpoint{5.400728in}{2.935236in}}%
\pgfpathlineto{\pgfqpoint{5.421437in}{2.925716in}}%
\pgfpathlineto{\pgfqpoint{5.433984in}{2.919918in}}%
\pgfpathlineto{\pgfqpoint{5.463326in}{2.906239in}}%
\pgfpathlineto{\pgfqpoint{5.467240in}{2.904402in}}%
\pgfpathlineto{\pgfqpoint{5.500496in}{2.888643in}}%
\pgfpathlineto{\pgfqpoint{5.504428in}{2.886761in}}%
\pgfpathlineto{\pgfqpoint{5.533752in}{2.872626in}}%
\pgfpathlineto{\pgfqpoint{5.544731in}{2.867283in}}%
\pgfpathlineto{\pgfqpoint{5.567008in}{2.856362in}}%
\pgfpathlineto{\pgfqpoint{5.584296in}{2.847806in}}%
\pgfpathlineto{\pgfqpoint{5.600265in}{2.839842in}}%
\pgfpathlineto{\pgfqpoint{5.623137in}{2.828328in}}%
\pgfpathlineto{\pgfqpoint{5.633521in}{2.823060in}}%
\pgfpathlineto{\pgfqpoint{5.661271in}{2.808850in}}%
\pgfpathlineto{\pgfqpoint{5.666777in}{2.806009in}}%
\pgfusepath{stroke}%
\end{pgfscope}%
\begin{pgfscope}%
\pgfpathrectangle{\pgfqpoint{0.711606in}{0.549444in}}{\pgfqpoint{4.955171in}{2.902168in}}%
\pgfusepath{clip}%
\pgfsetbuttcap%
\pgfsetroundjoin%
\pgfsetlinewidth{1.003750pt}%
\definecolor{currentstroke}{rgb}{0.886302,0.342586,0.202968}%
\pgfsetstrokecolor{currentstroke}%
\pgfsetdash{}{0pt}%
\pgfpathmoveto{\pgfqpoint{4.559039in}{3.451613in}}%
\pgfpathlineto{\pgfqpoint{4.569323in}{3.444583in}}%
\pgfpathlineto{\pgfqpoint{4.587423in}{3.432135in}}%
\pgfpathlineto{\pgfqpoint{4.602579in}{3.421635in}}%
\pgfpathlineto{\pgfqpoint{4.615484in}{3.412657in}}%
\pgfpathlineto{\pgfqpoint{4.635835in}{3.398420in}}%
\pgfpathlineto{\pgfqpoint{4.643322in}{3.393180in}}%
\pgfpathlineto{\pgfqpoint{4.669091in}{3.375100in}}%
\pgfpathlineto{\pgfqpoint{4.671095in}{3.373702in}}%
\pgfpathlineto{\pgfqpoint{4.699068in}{3.354224in}}%
\pgfpathlineto{\pgfqpoint{4.702348in}{3.351951in}}%
\pgfpathlineto{\pgfqpoint{4.727675in}{3.334747in}}%
\pgfpathlineto{\pgfqpoint{4.735604in}{3.329429in}}%
\pgfpathlineto{\pgfqpoint{4.757394in}{3.315269in}}%
\pgfpathlineto{\pgfqpoint{4.768860in}{3.307950in}}%
\pgfpathlineto{\pgfqpoint{4.788618in}{3.295791in}}%
\pgfpathlineto{\pgfqpoint{4.802116in}{3.287624in}}%
\pgfpathlineto{\pgfqpoint{4.821407in}{3.276314in}}%
\pgfpathlineto{\pgfqpoint{4.835372in}{3.268214in}}%
\pgfpathlineto{\pgfqpoint{4.855383in}{3.256836in}}%
\pgfpathlineto{\pgfqpoint{4.868629in}{3.249334in}}%
\pgfpathlineto{\pgfqpoint{4.889947in}{3.237359in}}%
\pgfpathlineto{\pgfqpoint{4.901885in}{3.230642in}}%
\pgfpathlineto{\pgfqpoint{4.924573in}{3.217881in}}%
\pgfpathlineto{\pgfqpoint{4.935141in}{3.211908in}}%
\pgfpathlineto{\pgfqpoint{4.958934in}{3.198403in}}%
\pgfpathlineto{\pgfqpoint{4.968397in}{3.192996in}}%
\pgfpathlineto{\pgfqpoint{4.992863in}{3.178926in}}%
\pgfpathlineto{\pgfqpoint{5.001653in}{3.173833in}}%
\pgfpathlineto{\pgfqpoint{5.026302in}{3.159448in}}%
\pgfpathlineto{\pgfqpoint{5.034909in}{3.154389in}}%
\pgfpathlineto{\pgfqpoint{5.059265in}{3.139970in}}%
\pgfpathlineto{\pgfqpoint{5.068166in}{3.134672in}}%
\pgfpathlineto{\pgfqpoint{5.091844in}{3.120493in}}%
\pgfpathlineto{\pgfqpoint{5.101422in}{3.114745in}}%
\pgfpathlineto{\pgfqpoint{5.124239in}{3.101015in}}%
\pgfpathlineto{\pgfqpoint{5.134678in}{3.094766in}}%
\pgfpathlineto{\pgfqpoint{5.156846in}{3.081537in}}%
\pgfpathlineto{\pgfqpoint{5.167934in}{3.075036in}}%
\pgfpathlineto{\pgfqpoint{5.190354in}{3.062060in}}%
\pgfpathlineto{\pgfqpoint{5.201190in}{3.055994in}}%
\pgfpathlineto{\pgfqpoint{5.225723in}{3.042582in}}%
\pgfpathlineto{\pgfqpoint{5.234447in}{3.038018in}}%
\pgfpathlineto{\pgfqpoint{5.263769in}{3.023104in}}%
\pgfpathlineto{\pgfqpoint{5.267703in}{3.021183in}}%
\pgfpathlineto{\pgfqpoint{5.300959in}{3.005265in}}%
\pgfpathlineto{\pgfqpoint{5.304435in}{3.003627in}}%
\pgfpathlineto{\pgfqpoint{5.334215in}{2.989873in}}%
\pgfpathlineto{\pgfqpoint{5.346692in}{2.984149in}}%
\pgfpathlineto{\pgfqpoint{5.367471in}{2.974689in}}%
\pgfpathlineto{\pgfqpoint{5.389447in}{2.964672in}}%
\pgfpathlineto{\pgfqpoint{5.400728in}{2.959529in}}%
\pgfpathlineto{\pgfqpoint{5.431995in}{2.945194in}}%
\pgfpathlineto{\pgfqpoint{5.433984in}{2.944278in}}%
\pgfpathlineto{\pgfqpoint{5.467240in}{2.928834in}}%
\pgfpathlineto{\pgfqpoint{5.473896in}{2.925716in}}%
\pgfpathlineto{\pgfqpoint{5.500496in}{2.913177in}}%
\pgfpathlineto{\pgfqpoint{5.515084in}{2.906239in}}%
\pgfpathlineto{\pgfqpoint{5.533752in}{2.897297in}}%
\pgfpathlineto{\pgfqpoint{5.555547in}{2.886761in}}%
\pgfpathlineto{\pgfqpoint{5.567008in}{2.881180in}}%
\pgfpathlineto{\pgfqpoint{5.595279in}{2.867283in}}%
\pgfpathlineto{\pgfqpoint{5.600265in}{2.864814in}}%
\pgfpathlineto{\pgfqpoint{5.633521in}{2.848188in}}%
\pgfpathlineto{\pgfqpoint{5.634278in}{2.847806in}}%
\pgfpathlineto{\pgfqpoint{5.666777in}{2.831273in}}%
\pgfusepath{stroke}%
\end{pgfscope}%
\begin{pgfscope}%
\pgfpathrectangle{\pgfqpoint{0.711606in}{0.549444in}}{\pgfqpoint{4.955171in}{2.902168in}}%
\pgfusepath{clip}%
\pgfsetbuttcap%
\pgfsetroundjoin%
\pgfsetlinewidth{1.003750pt}%
\definecolor{currentstroke}{rgb}{0.902003,0.364492,0.184116}%
\pgfsetstrokecolor{currentstroke}%
\pgfsetdash{}{0pt}%
\pgfpathmoveto{\pgfqpoint{4.601683in}{3.451613in}}%
\pgfpathlineto{\pgfqpoint{4.602579in}{3.450997in}}%
\pgfpathlineto{\pgfqpoint{4.629893in}{3.432135in}}%
\pgfpathlineto{\pgfqpoint{4.635835in}{3.428006in}}%
\pgfpathlineto{\pgfqpoint{4.657877in}{3.412657in}}%
\pgfpathlineto{\pgfqpoint{4.669091in}{3.404819in}}%
\pgfpathlineto{\pgfqpoint{4.685798in}{3.393180in}}%
\pgfpathlineto{\pgfqpoint{4.702348in}{3.381659in}}%
\pgfpathlineto{\pgfqpoint{4.713913in}{3.373702in}}%
\pgfpathlineto{\pgfqpoint{4.735604in}{3.358889in}}%
\pgfpathlineto{\pgfqpoint{4.742594in}{3.354224in}}%
\pgfpathlineto{\pgfqpoint{4.768860in}{3.336949in}}%
\pgfpathlineto{\pgfqpoint{4.772322in}{3.334747in}}%
\pgfpathlineto{\pgfqpoint{4.802116in}{3.316129in}}%
\pgfpathlineto{\pgfqpoint{4.803542in}{3.315269in}}%
\pgfpathlineto{\pgfqpoint{4.835372in}{3.296366in}}%
\pgfpathlineto{\pgfqpoint{4.836367in}{3.295791in}}%
\pgfpathlineto{\pgfqpoint{4.868629in}{3.277323in}}%
\pgfpathlineto{\pgfqpoint{4.870420in}{3.276314in}}%
\pgfpathlineto{\pgfqpoint{4.901885in}{3.258623in}}%
\pgfpathlineto{\pgfqpoint{4.905081in}{3.256836in}}%
\pgfpathlineto{\pgfqpoint{4.935141in}{3.239984in}}%
\pgfpathlineto{\pgfqpoint{4.939818in}{3.237359in}}%
\pgfpathlineto{\pgfqpoint{4.968397in}{3.221229in}}%
\pgfpathlineto{\pgfqpoint{4.974299in}{3.217881in}}%
\pgfpathlineto{\pgfqpoint{5.001653in}{3.202257in}}%
\pgfpathlineto{\pgfqpoint{5.008354in}{3.198403in}}%
\pgfpathlineto{\pgfqpoint{5.034909in}{3.183017in}}%
\pgfpathlineto{\pgfqpoint{5.041918in}{3.178926in}}%
\pgfpathlineto{\pgfqpoint{5.068166in}{3.163491in}}%
\pgfpathlineto{\pgfqpoint{5.074990in}{3.159448in}}%
\pgfpathlineto{\pgfqpoint{5.101422in}{3.143695in}}%
\pgfpathlineto{\pgfqpoint{5.107632in}{3.139970in}}%
\pgfpathlineto{\pgfqpoint{5.134678in}{3.123699in}}%
\pgfpathlineto{\pgfqpoint{5.139988in}{3.120493in}}%
\pgfpathlineto{\pgfqpoint{5.167934in}{3.103670in}}%
\pgfpathlineto{\pgfqpoint{5.172350in}{3.101015in}}%
\pgfpathlineto{\pgfqpoint{5.201190in}{3.083923in}}%
\pgfpathlineto{\pgfqpoint{5.205258in}{3.081537in}}%
\pgfpathlineto{\pgfqpoint{5.234447in}{3.064913in}}%
\pgfpathlineto{\pgfqpoint{5.239563in}{3.062060in}}%
\pgfpathlineto{\pgfqpoint{5.267703in}{3.047012in}}%
\pgfpathlineto{\pgfqpoint{5.276218in}{3.042582in}}%
\pgfpathlineto{\pgfqpoint{5.300959in}{3.030238in}}%
\pgfpathlineto{\pgfqpoint{5.315601in}{3.023104in}}%
\pgfpathlineto{\pgfqpoint{5.334215in}{3.014298in}}%
\pgfpathlineto{\pgfqpoint{5.357073in}{3.003627in}}%
\pgfpathlineto{\pgfqpoint{5.367471in}{2.998842in}}%
\pgfpathlineto{\pgfqpoint{5.399507in}{2.984149in}}%
\pgfpathlineto{\pgfqpoint{5.400728in}{2.983592in}}%
\pgfpathlineto{\pgfqpoint{5.433984in}{2.968325in}}%
\pgfpathlineto{\pgfqpoint{5.441903in}{2.964672in}}%
\pgfpathlineto{\pgfqpoint{5.467240in}{2.952942in}}%
\pgfpathlineto{\pgfqpoint{5.483855in}{2.945194in}}%
\pgfpathlineto{\pgfqpoint{5.500496in}{2.937390in}}%
\pgfpathlineto{\pgfqpoint{5.525178in}{2.925716in}}%
\pgfpathlineto{\pgfqpoint{5.533752in}{2.921634in}}%
\pgfpathlineto{\pgfqpoint{5.565799in}{2.906239in}}%
\pgfpathlineto{\pgfqpoint{5.567008in}{2.905654in}}%
\pgfpathlineto{\pgfqpoint{5.600265in}{2.889404in}}%
\pgfpathlineto{\pgfqpoint{5.605623in}{2.886761in}}%
\pgfpathlineto{\pgfqpoint{5.633521in}{2.872900in}}%
\pgfpathlineto{\pgfqpoint{5.644724in}{2.867283in}}%
\pgfpathlineto{\pgfqpoint{5.666777in}{2.856143in}}%
\pgfusepath{stroke}%
\end{pgfscope}%
\begin{pgfscope}%
\pgfpathrectangle{\pgfqpoint{0.711606in}{0.549444in}}{\pgfqpoint{4.955171in}{2.902168in}}%
\pgfusepath{clip}%
\pgfsetbuttcap%
\pgfsetroundjoin%
\pgfsetlinewidth{1.003750pt}%
\definecolor{currentstroke}{rgb}{0.912966,0.381636,0.169755}%
\pgfsetstrokecolor{currentstroke}%
\pgfsetdash{}{0pt}%
\pgfpathmoveto{\pgfqpoint{4.643716in}{3.451613in}}%
\pgfpathlineto{\pgfqpoint{4.669091in}{3.434068in}}%
\pgfpathlineto{\pgfqpoint{4.671885in}{3.432135in}}%
\pgfpathlineto{\pgfqpoint{4.699941in}{3.412657in}}%
\pgfpathlineto{\pgfqpoint{4.702348in}{3.410984in}}%
\pgfpathlineto{\pgfqpoint{4.728157in}{3.393180in}}%
\pgfpathlineto{\pgfqpoint{4.735604in}{3.388067in}}%
\pgfpathlineto{\pgfqpoint{4.756926in}{3.373702in}}%
\pgfpathlineto{\pgfqpoint{4.768860in}{3.365757in}}%
\pgfpathlineto{\pgfqpoint{4.786713in}{3.354224in}}%
\pgfpathlineto{\pgfqpoint{4.802116in}{3.344442in}}%
\pgfpathlineto{\pgfqpoint{4.817941in}{3.334747in}}%
\pgfpathlineto{\pgfqpoint{4.835372in}{3.324242in}}%
\pgfpathlineto{\pgfqpoint{4.850740in}{3.315269in}}%
\pgfpathlineto{\pgfqpoint{4.868629in}{3.304938in}}%
\pgfpathlineto{\pgfqpoint{4.884792in}{3.295791in}}%
\pgfpathlineto{\pgfqpoint{4.901885in}{3.286158in}}%
\pgfpathlineto{\pgfqpoint{4.919507in}{3.276314in}}%
\pgfpathlineto{\pgfqpoint{4.935141in}{3.267569in}}%
\pgfpathlineto{\pgfqpoint{4.954345in}{3.256836in}}%
\pgfpathlineto{\pgfqpoint{4.968397in}{3.248946in}}%
\pgfpathlineto{\pgfqpoint{4.988953in}{3.237359in}}%
\pgfpathlineto{\pgfqpoint{5.001653in}{3.230153in}}%
\pgfpathlineto{\pgfqpoint{5.023145in}{3.217881in}}%
\pgfpathlineto{\pgfqpoint{5.034909in}{3.211114in}}%
\pgfpathlineto{\pgfqpoint{5.056844in}{3.198403in}}%
\pgfpathlineto{\pgfqpoint{5.068166in}{3.191793in}}%
\pgfpathlineto{\pgfqpoint{5.090038in}{3.178926in}}%
\pgfpathlineto{\pgfqpoint{5.101422in}{3.172182in}}%
\pgfpathlineto{\pgfqpoint{5.122766in}{3.159448in}}%
\pgfpathlineto{\pgfqpoint{5.134678in}{3.152304in}}%
\pgfpathlineto{\pgfqpoint{5.155133in}{3.139970in}}%
\pgfpathlineto{\pgfqpoint{5.167934in}{3.132242in}}%
\pgfpathlineto{\pgfqpoint{5.187357in}{3.120493in}}%
\pgfpathlineto{\pgfqpoint{5.201190in}{3.112180in}}%
\pgfpathlineto{\pgfqpoint{5.219852in}{3.101015in}}%
\pgfpathlineto{\pgfqpoint{5.234447in}{3.092454in}}%
\pgfpathlineto{\pgfqpoint{5.253322in}{3.081537in}}%
\pgfpathlineto{\pgfqpoint{5.267703in}{3.073508in}}%
\pgfpathlineto{\pgfqpoint{5.288706in}{3.062060in}}%
\pgfpathlineto{\pgfqpoint{5.300959in}{3.055666in}}%
\pgfpathlineto{\pgfqpoint{5.326719in}{3.042582in}}%
\pgfpathlineto{\pgfqpoint{5.334215in}{3.038915in}}%
\pgfpathlineto{\pgfqpoint{5.367190in}{3.023104in}}%
\pgfpathlineto{\pgfqpoint{5.367471in}{3.022973in}}%
\pgfpathlineto{\pgfqpoint{5.400728in}{3.007484in}}%
\pgfpathlineto{\pgfqpoint{5.409050in}{3.003627in}}%
\pgfpathlineto{\pgfqpoint{5.433984in}{2.992139in}}%
\pgfpathlineto{\pgfqpoint{5.451288in}{2.984149in}}%
\pgfpathlineto{\pgfqpoint{5.467240in}{2.976777in}}%
\pgfpathlineto{\pgfqpoint{5.493278in}{2.964672in}}%
\pgfpathlineto{\pgfqpoint{5.500496in}{2.961301in}}%
\pgfpathlineto{\pgfqpoint{5.533752in}{2.945648in}}%
\pgfpathlineto{\pgfqpoint{5.534708in}{2.945194in}}%
\pgfpathlineto{\pgfqpoint{5.567008in}{2.929753in}}%
\pgfpathlineto{\pgfqpoint{5.575380in}{2.925716in}}%
\pgfpathlineto{\pgfqpoint{5.600265in}{2.913632in}}%
\pgfpathlineto{\pgfqpoint{5.615354in}{2.906239in}}%
\pgfpathlineto{\pgfqpoint{5.633521in}{2.897273in}}%
\pgfpathlineto{\pgfqpoint{5.654628in}{2.886761in}}%
\pgfpathlineto{\pgfqpoint{5.666777in}{2.880666in}}%
\pgfusepath{stroke}%
\end{pgfscope}%
\begin{pgfscope}%
\pgfpathrectangle{\pgfqpoint{0.711606in}{0.549444in}}{\pgfqpoint{4.955171in}{2.902168in}}%
\pgfusepath{clip}%
\pgfsetbuttcap%
\pgfsetroundjoin%
\pgfsetlinewidth{1.003750pt}%
\definecolor{currentstroke}{rgb}{0.926470,0.405389,0.150292}%
\pgfsetstrokecolor{currentstroke}%
\pgfsetdash{}{0pt}%
\pgfpathmoveto{\pgfqpoint{4.685300in}{3.451613in}}%
\pgfpathlineto{\pgfqpoint{4.702348in}{3.439831in}}%
\pgfpathlineto{\pgfqpoint{4.713513in}{3.432135in}}%
\pgfpathlineto{\pgfqpoint{4.735604in}{3.416914in}}%
\pgfpathlineto{\pgfqpoint{4.741849in}{3.412657in}}%
\pgfpathlineto{\pgfqpoint{4.768860in}{3.394369in}}%
\pgfpathlineto{\pgfqpoint{4.770655in}{3.393180in}}%
\pgfpathlineto{\pgfqpoint{4.800429in}{3.373702in}}%
\pgfpathlineto{\pgfqpoint{4.802116in}{3.372614in}}%
\pgfpathlineto{\pgfqpoint{4.831642in}{3.354224in}}%
\pgfpathlineto{\pgfqpoint{4.835372in}{3.351941in}}%
\pgfpathlineto{\pgfqpoint{4.864431in}{3.334747in}}%
\pgfpathlineto{\pgfqpoint{4.868629in}{3.332296in}}%
\pgfpathlineto{\pgfqpoint{4.898510in}{3.315269in}}%
\pgfpathlineto{\pgfqpoint{4.901885in}{3.313358in}}%
\pgfpathlineto{\pgfqpoint{4.933306in}{3.295791in}}%
\pgfpathlineto{\pgfqpoint{4.935141in}{3.294766in}}%
\pgfpathlineto{\pgfqpoint{4.968264in}{3.276314in}}%
\pgfpathlineto{\pgfqpoint{4.968397in}{3.276239in}}%
\pgfpathlineto{\pgfqpoint{5.001653in}{3.257593in}}%
\pgfpathlineto{\pgfqpoint{5.002997in}{3.256836in}}%
\pgfpathlineto{\pgfqpoint{5.034909in}{3.238736in}}%
\pgfpathlineto{\pgfqpoint{5.037323in}{3.237359in}}%
\pgfpathlineto{\pgfqpoint{5.068166in}{3.219616in}}%
\pgfpathlineto{\pgfqpoint{5.071159in}{3.217881in}}%
\pgfpathlineto{\pgfqpoint{5.101422in}{3.200207in}}%
\pgfpathlineto{\pgfqpoint{5.104486in}{3.198403in}}%
\pgfpathlineto{\pgfqpoint{5.134678in}{3.180507in}}%
\pgfpathlineto{\pgfqpoint{5.137327in}{3.178926in}}%
\pgfpathlineto{\pgfqpoint{5.167934in}{3.160548in}}%
\pgfpathlineto{\pgfqpoint{5.169756in}{3.159448in}}%
\pgfpathlineto{\pgfqpoint{5.201190in}{3.140418in}}%
\pgfpathlineto{\pgfqpoint{5.201928in}{3.139970in}}%
\pgfpathlineto{\pgfqpoint{5.234144in}{3.120493in}}%
\pgfpathlineto{\pgfqpoint{5.234447in}{3.120312in}}%
\pgfpathlineto{\pgfqpoint{5.266961in}{3.101015in}}%
\pgfpathlineto{\pgfqpoint{5.267703in}{3.100586in}}%
\pgfpathlineto{\pgfqpoint{5.300959in}{3.081692in}}%
\pgfpathlineto{\pgfqpoint{5.301237in}{3.081537in}}%
\pgfpathlineto{\pgfqpoint{5.334215in}{3.063955in}}%
\pgfpathlineto{\pgfqpoint{5.337866in}{3.062060in}}%
\pgfpathlineto{\pgfqpoint{5.367471in}{3.047288in}}%
\pgfpathlineto{\pgfqpoint{5.377109in}{3.042582in}}%
\pgfpathlineto{\pgfqpoint{5.400728in}{3.031349in}}%
\pgfpathlineto{\pgfqpoint{5.418255in}{3.023104in}}%
\pgfpathlineto{\pgfqpoint{5.433984in}{3.015796in}}%
\pgfpathlineto{\pgfqpoint{5.460218in}{3.003627in}}%
\pgfpathlineto{\pgfqpoint{5.467240in}{3.000379in}}%
\pgfpathlineto{\pgfqpoint{5.500496in}{2.984930in}}%
\pgfpathlineto{\pgfqpoint{5.502165in}{2.984149in}}%
\pgfpathlineto{\pgfqpoint{5.533752in}{2.969325in}}%
\pgfpathlineto{\pgfqpoint{5.543593in}{2.964672in}}%
\pgfpathlineto{\pgfqpoint{5.567008in}{2.953538in}}%
\pgfpathlineto{\pgfqpoint{5.584409in}{2.945194in}}%
\pgfpathlineto{\pgfqpoint{5.600265in}{2.937541in}}%
\pgfpathlineto{\pgfqpoint{5.624549in}{2.925716in}}%
\pgfpathlineto{\pgfqpoint{5.633521in}{2.921317in}}%
\pgfpathlineto{\pgfqpoint{5.663997in}{2.906239in}}%
\pgfpathlineto{\pgfqpoint{5.666777in}{2.904853in}}%
\pgfusepath{stroke}%
\end{pgfscope}%
\begin{pgfscope}%
\pgfpathrectangle{\pgfqpoint{0.711606in}{0.549444in}}{\pgfqpoint{4.955171in}{2.902168in}}%
\pgfusepath{clip}%
\pgfsetbuttcap%
\pgfsetroundjoin%
\pgfsetlinewidth{1.003750pt}%
\definecolor{currentstroke}{rgb}{0.938675,0.430091,0.130438}%
\pgfsetstrokecolor{currentstroke}%
\pgfsetdash{}{0pt}%
\pgfpathmoveto{\pgfqpoint{4.726547in}{3.451613in}}%
\pgfpathlineto{\pgfqpoint{4.735604in}{3.445379in}}%
\pgfpathlineto{\pgfqpoint{4.754992in}{3.432135in}}%
\pgfpathlineto{\pgfqpoint{4.768860in}{3.422700in}}%
\pgfpathlineto{\pgfqpoint{4.783885in}{3.412657in}}%
\pgfpathlineto{\pgfqpoint{4.802116in}{3.400605in}}%
\pgfpathlineto{\pgfqpoint{4.813676in}{3.393180in}}%
\pgfpathlineto{\pgfqpoint{4.835372in}{3.379470in}}%
\pgfpathlineto{\pgfqpoint{4.844825in}{3.373702in}}%
\pgfpathlineto{\pgfqpoint{4.868629in}{3.359413in}}%
\pgfpathlineto{\pgfqpoint{4.877548in}{3.354224in}}%
\pgfpathlineto{\pgfqpoint{4.901885in}{3.340225in}}%
\pgfpathlineto{\pgfqpoint{4.911607in}{3.334747in}}%
\pgfpathlineto{\pgfqpoint{4.935141in}{3.321546in}}%
\pgfpathlineto{\pgfqpoint{4.946435in}{3.315269in}}%
\pgfpathlineto{\pgfqpoint{4.968397in}{3.303053in}}%
\pgfpathlineto{\pgfqpoint{4.981469in}{3.295791in}}%
\pgfpathlineto{\pgfqpoint{5.001653in}{3.284529in}}%
\pgfpathlineto{\pgfqpoint{5.016323in}{3.276314in}}%
\pgfpathlineto{\pgfqpoint{5.034909in}{3.265840in}}%
\pgfpathlineto{\pgfqpoint{5.050789in}{3.256836in}}%
\pgfpathlineto{\pgfqpoint{5.068166in}{3.246913in}}%
\pgfpathlineto{\pgfqpoint{5.084772in}{3.237359in}}%
\pgfpathlineto{\pgfqpoint{5.101422in}{3.227707in}}%
\pgfpathlineto{\pgfqpoint{5.118242in}{3.217881in}}%
\pgfpathlineto{\pgfqpoint{5.134678in}{3.208207in}}%
\pgfpathlineto{\pgfqpoint{5.151209in}{3.198403in}}%
\pgfpathlineto{\pgfqpoint{5.167934in}{3.188418in}}%
\pgfpathlineto{\pgfqpoint{5.183726in}{3.178926in}}%
\pgfpathlineto{\pgfqpoint{5.201190in}{3.168379in}}%
\pgfpathlineto{\pgfqpoint{5.215907in}{3.159448in}}%
\pgfpathlineto{\pgfqpoint{5.234447in}{3.148194in}}%
\pgfpathlineto{\pgfqpoint{5.247977in}{3.139970in}}%
\pgfpathlineto{\pgfqpoint{5.267703in}{3.128080in}}%
\pgfpathlineto{\pgfqpoint{5.280362in}{3.120493in}}%
\pgfpathlineto{\pgfqpoint{5.300959in}{3.108412in}}%
\pgfpathlineto{\pgfqpoint{5.313768in}{3.101015in}}%
\pgfpathlineto{\pgfqpoint{5.334215in}{3.089630in}}%
\pgfpathlineto{\pgfqpoint{5.349112in}{3.081537in}}%
\pgfpathlineto{\pgfqpoint{5.367471in}{3.071979in}}%
\pgfpathlineto{\pgfqpoint{5.387019in}{3.062060in}}%
\pgfpathlineto{\pgfqpoint{5.400728in}{3.055342in}}%
\pgfpathlineto{\pgfqpoint{5.427226in}{3.042582in}}%
\pgfpathlineto{\pgfqpoint{5.433984in}{3.039394in}}%
\pgfpathlineto{\pgfqpoint{5.467240in}{3.023803in}}%
\pgfpathlineto{\pgfqpoint{5.468733in}{3.023104in}}%
\pgfpathlineto{\pgfqpoint{5.500496in}{3.008298in}}%
\pgfpathlineto{\pgfqpoint{5.510487in}{3.003627in}}%
\pgfpathlineto{\pgfqpoint{5.533752in}{2.992733in}}%
\pgfpathlineto{\pgfqpoint{5.551970in}{2.984149in}}%
\pgfpathlineto{\pgfqpoint{5.567008in}{2.977031in}}%
\pgfpathlineto{\pgfqpoint{5.592916in}{2.964672in}}%
\pgfpathlineto{\pgfqpoint{5.600265in}{2.961145in}}%
\pgfpathlineto{\pgfqpoint{5.633218in}{2.945194in}}%
\pgfpathlineto{\pgfqpoint{5.633521in}{2.945046in}}%
\pgfpathlineto{\pgfqpoint{5.666777in}{2.928685in}}%
\pgfusepath{stroke}%
\end{pgfscope}%
\begin{pgfscope}%
\pgfpathrectangle{\pgfqpoint{0.711606in}{0.549444in}}{\pgfqpoint{4.955171in}{2.902168in}}%
\pgfusepath{clip}%
\pgfsetbuttcap%
\pgfsetroundjoin%
\pgfsetlinewidth{1.003750pt}%
\definecolor{currentstroke}{rgb}{0.946965,0.449191,0.115272}%
\pgfsetstrokecolor{currentstroke}%
\pgfsetdash{}{0pt}%
\pgfpathmoveto{\pgfqpoint{4.767604in}{3.451613in}}%
\pgfpathlineto{\pgfqpoint{4.768860in}{3.450756in}}%
\pgfpathlineto{\pgfqpoint{4.796538in}{3.432135in}}%
\pgfpathlineto{\pgfqpoint{4.802116in}{3.428413in}}%
\pgfpathlineto{\pgfqpoint{4.826328in}{3.412657in}}%
\pgfpathlineto{\pgfqpoint{4.835372in}{3.406858in}}%
\pgfpathlineto{\pgfqpoint{4.857431in}{3.393180in}}%
\pgfpathlineto{\pgfqpoint{4.868629in}{3.386353in}}%
\pgfpathlineto{\pgfqpoint{4.890087in}{3.373702in}}%
\pgfpathlineto{\pgfqpoint{4.901885in}{3.366838in}}%
\pgfpathlineto{\pgfqpoint{4.924111in}{3.354224in}}%
\pgfpathlineto{\pgfqpoint{4.935141in}{3.348006in}}%
\pgfpathlineto{\pgfqpoint{4.958972in}{3.334747in}}%
\pgfpathlineto{\pgfqpoint{4.968397in}{3.329507in}}%
\pgfpathlineto{\pgfqpoint{4.994099in}{3.315269in}}%
\pgfpathlineto{\pgfqpoint{5.001653in}{3.311071in}}%
\pgfpathlineto{\pgfqpoint{5.029084in}{3.295791in}}%
\pgfpathlineto{\pgfqpoint{5.034909in}{3.292528in}}%
\pgfpathlineto{\pgfqpoint{5.063698in}{3.276314in}}%
\pgfpathlineto{\pgfqpoint{5.068166in}{3.273780in}}%
\pgfpathlineto{\pgfqpoint{5.097835in}{3.256836in}}%
\pgfpathlineto{\pgfqpoint{5.101422in}{3.254772in}}%
\pgfpathlineto{\pgfqpoint{5.131455in}{3.237359in}}%
\pgfpathlineto{\pgfqpoint{5.134678in}{3.235476in}}%
\pgfpathlineto{\pgfqpoint{5.164561in}{3.217881in}}%
\pgfpathlineto{\pgfqpoint{5.167934in}{3.215881in}}%
\pgfpathlineto{\pgfqpoint{5.197189in}{3.198403in}}%
\pgfpathlineto{\pgfqpoint{5.201190in}{3.195998in}}%
\pgfpathlineto{\pgfqpoint{5.229419in}{3.178926in}}%
\pgfpathlineto{\pgfqpoint{5.234447in}{3.175875in}}%
\pgfpathlineto{\pgfqpoint{5.261418in}{3.159448in}}%
\pgfpathlineto{\pgfqpoint{5.267703in}{3.155630in}}%
\pgfpathlineto{\pgfqpoint{5.293503in}{3.139970in}}%
\pgfpathlineto{\pgfqpoint{5.300959in}{3.135503in}}%
\pgfpathlineto{\pgfqpoint{5.326239in}{3.120493in}}%
\pgfpathlineto{\pgfqpoint{5.334215in}{3.115885in}}%
\pgfpathlineto{\pgfqpoint{5.360461in}{3.101015in}}%
\pgfpathlineto{\pgfqpoint{5.367471in}{3.097200in}}%
\pgfpathlineto{\pgfqpoint{5.397005in}{3.081537in}}%
\pgfpathlineto{\pgfqpoint{5.400728in}{3.079642in}}%
\pgfpathlineto{\pgfqpoint{5.433984in}{3.063077in}}%
\pgfpathlineto{\pgfqpoint{5.436060in}{3.062060in}}%
\pgfpathlineto{\pgfqpoint{5.467240in}{3.047145in}}%
\pgfpathlineto{\pgfqpoint{5.476866in}{3.042582in}}%
\pgfpathlineto{\pgfqpoint{5.500496in}{3.031493in}}%
\pgfpathlineto{\pgfqpoint{5.518382in}{3.023104in}}%
\pgfpathlineto{\pgfqpoint{5.533752in}{3.015907in}}%
\pgfpathlineto{\pgfqpoint{5.559862in}{3.003627in}}%
\pgfpathlineto{\pgfqpoint{5.567008in}{3.000255in}}%
\pgfpathlineto{\pgfqpoint{5.600265in}{2.984456in}}%
\pgfpathlineto{\pgfqpoint{5.600905in}{2.984149in}}%
\pgfpathlineto{\pgfqpoint{5.633521in}{2.968434in}}%
\pgfpathlineto{\pgfqpoint{5.641265in}{2.964672in}}%
\pgfpathlineto{\pgfqpoint{5.666777in}{2.952196in}}%
\pgfusepath{stroke}%
\end{pgfscope}%
\begin{pgfscope}%
\pgfpathrectangle{\pgfqpoint{0.711606in}{0.549444in}}{\pgfqpoint{4.955171in}{2.902168in}}%
\pgfusepath{clip}%
\pgfsetbuttcap%
\pgfsetroundjoin%
\pgfsetlinewidth{1.003750pt}%
\definecolor{currentstroke}{rgb}{0.956852,0.475356,0.094695}%
\pgfsetstrokecolor{currentstroke}%
\pgfsetdash{}{0pt}%
\pgfpathmoveto{\pgfqpoint{4.808700in}{3.451613in}}%
\pgfpathlineto{\pgfqpoint{4.835372in}{3.434099in}}%
\pgfpathlineto{\pgfqpoint{4.838447in}{3.432135in}}%
\pgfpathlineto{\pgfqpoint{4.868629in}{3.413154in}}%
\pgfpathlineto{\pgfqpoint{4.869445in}{3.412657in}}%
\pgfpathlineto{\pgfqpoint{4.901885in}{3.393254in}}%
\pgfpathlineto{\pgfqpoint{4.902013in}{3.393180in}}%
\pgfpathlineto{\pgfqpoint{4.935141in}{3.374192in}}%
\pgfpathlineto{\pgfqpoint{4.936015in}{3.373702in}}%
\pgfpathlineto{\pgfqpoint{4.968397in}{3.355616in}}%
\pgfpathlineto{\pgfqpoint{4.970914in}{3.354224in}}%
\pgfpathlineto{\pgfqpoint{5.001653in}{3.337216in}}%
\pgfpathlineto{\pgfqpoint{5.006123in}{3.334747in}}%
\pgfpathlineto{\pgfqpoint{5.034909in}{3.318780in}}%
\pgfpathlineto{\pgfqpoint{5.041218in}{3.315269in}}%
\pgfpathlineto{\pgfqpoint{5.068166in}{3.300183in}}%
\pgfpathlineto{\pgfqpoint{5.075963in}{3.295791in}}%
\pgfpathlineto{\pgfqpoint{5.101422in}{3.281353in}}%
\pgfpathlineto{\pgfqpoint{5.110243in}{3.276314in}}%
\pgfpathlineto{\pgfqpoint{5.134678in}{3.262251in}}%
\pgfpathlineto{\pgfqpoint{5.144014in}{3.256836in}}%
\pgfpathlineto{\pgfqpoint{5.167934in}{3.242857in}}%
\pgfpathlineto{\pgfqpoint{5.177270in}{3.237359in}}%
\pgfpathlineto{\pgfqpoint{5.201190in}{3.223167in}}%
\pgfpathlineto{\pgfqpoint{5.210034in}{3.217881in}}%
\pgfpathlineto{\pgfqpoint{5.234447in}{3.203197in}}%
\pgfpathlineto{\pgfqpoint{5.242365in}{3.198403in}}%
\pgfpathlineto{\pgfqpoint{5.267703in}{3.183002in}}%
\pgfpathlineto{\pgfqpoint{5.274380in}{3.178926in}}%
\pgfpathlineto{\pgfqpoint{5.300959in}{3.162711in}}%
\pgfpathlineto{\pgfqpoint{5.306307in}{3.159448in}}%
\pgfpathlineto{\pgfqpoint{5.334215in}{3.142585in}}%
\pgfpathlineto{\pgfqpoint{5.338572in}{3.139970in}}%
\pgfpathlineto{\pgfqpoint{5.367471in}{3.123035in}}%
\pgfpathlineto{\pgfqpoint{5.371884in}{3.120493in}}%
\pgfpathlineto{\pgfqpoint{5.400728in}{3.104483in}}%
\pgfpathlineto{\pgfqpoint{5.407132in}{3.101015in}}%
\pgfpathlineto{\pgfqpoint{5.433984in}{3.087064in}}%
\pgfpathlineto{\pgfqpoint{5.444879in}{3.081537in}}%
\pgfpathlineto{\pgfqpoint{5.467240in}{3.070553in}}%
\pgfpathlineto{\pgfqpoint{5.484795in}{3.062060in}}%
\pgfpathlineto{\pgfqpoint{5.500496in}{3.054597in}}%
\pgfpathlineto{\pgfqpoint{5.525905in}{3.042582in}}%
\pgfpathlineto{\pgfqpoint{5.533752in}{3.038895in}}%
\pgfpathlineto{\pgfqpoint{5.567008in}{3.023239in}}%
\pgfpathlineto{\pgfqpoint{5.567292in}{3.023104in}}%
\pgfpathlineto{\pgfqpoint{5.600265in}{3.007469in}}%
\pgfpathlineto{\pgfqpoint{5.608315in}{3.003627in}}%
\pgfpathlineto{\pgfqpoint{5.633521in}{2.991539in}}%
\pgfpathlineto{\pgfqpoint{5.648812in}{2.984149in}}%
\pgfpathlineto{\pgfqpoint{5.666777in}{2.975414in}}%
\pgfusepath{stroke}%
\end{pgfscope}%
\begin{pgfscope}%
\pgfpathrectangle{\pgfqpoint{0.711606in}{0.549444in}}{\pgfqpoint{4.955171in}{2.902168in}}%
\pgfusepath{clip}%
\pgfsetbuttcap%
\pgfsetroundjoin%
\pgfsetlinewidth{1.003750pt}%
\definecolor{currentstroke}{rgb}{0.963387,0.495462,0.079073}%
\pgfsetstrokecolor{currentstroke}%
\pgfsetdash{}{0pt}%
\pgfpathmoveto{\pgfqpoint{4.850118in}{3.451613in}}%
\pgfpathlineto{\pgfqpoint{4.868629in}{3.439806in}}%
\pgfpathlineto{\pgfqpoint{4.881052in}{3.432135in}}%
\pgfpathlineto{\pgfqpoint{4.901885in}{3.419482in}}%
\pgfpathlineto{\pgfqpoint{4.913499in}{3.412657in}}%
\pgfpathlineto{\pgfqpoint{4.935141in}{3.400110in}}%
\pgfpathlineto{\pgfqpoint{4.947397in}{3.393180in}}%
\pgfpathlineto{\pgfqpoint{4.968397in}{3.381389in}}%
\pgfpathlineto{\pgfqpoint{4.982274in}{3.373702in}}%
\pgfpathlineto{\pgfqpoint{5.001653in}{3.362980in}}%
\pgfpathlineto{\pgfqpoint{5.017543in}{3.354224in}}%
\pgfpathlineto{\pgfqpoint{5.034909in}{3.344627in}}%
\pgfpathlineto{\pgfqpoint{5.052754in}{3.334747in}}%
\pgfpathlineto{\pgfqpoint{5.068166in}{3.326167in}}%
\pgfpathlineto{\pgfqpoint{5.087642in}{3.315269in}}%
\pgfpathlineto{\pgfqpoint{5.101422in}{3.307507in}}%
\pgfpathlineto{\pgfqpoint{5.122079in}{3.295791in}}%
\pgfpathlineto{\pgfqpoint{5.134678in}{3.288594in}}%
\pgfpathlineto{\pgfqpoint{5.156010in}{3.276314in}}%
\pgfpathlineto{\pgfqpoint{5.167934in}{3.269397in}}%
\pgfpathlineto{\pgfqpoint{5.189421in}{3.256836in}}%
\pgfpathlineto{\pgfqpoint{5.201190in}{3.249905in}}%
\pgfpathlineto{\pgfqpoint{5.222329in}{3.237359in}}%
\pgfpathlineto{\pgfqpoint{5.234447in}{3.230116in}}%
\pgfpathlineto{\pgfqpoint{5.254773in}{3.217881in}}%
\pgfpathlineto{\pgfqpoint{5.267703in}{3.210055in}}%
\pgfpathlineto{\pgfqpoint{5.286840in}{3.198403in}}%
\pgfpathlineto{\pgfqpoint{5.300959in}{3.189786in}}%
\pgfpathlineto{\pgfqpoint{5.318696in}{3.178926in}}%
\pgfpathlineto{\pgfqpoint{5.334215in}{3.169459in}}%
\pgfpathlineto{\pgfqpoint{5.350661in}{3.159448in}}%
\pgfpathlineto{\pgfqpoint{5.367471in}{3.149364in}}%
\pgfpathlineto{\pgfqpoint{5.383295in}{3.139970in}}%
\pgfpathlineto{\pgfqpoint{5.400728in}{3.129921in}}%
\pgfpathlineto{\pgfqpoint{5.417419in}{3.120493in}}%
\pgfpathlineto{\pgfqpoint{5.433984in}{3.111506in}}%
\pgfpathlineto{\pgfqpoint{5.453820in}{3.101015in}}%
\pgfpathlineto{\pgfqpoint{5.467240in}{3.094187in}}%
\pgfpathlineto{\pgfqpoint{5.492632in}{3.081537in}}%
\pgfpathlineto{\pgfqpoint{5.500496in}{3.077721in}}%
\pgfpathlineto{\pgfqpoint{5.533129in}{3.062060in}}%
\pgfpathlineto{\pgfqpoint{5.533752in}{3.061764in}}%
\pgfpathlineto{\pgfqpoint{5.567008in}{3.046007in}}%
\pgfpathlineto{\pgfqpoint{5.574231in}{3.042582in}}%
\pgfpathlineto{\pgfqpoint{5.600265in}{3.030246in}}%
\pgfpathlineto{\pgfqpoint{5.615264in}{3.023104in}}%
\pgfpathlineto{\pgfqpoint{5.633521in}{3.014382in}}%
\pgfpathlineto{\pgfqpoint{5.655875in}{3.003627in}}%
\pgfpathlineto{\pgfqpoint{5.666777in}{2.998354in}}%
\pgfusepath{stroke}%
\end{pgfscope}%
\begin{pgfscope}%
\pgfpathrectangle{\pgfqpoint{0.711606in}{0.549444in}}{\pgfqpoint{4.955171in}{2.902168in}}%
\pgfusepath{clip}%
\pgfsetbuttcap%
\pgfsetroundjoin%
\pgfsetlinewidth{1.003750pt}%
\definecolor{currentstroke}{rgb}{0.970919,0.522853,0.058367}%
\pgfsetstrokecolor{currentstroke}%
\pgfsetdash{}{0pt}%
\pgfpathmoveto{\pgfqpoint{4.892107in}{3.451613in}}%
\pgfpathlineto{\pgfqpoint{4.901885in}{3.445581in}}%
\pgfpathlineto{\pgfqpoint{4.924434in}{3.432135in}}%
\pgfpathlineto{\pgfqpoint{4.935141in}{3.425844in}}%
\pgfpathlineto{\pgfqpoint{4.958235in}{3.412657in}}%
\pgfpathlineto{\pgfqpoint{4.968397in}{3.406908in}}%
\pgfpathlineto{\pgfqpoint{4.993094in}{3.393180in}}%
\pgfpathlineto{\pgfqpoint{5.001653in}{3.388436in}}%
\pgfpathlineto{\pgfqpoint{5.028427in}{3.373702in}}%
\pgfpathlineto{\pgfqpoint{5.034909in}{3.370128in}}%
\pgfpathlineto{\pgfqpoint{5.063757in}{3.354224in}}%
\pgfpathlineto{\pgfqpoint{5.068166in}{3.351783in}}%
\pgfpathlineto{\pgfqpoint{5.098794in}{3.334747in}}%
\pgfpathlineto{\pgfqpoint{5.101422in}{3.333276in}}%
\pgfpathlineto{\pgfqpoint{5.133392in}{3.315269in}}%
\pgfpathlineto{\pgfqpoint{5.134678in}{3.314540in}}%
\pgfpathlineto{\pgfqpoint{5.167488in}{3.295791in}}%
\pgfpathlineto{\pgfqpoint{5.167934in}{3.295534in}}%
\pgfpathlineto{\pgfqpoint{5.201062in}{3.276314in}}%
\pgfpathlineto{\pgfqpoint{5.201190in}{3.276239in}}%
\pgfpathlineto{\pgfqpoint{5.234123in}{3.256836in}}%
\pgfpathlineto{\pgfqpoint{5.234447in}{3.256644in}}%
\pgfpathlineto{\pgfqpoint{5.266700in}{3.237359in}}%
\pgfpathlineto{\pgfqpoint{5.267703in}{3.236755in}}%
\pgfpathlineto{\pgfqpoint{5.298854in}{3.217881in}}%
\pgfpathlineto{\pgfqpoint{5.300959in}{3.216599in}}%
\pgfpathlineto{\pgfqpoint{5.330705in}{3.198403in}}%
\pgfpathlineto{\pgfqpoint{5.334215in}{3.196255in}}%
\pgfpathlineto{\pgfqpoint{5.362487in}{3.178926in}}%
\pgfpathlineto{\pgfqpoint{5.367471in}{3.175892in}}%
\pgfpathlineto{\pgfqpoint{5.394629in}{3.159448in}}%
\pgfpathlineto{\pgfqpoint{5.400728in}{3.155828in}}%
\pgfpathlineto{\pgfqpoint{5.427830in}{3.139970in}}%
\pgfpathlineto{\pgfqpoint{5.433984in}{3.136491in}}%
\pgfpathlineto{\pgfqpoint{5.462944in}{3.120493in}}%
\pgfpathlineto{\pgfqpoint{5.467240in}{3.118215in}}%
\pgfpathlineto{\pgfqpoint{5.500492in}{3.101015in}}%
\pgfpathlineto{\pgfqpoint{5.500496in}{3.101013in}}%
\pgfpathlineto{\pgfqpoint{5.533752in}{3.084628in}}%
\pgfpathlineto{\pgfqpoint{5.540107in}{3.081537in}}%
\pgfpathlineto{\pgfqpoint{5.567008in}{3.068653in}}%
\pgfpathlineto{\pgfqpoint{5.580825in}{3.062060in}}%
\pgfpathlineto{\pgfqpoint{5.600265in}{3.052827in}}%
\pgfpathlineto{\pgfqpoint{5.621783in}{3.042582in}}%
\pgfpathlineto{\pgfqpoint{5.633521in}{3.036986in}}%
\pgfpathlineto{\pgfqpoint{5.662471in}{3.023104in}}%
\pgfpathlineto{\pgfqpoint{5.666777in}{3.021031in}}%
\pgfusepath{stroke}%
\end{pgfscope}%
\begin{pgfscope}%
\pgfpathrectangle{\pgfqpoint{0.711606in}{0.549444in}}{\pgfqpoint{4.955171in}{2.902168in}}%
\pgfusepath{clip}%
\pgfsetbuttcap%
\pgfsetroundjoin%
\pgfsetlinewidth{1.003750pt}%
\definecolor{currentstroke}{rgb}{0.977092,0.550850,0.039050}%
\pgfsetstrokecolor{currentstroke}%
\pgfsetdash{}{0pt}%
\pgfpathmoveto{\pgfqpoint{4.934831in}{3.451613in}}%
\pgfpathlineto{\pgfqpoint{4.935141in}{3.451428in}}%
\pgfpathlineto{\pgfqpoint{4.968397in}{3.432211in}}%
\pgfpathlineto{\pgfqpoint{4.968532in}{3.432135in}}%
\pgfpathlineto{\pgfqpoint{5.001653in}{3.413606in}}%
\pgfpathlineto{\pgfqpoint{5.003374in}{3.412657in}}%
\pgfpathlineto{\pgfqpoint{5.034909in}{3.395288in}}%
\pgfpathlineto{\pgfqpoint{5.038755in}{3.393180in}}%
\pgfpathlineto{\pgfqpoint{5.068166in}{3.377012in}}%
\pgfpathlineto{\pgfqpoint{5.074177in}{3.373702in}}%
\pgfpathlineto{\pgfqpoint{5.101422in}{3.358626in}}%
\pgfpathlineto{\pgfqpoint{5.109337in}{3.354224in}}%
\pgfpathlineto{\pgfqpoint{5.134678in}{3.340042in}}%
\pgfpathlineto{\pgfqpoint{5.144077in}{3.334747in}}%
\pgfpathlineto{\pgfqpoint{5.167934in}{3.321210in}}%
\pgfpathlineto{\pgfqpoint{5.178327in}{3.315269in}}%
\pgfpathlineto{\pgfqpoint{5.201190in}{3.302102in}}%
\pgfpathlineto{\pgfqpoint{5.212063in}{3.295791in}}%
\pgfpathlineto{\pgfqpoint{5.234447in}{3.282702in}}%
\pgfpathlineto{\pgfqpoint{5.245287in}{3.276314in}}%
\pgfpathlineto{\pgfqpoint{5.267703in}{3.263006in}}%
\pgfpathlineto{\pgfqpoint{5.278019in}{3.256836in}}%
\pgfpathlineto{\pgfqpoint{5.300959in}{3.243023in}}%
\pgfpathlineto{\pgfqpoint{5.310303in}{3.237359in}}%
\pgfpathlineto{\pgfqpoint{5.334215in}{3.222787in}}%
\pgfpathlineto{\pgfqpoint{5.342224in}{3.217881in}}%
\pgfpathlineto{\pgfqpoint{5.367471in}{3.202387in}}%
\pgfpathlineto{\pgfqpoint{5.373947in}{3.198403in}}%
\pgfpathlineto{\pgfqpoint{5.400728in}{3.182012in}}%
\pgfpathlineto{\pgfqpoint{5.405786in}{3.178926in}}%
\pgfpathlineto{\pgfqpoint{5.433984in}{3.162003in}}%
\pgfpathlineto{\pgfqpoint{5.438292in}{3.159448in}}%
\pgfpathlineto{\pgfqpoint{5.467240in}{3.142804in}}%
\pgfpathlineto{\pgfqpoint{5.472273in}{3.139970in}}%
\pgfpathlineto{\pgfqpoint{5.500496in}{3.124708in}}%
\pgfpathlineto{\pgfqpoint{5.508485in}{3.120493in}}%
\pgfpathlineto{\pgfqpoint{5.533752in}{3.107638in}}%
\pgfpathlineto{\pgfqpoint{5.547022in}{3.101015in}}%
\pgfpathlineto{\pgfqpoint{5.567008in}{3.091270in}}%
\pgfpathlineto{\pgfqpoint{5.587152in}{3.081537in}}%
\pgfpathlineto{\pgfqpoint{5.600265in}{3.075267in}}%
\pgfpathlineto{\pgfqpoint{5.627912in}{3.062060in}}%
\pgfpathlineto{\pgfqpoint{5.633521in}{3.059386in}}%
\pgfpathlineto{\pgfqpoint{5.666777in}{3.043460in}}%
\pgfusepath{stroke}%
\end{pgfscope}%
\begin{pgfscope}%
\pgfpathrectangle{\pgfqpoint{0.711606in}{0.549444in}}{\pgfqpoint{4.955171in}{2.902168in}}%
\pgfusepath{clip}%
\pgfsetbuttcap%
\pgfsetroundjoin%
\pgfsetlinewidth{1.003750pt}%
\definecolor{currentstroke}{rgb}{0.980824,0.572209,0.028508}%
\pgfsetstrokecolor{currentstroke}%
\pgfsetdash{}{0pt}%
\pgfpathmoveto{\pgfqpoint{4.978396in}{3.451613in}}%
\pgfpathlineto{\pgfqpoint{5.001653in}{3.438499in}}%
\pgfpathlineto{\pgfqpoint{5.013146in}{3.432135in}}%
\pgfpathlineto{\pgfqpoint{5.034909in}{3.420123in}}%
\pgfpathlineto{\pgfqpoint{5.048539in}{3.412657in}}%
\pgfpathlineto{\pgfqpoint{5.068166in}{3.401892in}}%
\pgfpathlineto{\pgfqpoint{5.084053in}{3.393180in}}%
\pgfpathlineto{\pgfqpoint{5.101422in}{3.383614in}}%
\pgfpathlineto{\pgfqpoint{5.119349in}{3.373702in}}%
\pgfpathlineto{\pgfqpoint{5.134678in}{3.365175in}}%
\pgfpathlineto{\pgfqpoint{5.154246in}{3.354224in}}%
\pgfpathlineto{\pgfqpoint{5.167934in}{3.346511in}}%
\pgfpathlineto{\pgfqpoint{5.188663in}{3.334747in}}%
\pgfpathlineto{\pgfqpoint{5.201190in}{3.327585in}}%
\pgfpathlineto{\pgfqpoint{5.222569in}{3.315269in}}%
\pgfpathlineto{\pgfqpoint{5.234447in}{3.308375in}}%
\pgfpathlineto{\pgfqpoint{5.255960in}{3.295791in}}%
\pgfpathlineto{\pgfqpoint{5.267703in}{3.288871in}}%
\pgfpathlineto{\pgfqpoint{5.288849in}{3.276314in}}%
\pgfpathlineto{\pgfqpoint{5.300959in}{3.269071in}}%
\pgfpathlineto{\pgfqpoint{5.321269in}{3.256836in}}%
\pgfpathlineto{\pgfqpoint{5.334215in}{3.248989in}}%
\pgfpathlineto{\pgfqpoint{5.353282in}{3.237359in}}%
\pgfpathlineto{\pgfqpoint{5.367471in}{3.228668in}}%
\pgfpathlineto{\pgfqpoint{5.385006in}{3.217881in}}%
\pgfpathlineto{\pgfqpoint{5.400728in}{3.208213in}}%
\pgfpathlineto{\pgfqpoint{5.416670in}{3.198403in}}%
\pgfpathlineto{\pgfqpoint{5.433984in}{3.187844in}}%
\pgfpathlineto{\pgfqpoint{5.448697in}{3.178926in}}%
\pgfpathlineto{\pgfqpoint{5.467240in}{3.167930in}}%
\pgfpathlineto{\pgfqpoint{5.481767in}{3.159448in}}%
\pgfpathlineto{\pgfqpoint{5.500496in}{3.148894in}}%
\pgfpathlineto{\pgfqpoint{5.516707in}{3.139970in}}%
\pgfpathlineto{\pgfqpoint{5.533752in}{3.130956in}}%
\pgfpathlineto{\pgfqpoint{5.554002in}{3.120493in}}%
\pgfpathlineto{\pgfqpoint{5.567008in}{3.113982in}}%
\pgfpathlineto{\pgfqpoint{5.593308in}{3.101015in}}%
\pgfpathlineto{\pgfqpoint{5.600265in}{3.097646in}}%
\pgfpathlineto{\pgfqpoint{5.633521in}{3.081626in}}%
\pgfpathlineto{\pgfqpoint{5.633705in}{3.081537in}}%
\pgfpathlineto{\pgfqpoint{5.666777in}{3.065666in}}%
\pgfusepath{stroke}%
\end{pgfscope}%
\begin{pgfscope}%
\pgfpathrectangle{\pgfqpoint{0.711606in}{0.549444in}}{\pgfqpoint{4.955171in}{2.902168in}}%
\pgfusepath{clip}%
\pgfsetbuttcap%
\pgfsetroundjoin%
\pgfsetlinewidth{1.003750pt}%
\definecolor{currentstroke}{rgb}{0.984591,0.601122,0.023606}%
\pgfsetstrokecolor{currentstroke}%
\pgfsetdash{}{0pt}%
\pgfpathmoveto{\pgfqpoint{5.022431in}{3.451613in}}%
\pgfpathlineto{\pgfqpoint{5.034909in}{3.444698in}}%
\pgfpathlineto{\pgfqpoint{5.057830in}{3.432135in}}%
\pgfpathlineto{\pgfqpoint{5.068166in}{3.426470in}}%
\pgfpathlineto{\pgfqpoint{5.093431in}{3.412657in}}%
\pgfpathlineto{\pgfqpoint{5.101422in}{3.408274in}}%
\pgfpathlineto{\pgfqpoint{5.128863in}{3.393180in}}%
\pgfpathlineto{\pgfqpoint{5.134678in}{3.389964in}}%
\pgfpathlineto{\pgfqpoint{5.163921in}{3.373702in}}%
\pgfpathlineto{\pgfqpoint{5.167934in}{3.371456in}}%
\pgfpathlineto{\pgfqpoint{5.198508in}{3.354224in}}%
\pgfpathlineto{\pgfqpoint{5.201190in}{3.352702in}}%
\pgfpathlineto{\pgfqpoint{5.232587in}{3.334747in}}%
\pgfpathlineto{\pgfqpoint{5.234447in}{3.333676in}}%
\pgfpathlineto{\pgfqpoint{5.266150in}{3.315269in}}%
\pgfpathlineto{\pgfqpoint{5.267703in}{3.314361in}}%
\pgfpathlineto{\pgfqpoint{5.299204in}{3.295791in}}%
\pgfpathlineto{\pgfqpoint{5.300959in}{3.294749in}}%
\pgfpathlineto{\pgfqpoint{5.331774in}{3.276314in}}%
\pgfpathlineto{\pgfqpoint{5.334215in}{3.274844in}}%
\pgfpathlineto{\pgfqpoint{5.363905in}{3.256836in}}%
\pgfpathlineto{\pgfqpoint{5.367471in}{3.254661in}}%
\pgfpathlineto{\pgfqpoint{5.395682in}{3.237359in}}%
\pgfpathlineto{\pgfqpoint{5.400728in}{3.234256in}}%
\pgfpathlineto{\pgfqpoint{5.427269in}{3.217881in}}%
\pgfpathlineto{\pgfqpoint{5.433984in}{3.213751in}}%
\pgfpathlineto{\pgfqpoint{5.458977in}{3.198403in}}%
\pgfpathlineto{\pgfqpoint{5.467240in}{3.193397in}}%
\pgfpathlineto{\pgfqpoint{5.491350in}{3.178926in}}%
\pgfpathlineto{\pgfqpoint{5.500496in}{3.173585in}}%
\pgfpathlineto{\pgfqpoint{5.525168in}{3.159448in}}%
\pgfpathlineto{\pgfqpoint{5.533752in}{3.154716in}}%
\pgfpathlineto{\pgfqpoint{5.561156in}{3.139970in}}%
\pgfpathlineto{\pgfqpoint{5.567008in}{3.136938in}}%
\pgfpathlineto{\pgfqpoint{5.599412in}{3.120493in}}%
\pgfpathlineto{\pgfqpoint{5.600265in}{3.120071in}}%
\pgfpathlineto{\pgfqpoint{5.633521in}{3.103777in}}%
\pgfpathlineto{\pgfqpoint{5.639197in}{3.101015in}}%
\pgfpathlineto{\pgfqpoint{5.666777in}{3.087710in}}%
\pgfusepath{stroke}%
\end{pgfscope}%
\begin{pgfscope}%
\pgfpathrectangle{\pgfqpoint{0.711606in}{0.549444in}}{\pgfqpoint{4.955171in}{2.902168in}}%
\pgfusepath{clip}%
\pgfsetbuttcap%
\pgfsetroundjoin%
\pgfsetlinewidth{1.003750pt}%
\definecolor{currentstroke}{rgb}{0.986502,0.623105,0.027814}%
\pgfsetstrokecolor{currentstroke}%
\pgfsetdash{}{0pt}%
\pgfpathmoveto{\pgfqpoint{5.066630in}{3.451613in}}%
\pgfpathlineto{\pgfqpoint{5.068166in}{3.450770in}}%
\pgfpathlineto{\pgfqpoint{5.101422in}{3.432620in}}%
\pgfpathlineto{\pgfqpoint{5.102310in}{3.432135in}}%
\pgfpathlineto{\pgfqpoint{5.134678in}{3.414404in}}%
\pgfpathlineto{\pgfqpoint{5.137854in}{3.412657in}}%
\pgfpathlineto{\pgfqpoint{5.167934in}{3.396025in}}%
\pgfpathlineto{\pgfqpoint{5.173049in}{3.393180in}}%
\pgfpathlineto{\pgfqpoint{5.201190in}{3.377423in}}%
\pgfpathlineto{\pgfqpoint{5.207790in}{3.373702in}}%
\pgfpathlineto{\pgfqpoint{5.234447in}{3.358564in}}%
\pgfpathlineto{\pgfqpoint{5.242032in}{3.354224in}}%
\pgfpathlineto{\pgfqpoint{5.267703in}{3.339429in}}%
\pgfpathlineto{\pgfqpoint{5.275765in}{3.334747in}}%
\pgfpathlineto{\pgfqpoint{5.300959in}{3.320005in}}%
\pgfpathlineto{\pgfqpoint{5.308992in}{3.315269in}}%
\pgfpathlineto{\pgfqpoint{5.334215in}{3.300288in}}%
\pgfpathlineto{\pgfqpoint{5.341731in}{3.295791in}}%
\pgfpathlineto{\pgfqpoint{5.367471in}{3.280283in}}%
\pgfpathlineto{\pgfqpoint{5.374014in}{3.276314in}}%
\pgfpathlineto{\pgfqpoint{5.400728in}{3.260012in}}%
\pgfpathlineto{\pgfqpoint{5.405901in}{3.256836in}}%
\pgfpathlineto{\pgfqpoint{5.433984in}{3.239537in}}%
\pgfpathlineto{\pgfqpoint{5.437507in}{3.237359in}}%
\pgfpathlineto{\pgfqpoint{5.467240in}{3.218999in}}%
\pgfpathlineto{\pgfqpoint{5.469052in}{3.217881in}}%
\pgfpathlineto{\pgfqpoint{5.500496in}{3.198673in}}%
\pgfpathlineto{\pgfqpoint{5.500940in}{3.198403in}}%
\pgfpathlineto{\pgfqpoint{5.533752in}{3.178976in}}%
\pgfpathlineto{\pgfqpoint{5.533838in}{3.178926in}}%
\pgfpathlineto{\pgfqpoint{5.567008in}{3.160300in}}%
\pgfpathlineto{\pgfqpoint{5.568561in}{3.159448in}}%
\pgfpathlineto{\pgfqpoint{5.600265in}{3.142717in}}%
\pgfpathlineto{\pgfqpoint{5.605584in}{3.139970in}}%
\pgfpathlineto{\pgfqpoint{5.633521in}{3.125955in}}%
\pgfpathlineto{\pgfqpoint{5.644553in}{3.120493in}}%
\pgfpathlineto{\pgfqpoint{5.666777in}{3.109657in}}%
\pgfusepath{stroke}%
\end{pgfscope}%
\begin{pgfscope}%
\pgfpathrectangle{\pgfqpoint{0.711606in}{0.549444in}}{\pgfqpoint{4.955171in}{2.902168in}}%
\pgfusepath{clip}%
\pgfsetbuttcap%
\pgfsetroundjoin%
\pgfsetlinewidth{1.003750pt}%
\definecolor{currentstroke}{rgb}{0.987819,0.652773,0.045581}%
\pgfsetstrokecolor{currentstroke}%
\pgfsetdash{}{0pt}%
\pgfpathmoveto{\pgfqpoint{5.110657in}{3.451613in}}%
\pgfpathlineto{\pgfqpoint{5.134678in}{3.438501in}}%
\pgfpathlineto{\pgfqpoint{5.146314in}{3.432135in}}%
\pgfpathlineto{\pgfqpoint{5.167934in}{3.420245in}}%
\pgfpathlineto{\pgfqpoint{5.181659in}{3.412657in}}%
\pgfpathlineto{\pgfqpoint{5.201190in}{3.401791in}}%
\pgfpathlineto{\pgfqpoint{5.216568in}{3.393180in}}%
\pgfpathlineto{\pgfqpoint{5.234447in}{3.383097in}}%
\pgfpathlineto{\pgfqpoint{5.250985in}{3.373702in}}%
\pgfpathlineto{\pgfqpoint{5.267703in}{3.364136in}}%
\pgfpathlineto{\pgfqpoint{5.284895in}{3.354224in}}%
\pgfpathlineto{\pgfqpoint{5.300959in}{3.344894in}}%
\pgfpathlineto{\pgfqpoint{5.318297in}{3.334747in}}%
\pgfpathlineto{\pgfqpoint{5.334215in}{3.325362in}}%
\pgfpathlineto{\pgfqpoint{5.351204in}{3.315269in}}%
\pgfpathlineto{\pgfqpoint{5.367471in}{3.305536in}}%
\pgfpathlineto{\pgfqpoint{5.383641in}{3.295791in}}%
\pgfpathlineto{\pgfqpoint{5.400728in}{3.285427in}}%
\pgfpathlineto{\pgfqpoint{5.415652in}{3.276314in}}%
\pgfpathlineto{\pgfqpoint{5.433984in}{3.265063in}}%
\pgfpathlineto{\pgfqpoint{5.447318in}{3.256836in}}%
\pgfpathlineto{\pgfqpoint{5.467240in}{3.244521in}}%
\pgfpathlineto{\pgfqpoint{5.478798in}{3.237359in}}%
\pgfpathlineto{\pgfqpoint{5.500496in}{3.223971in}}%
\pgfpathlineto{\pgfqpoint{5.510391in}{3.217881in}}%
\pgfpathlineto{\pgfqpoint{5.533752in}{3.203721in}}%
\pgfpathlineto{\pgfqpoint{5.542620in}{3.198403in}}%
\pgfpathlineto{\pgfqpoint{5.567008in}{3.184199in}}%
\pgfpathlineto{\pgfqpoint{5.576243in}{3.178926in}}%
\pgfpathlineto{\pgfqpoint{5.600265in}{3.165729in}}%
\pgfpathlineto{\pgfqpoint{5.611970in}{3.159448in}}%
\pgfpathlineto{\pgfqpoint{5.633521in}{3.148291in}}%
\pgfpathlineto{\pgfqpoint{5.649897in}{3.139970in}}%
\pgfpathlineto{\pgfqpoint{5.666777in}{3.131592in}}%
\pgfusepath{stroke}%
\end{pgfscope}%
\begin{pgfscope}%
\pgfpathrectangle{\pgfqpoint{0.711606in}{0.549444in}}{\pgfqpoint{4.955171in}{2.902168in}}%
\pgfusepath{clip}%
\pgfsetbuttcap%
\pgfsetroundjoin%
\pgfsetlinewidth{1.003750pt}%
\definecolor{currentstroke}{rgb}{0.987714,0.682807,0.072489}%
\pgfsetstrokecolor{currentstroke}%
\pgfsetdash{}{0pt}%
\pgfpathmoveto{\pgfqpoint{5.154310in}{3.451613in}}%
\pgfpathlineto{\pgfqpoint{5.167934in}{3.444156in}}%
\pgfpathlineto{\pgfqpoint{5.189803in}{3.432135in}}%
\pgfpathlineto{\pgfqpoint{5.201190in}{3.425838in}}%
\pgfpathlineto{\pgfqpoint{5.224880in}{3.412657in}}%
\pgfpathlineto{\pgfqpoint{5.234447in}{3.407299in}}%
\pgfpathlineto{\pgfqpoint{5.259474in}{3.393180in}}%
\pgfpathlineto{\pgfqpoint{5.267703in}{3.388505in}}%
\pgfpathlineto{\pgfqpoint{5.293564in}{3.373702in}}%
\pgfpathlineto{\pgfqpoint{5.300959in}{3.369438in}}%
\pgfpathlineto{\pgfqpoint{5.327144in}{3.354224in}}%
\pgfpathlineto{\pgfqpoint{5.334215in}{3.350086in}}%
\pgfpathlineto{\pgfqpoint{5.360226in}{3.334747in}}%
\pgfpathlineto{\pgfqpoint{5.367471in}{3.330443in}}%
\pgfpathlineto{\pgfqpoint{5.392827in}{3.315269in}}%
\pgfpathlineto{\pgfqpoint{5.400728in}{3.310508in}}%
\pgfpathlineto{\pgfqpoint{5.424980in}{3.295791in}}%
\pgfpathlineto{\pgfqpoint{5.433984in}{3.290295in}}%
\pgfpathlineto{\pgfqpoint{5.456744in}{3.276314in}}%
\pgfpathlineto{\pgfqpoint{5.467240in}{3.269840in}}%
\pgfpathlineto{\pgfqpoint{5.488231in}{3.256836in}}%
\pgfpathlineto{\pgfqpoint{5.500496in}{3.249239in}}%
\pgfpathlineto{\pgfqpoint{5.519657in}{3.237359in}}%
\pgfpathlineto{\pgfqpoint{5.533752in}{3.228689in}}%
\pgfpathlineto{\pgfqpoint{5.551418in}{3.217881in}}%
\pgfpathlineto{\pgfqpoint{5.567008in}{3.208535in}}%
\pgfpathlineto{\pgfqpoint{5.584153in}{3.198403in}}%
\pgfpathlineto{\pgfqpoint{5.600265in}{3.189198in}}%
\pgfpathlineto{\pgfqpoint{5.618643in}{3.178926in}}%
\pgfpathlineto{\pgfqpoint{5.633521in}{3.170927in}}%
\pgfpathlineto{\pgfqpoint{5.655359in}{3.159448in}}%
\pgfpathlineto{\pgfqpoint{5.666777in}{3.153632in}}%
\pgfusepath{stroke}%
\end{pgfscope}%
\begin{pgfscope}%
\pgfpathrectangle{\pgfqpoint{0.711606in}{0.549444in}}{\pgfqpoint{4.955171in}{2.902168in}}%
\pgfusepath{clip}%
\pgfsetbuttcap%
\pgfsetroundjoin%
\pgfsetlinewidth{1.003750pt}%
\definecolor{currentstroke}{rgb}{0.986694,0.705540,0.095694}%
\pgfsetstrokecolor{currentstroke}%
\pgfsetdash{}{0pt}%
\pgfpathmoveto{\pgfqpoint{5.197486in}{3.451613in}}%
\pgfpathlineto{\pgfqpoint{5.201190in}{3.449576in}}%
\pgfpathlineto{\pgfqpoint{5.232732in}{3.432135in}}%
\pgfpathlineto{\pgfqpoint{5.234447in}{3.431180in}}%
\pgfpathlineto{\pgfqpoint{5.267505in}{3.412657in}}%
\pgfpathlineto{\pgfqpoint{5.267703in}{3.412546in}}%
\pgfpathlineto{\pgfqpoint{5.300959in}{3.393642in}}%
\pgfpathlineto{\pgfqpoint{5.301767in}{3.393180in}}%
\pgfpathlineto{\pgfqpoint{5.334215in}{3.374463in}}%
\pgfpathlineto{\pgfqpoint{5.335524in}{3.373702in}}%
\pgfpathlineto{\pgfqpoint{5.367471in}{3.354999in}}%
\pgfpathlineto{\pgfqpoint{5.368784in}{3.354224in}}%
\pgfpathlineto{\pgfqpoint{5.400728in}{3.335247in}}%
\pgfpathlineto{\pgfqpoint{5.401563in}{3.334747in}}%
\pgfpathlineto{\pgfqpoint{5.433883in}{3.315269in}}%
\pgfpathlineto{\pgfqpoint{5.433984in}{3.315208in}}%
\pgfpathlineto{\pgfqpoint{5.465772in}{3.295791in}}%
\pgfpathlineto{\pgfqpoint{5.467240in}{3.294890in}}%
\pgfpathlineto{\pgfqpoint{5.497321in}{3.276314in}}%
\pgfpathlineto{\pgfqpoint{5.500496in}{3.274348in}}%
\pgfpathlineto{\pgfqpoint{5.528681in}{3.256836in}}%
\pgfpathlineto{\pgfqpoint{5.533752in}{3.253694in}}%
\pgfpathlineto{\pgfqpoint{5.560142in}{3.237359in}}%
\pgfpathlineto{\pgfqpoint{5.567008in}{3.233160in}}%
\pgfpathlineto{\pgfqpoint{5.592210in}{3.217881in}}%
\pgfpathlineto{\pgfqpoint{5.600265in}{3.213122in}}%
\pgfpathlineto{\pgfqpoint{5.625616in}{3.198403in}}%
\pgfpathlineto{\pgfqpoint{5.633521in}{3.193981in}}%
\pgfpathlineto{\pgfqpoint{5.661057in}{3.178926in}}%
\pgfpathlineto{\pgfqpoint{5.666777in}{3.175912in}}%
\pgfusepath{stroke}%
\end{pgfscope}%
\begin{pgfscope}%
\pgfpathrectangle{\pgfqpoint{0.711606in}{0.549444in}}{\pgfqpoint{4.955171in}{2.902168in}}%
\pgfusepath{clip}%
\pgfsetbuttcap%
\pgfsetroundjoin%
\pgfsetlinewidth{1.003750pt}%
\definecolor{currentstroke}{rgb}{0.984075,0.736087,0.129527}%
\pgfsetstrokecolor{currentstroke}%
\pgfsetdash{}{0pt}%
\pgfpathmoveto{\pgfqpoint{5.240067in}{3.451613in}}%
\pgfpathlineto{\pgfqpoint{5.267703in}{3.436225in}}%
\pgfpathlineto{\pgfqpoint{5.274998in}{3.432135in}}%
\pgfpathlineto{\pgfqpoint{5.300959in}{3.417480in}}%
\pgfpathlineto{\pgfqpoint{5.309439in}{3.412657in}}%
\pgfpathlineto{\pgfqpoint{5.334215in}{3.398467in}}%
\pgfpathlineto{\pgfqpoint{5.343379in}{3.393180in}}%
\pgfpathlineto{\pgfqpoint{5.367471in}{3.379177in}}%
\pgfpathlineto{\pgfqpoint{5.376821in}{3.373702in}}%
\pgfpathlineto{\pgfqpoint{5.400728in}{3.359601in}}%
\pgfpathlineto{\pgfqpoint{5.409776in}{3.354224in}}%
\pgfpathlineto{\pgfqpoint{5.433984in}{3.339738in}}%
\pgfpathlineto{\pgfqpoint{5.442264in}{3.334747in}}%
\pgfpathlineto{\pgfqpoint{5.467240in}{3.319591in}}%
\pgfpathlineto{\pgfqpoint{5.474315in}{3.315269in}}%
\pgfpathlineto{\pgfqpoint{5.500496in}{3.299184in}}%
\pgfpathlineto{\pgfqpoint{5.505986in}{3.295791in}}%
\pgfpathlineto{\pgfqpoint{5.533752in}{3.278572in}}%
\pgfpathlineto{\pgfqpoint{5.537380in}{3.276314in}}%
\pgfpathlineto{\pgfqpoint{5.567008in}{3.257888in}}%
\pgfpathlineto{\pgfqpoint{5.568700in}{3.256836in}}%
\pgfpathlineto{\pgfqpoint{5.600265in}{3.237391in}}%
\pgfpathlineto{\pgfqpoint{5.600318in}{3.237359in}}%
\pgfpathlineto{\pgfqpoint{5.632847in}{3.217881in}}%
\pgfpathlineto{\pgfqpoint{5.633521in}{3.217490in}}%
\pgfpathlineto{\pgfqpoint{5.666777in}{3.198564in}}%
\pgfusepath{stroke}%
\end{pgfscope}%
\begin{pgfscope}%
\pgfpathrectangle{\pgfqpoint{0.711606in}{0.549444in}}{\pgfqpoint{4.955171in}{2.902168in}}%
\pgfusepath{clip}%
\pgfsetbuttcap%
\pgfsetroundjoin%
\pgfsetlinewidth{1.003750pt}%
\definecolor{currentstroke}{rgb}{0.981173,0.759135,0.156863}%
\pgfsetstrokecolor{currentstroke}%
\pgfsetdash{}{0pt}%
\pgfpathmoveto{\pgfqpoint{5.282049in}{3.451613in}}%
\pgfpathlineto{\pgfqpoint{5.300959in}{3.441009in}}%
\pgfpathlineto{\pgfqpoint{5.316672in}{3.432135in}}%
\pgfpathlineto{\pgfqpoint{5.334215in}{3.422158in}}%
\pgfpathlineto{\pgfqpoint{5.350796in}{3.412657in}}%
\pgfpathlineto{\pgfqpoint{5.367471in}{3.403035in}}%
\pgfpathlineto{\pgfqpoint{5.384422in}{3.393180in}}%
\pgfpathlineto{\pgfqpoint{5.400728in}{3.383632in}}%
\pgfpathlineto{\pgfqpoint{5.417559in}{3.373702in}}%
\pgfpathlineto{\pgfqpoint{5.433984in}{3.363943in}}%
\pgfpathlineto{\pgfqpoint{5.450221in}{3.354224in}}%
\pgfpathlineto{\pgfqpoint{5.467240in}{3.343967in}}%
\pgfpathlineto{\pgfqpoint{5.482432in}{3.334747in}}%
\pgfpathlineto{\pgfqpoint{5.500496in}{3.323714in}}%
\pgfpathlineto{\pgfqpoint{5.514234in}{3.315269in}}%
\pgfpathlineto{\pgfqpoint{5.533752in}{3.303211in}}%
\pgfpathlineto{\pgfqpoint{5.545700in}{3.295791in}}%
\pgfpathlineto{\pgfqpoint{5.567008in}{3.282531in}}%
\pgfpathlineto{\pgfqpoint{5.576973in}{3.276314in}}%
\pgfpathlineto{\pgfqpoint{5.600265in}{3.261836in}}%
\pgfpathlineto{\pgfqpoint{5.608325in}{3.256836in}}%
\pgfpathlineto{\pgfqpoint{5.633521in}{3.241424in}}%
\pgfpathlineto{\pgfqpoint{5.640232in}{3.237359in}}%
\pgfpathlineto{\pgfqpoint{5.666777in}{3.221714in}}%
\pgfusepath{stroke}%
\end{pgfscope}%
\begin{pgfscope}%
\pgfpathrectangle{\pgfqpoint{0.711606in}{0.549444in}}{\pgfqpoint{4.955171in}{2.902168in}}%
\pgfusepath{clip}%
\pgfsetbuttcap%
\pgfsetroundjoin%
\pgfsetlinewidth{1.003750pt}%
\definecolor{currentstroke}{rgb}{0.976108,0.789974,0.196018}%
\pgfsetstrokecolor{currentstroke}%
\pgfsetdash{}{0pt}%
\pgfpathmoveto{\pgfqpoint{5.323471in}{3.451613in}}%
\pgfpathlineto{\pgfqpoint{5.334215in}{3.445544in}}%
\pgfpathlineto{\pgfqpoint{5.357781in}{3.432135in}}%
\pgfpathlineto{\pgfqpoint{5.367471in}{3.426583in}}%
\pgfpathlineto{\pgfqpoint{5.391594in}{3.412657in}}%
\pgfpathlineto{\pgfqpoint{5.400728in}{3.407347in}}%
\pgfpathlineto{\pgfqpoint{5.424915in}{3.393180in}}%
\pgfpathlineto{\pgfqpoint{5.433984in}{3.387830in}}%
\pgfpathlineto{\pgfqpoint{5.457757in}{3.373702in}}%
\pgfpathlineto{\pgfqpoint{5.467240in}{3.368027in}}%
\pgfpathlineto{\pgfqpoint{5.490138in}{3.354224in}}%
\pgfpathlineto{\pgfqpoint{5.500496in}{3.347939in}}%
\pgfpathlineto{\pgfqpoint{5.522089in}{3.334747in}}%
\pgfpathlineto{\pgfqpoint{5.533752in}{3.327579in}}%
\pgfpathlineto{\pgfqpoint{5.553664in}{3.315269in}}%
\pgfpathlineto{\pgfqpoint{5.567008in}{3.306985in}}%
\pgfpathlineto{\pgfqpoint{5.584961in}{3.295791in}}%
\pgfpathlineto{\pgfqpoint{5.600265in}{3.286247in}}%
\pgfpathlineto{\pgfqpoint{5.616175in}{3.276314in}}%
\pgfpathlineto{\pgfqpoint{5.633521in}{3.265561in}}%
\pgfpathlineto{\pgfqpoint{5.647660in}{3.256836in}}%
\pgfpathlineto{\pgfqpoint{5.666777in}{3.245262in}}%
\pgfusepath{stroke}%
\end{pgfscope}%
\begin{pgfscope}%
\pgfpathrectangle{\pgfqpoint{0.711606in}{0.549444in}}{\pgfqpoint{4.955171in}{2.902168in}}%
\pgfusepath{clip}%
\pgfsetbuttcap%
\pgfsetroundjoin%
\pgfsetlinewidth{1.003750pt}%
\definecolor{currentstroke}{rgb}{0.969783,0.820825,0.238686}%
\pgfsetstrokecolor{currentstroke}%
\pgfsetdash{}{0pt}%
\pgfpathmoveto{\pgfqpoint{5.364339in}{3.451613in}}%
\pgfpathlineto{\pgfqpoint{5.367471in}{3.449830in}}%
\pgfpathlineto{\pgfqpoint{5.398340in}{3.432135in}}%
\pgfpathlineto{\pgfqpoint{5.400728in}{3.430757in}}%
\pgfpathlineto{\pgfqpoint{5.431849in}{3.412657in}}%
\pgfpathlineto{\pgfqpoint{5.433984in}{3.411407in}}%
\pgfpathlineto{\pgfqpoint{5.464874in}{3.393180in}}%
\pgfpathlineto{\pgfqpoint{5.467240in}{3.391774in}}%
\pgfpathlineto{\pgfqpoint{5.497433in}{3.373702in}}%
\pgfpathlineto{\pgfqpoint{5.500496in}{3.371856in}}%
\pgfpathlineto{\pgfqpoint{5.529548in}{3.354224in}}%
\pgfpathlineto{\pgfqpoint{5.533752in}{3.351656in}}%
\pgfpathlineto{\pgfqpoint{5.561256in}{3.334747in}}%
\pgfpathlineto{\pgfqpoint{5.567008in}{3.331192in}}%
\pgfpathlineto{\pgfqpoint{5.592630in}{3.315269in}}%
\pgfpathlineto{\pgfqpoint{5.600265in}{3.310511in}}%
\pgfpathlineto{\pgfqpoint{5.623803in}{3.295791in}}%
\pgfpathlineto{\pgfqpoint{5.633521in}{3.289727in}}%
\pgfpathlineto{\pgfqpoint{5.655033in}{3.276314in}}%
\pgfpathlineto{\pgfqpoint{5.666777in}{3.269072in}}%
\pgfusepath{stroke}%
\end{pgfscope}%
\begin{pgfscope}%
\pgfpathrectangle{\pgfqpoint{0.711606in}{0.549444in}}{\pgfqpoint{4.955171in}{2.902168in}}%
\pgfusepath{clip}%
\pgfsetbuttcap%
\pgfsetroundjoin%
\pgfsetlinewidth{1.003750pt}%
\definecolor{currentstroke}{rgb}{0.964394,0.843848,0.273391}%
\pgfsetstrokecolor{currentstroke}%
\pgfsetdash{}{0pt}%
\pgfpathmoveto{\pgfqpoint{5.404621in}{3.451613in}}%
\pgfpathlineto{\pgfqpoint{5.433984in}{3.434655in}}%
\pgfpathlineto{\pgfqpoint{5.438315in}{3.432135in}}%
\pgfpathlineto{\pgfqpoint{5.467240in}{3.415189in}}%
\pgfpathlineto{\pgfqpoint{5.471529in}{3.412657in}}%
\pgfpathlineto{\pgfqpoint{5.500496in}{3.395443in}}%
\pgfpathlineto{\pgfqpoint{5.504277in}{3.393180in}}%
\pgfpathlineto{\pgfqpoint{5.533752in}{3.375415in}}%
\pgfpathlineto{\pgfqpoint{5.536575in}{3.373702in}}%
\pgfpathlineto{\pgfqpoint{5.567008in}{3.355113in}}%
\pgfpathlineto{\pgfqpoint{5.568453in}{3.354224in}}%
\pgfpathlineto{\pgfqpoint{5.599955in}{3.334747in}}%
\pgfpathlineto{\pgfqpoint{5.600265in}{3.334555in}}%
\pgfpathlineto{\pgfqpoint{5.631160in}{3.315269in}}%
\pgfpathlineto{\pgfqpoint{5.633521in}{3.313794in}}%
\pgfpathlineto{\pgfqpoint{5.662265in}{3.295791in}}%
\pgfpathlineto{\pgfqpoint{5.666777in}{3.292979in}}%
\pgfusepath{stroke}%
\end{pgfscope}%
\begin{pgfscope}%
\pgfpathrectangle{\pgfqpoint{0.711606in}{0.549444in}}{\pgfqpoint{4.955171in}{2.902168in}}%
\pgfusepath{clip}%
\pgfsetbuttcap%
\pgfsetroundjoin%
\pgfsetlinewidth{1.003750pt}%
\definecolor{currentstroke}{rgb}{0.956834,0.874129,0.323974}%
\pgfsetstrokecolor{currentstroke}%
\pgfsetdash{}{0pt}%
\pgfpathmoveto{\pgfqpoint{5.444348in}{3.451613in}}%
\pgfpathlineto{\pgfqpoint{5.467240in}{3.438295in}}%
\pgfpathlineto{\pgfqpoint{5.477751in}{3.432135in}}%
\pgfpathlineto{\pgfqpoint{5.500496in}{3.418713in}}%
\pgfpathlineto{\pgfqpoint{5.510684in}{3.412657in}}%
\pgfpathlineto{\pgfqpoint{5.533752in}{3.398851in}}%
\pgfpathlineto{\pgfqpoint{5.543162in}{3.393180in}}%
\pgfpathlineto{\pgfqpoint{5.567008in}{3.378710in}}%
\pgfpathlineto{\pgfqpoint{5.575206in}{3.373702in}}%
\pgfpathlineto{\pgfqpoint{5.600265in}{3.358298in}}%
\pgfpathlineto{\pgfqpoint{5.606852in}{3.354224in}}%
\pgfpathlineto{\pgfqpoint{5.633521in}{3.337647in}}%
\pgfpathlineto{\pgfqpoint{5.638164in}{3.334747in}}%
\pgfpathlineto{\pgfqpoint{5.666777in}{3.316831in}}%
\pgfusepath{stroke}%
\end{pgfscope}%
\begin{pgfscope}%
\pgfpathrectangle{\pgfqpoint{0.711606in}{0.549444in}}{\pgfqpoint{4.955171in}{2.902168in}}%
\pgfusepath{clip}%
\pgfsetbuttcap%
\pgfsetroundjoin%
\pgfsetlinewidth{1.003750pt}%
\definecolor{currentstroke}{rgb}{0.951546,0.896226,0.365627}%
\pgfsetstrokecolor{currentstroke}%
\pgfsetdash{}{0pt}%
\pgfpathmoveto{\pgfqpoint{5.483571in}{3.451613in}}%
\pgfpathlineto{\pgfqpoint{5.500496in}{3.441694in}}%
\pgfpathlineto{\pgfqpoint{5.516692in}{3.432135in}}%
\pgfpathlineto{\pgfqpoint{5.533752in}{3.421995in}}%
\pgfpathlineto{\pgfqpoint{5.549353in}{3.412657in}}%
\pgfpathlineto{\pgfqpoint{5.567008in}{3.402017in}}%
\pgfpathlineto{\pgfqpoint{5.581572in}{3.393180in}}%
\pgfpathlineto{\pgfqpoint{5.600265in}{3.381763in}}%
\pgfpathlineto{\pgfqpoint{5.613376in}{3.373702in}}%
\pgfpathlineto{\pgfqpoint{5.633521in}{3.361245in}}%
\pgfpathlineto{\pgfqpoint{5.644809in}{3.354224in}}%
\pgfpathlineto{\pgfqpoint{5.666777in}{3.340505in}}%
\pgfusepath{stroke}%
\end{pgfscope}%
\begin{pgfscope}%
\pgfpathrectangle{\pgfqpoint{0.711606in}{0.549444in}}{\pgfqpoint{4.955171in}{2.902168in}}%
\pgfusepath{clip}%
\pgfsetbuttcap%
\pgfsetroundjoin%
\pgfsetlinewidth{1.003750pt}%
\definecolor{currentstroke}{rgb}{0.946809,0.924168,0.426373}%
\pgfsetstrokecolor{currentstroke}%
\pgfsetdash{}{0pt}%
\pgfpathmoveto{\pgfqpoint{5.522303in}{3.451613in}}%
\pgfpathlineto{\pgfqpoint{5.533752in}{3.444855in}}%
\pgfpathlineto{\pgfqpoint{5.555153in}{3.432135in}}%
\pgfpathlineto{\pgfqpoint{5.567008in}{3.425039in}}%
\pgfpathlineto{\pgfqpoint{5.587553in}{3.412657in}}%
\pgfpathlineto{\pgfqpoint{5.600265in}{3.404945in}}%
\pgfpathlineto{\pgfqpoint{5.619526in}{3.393180in}}%
\pgfpathlineto{\pgfqpoint{5.633521in}{3.384577in}}%
\pgfpathlineto{\pgfqpoint{5.651103in}{3.373702in}}%
\pgfpathlineto{\pgfqpoint{5.666777in}{3.363956in}}%
\pgfusepath{stroke}%
\end{pgfscope}%
\begin{pgfscope}%
\pgfpathrectangle{\pgfqpoint{0.711606in}{0.549444in}}{\pgfqpoint{4.955171in}{2.902168in}}%
\pgfusepath{clip}%
\pgfsetbuttcap%
\pgfsetroundjoin%
\pgfsetlinewidth{1.003750pt}%
\definecolor{currentstroke}{rgb}{0.947937,0.949318,0.491426}%
\pgfsetstrokecolor{currentstroke}%
\pgfsetdash{}{0pt}%
\pgfpathmoveto{\pgfqpoint{5.560563in}{3.451613in}}%
\pgfpathlineto{\pgfqpoint{5.567008in}{3.447781in}}%
\pgfpathlineto{\pgfqpoint{5.593149in}{3.432135in}}%
\pgfpathlineto{\pgfqpoint{5.600265in}{3.427847in}}%
\pgfpathlineto{\pgfqpoint{5.625299in}{3.412657in}}%
\pgfpathlineto{\pgfqpoint{5.633521in}{3.407636in}}%
\pgfpathlineto{\pgfqpoint{5.657038in}{3.393180in}}%
\pgfpathlineto{\pgfqpoint{5.666777in}{3.387157in}}%
\pgfusepath{stroke}%
\end{pgfscope}%
\begin{pgfscope}%
\pgfpathrectangle{\pgfqpoint{0.711606in}{0.549444in}}{\pgfqpoint{4.955171in}{2.902168in}}%
\pgfusepath{clip}%
\pgfsetbuttcap%
\pgfsetroundjoin%
\pgfsetlinewidth{1.003750pt}%
\definecolor{currentstroke}{rgb}{0.954529,0.965896,0.540361}%
\pgfsetstrokecolor{currentstroke}%
\pgfsetdash{}{0pt}%
\pgfpathmoveto{\pgfqpoint{5.598363in}{3.451613in}}%
\pgfpathlineto{\pgfqpoint{5.600265in}{3.450474in}}%
\pgfpathlineto{\pgfqpoint{5.630697in}{3.432135in}}%
\pgfpathlineto{\pgfqpoint{5.633521in}{3.430422in}}%
\pgfpathlineto{\pgfqpoint{5.662608in}{3.412657in}}%
\pgfpathlineto{\pgfqpoint{5.666777in}{3.410095in}}%
\pgfusepath{stroke}%
\end{pgfscope}%
\begin{pgfscope}%
\pgfpathrectangle{\pgfqpoint{0.711606in}{0.549444in}}{\pgfqpoint{4.955171in}{2.902168in}}%
\pgfusepath{clip}%
\pgfsetbuttcap%
\pgfsetroundjoin%
\pgfsetlinewidth{1.003750pt}%
\definecolor{currentstroke}{rgb}{0.971162,0.985282,0.602154}%
\pgfsetstrokecolor{currentstroke}%
\pgfsetdash{}{0pt}%
\pgfpathmoveto{\pgfqpoint{5.635696in}{3.451613in}}%
\pgfpathlineto{\pgfqpoint{5.666777in}{3.432760in}}%
\pgfusepath{stroke}%
\end{pgfscope}%
\begin{pgfscope}%
\pgfsetrectcap%
\pgfsetmiterjoin%
\pgfsetlinewidth{0.803000pt}%
\definecolor{currentstroke}{rgb}{0.000000,0.000000,0.000000}%
\pgfsetstrokecolor{currentstroke}%
\pgfsetdash{}{0pt}%
\pgfpathmoveto{\pgfqpoint{0.711606in}{0.549444in}}%
\pgfpathlineto{\pgfqpoint{0.711606in}{3.451613in}}%
\pgfusepath{stroke}%
\end{pgfscope}%
\begin{pgfscope}%
\pgfsetrectcap%
\pgfsetmiterjoin%
\pgfsetlinewidth{0.803000pt}%
\definecolor{currentstroke}{rgb}{0.000000,0.000000,0.000000}%
\pgfsetstrokecolor{currentstroke}%
\pgfsetdash{}{0pt}%
\pgfpathmoveto{\pgfqpoint{0.711606in}{0.549444in}}%
\pgfpathlineto{\pgfqpoint{5.666777in}{0.549444in}}%
\pgfusepath{stroke}%
\end{pgfscope}%
\end{pgfpicture}%
\makeatother%
\endgroup%

    \caption{Contour plot of the energy on the plane defined by three random patterns. Three minima -- each of which correspond to the three patterns used for building the plane -- can be observed.}
\end{figure}

\subbibliography
\end{document}